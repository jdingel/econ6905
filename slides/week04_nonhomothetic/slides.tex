\documentclass[11pt,notes=hide,aspectratio=169]{beamer}
%Jonathan Dingel; PhD trade course

% PACKAGES
\usepackage{graphics}  % Support for images/figures
\usepackage{graphicx}  % Includes the \resizebox command
\usepackage{url}	   % Includes \urldef and \url commands
\usepackage{soul}      % Includes the underline \ul command
%\usepackage{framed}	   % Includes the \framed command for box around text
\usepackage{booktabs} %\toprule,\bottomrule
%\usepackage{natbib}
\usepackage{bibentry}  % Includes the \nobibliography command
\usepackage{bbm}       %
%\usepackage{pgfpages}  %Supports "notes on second screen" option for beamer
\usepackage{verbatim}  %Supports comments
\usepackage{tikz}		%Supports graphing/drawing
\usepackage{pgfplots} %Supports graphing/drawing
\usepackage{amsfonts}  % Lots of stuff, including \mathbb 
\usepackage{amsmath}   % Standard math package
\usepackage{amsthm}    % Includes the comment functions
\usepackage{physics}

% CUSTOM DEFINITIONS
\def\newblock{} %Get beamer to cooperate with BibTeX
\linespread{1.2}
\hypersetup{backref,pdfpagemode=FullScreen,colorlinks=true,linkcolor=blue,urlcolor=blue}
\newtheorem{proposition}{Proposition}
\newtheorem{assumption}{Assumption}
\newtheorem{condition}{Condition}

% IDENTIFYING INFORMATION
\title{Topics in Trade}
\author{Jonathan I. Dingel}
\date{Fall \the\year}

% BEAMER TEACHING STUFF
\setbeamertemplate{navigation symbols}{}  %Turn off navigation bar

% THEMATIC OPTIONS
\definecolor{columbiablue}{RGB}{185,217,235}  %Columbia blue defined at https://visualidentity.columbia.edu/branding
\definecolor{columbiadarkblue}{RGB}{0,48,135}  %Columbia dark blue defined at https://visualidentity.columbia.edu/branding
\setbeamercovered{transparent=5}
\setbeamercolor{frametitle}{fg=columbiadarkblue}
\setbeamercolor{item}{fg=columbiadarkblue}
\usefonttheme{serif}
\setbeamercolor{button}{bg = white,fg = columbiadarkblue}
\setbeamercolor{button border}{fg = columbiadarkblue}

\setbeamertemplate{footline}{\begin{center}\textcolor{gray}{Dingel -- Topics in Trade -- Week 4 -- \insertframenumber}\end{center}}
\begin{document}
% -----------------------------------------
\begin{frame}[plain]
\begin{center}
\large
\textcolor{columbiadarkblue}{ECON G6905\\
Topics in Trade\\ 
Jonathan Dingel\\
Spring \the\year, Week 4}
\vfill 
\includegraphics[width=0.4\textwidth]{../images/Columbia_logo.png}
\end{center}
\end{frame}
% -----------------------------------------
\begin{frame}{Today: Non-homothetic preferences in trade}
\begin{itemize}
	\item Are income levels relevant for consumer expenditure allocations?
	\item Non-homothetic preferences: The income elasticity of demand differs from one
	\item {\small \href{https://www.nber.org/papers/w8675}{Harrigan (2001)}: ``The assumption of identical homothetic preferences is implausible, and uninteresting in the sense that there is no real theory behind it''}
	\item Linder posited that income composition affects demand composition, which is what every household budget study finds (\href{http://www.cambridge.org/us/academic/subjects/economics/microeconomics/economics-and-consumer-behavior?format=PB&isbn=9780521296762}{Deaton and Muellbauer 1980})
	\item There may not be a representative consumer when income distribution matters
	\item This heterogeneous demand might predict intersectoral or intrasectoral variation in trade flows
	\item {High-income economies might have comparative advantage in income-elastic goods due to a coincidental correlation or a causal home-market effect\par}
\end{itemize}
\end{frame}
% -----------------------------------------
\begin{frame}{Today}
\begin{itemize}
\item Caron, Fally, Markusen (2014): Intersectoral differences and correlation
\item Matsuyama (2019): Intersectoral differences and causation
\item A tangent: Structural transformation in closed and open economies
\item Dingel (2017): Intrasectoral (quality) differences and causation
\end{itemize}
\end{frame}
% -----------------------------------------
\begin{frame}{Caron, Fally, Markusen (2014)}
``\href{http://qje.oxfordjournals.org/content/129/3/1501}{International Trade Puzzles - A Solution Linking Production and Preferences}''
\begin{enumerate}
\item The factor content of a country's production and consumption are very similar
(factor-service trade is too small; Trefler's ``case of the missing trade'')
\item Aggregate trade-to-GDP ratios are low (``home bias puzzle'').
\item Bilateral trade between countries with similar incomes is higher than predicted by supply-driven theories.
\item Trade-to-GDP ratios are higher in higher-income countries.
\end{enumerate}
Income elasticities can help explain these patterns
\begin{itemize}
\item Sectoral income elasticity and skill intensity are positively correlated: countries have comparative advantage in what they demand (explains 1, 2, 3)
\item Sectoral income elasticity and tradability are positively correlated: high-income countries have higher trade-to-GDP ratios
\end{itemize}
\end{frame}
% -----------------------------------------
\begin{frame}{Caron, Fally, Markusen (2014): Coincidental correlation}
\begin{itemize}
	\item Constant relative income elasticity (CRIE) preferences (Hanoch 1975):
	$$U = \sum_k \alpha_{1,k} Q_{k}^{\frac{\sigma_k-1}{\sigma_k}} \implies x_{nk} = \lambda_n^{-\sigma_k} \alpha_{2,k}P_{nk}^{1-\sigma_k}$$
	\item EK-CDK-CP multi-sector production with intermediates (no HME)
	\item Sectoral gravity (with EK notation for $X_{ni}$ subscript order):
	\begin{align*}
	\frac{X_{nik}}{X_{nk}}
	= 
	\frac{S_{ik} d_{nik}^{-\theta_k}}{\Phi_{nk}} ; \quad \Phi_{nk} = \sum_i S_{ik} d_{nik}^{-\theta_k}
	\end{align*}
	\item Gravity with zero trade costs ($d_{nik}=1$) and no intermediates shows interaction of supply and demand characteristics
\begin{align*}
	\frac{X_{ni}}{X_n} 
	= 
	\sum_{k} \frac{X_{nik}}{X_{nk}} \frac{X_{nk}}{X_n} 
	=
	\sum_k \underbrace{\left(\frac{S_{ik}}{\sum_j S_{jk}}\right)}_{\textnormal{supply shifters}} \underbrace{\left(\frac{\alpha_{4,k} \lambda_n^{-\sigma_k}} {\sum_{k'} \alpha_{4,k'} \lambda_n^{-\sigma_{k'}}} \right)}_{\textnormal{demand shifters}}
\end{align*}
\end{itemize}
\end{frame}
% -----------------------------------------
\begin{frame}{Caron, Fally, Markusen (2014): Demand estimation}
\linespread{1.0}
\begin{itemize}
	\item Need estimated model to distinguish roles of trade costs and nonhomotheticity
	\item Use GTAP data on 94 countries and 56 broad sectors (and 5 factor inputs)
	\item Sectoral gravity to obtain structural proxy for $\Phi_{nk}$
	\begin{equation*}
	\ln X_{nik} = \underbrace{\ln S_{ik}}_{ik \textnormal{ FE}} - \theta_k \underbrace{\ln d_{nik}}_{\textnormal{proxies}} + \underbrace{\ln\left(\frac{X_{nk}}{\Phi_{nk}} \right)}_{nk \textnormal{ FE}}
	\Rightarrow
	\hat{\Phi}_{nk} = \sum_i \exp \left( \widehat{\ln S_{ik}} - \hat{\theta}_k \ln d_{nik} \right)
	\end{equation*}
	\item Price indices closely related to inward MR, $P_{nk} = \alpha_{3,k} \Phi_{nk}^{-1/\theta_k}$
	\item Sectoral expenditures to estimate $\{\sigma_k, \lambda_n, \alpha_{5,k}, \theta_k\}$ via constrained NLS
	\begin{align*}
	\ln x_{nk} = {-\sigma_k}\ln \lambda_n + \ln \alpha_{5,k} + \frac{\sigma_k-1}{\theta_k} \ln \hat{\Phi}_{nk} + \varepsilon_{nk}^{D} 
	& \qquad (24)
	\\
	\sum_k \exp \left[ {-\sigma_k}\ln \lambda_n + \ln \alpha_{5,k} + \frac{\sigma_k-1}{\theta_k} \ln \hat{\Phi}_{nk} \right] = e_n
	& \qquad (25)
	\end{align*}
	minimize sum of $(\epsilon_{nk}^{D})^{2}$ while satisfying equation (25)
\end{itemize}
\end{frame}
% -----------------------------------------
\begin{frame}{Caron, Fally, Markusen (2014): Application to puzzles}
\begin{itemize}
	\item Compute factor intensities as the ratio of skilled labor, capital, or natural resources to total labor input
	\item Correlation between a good's income elasticity of demand and its skilled-labor intensity in production is about 50\%
	\item Income-elastic goods are systematically more traded (variation in the elasticity of trade costs to distance rather than $\theta_k$)
	\item Quantitative account of the four puzzles (missing factor content, income-openness relationship, etc.)
\end{itemize}
\end{frame}
% -----------------------------------------
\begin{frame}{Causal HME in theory: Matsuyama (2019)}
\begin{itemize}
	\item \href{https://doi.org/10.3982/ECTA13765}{Matsuyama (2019)}: Intersectoral specialization, causal home-market effect
	\item Continuum of SDS sectors -- lower tier looks like Krugman (1980)
	\item Direct implicitly additive CES utility with sector-specific income elasticity parameters (see \href{https://ideas.repec.org/a/ecm/emetrp/v43y1975i3p395-419.html}{Hanoch 1975} or \href{http://faculty.wcas.northwestern.edu/~kmatsu/}{Matsuyama's Canon lecture})
	\item Ranking sectors by income elasticities, the economy with higher standard of living is a net exporter in higher-ranked sectors
	\item Comparative statics for productivity improvements and trade costs
\end{itemize}
\end{frame}
% -----------------------------------------
\begin{frame}{Brief discussion of implicit additive separability}
\begin{itemize}
	\item Recall utility functions $U(x)$ and indirect utility $V(p,y)$
	\item Explicitly additively separable function $U: \mathbb{R}_{+}^{J} \to \mathbb{R}$ is $U(x) = \sum_j u_j(x_j)$
	\item Bergson (\href{https://www.jstor.org/stable/2967658}{[Burk] 1936}): If $U$ is quasi-concave, increasing, and explicitly additively separable,
	then it is homothetic if and only if
	$u_j(x_j) = \alpha_j \frac{x_j^{\rho}}{\rho} + \beta_j \ \forall j$
	\item Pigou's Law (1910) (\href{https://www.jstor.org/stable/2231258}{Deaton 1974}): If the (direct) utility function is additively separable, then the income elasticity of a good is (approximately) proportionate to the price elasticity of that good
	\item An implicitly additively separable utility function is
	$\sum_{j=1}^{J} f_j\left(x_j; U\right)=1$
	\item \href{https://ideas.repec.org/a/mcb/jmoncb/v27y1995i4p1241-77.html}{Kimball (1995)} preferences are popular example:
	$$
	\min_{y_j} \int_{0}^{1} p_j y_j \textrm{d}j \text{ s.t. } \int_{0}^{1} \Upsilon \left(\frac{y_j}{Y}\right) \textrm{d}j = 1 
	\quad \Upsilon(1) = 1, \Upsilon^{'}>0, \Upsilon^{''}<0
	$$
\end{itemize}
\end{frame}
% -----------------------------------------
\begin{frame}{Matsuyama (2019): Non-homothetic CES preferences}
``Non-homothetic CES'' is an implicitly additive direct utility function with a constant elasticity of substitution (Hanoch 1975). Popularized by Comin, Lashkari, Mestieri (2021)
\begin{equation*}
\left[\int_{I}\left(\beta_{s}\right)^{1/\eta}\left(\tilde{U}^{k}\right)^{\frac{\epsilon(s) - \eta}{\eta}}\left(C^k_s\right)^{\frac{\eta-1}{\eta}}\textrm{d} s\right]^{\frac{\eta}{\eta-1}}
\equiv 1
\qquad
\beta_{s} > 0, \eta > 0, \eta \neq 1, \frac{\epsilon(s)-\eta}{1-\eta} >0
\end{equation*}
$\epsilon(s) = 1 \ \forall s$ is homothetic CES.\\
$\epsilon(s)$ increasing in $s$ means
$\left(\beta_{s}\right)^{1/\eta}\left(\tilde{U}^{k}\right)^{\frac{\epsilon(s) - \eta}{\eta}}$
is isoelastic in $\tilde{U}^{k}$ and LSM in $(s,\tilde{U}^{k})$
\smallskip
Other applications:
\begin{itemize}
\item {\small Finlay and Williams ``Housing Demand, Inequality, and Spatial Sorting'' (2023)}
\item {\small Comin, Danieli, Mestieri ``\href{https://www.nber.org/papers/w27455}{Income-driven Labor Market Polarization}'' (2020)}
\end{itemize}
Go to \href{https://tinyurl.com/MatsuyamaEngeldeck}{Matsuyama's 2018 slide deck}
\end{frame}
% -----------------------------------------
\begin{frame}{Structural transformation in closed and open economies}
A warning to empiricists (Matsuyama \textit{JEEA} 2009)
\begin{quote}
This paper presents a simple model of the world economy, in which productivity gains in manufacturing are responsible for the global trend of manufacturing decline, and yet, in a cross-section of countries, faster productivity gains in manufacturing do not necessarily imply faster declines in manufacturing.
In doing so, it aims to draw attention to the common pitfall of using the cross-country evidence to test a closed economy model, and argues for a global perspective; in order to understand cross-country patterns of structural change, one needs a world economy model in which the interdependence across countries is explicitly spelled out.
\end{quote}
\end{frame}
% -----------------------------------------
\begin{frame}{Matsuyama (\textit{JET} 1992) overview}
{\small ``Agricultural productivity, comparative advantage, and economic growth''}
\begin{itemize}
\item Model of endogenous growth driven by learning by doing in the manufacturing sector
\item Income elasticity of demand for agricultural output $<1$
\item Closed economy: Agricultural productivity raises growth
\item Small open economy: Agricultural productivity lowers growth
\end{itemize}
\end{frame}
% -----------------------------------------
\begin{frame}{Matsuyama (1992) setup/mechanics}
Learning by doing in manufacturing
\begin{align*}
X_t^A 
&=
A G(1-n_t)
\quad
X_t^M 
=
M_t F(n_t)
\quad
\dot{M}_t 
=
\delta X_t^M \quad \delta>0
\\
A G'(1-n_t)
&=
p_t M_t F'(n_t)
&(4)
\end{align*}
Stone-Geary preferences with necessity $\gamma$ where $AG(1)>\gamma L > 0$.
\begin{align*}
C_t^A &= \gamma L + \beta p_t C_t^M 
& (7)
\end{align*}
Closed-economy has $C_t^M=X_t^M$ and $C_t^A=X_t^A$. Combining with equations (4) and (7), equilibrium $n_t$ satisfies
$$
\phi(n_t)
\equiv
G(1-n_t)-\beta G'(1-n_t)F(n_t)/F'(n_t)
=
\gamma L / A
$$
Unique solution $n_t = v(A)$ with $v'()>0$.
Higher $A$ raises \textit{level} of manufacturing and thus economic growth \textit{rate}
\end{frame}
% -----------------------------------------
\begin{frame}{Matsuyama (1992): small open economy}
World economy has $A^{*}$ and $M_{0}^{*}$.
World relative price satisfies
$$
A^{*} G'(1-n^{*})
=
p_t M_t^{*} F'(n^{*})
$$
SOE allocation must satisfy
$$
A G'(1-n_t)
=
p_t M_t F'(n_t)
$$
Take ratio, set $t=0$, find $n_0 \lessgtr n^{*}$.
\smallskip
Growth rate result: When the Home initially has a comparative advantage in manufacturing,
its manufacturing productivity will grow faster than the rest of the world
and accelerate over time.
\smallskip
{\small ``a caution to the readers of the recent empirical work, which, in order to test implications of closed economy models of endogenous growth,
uses cross-country data and treats all economies in the sample as if they were isolated from each other''\par}
\end{frame}
% -----------------------------------------
\begin{frame}{Quality specialization}
\begin{itemize}
	\item High-income countries export products at higher prices
	\item High-income countries import products at higher prices
	\item Is it correlated comparative advantage or a causal home-market effect?
	\item \href{https://academic.oup.com/restud/article/84/4/1551/2684498}{Dingel (2017)} extends \href{http://www.journals.uchicago.edu/doi/abs/10.1086/662628}{Fajgelbaum, Grossman, Helpman (2011)} to empirically pursue this question
\end{itemize}
\end{frame}
% -----------------------------------------
\begin{frame}{Next week}
Up next: Agglomeration economies
\end{frame}
% -----------------------------------------
\end{document}
