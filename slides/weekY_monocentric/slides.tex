\documentclass[11pt,notes=hide,aspectratio=169]{beamer}
%Jonathan Dingel; PhD trade course

% PACKAGES
\usepackage{graphics}  % Support for images/figures
\usepackage{graphicx}  % Includes the \resizebox command
\usepackage{url}	   % Includes \urldef and \url commands
\usepackage{soul}      % Includes the underline \ul command
%\usepackage{framed}	   % Includes the \framed command for box around text
\usepackage{booktabs} %\toprule,\bottomrule
%\usepackage{natbib}
\usepackage{bibentry}  % Includes the \nobibliography command
\usepackage{bbm}       %
%\usepackage{pgfpages}  %Supports "notes on second screen" option for beamer
\usepackage{verbatim}  %Supports comments
\usepackage{tikz}		%Supports graphing/drawing
\usepackage{pgfplots} %Supports graphing/drawing
\usepackage{amsfonts}  % Lots of stuff, including \mathbb 
\usepackage{amsmath}   % Standard math package
\usepackage{amsthm}    % Includes the comment functions
\usepackage{physics}

% CUSTOM DEFINITIONS
\def\newblock{} %Get beamer to cooperate with BibTeX
\linespread{1.2}
\hypersetup{backref,pdfpagemode=FullScreen,colorlinks=true,linkcolor=blue,urlcolor=blue}
\newtheorem{proposition}{Proposition}
\newtheorem{assumption}{Assumption}
\newtheorem{condition}{Condition}

% IDENTIFYING INFORMATION
\title{Topics in Trade}
\author{Jonathan I. Dingel}
\date{Fall \the\year}

% BEAMER TEACHING STUFF
\setbeamertemplate{navigation symbols}{}  %Turn off navigation bar

% THEMATIC OPTIONS
\definecolor{columbiablue}{RGB}{185,217,235}  %Columbia blue defined at https://visualidentity.columbia.edu/branding
\definecolor{columbiadarkblue}{RGB}{0,48,135}  %Columbia dark blue defined at https://visualidentity.columbia.edu/branding
\setbeamercovered{transparent=5}
\setbeamercolor{frametitle}{fg=columbiadarkblue}
\setbeamercolor{item}{fg=columbiadarkblue}
\usefonttheme{serif}
\setbeamercolor{button}{bg = white,fg = columbiadarkblue}
\setbeamercolor{button border}{fg = columbiadarkblue}

\setbeamertemplate{footline}{\begin{center}\textcolor{gray}{Dingel -- Topics in Trade -- Week 9 -- \insertframenumber}\end{center}}
\begin{document}
% -----------------------------------------
\begin{frame}[plain]
\begin{center}
\large
\textcolor{columbiadarkblue}{ECON G6905\\
Topics in Trade\\ 
Jonathan Dingel\\
Spring \the\year, Week 9}
\vfill 
\includegraphics[width=0.4\textwidth]{../images/Columbia_logo.png}
\end{center}
\end{frame}
% -----------------------------------------
\begin{frame}{This week: The canonical urban model}
\begin{itemize}
\item Monocentric city model
\item Contrast with quantitative spatial models
\item Embedded in a system of cities
\end{itemize}
\end{frame}
% -----------------------------------------
\begin{frame}{Overview of monocentric city models}
The monocentric city model is spatial equilibrium in the simplest commuting geography: a single, exogenous central workplace
\begin{itemize}
\item Basic 1: linear geography with continuum of locations and unit housing demand
\item Basic 2: linear geography with continuum of locations and continuum of identical individuals 
\item Multiple commuting modes/technologies (LeRoy and Sonstelie 1983)
\item Endogenize firm location (Fujita and Ogawa 1982)
\end{itemize}
\end{frame}
% -----------------------------------------
\begin{frame}{The basic monocentric city model}
All employment is at the city center, $\tau = 0$.
Commuting costs rise with distance to the center, $\tau$.
What is the equilibrium rent at $\tau$?
Better presentations of this material:
\begin{itemize}
\item Video: Kevin Murphy. ``\href{https://www.youtube.com/playlist?list=PLp2AOdiHSxGeV8AwAwm7nye_2QsQHiC6D}{Location Choice: An Introduction to Compensating Differences}''
\item Handbook chapter: Gilles Duranton and Diego Puga. Section 8.2 of ``\href{https://doi.org/10.1016/B978-0-444-59517-1.00008-8}{Urban Land Use}'' 
\end{itemize}
Kevin uses unit housing demand, opportunity cost of time, and heterogeneous workers.
Gilles and Diego use general preferences, commuting costs in units of the numeraire good, and homogenous workers.
The beauty of the economic logic comes through in both presentations,
but these details matter for empirical investigations.
\end{frame}
% -----------------------------------------
\begin{frame}{Basic monocentric city model: Setup}
Like Murphy, our formulation in Davis and Dingel (2020)
uses unit housing demand, opportunity cost of time, and heterogeneous workers.
\begin{itemize}
\item Locations within cities vary in their desirability $\tau$; schedule is $S(\tau)$ 
\item Relative to DD (2020), assume only one city and ignore sectoral choice $\sigma$
\item Utility is consumption of the numeraire final good, which is income minus locational cost:
\begin{equation*}
U(c,\tau;\omega)
=
T(\tau)G(\omega)-r(c,\tau)
\end{equation*}
\item In DD (2020), $T'(\tau)<0$ may be interpreted as commuting to CBD, proximity to
productive opportunities, or consumption value
\item Working time $T(\tau) = 24 - L - \tau$ and $G(\omega)$ is wage of skill $\omega$
\item With uniform housing density: Linear city $S(\tau) = \tau$; disc city $S(\tau) = \pi \tau^{2}$
\item More skilled are more willing to pay for more attractive locations
\textcolor{gray}{(Net income is supermodular in $G(\omega)$ and $T(\tau)$)}
\end{itemize}
\end{frame}
% -----------------------------------------
\begin{frame}{Basic monocentric city model: Equilibrium conditions}
Individuals maximize their utility by their choices of city, location, and sector such that
\begin{equation*}
f(c,\omega,\tau)>0\iff\{c,\tau\}\in\arg\max U(c,\tau;\omega)
\end{equation*}
Profit maximization by absentee landlords engaged in Bertrand competition causes unoccupied locations to have rental prices of zero,
\begin{equation*}
r(c,\tau)\times\left(S'(\tau)-L \int_{\omega\in\Omega}f(\omega,c,\tau)\textrm{d}\omega\right)=0\ \forall c\ \forall\tau
\end{equation*}
Single (closed) city has an exogenous population $L(c)$ and skill distribution $F(\omega)$.
\medskip
Prior to using a differential equation,
sketch the outcome for homogenous workers using the dual approach 
$\bar{U} = U(\tau; \omega) = T(\tau) G(\omega) - r(\tau)$.
Then consider discrete types of workers.
\textcolor{gray}{(The dual approach is powerful here, as it is in neoclassical trade theory)}
\end{frame}
% -----------------------------------------
\begin{frame}{Basic monocentric city model: Equilibrium solution}
\textbf{Lemma 6}.
In autarkic equilibrium,
there exists a continuous and strictly decreasing locational assignment
function $N:\bar{\mathcal{T}}(c)\to\Omega$ such that $f(\omega,c,\tau)>0\iff N(\tau)=\omega$,
$N(0)=\bar{\omega}$ and $N(\bar{\tau}(c))=\underline{\omega}$.
This assignment function is obtained by equating supply and demand
of locations: 
\begin{equation*}
S(\tau) 
=
L\int_{0}^{\tau}\int_{\omega\in\Omega}f(\omega,c,x,\sigma)\textrm{d}\omega\textrm{d} x
\implies
N(\tau) 
=
F^{-1}\left(\frac{L(c)-S(\tau)}{L(c)}\right)
\end{equation*}
\textbf{Lemma 7}.
In autarkic equilibrium, $r(c,\tau)$
is continuously differentiable on $\tau\geq0$ and given by \textup{$r(c,\tau)=-A(c)\int_{\tau}^{\bar{\tau}(c)}T'(t)G(N(t))\textrm{d} t$
for $\tau\leq\bar{\tau}(c)$.}
The properties of interest in a competitive equilibrium are characterized
by the assignment function $N:\bar{\mathcal{T}}(c)\to\Omega$.
\end{frame}
% -----------------------------------------
\begin{frame}{Rent and density gradients}
The monocentric geography produces a negative, convex rent gradient
\only<1>{(works qualitatively, what about quantitatively?)}
\only<2>{and 
not yet an adequate account of the density gradient}
\begin{center}
\only<1>{
    \includegraphics[height=0.65\textheight]{../images/CombesDurantonGobillon2019_fig1ab.pdf}\\
    {\scriptsize (a) Paris (b) Toulouse; (.1) house prices (.2) land prices\\
	Combes, Duranton, Gobillon - \href{https://doi.org/10.1093/restud/rdy063}{The Costs of Agglomeration: House and Land Prices in French Cities}\par}
}
\only<2>{\includegraphics[height=0.70\textheight]{../images/DurantonPuga2015_fig1.pdf}\\
    {\scriptsize \href{https://doi.org/10.1016/B978-0-444-59517-1.00008-8}{Duranton and Puga (2015)}: ``Since monocentricity can always be rejected, the more interesting question is: How monocentric are cities?''\par}
}
\end{center}
\only<1>{}
\end{frame}
% -----------------------------------------
\begin{frame}{Introducing housing quantities}
Homogeneous workers, commuting costs in terms of goods, and general preferences:
$$
\max U(q,z) \text{ s.t. } r(\tau) q + z + \tau \leq w 
$$
The indirect utility function is $V(r(\tau),w - \tau)$.
Spatial equilibrium means
$$
V(r(\tau),w - \tau) = \bar{V}
\implies
\frac{\partial V}{\partial r}\frac{\textrm{d} r}{\textrm{d} \tau} 
+
\frac{\partial V}{\partial y}\frac{\partial y}{\partial \tau}
=
0
\implies
\frac{\textrm{d} r}{\textrm{d} \tau}
=
\frac{\partial V/\partial y}{\partial V/\partial r}
=
\frac{-1}{q(\tau)}
< 0
$$
by Roy's identity.
The Alonso-Becker insight: transform consumption problem into production problem by putting commuting costs in the budget constraint.
{\footnotesize 
\textcolor{gray}{(Note $\tau$ is commuting cost, so $\frac{\partial y}{\partial \tau} = -1$. If using distances, $\frac{\partial (w-\tau)}{\partial \text{distance}}$ is in numerator.)}
\par}
Substitution effect from Hicksian demand $h$ for housing:
$$
\frac{\partial q(r(\tau),\bar{V})}{\partial \tau}
=
\frac{\partial q(r(\tau),\bar{V})}{\partial r(\tau)}
\frac{\textrm{d} r}{\textrm{d} \tau}
\geq 0
$$
{\small 
See Brueckner (1987) and Duranton and Puga (2015) chapters for details,
housing supply, and 3 more gradients:
land price, housing capital intensity, population density.
\par}
\end{frame}
% -----------------------------------------
\begin{frame}{Density gradients have flattened over time}
\vspace{-4mm}
$$\ln D (\tau) = \ln D(0) - \gamma \tau$$
\begin{center}
\includegraphics{../images/Macauley1985_tab3.pdf}\\ \vspace{-2mm}
\href{https://doi.org/10.1016/0094-1190(85)90021-X}{Macauley (1985)}
\end{center}
Population density gradient is flatter than employment density gradient
See Ken Jackson's \href{https://en.wikipedia.org/wiki/Crabgrass_Frontier}{\textit{Crabgrass Frontier: The Suburbanization of the United States}}
\end{frame}
% -----------------------------------------
\begin{frame}{Lower commuting costs flatten density gradients}
Baum-Snow - \href{https://doi.org/10.1162/qjec.122.2.775}{Did Highways Cause Suburbanization?} (\textit{QJE} 2007)
\begin{itemize}
\item 1950-1990: Central city populations $\downarrow 17\%$, metro area populations $\uparrow 72\%$
\item Use 1947 interstate highway plan as instrumental variable for number of rays emanating from central city constructed between 1950 and 1990
\item Regressing 1950-1990 change in rays on 1940-1950 change in log MSA population yields significant positive coefficient; regressing planned rays yields insignificant negative point estimate
\item 2SLS regression of central city population change on change in rays (plus some extra calculations) says highways can explain about 1/3 of the observed decline
\end{itemize}
Not just roads: \href{https://doi.org/10.1016/j.jue.2018.09.002}{Gonzalez-Navarro, Turner (2018)}: subways $\to$ decentralize\\
These findings match the monocentric model's story about commuting costs, but
\begin{itemize}
\item this model doesn't let jobs decentralize
\item lower commuting costs do not increase total city population (much)
\end{itemize}
\end{frame}
% -----------------------------------------
\begin{frame}{Multiple transportation technologies}
\begin{itemize}
\item Consider a set of transportation technologies with a trade-off between fixed and variable costs (as a function of distance).
\item For example, walking has the highest variable cost and no fixed cost,
buses have a lower variable cost and a higher fixed cost, and cars
have the lowest variable cost and the highest fixed cost.
\item This makes $\tau$ as a function of distance the upper envelope of these technologies' costs as a function of distance.
\end{itemize}
\end{frame}
% -----------------------------------------
\begin{frame}{Multiple transport modes and multiple income levels}
LeRoy and Sonstelie (1983):
Use the monocentric city model to think about changing income gradients
(poor centers and rich suburbs vs urban revivals of 1970s and 2000s)
\begin{itemize}
\item Contrast cars and buses in terms of fixed monetary cost $f^i$, variable monetary cost $v^i$, and variable time cost $t^i$. Total cost is $f^i + (w t^i + c^i) d$ where $d$ is distance.
\item Assume $f^{\text{car}} > f^{\text{bus}}$, $v^{\text{car}} > v^{\text{bus}}$, $t^{\text{car}} < t^{\text{bus}}$.
\item If someone uses cars, it will be the high-wage workers (higher value of time).
\end{itemize}
Bid-rent function is maximum price one would pay for housing at distance $d$ while achieving utility $\bar{u}$:
$$
\Psi(w,t^{i},d,\bar{u})
=
\frac{1}{\bar{h}} \left(w - w t^i d - f^i - u^{-1}(\bar{u})\right)
$$
where $\bar{h}$ is fixed housing consumption
[see \href{https://matthewturner.org/ec2410/lectures/4_LeRoySonstelie_v1.pdf}{Matt Turner's notes}]
\end{frame}
% -----------------------------------------
\begin{frame}{LeRoy and Sonstelie (1983): Possible equilibria}
\only<1>{A world without cars (``paradise'')}
\only<2>{A world with cars has rich at center and in suburbs}
\only<3>{A center without rich if they demand more housing ($\bar{h}_r > \bar{h}_p$)}
\begin{center}
\only<1>{
\includegraphics[height=0.70\textheight]{../images/Turner2023_fig1.pdf}
}
\only<2>{
\includegraphics[height=0.65\textheight]{../images/Turner2023_fig2.pdf}
}
\only<3>{
\includegraphics[height=0.70\textheight]{../images/Turner2023_fig3.pdf}
}
\end{center}
\only<1>{$w_r > w_p$ means rich bus riders outbid poor bus riders for central locations}
\only<2>{Rich bus riders at center (small $d$):{\scriptsize
$$
\frac{1}{\bar{h}} \left(w_r - w_r t^\text{bus} d - f^\text{bus} - u^{-1}(\bar{u})\right)
>
\frac{1}{\bar{h}} \left(w_r - w_r t^\text{car} d - f^\text{car} - u^{-1}(\bar{u})\right)
\iff
- w_r t^\text{bus} d - f^\text{bus}
>
- w_r t^\text{car} d - f^\text{car}
$$
}}
\end{frame}
% -----------------------------------------
\begin{frame}{The evolution of income gradients and natural amenities as anchors}
\begin{center}
\includegraphics[height=0.90\textheight]{../images/LeeLin2018_fig7.pdf}\\
\vspace{-5mm}
{\scriptsize \href{https://doi.org/10.1093/restud/rdx018}{Lee and Lin (2018)}}
\end{center}
\end{frame}
% -----------------------------------------
\begin{frame}{Urban revival and income sorting}
\begin{center}
\includegraphics[height=0.9\textheight]{../images/CGHH2023_fig1.pdf}
\end{center}
\vspace{-4mm}
{\footnotesize Model with 2 neighborhoods, 2 income levels, and endogenous amenities}
\end{frame}
% -----------------------------------------
\begin{frame}{Agglomeration economies and firm locations}
\begin{itemize}
\item
We might want $w$ at the CBD to depend on the number of workers (agglomeration economies at a single point)
\item
Straightforward to have firms use land at center (disc and donut of land use)
\item
More complex is producing commercial, residential, and mixed-use locations endogenously 
as a function of local productivity spillovers and commuting costs
(Ogawa and Fujita 1980; Fujita and Ogawa 1982; Lucas and Rossi-Hansberg 2002)
\item
For Ogawa and Fujita (1980) and Fujita and Ogawa (1982),
see Fujita and Thisse (2003) monograph and Duranton and Puga (2016) chapter, respectively.
\item Leverage the bid-rent logic of land allocation across multiple types when firms are one of the bidders
\item Monocentric city is outcome when productivity spillovers are large relative to commuting costs
\end{itemize}
\end{frame}
% -----------------------------------------
\begin{frame}{Contrast with quantitative spatial models}
In the canonical model, location does not enter the direct utility function
\begin{itemize}
\item I do not directly care about location, but location matters because time commuting appears in my budget constraint
\item Hallmark of price theory:
transform a consumption problem into a production problem (I work at work or I work as a driver for myself)
\end{itemize}
\href{https://matthewturner.org/papers/unpublished/Thisse_Turner_Ushchev_unp_2021.pdf}{Thisse, Turner, Ushchev (2021)}
contrast quantitative spatial models and canonical urban model
\begin{itemize}
\item QSM features cross-hauling of homogeneous labor, canonical model does not
\item Canonical model: Commuting costs are a centralizing force when employment is concentrated (e.g., the monocentric case)
\item Quantitative spatial models: Commuting costs are a centralizing force even when net labor flows are zero
\end{itemize}
\end{frame}
% -----------------------------------------
\begin{frame}{Cross hauling is prevalent in commuting matrices}
Grubel-Lloyd (1971) index for intraindustry trade with flows $X_{ij}$:
$$
\textrm{GL}_{i}
\equiv
1 - \frac{\vert \sum_{j} X_{ij} - \sum_{j} X_{ji} \vert}{\sum_{j} X_{ij} + \sum_{j} X_{ji}}
$$
\vspace{-3mm}
\begin{center}
\includegraphics[height=0.7\textheight]{../images/grubel_lloyd_counties.png}
\end{center}
\end{frame}
% -----------------------------------------
\begin{frame}{Cross-hauling example: Alameda County CA}
Alameda (home to Oakland) has a Grubel-Lloyd index of 0.995.
\begin{center}
\resizebox{0.7\textwidth}{!}{
\input{../images/Alameda_county.tex}
}
\end{center}
American Community Survey 2006-2010 county-to-county commuting flows
\end{frame}
% -----------------------------------------
\begin{frame}{Cross hauling in NYC tract-to-tract matrix}
Looking at tracts, employment is much more concentrated.
\begin{center}
\includegraphics[height=0.80\textheight]{../images/grubel_lloyd_NYCtracts.png}
\end{center}
\end{frame}
% -----------------------------------------
\begin{frame}{Locational assignments with multiple cities}
Davis and Dingel (2020) solve an assignment model in which locational choice is between and within cities.
The key is to define assignments in terms of a single index and then unveil quantities
\medskip
Let's start by the reviewing differential rents model.
\textcolor{gray}{(see Sattinger \textit{JEL} 1993)}
In the spirit of Ricardo's analysis of rent, start with land and labor:
\begin{itemize}
	\item A plot of land has fertility $\gamma \in\mathbb{R}$
	\item A farmer has skill $\omega \in\mathbb{R}$
	\item Profits are $\pi(\gamma,\omega) = p\cdot y(\gamma,\omega) - r(\gamma)$
\end{itemize}
Which farmer will use which plot of land?
\begin{itemize}
	\item Farmers optimize: $\gamma^{*}(\omega) \equiv \arg\max_{\gamma} \pi(\gamma,\omega)$
	\item Equilibrium prices $r(\gamma)$ must support the equilibrium assignment of farmers to plots
\end{itemize}
\end{frame}
% -----------------------------------------
\begin{frame}{Supermodularity}
\begin{definition}[Supermodularity]
A function $g:\mathbb{R}^n\to\mathbb{R}$ is \emph{supermodular} if $\forall x,x'\in\mathbb{R}^n$
\begin{align*}
g\left(\max\left(x,x'\right)\right) + g\left(\min\left(x,x'\right)\right)\geq g(x) + g(x')
\end{align*}
where $\max$ and $\min$ are component-wise operators.
\end{definition}
\begin{itemize}
	\item Supermodularity means the arguments of $g(\cdot)$ are complements
	\item $g(x)$ is SM in $(x_i,x_j)$ if $g(x_i,x_j;x_{-i,-j})$ is SM 
	\item $g(x)$ is SM $\iff g(x)$ is SM in $(x_i,x_j)$  $\forall i,j$
	\item If $g$ is $C^2$, $\frac{\partial^2 g}{\partial x_i \partial x_j}\geq 0 \iff g(x)$ is SM in $(x_i,x_j)$
\end{itemize}
\end{frame}
% -----------------------------------------
\begin{frame}{Supermodularity implies PAM}
Positive assortative matching:
\begin{itemize}
	\item If $g(x,t)$ is supermodular in $(x,t)$, then $x^{*}(t)\equiv \arg\max_{x\in X} g(x,t)$ is increasing in $t$
	\item If $y(\gamma,\omega)$ is strictly supermodular (fertility and skill are complements), then $\gamma^{*}(\omega)$ is increasing
	\item More skilled farmers are assigned to more fertile land
\end{itemize}
Why? Suppose not:
\begin{itemize}
	\item Suppose $\exists \omega > \omega', \gamma>\gamma'$ where $\gamma' \in \gamma^{*}(\omega), \gamma \in \gamma^{*}(\omega')$
	\item $\gamma' \in \gamma^{*}(\omega) \Rightarrow p\cdot y(\gamma',\omega) - r(\gamma') \geq p\cdot y(\gamma,\omega) - r(\gamma) \ \forall \gamma$
	\item $\gamma \in \gamma^{*}(\omega') \Rightarrow p\cdot y(\gamma,\omega') - r(\gamma) \geq p\cdot y(\gamma',\omega') - r(\gamma') \ \forall \gamma'$
	\item Summing: $p \cdot \left( y(\gamma',\omega) + y(\gamma,\omega') \right) \geq p\cdot \left(y(\gamma,\omega) + y(\gamma',\omega') \right)$
	\item Would contradict strict supermodularity of $y(\cdot)$
\end{itemize}
\end{frame}
% -----------------------------------------
% -----------------------------------------
\begin{frame}{Log-supermodularity (1/2)}
\begin{definition}[Log-supermodularity]
A function $g:\mathbb{R}^n\to\mathbb{R}^{+}$ is \emph{log-supermodular} if $\forall x,x'\in\mathbb{R}^n$
\begin{align*}
g\left(\max\left(x,x'\right)\right)\cdot g\left(\min\left(x,x'\right)\right)\geq g(x)\cdot g(x')
\end{align*}
where $\max$ and $\min$ are component-wise operators.
\end{definition}
\begin{itemize}
	\item Example: $A: \Sigma\times\mathbb{C}\to\mathbb{R}^{+}$, where $\Sigma\subseteq\mathbb{R}$ and $\mathbb{C}\subseteq\mathbb{R}$, with $\sigma>\sigma'$ and $c>c'$
		\begin{align*}
		A(\sigma,c)A(\sigma',c')\geq A(\sigma',c)A(\sigma,c')
		\end{align*}
	\item $g(x)$ is LSM in $(x_i,x_j)$ if $g(x_i,x_j;x_{-i,-j})$ is LSM 
	\item $g(x)$ is LSM $\iff g(x)$ is LSM in $(x_i,x_j)$  $\forall i,j$
	\item $g>0$ and $g$ is $C^2$ $\Rightarrow \frac{\partial^2 \ln g}{\partial x_i \partial x_j}\geq 0 \iff g(x)$ is LSM in $(x_i,x_j)$
\end{itemize}
\end{frame}
% -----------------------------------------
\begin{frame}{Log-supermodularity (2/2)}
Three handy properties:
\begin{enumerate}
\item If $g,h:\mathbb{R}^n\to\mathbb{R}^{+}$ are log-supermodular, then $gh$ is log-supermodular.
\item If $g:\mathbb{R}^n\to\mathbb{R}^{+}$ is log-supermodular, then $G(x_{-i})\equiv \int g(x_i,x_{-i})dx_i$ is log-supermodular.
\item If $g:\mathbb{R}^n\to\mathbb{R}^{+}$ is log-supermodular, then $x_i^* (x_{-i}) \equiv \arg\max_{x_i\in\mathbb{R}} g(x_i,x_{-i})$ is increasing in $x_{-i}$.
\end{enumerate}
\end{frame}
% -----------------------------------------
\begin{frame}{Individual optimization}
Perfectly mobile individuals simultaneously choose
\begin{itemize}
\item A sector $\sigma$ of employment
\item A city with total factor productivity $A(c)$
\item A location $\tau$ (distance from ideal) within city $c$
\end{itemize}
The productivity of an individual of skill $\omega$ is
\[
q(c,\tau,\sigma;\omega)=A(c)T(\tau)H(\omega,\sigma)
\]
Utility is consumption of the numeraire final good, which is income minus locational cost:
\begin{align*}
U(c,\tau,\sigma;\omega) & =q(c,\tau,\sigma;\omega)p(\sigma)-r(c,\tau)\\
 & =A(c)T(\tau)H(\omega,\sigma)p(\sigma)-r(c,\tau)
\end{align*}
\end{frame}
% -----------------------------------------
\begin{frame}{Sectoral choice}
\begin{itemize}
\item Individuals' choices of locations and sectors are separable: 
\begin{align*}
\arg\max_{\sigma}\underbrace{A(c)T(\tau)}_{\textnormal{locational}}\underbrace{H(\omega,\sigma)p(\sigma)}_{\textnormal{sectoral}}-r(c,\tau)=\arg\max_{\sigma}H(\omega,\sigma)p(\sigma)
\end{align*}
\item $H(\omega,\sigma)$ is log-supermodular in $\omega,\sigma$ and strictly increasing in $\omega$
\item Comparative advantage assigns high-$\omega$ individuals to high-$\sigma$ sectors
\item Absolute advantage makes more skilled have higher incomes ($G(\omega)=\max_{\sigma}H(\omega,\sigma)p(\sigma)$
is increasing)
\end{itemize}
\end{frame}
% -----------------------------------------
\begin{frame}{Locational choice}
\begin{itemize}
\item \textcolor{black}{A location's attractiveness ${\color{red}\gamma}=A(c)T(\tau)$
depends on $c$ and $\tau$}
\item $T'(\tau)<0$ may be interpreted as commuting to CBD, proximity to
productive opportunities, or consumption value
\item More skilled are more willing to pay for more attractive locations
\item Equally attractive locations have same rental price and skill type
\item Location in higher-TFP city is farther from ideal desirability
\begin{align*}
\gamma=A(c)T(\tau)=A(c')T(\tau')\\
A(c)>A(c')\Rightarrow\tau>\tau'
\end{align*}
\item Locational hierarchy: A smaller city's locations are a subset of larger
city's in terms of attractiveness: $A(c)T(0)>A(c')T(0)$ 
\end{itemize}
\end{frame}
% -----------------------------------------
\begin{frame}{Single-index assignment function}
\begin{itemize}
\item Label cities from $1$ to $C$ so $A(C)\geq A(C-1)\geq\dots\geq A(2)\geq A(1)$.
\item Denote attractiveness levels occupied in equilibrium by
    $\Gamma\equiv[\underline{\gamma},\bar{\gamma}]$, where $\underline{\gamma}\equiv A(C)T(\bar{\tau}(C)$
    and $\bar{\gamma}\equiv A(C)T(0)$.
\item In equilibrium, there exists a continuous and strictly increasing locational assignment function $K:\Gamma\to\Omega$ such that (i) $f(\omega,c,\tau)>0\iff A(c)T(\tau)=\gamma$ and $K(\gamma)=\omega$, and (ii) $K(\underline{\gamma})=\underline{\omega}$ and $K(\bar{\gamma})=\bar{\omega}$.
\item Denote the supply of locations across all cities combined with
attractiveness $\gamma$ or greater by
\[
S_{\Gamma}(\gamma)=\sum_{c:A(c)T(0)\geq\gamma}S\left(T^{-1}\left(\frac{\gamma}{A(c)}\right)\right).
\]
$S_{\Gamma}(\bar{\gamma})=0$ and $S_{\Gamma}(\underline{\gamma})=L$.
$S_{\Gamma}(\gamma)=L\int_{\gamma}^{\bar{\gamma}}f(K(x))K'(x) \textrm{d} x$,
so $K(\gamma)=F^{-1}\left(\frac{L-S_{\Gamma}(\gamma)}{L}\right)$.
\end{itemize}
\end{frame}
% -----------------------------------------
\begin{frame}{Equilibrium distributions}
\begin{itemize}
\item Skill and sectoral distributions reflect distribution of locational
attractiveness: Higher-$\gamma$ locations occupied by higher-$\omega$
individuals who work in higher-$\sigma$ sectors
\item Locational hierarchy $\Rightarrow$ hierarchy of skills and sectors
\item The distributions $f(\omega,c)$ and $f(\sigma,c)$ are log-supermodular
if and only if the supply of locations with attractiveness $\gamma$
in city $c$, $s(\gamma,c)$, is log-supermodular 
\begin{align*}
s(\gamma,c)= & \begin{cases}
\frac{1}{A(c)}V\left(\frac{\gamma}{A(c)}\right) & \textnormal{if }\gamma\leq A(c)T(0)\\
0 & \textnormal{otherwise}
\end{cases}
\end{align*}
where\textrm{ $V(z)\equiv-\frac{\partial}{\partial z}S\left(T^{-1}(z)\right)$
is }the supply of locations with innate desirability $\tau$ such
that $T(\tau)=z$
\end{itemize}
\end{frame}
% -----------------------------------------
\begin{frame}{When is $s(\gamma,c)$ log-supermodular?}
\begin{proposition}[Locational attractiveness distribution]
\label{prop:LocationDistribution}The supply of locations of attractiveness
$\gamma$ in city $c$, $s(\gamma,c)$, is log-supermodular if and
only if the supply of locations with innate desirability \textrm{\textup{$T^{-1}(z)$}}
within each city, $V(z)$, has a decreasing elasticity.\end{proposition}
\begin{itemize}
\item Links each city's exogeneous distribution of locations, $V(z)$, to
endogenous equilibrium locational supplies $s(\gamma,c)$
\item Informally, ranking relative supplies is ranking elasticities of $V(z)$
\[
s(\gamma,c)\propto V\left(\frac{\gamma}{A(c)}\right)\Rightarrow\frac{\partial\ln s(\gamma,c)}{\partial\ln\gamma}=\frac{\partial\ln V\left(\frac{\gamma}{A(c)}\right)}{\partial\ln z}
\]
\item Satisfied by the canonical von Th\"{u}nen/monocentric geography
\end{itemize}
\end{frame}
% -----------------------------------------
\begin{frame}{Next week}
Next week: Spatial sorting of skills and sectors
\end{frame}
% -----------------------------------------
\end{document}
