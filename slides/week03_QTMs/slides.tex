\documentclass[11pt,notes=hide,aspectratio=169]{beamer}
%Jonathan Dingel; PhD trade course

% PACKAGES
\usepackage{graphics}  % Support for images/figures
\usepackage{graphicx}  % Includes the \resizebox command
\usepackage{url}	   % Includes \urldef and \url commands
\usepackage{soul}      % Includes the underline \ul command
%\usepackage{framed}	   % Includes the \framed command for box around text
\usepackage{booktabs} %\toprule,\bottomrule
%\usepackage{natbib}
\usepackage{bibentry}  % Includes the \nobibliography command
\usepackage{bbm}       %
%\usepackage{pgfpages}  %Supports "notes on second screen" option for beamer
\usepackage{verbatim}  %Supports comments
\usepackage{tikz}		%Supports graphing/drawing
\usepackage{pgfplots} %Supports graphing/drawing
\usepackage{amsfonts}  % Lots of stuff, including \mathbb 
\usepackage{amsmath}   % Standard math package
\usepackage{amsthm}    % Includes the comment functions
\usepackage{physics}

% CUSTOM DEFINITIONS
\def\newblock{} %Get beamer to cooperate with BibTeX
\linespread{1.2}
\hypersetup{backref,pdfpagemode=FullScreen,colorlinks=true,linkcolor=blue,urlcolor=blue}
\newtheorem{proposition}{Proposition}
\newtheorem{assumption}{Assumption}
\newtheorem{condition}{Condition}

% IDENTIFYING INFORMATION
\title{Topics in Trade}
\author{Jonathan I. Dingel}
\date{Fall \the\year}

% BEAMER TEACHING STUFF
\setbeamertemplate{navigation symbols}{}  %Turn off navigation bar

% THEMATIC OPTIONS
\definecolor{columbiablue}{RGB}{185,217,235}  %Columbia blue defined at https://visualidentity.columbia.edu/branding
\definecolor{columbiadarkblue}{RGB}{0,48,135}  %Columbia dark blue defined at https://visualidentity.columbia.edu/branding
\setbeamercovered{transparent=5}
\setbeamercolor{frametitle}{fg=columbiadarkblue}
\setbeamercolor{item}{fg=columbiadarkblue}
\usefonttheme{serif}
\setbeamercolor{button}{bg = white,fg = columbiadarkblue}
\setbeamercolor{button border}{fg = columbiadarkblue}

\setbeamertemplate{footline}{\begin{center}\textcolor{gray}{Dingel -- Topics in Trade -- \semester -- Week 3 -- \insertframenumber}\end{center}}
\begin{document}
% -----------------------------------------
\begin{frame}[plain]
\begin{center}
\large
\textcolor{columbiadarkblue}{ECON G6905\\
Topics in Trade\\ 
Jonathan Dingel\\
\semester, Week 3}
\vfill 
\includegraphics[width=0.4\textwidth]{../images/Columbia_logo.png}
\end{center}
\end{frame}
% -----------------------------------------
\begin{frame}{Today: Quantitative Ricardian models}
We study a class of Ricardian models that make more structural assumptions to facilitate quantitative exercises
\begin{itemize}
	\item \href{https://www.aeaweb.org/articles?id=10.1257/jep.26.2.65}{Eaton and Kortum (2012)} call this ``putting Ricardo to work''
	\item Probabilistic Ricardian models do not attempt to predict which countries make which goods
	\item Akin to Wilson (1980), we focus on what account of trade patterns is needed to conduct certain counterfactuals
	\item Borrow tools from discrete-choice models to deliver closed-form solutions that depend on few key parameters
	\item Over time, we've learned that some of this is not so different -- an Armington or CGE model in Fr\'{e}chet clothing
\end{itemize}
\end{frame}
% -----------------------------------------
\begin{frame}{The logit model of discrete choice}
Individual $i$ considers choice $j$ (see \href{https://eml.berkeley.edu/books/choice2.html}{Train 2009})
\begin{itemize}
	\item Utility $U_{ij} = V_{ij} + \epsilon_{ij}$
	\item Assume error is iid T1EV: $F\left(\epsilon_{ij}\right)=\exp(-\exp(-\epsilon_{ij}))$
	\item Choice probabilities are
	\begin{equation*}\Pr(U_{ij}>U_{ij'} \ \forall j' \neq j) = \frac{\exp(V_{ij})}{\sum_{j'}\exp(V_{ij'})} \end{equation*}
\end{itemize}
Now try a cost-minimization problem with multiplicative error term
\begin{itemize}
	\item Cost $\ln c_{ji} = \ln c_j + \ln \tau_{ji} - \epsilon_{j}$
	\item Least-cost probability
	\begin{align*}
	\Pr(-\ln c_{ji}> -\ln c_{j'i} \ \forall j' \neq j) \\
	= \Pr(\ln c_{ji}<\ln c_{j'i} \ \forall j' \neq j) 
	& 
	= \frac{1/(c_j\tau_{ji})}{\sum_{j'}1/(c_{j'}\tau_{j'i})} \\
	\end{align*}
\end{itemize}
\end{frame}
% -----------------------------------------
\begin{frame}{Eaton \& Kortum (Ecma 2002): Environment}
Before we start: Questions about the big picture?
\begin{itemize}
	\item $N$ countries indexed by $i=1,\dots,N$
	\item Continuum of goods indexed by $u \in [0,1]$ [EK use ``$j$'']
	\item CES preferences
	\begin{equation*}
	U=\left( \int_{0}^{1}Q(u) ^{\frac{\sigma -1}{\sigma }}du\right)
	^{\frac{\sigma }{\sigma -1}}
	\end{equation*}
	\item Trade costs ``$d_{ni}$'' (as in Train book) from $i$ to $n$ (contra ``$\tau_{ij}$'')
	\item One factor of production labor with wage $w_i$ (perhaps intermediate goods too)
	\item $c_i$ is the unit cost of sole (or composite) input in $i$
	\item Perfect competition with good-specific idiosyncratic productivity $z_i(u)$ so that cost of delivering $u$ to $n$ from $i$ is 
	\begin{equation*}c_{ni}(u) = c_i d_{ni} / z_i(u)\end{equation*}
\end{itemize}
\end{frame}
% -----------------------------------------
\begin{frame}{Eaton \& Kortum (2002): Probabilistic technology}
\begin{itemize}
	\item $Z_i(u)$ is drawn independently from a Fr\'{e}chet distribution
	\begin{equation*}
	F_{i}\left( z\right) =\exp \left( -T_{i}z^{-\theta }\right) \text{, \ \ \ \ }%
	T_{i}>0\text{, } \theta>\sigma-1
	\end{equation*}
	\item $T_i$ is distribution's location parameter (shifts absolute advantage for all goods)
	\item $\theta$ is distribution's shape parameter (scope of comparative advantage)
	\item The Fr\'{e}chet distribution is a ``max stable'' distribution, and this is the key property that delivers the results that follow
\end{itemize}
\end{frame}
% -----------------------------------------
\begin{frame}{Price distribution}
The distribution of prices offered by $i$ to $n$ is
\begin{align*}
G_{ni}\left( p\right) 
&=\Pr \left( p_{ni}(u) \leq p\right)
=\Pr \left( \frac{d _{ni}c_{i}}{z_{i}(u) }\leq p\right)
=1-\Pr \left( z_{i}(u) \leq \frac{d _{ni}c_{i}}{p}\right) \\
&=1-\exp \left( -T_{i}\left( d _{ni}c_{i}\right) ^{-\theta}p^{\theta}\right)
\end{align*}
The distribution of minimum prices in $n$ is
\begin{align*}
G_{n}\left( p\right) 
&=1-\prod_{i=1}^{N}\left( 1-G_{ni}\left( p\right) \right)
=1-\prod_{i=1}^{N}\exp \left( -T_{i}\left(d_{ni}c_{i}\right) ^{-\theta}p^{\theta }\right) \\
&=1-\exp \left(-{\sum_{i=1}^{N}T_{i}\left(d_{ni}c_{i}\right) ^{-\theta}}p^{\theta }\right) \\
&=1-\exp \left( -\Phi _{n}p^{\theta }\right)
\end{align*}
where $\Phi_{n}$ summarizes $n$'s ``market access'',
which depends on all partners' technologies, input costs, and bilateral trade costs
\end{frame}
% -----------------------------------------
\begin{frame}{Allocation of purchases}
Probability that $i$ provides good $u$ at the lowest price in $n$ is
\begin{equation*}
\pi_{ni}=\frac{T_{i}\left(d_{ni}c_{i}\right) ^{-\theta}}{\Phi_{n}}
\end{equation*}
\begin{align*}
\pi_{ni}
 = \Pr\left(p_{ni} \leq \min_{i'\neq i} p_{ni'}\right) 
 = \prod_{i' \neq i} \Pr \left(p_{ni'} \geq p_{ni}\right)
 = \prod_{i' \neq i} \left[1 - G_{ni'}(p)\right]
 = \exp\left(-\Phi_{n,\neg i}p_{ni}^{\theta}\right)
\end{align*}
where $\Phi_{n,\neg i} \equiv {\sum_{i' \neq i}T_{i'}\left(d_{ni'}c_{i'}\right) ^{-\theta}}$.
Integrate over all $p$. % with density $dG_{ni}(p)$
\begin{align*}
&\int_{0}^{\infty} \exp\left(-\Phi_{n,\neg i}p^{\theta}\right) dG_{ni}(p) \\
&=
\int_{0}^{\infty} \exp\left(-\Phi_{n,\neg i}p^{\theta}\right) \theta p^{\theta-1} T_{i}\left( d _{ni}c_{i}\right) ^{-\theta} \exp \left( -T_{i}\left( d _{ni}c_{i}\right) ^{-\theta}p^{\theta}\right) dp \\
&=
T_{i}\left(d _{ni}c_{i}\right)^{-\theta}
\int_{0}^{\infty} \exp\left(-\Phi_{n}p^{\theta}\right) \theta p^{\theta-1} dp
= \pi_{ni} \int_{0}^{\infty} dG_n(p) = \pi_{ni}
\end{align*}
\end{frame}
% -----------------------------------------
\begin{frame}{Bilateral import price distributions}
\begin{itemize}
	\item Goods imported by $n$ from $i$ also have price distribution $G_n(p)$.
	\item What is probability $i$ is least-cost supplier given its price is $q$? 
	$\Pr\left(q \leq \min_{i'\neq i} p_{ni'}\right) = \exp\left(-\Phi_{n,\neg i}q^{\theta}\right)$
	\item Joint probability that $i$ is least-cost supplier at price $q$ is
	$\exp\left(-\Phi_{n,\neg i}q^{\theta}\right)d G_{ni}(q)$
	\item Integrate this probability from $0$ to $p$ to find CDF of bilateral import prices
	\begin{equation*}
	\int_{0}^{p} \exp\left(-\Phi_{n,\neg i}q^{\theta}\right)d G_{ni}(q)
	= \pi_{ni} G_n(p)
	\end{equation*}
	\item Since $\pi_{ni}$ is the probability that any good is imported from $i$, the conditional distribution of the price paid by $n$ for goods actually imported from $i$ is $G_n(p)$
\end{itemize}
All action is on the extensive margin of varieties purchased. Price distributions are independent of exporter. Expenditure shares are $\pi_{ni}$.
\end{frame}
% -----------------------------------------
\begin{frame}{The CES price index}
\begin{itemize}
\item The CES price index in country $n$ is 
\begin{align*}
P_{n}^{1-\sigma } 
&=\int_{0}^{1}p_{n}^{1-\sigma }(u) du 
=\int_{0}^{\infty }p^{1-\sigma }dG_{n}\left( p\right) \\
&=\int_{0}^{\infty }p^{1-\sigma }e^{-\Phi _{n}p^{\theta }}\Phi _{n}\theta
p^{\theta -1}dp
\end{align*}
\item Change of variable $x=\Phi _{n}p^{\theta }\Rightarrow
dx=\Phi _{n}\theta p^{\theta -1}dp$
\begin{equation*}
P_{n}^{1-\sigma }
=\int_{0}^{\infty }\left( \frac{x}{\Phi _{n}}\right) ^{\frac{1-\sigma }{\theta }}e^{-x}dx
=\Phi _{n}^{-\frac{1-\sigma }{\theta }}\int_{0}^{\infty }x^{\frac{1-\sigma 
}{\theta }}e^{-x}dx
\end{equation*}
\item Integral is finite if $\theta>\sigma-1$
\begin{equation*}
P_{n} = \Phi _{n}^{-\frac{1}{\theta }} \Gamma \left( \frac{\theta +1-\sigma }{\theta }\right)
\end{equation*}
where $\Gamma()$ is the gamma function, $\Gamma(a)\equiv \int_{0}^{\infty} x^{a-1} \exp(-x)dx$
\end{itemize}
\end{frame}
% -----------------------------------------
\begin{frame}{Equilibrium}
\begin{itemize}
	\item Let $X_{ni}$ be $n$'s expenditure on imports from $i$
	\item $X_n \equiv \sum_{i} X_{ni}$ is $n$'s total expenditure
	\item Since $X_{ni} / X_n = \pi_{ni}$, we obtain a gravity equation
	\begin{equation*} 
	X_{ni} 
	= \frac{T_{i}\left( d _{ni}c_{i}\right) ^{-\theta}}{\Phi_n} X_n 
	= T_{i}c_{i}^{-\theta} \frac{X_n}{\Phi_n} d_{ni}^{-\theta} 
	\end{equation*}
	\item Suppose no intermediate goods so that $c_i = w_i$.
	\item The value of sales by $i$ is $w_i L_i = \sum_{n} X_{ni}$
	\item By budget balance, $w_i L_i = X_i$.
	\item We get a system of equations in the wage vector
		\begin{equation*}w_i L_i = \sum_n \frac{T_{i}\left( d _{ni}w_{i}\right) ^{-\theta}}{\sum _j T_{i}\left( d _{nj}w_{ij}\right) ^{-\theta}} w_n L_n
		\end{equation*}
\end{itemize}
We will discuss gravity more in week 4. Note the similarity to Armington system of equations.
\end{frame}
% -----------------------------------------
\begin{frame}{Estimating the trade elasticity $\theta$}
\begin{columns}
\begin{column}{.60\textwidth}
Write (relative) bilateral trade flows as
\begin{align*}
\frac{X_{ni}/X_{n}}{X_{ii}/X_{i}} 
&=
\left( \frac{P_{i}d_{ni}}{P_{n}}\right) ^{-\theta} \\
\ln \left(X_{ni}X_{i}/X_{n} X_{ii} \right)
&=
-\theta \ln P_{i}+\theta \ln P_{n}-\theta \ln d_{ni}
\end{align*}
\end{column}
\begin{column}{.38\textwidth}
The trade elasticity $-\theta$ governs how bilateral trade flows respond to bilateral trade costs.\\
Note the absence of preference parameter $\sigma$.
\end{column}
\end{columns}
\begin{columns}
\begin{column}{.60\textwidth}
\includegraphics[width=\textwidth]{../images/EatonKortum2002_fig1.pdf}
\end{column}
\begin{column}{.38\textwidth}
A proxy for $d_{ni}$ (e.g., distance) won't deliver an estimate of $\theta$
\end{column}
\end{columns}
\end{frame}
% -----------------------------------------
\begin{frame}{Estimating the trade elasticity $\theta$}
Eaton and Kortum (2002) infer $d_{ni}$ from price data for 50 goods and a price-differential inequality:
$%\begin{equation*}
{p_{n}(u)} / {p_{i}(u)}\leq d_{ni} 
$%\end{equation*}
\begin{equation*}
\widehat{\ln d_{ni}}
=
\max2_{u}\left\{ \ln p_{n}(u) -\ln p_{i}(u) \right\}
\end{equation*}
\begin{columns}
\begin{column}{.55\textwidth}
\includegraphics[width=\textwidth]{../images/EatonKortum2002_fig2.pdf}
\end{column}
\begin{column}{.43\textwidth}
\begin{itemize}
\item EK (2002) estimate $\hat{\theta} =8.28$.
\item \href{https://ideas.repec.org/a/eee/inecon/v92y2014i1p34-50.html}{Simonovska and Waugh (2014)} refine max estimator of the inequality: $\hat{\theta} \in \left[3,5
\right]$
\item See \href{https://ideas.repec.org/p/nbr/nberwo/21439.html}{Atkin and Donaldson (2015)} on inferring trade costs from price gaps
\end{itemize}
\end{column}
\end{columns}
\end{frame}
% -----------------------------------------
\begin{frame}{Intermediates \`{a} la Ethier (1982)}
\begin{itemize}
	\item Imagine that goods are produced using labor and a composite intermediate that is a CES aggregate coinciding with the consumption good (as in Ethier AER 1982)
	\item Thus, the composite may be consumed or used as an intermediate
	\item Assume Cobb-Douglas: $c_i = w_i^\beta p_i^{1-\beta}$
\end{itemize}
\end{frame}
% -----------------------------------------
\begin{frame}{Eaton and Kortum (2002) counterfactuals}
Three counterfactuals in Eaton and Kortum (2002):
\begin{itemize}
	\item Autarky: $d_{ni} \to \infty \ \forall i\neq n$. Single-digit percentage-point welfare loss for most countries.
	\item Free trade: $d_{ni}=1 \ \forall n,i$. Welfare gains on order of 20\%.
	\item Technological advances: Increase $T_{\text{US}}$ and  $T_{\text{Germany}}$ 20\%. Favors trading partners.
\end{itemize}
Some of this model's counterfactual predictions can be obtained by ``\href{https://tradediversion.net/2018/05/07/on-hat-algebra/}{exact hat algebra}'', (i.e., you might not have to separate productivities and trade costs)
\end{frame}
% -----------------------------------------
\begin{frame}{Empirical estimation of Ricardian predictions}
Alan Deardorff (\textit{Handbook}, 1984) on ``Testing Trade Theories and Predicting Trade Flows'':
\begin{quote}
The intuitive content of most trade theories is quite simple and straightforward\dots 
seldom stated in forms that are compatible with the real world complexities that empirical research cannot escape.
\end{quote}
\begin{itemize}
	\item Given difficulty of testing $p^a \cdot T \leq 0$,
	we typically model $p^a$ as function of primitives
	\item In the Ricardian model, this seems simple: 
	relative prices equal relative labor costs (in both trade and autarky)
	\item Model predicts which goods countries trade (not with whom or how much)
\end{itemize}
\end{frame}
% -----------------------------------------
\begin{frame}{Empirical challenges for Ricardian models}
\begin{itemize}
	\item Specialization is selection:
	If countries don't produce some goods in the trade equilibrium, 
	we cannot infer relative labor costs.\footnote{
		\href{https://www.jstor.org/stable/2728516}{Sattinger (1993, p.832)}: 
		``Empirical modeling of the distribution of earnings requires the econometric specification of worker alternatives, even though only the chosen sector or job is observed.
		This generates a set of econometric problems that have been addressed in applications of Roy's and Tinbergen's models.''
	}
	\begin{itemize}
		\item In fact, data suggests that countries aren't specializing at commodity-code level (intraindustry trade)
	\end{itemize}
	\item Suspicions that relative labor costs in trade equilibrium do not reveal relative labor costs in autarky
	\begin{itemize}
		\item Multiple factors of production
		\item Relative costs endogenous to trade flows
	\end{itemize}
	\item Dimensionality mismatch:
	How to take two-country predictions to many-country data?
\end{itemize}
\end{frame}
% -----------------------------------------
\begin{frame}{Ad hoc regressions}
Challenges evident in early empirical work (Deardorff 1984)
\begin{itemize}
	\item MacDougall (1951, 1952), Stern (1962), and Balassa (1963) regress relative exports volumes on relative productivities
	\item 95\% of US and UK trade are with third markets, so regress relative exports to third markets on relative productivities
	\item Absent trade costs, Ricardian model predicts no overlap in exports to third markets
\end{itemize}
\begin{center}
\includegraphics[height=.45\textheight]{../images/Stern1962_fig1.pdf}
\end{center}
\end{frame}
% -----------------------------------------
\begin{frame}{Regressions of exports on sector-country interactions}
A wave of papers in the 2000s examined sector-country interactions to test 
whether observable sources of comparative advantage govern trade patterns
\begin{itemize}
	\item \href{https://ideas.repec.org/a/aea/aecrev/v88y1998i3p559-86.html}{Rajan and Zingales (1998)} is early, non-trade exemplar
	\item \href{https://www.aeaweb.org/articles?id=10.1257/000282804322970715}{Romalis (2004)} interacts sectoral factor intensity with country factor abundance (we discuss factor-proportions theory in week 5)
	\item \href{https://doi.org/10.3982/ECTA7636}{Costinot (2009)} provides the general logic: log-supermodularity
	\item This has been described as ``the typical way trade economists would explore'' comparative advantage
	\item \href{https://doi.org/10.1093/ej/ueac047}{Ciccone and Papaioannou (2023)} raise concern that sectoral characteristics aren't commonly ordered across countries
	\item As one example of these trade papers, \href{https://academic.oup.com/qje/article/122/2/569/1942086}{Nunn (2007)} looks at countries' contract enforcement and the relationship specificity of goods' intermediate inputs
\end{itemize}
%
\end{frame}
% -----------------------------------------
\begin{frame}{Nunn (2007)}
``I find that countries with better contract enforcement export relatively more in industries for which relationship-specific investments are most important''
\begin{equation*}
\ln x_{ic} = 
\alpha_i + \alpha_c 
+ \beta_1 z_i Q_c + \beta_2 h_i H_c + \beta_3 k_i K_c 
+ \epsilon_{ic},
\end{equation*}
{\small where $z_i$ is inputs' relationship specificity and $h_i$ and $k_i$ are skill and capital intensities}
\begin{itemize}
	\item $z_i$ is average of inputs' Rauch (1999) indicators weighted by input cost shares from US input-output table
	\item Rauch (1999) indicator say commodity is neither `sold on an organized exchange' nor `reference priced in industry journals'
	\item $Q_c$ is investor perception (World Bank survey); uses legal origin as IV for $Q_c$
	\item No explicit model linking lower relative unit costs to export volume; appeals to Romalis (2004) two-country model
\end{itemize}
\end{frame}
% -----------------------------------------
\begin{frame}{Nunn (2007): Sectoral characteristic}
\begin{center}\includegraphics[height=.93\textheight]{../images/Nunn2007_tab2.pdf}\end{center}
\end{frame}
% -----------------------------------------
\begin{frame}{Nunn (2007): ``Determinants of Comparative Advantage''}
\begin{center}\includegraphics[height=.92\textheight]{../images/Nunn2007_tab4.pdf}\end{center}
\end{frame}
% -----------------------------------------
\begin{frame}{CDK: ``A Quantitative Exploration of Ricardo's Ideas''}
\href{https://doi.org/10.1093/restud/rdr033}{Costinot, Donaldson, and Komunjer (2012)} introduce a multi-sector model in which each sector behaves like Eaton and Kortum (2002)
\begin{itemize}
	\item EK's Ricardian model says nothing about a key Ricardian question: what's the pattern of specialization and trade?
	\item In CDK, while specialization within industries is indeterminate, now model predicts (aggregate) sectoral trade flows
\end{itemize}
The structural model guides the empirical estimation
\begin{itemize}
	\item Derive estimating equation from theory {\footnotesize (and contrast with ad hoc regressions)}
	\item Think about contents of error term and plausibility of orthogonality requirements
	\item Explicitly tackle the selection problem associated with unobserved productivities
	\item Quantify welfare importance of Ricardian comparative advantage
\end{itemize}
\end{frame}
% -----------------------------------------
\begin{frame}{CDK model: Technology}
\begin{itemize}
	\item Index countries by $i$, industries by $k$, and varieties by $\omega$
	\item Labor is sole factor of production, endowed in quantity $L_i$ and paid wage $w_i$
	\item Unit cost is $w_i/z_i^k(\omega)$ with productivity $z_i^k(\omega)$ randomly drawn
	\item CDK's notation for the Fr\'{e}chet distribution is
	\begin{equation*}
		F_i^k(z) = \exp\left[-\left(z/z_i^k\right)^{-\theta}\right]
	\end{equation*}
	\item $z_i^k$ is the ``fundamental productivity'', an industry-country location parameter, that generates Ricardian comparative advantage by sector
	\item $\theta$ governs idiosyncratic comparative advantage across varieties within sector, as in EK (2002). Note that it does not vary.
\end{itemize}
\end{frame}
% -----------------------------------------
\begin{frame}{CDK model: Rest of the setup}
\begin{itemize}
	\item Iceberg trade costs $d_{ij}^k$ from $i$ to $j$ with $d_{ii}^k=1$ and triangle inequality
	\item Perfect competition: $p_{j}^k(\omega) = \min_i \left[c_{ij}^k(\omega)\right] = \min_i \left[w_i d_{ij}^k/z_i^k(\omega)\right] $
	\item See paper for Bertrand competition case (\`{a} la BEJK 2003)
	\item Preferences: Cobb-Douglas upper tier and CES lower tier
	\begin{equation*}
	x_j^k(\omega) = \left[\frac{p_j^k(\omega)}{p_j^k} \right]^{1-\sigma_j^k} \alpha_j^k w_j L_j
	\end{equation*}
	\item Trade is balanced (\textit{not} sector by sector!)
\end{itemize}
\end{frame}
% -----------------------------------------
\begin{frame}{CDK Lemma 1: Trade and fundamental productivities}
Start from gravity equation:
\begin{equation*}
	x_{ij}^{k}
	= \frac{  \left(w_i d_{ij}^k/z_i^k\right)^{-\theta}}
	{\sum_{i'}\left(w_i d_{ij}^k/z_i^k\right)^{-\theta}} 
	\alpha_j^k w_j L_j
\end{equation*}
This implies a difference-in-differences version:
\begin{equation*}
\ln \left(\frac{x_{ij}^{k}x_{i'j}^{k'}}{x_{ij}^{k'}x_{i'j}^{k}}\right)
=
\theta \ln \left(\frac{z_{i}^{k}z_{i'}^{k'}}{z_{i}^{k'}z_{i'}^{k}}\right)
-
\theta \ln \left(\frac{d_{ij}^{k}d_{i'j}^{k'}}{d_{ij}^{k'}d_{i'j}^{k}}\right)
\end{equation*}
But we don't observe $z_{i}^k$ and we need a measure of $d_{ij}^k$
\begin{itemize}
	\item We cannot observe fundamental productivity
	$z_{i}^k = \mathbb{E}\left[z_i^k(\omega)\right]$
	\item We observe the endogenous object
	$\tilde{z}_{i}^k = \mathbb{E}\left[z_i^k(\omega) \vert \Omega_{i}^k\right]$
	where $\Omega_{i}^k$ is the set of varieties produced in equilibrium
\end{itemize}
\end{frame}
% -----------------------------------------
\begin{frame}{CDK Theorem 1: Trade and observed productivities}
CDK show that
\begin{equation*}
\frac{\tilde{z}_i^k}{\tilde{z}_{i'}^k} = \left(\frac{z_i^k}{z_{i'}^k}\right)  \left(\frac{\pi_{ii}^k}{\pi_{i'i'}^k}\right)^{-1/\theta}
\end{equation*}
Plug that in
\begin{equation*}
\ln \left(\frac{\tilde{x}_{ij}^{k}\tilde{x}_{i'j}^{k'}}{\tilde{x}_{ij}^{k'}\tilde{x}_{i'j}^{k}}\right)
=
\theta \ln \left(\frac{\tilde{z}_{i}^{k}\tilde{z}_{i'}^{k'}}{\tilde{z}_{i}^{k'}\tilde{z}_{i'}^{k}}\right)
-
\theta \ln \left(\frac{d_{ij}^{k}d_{i'j}^{k'}}{d_{ij}^{k'}d_{i'j}^{k}}\right)
\end{equation*}
where $\tilde{x}_{ij}^{k} = {x}_{ij}^{k} / \pi_{ii}^k$
\begin{itemize}
	\item Special case of $d_{ij}^{k} = d_{ij} d_j^k$ is illuminating
	\item We get pairwise predictions that feel like 2-by-2 Ricardian story but are for quantities, destination by destination
\end{itemize}
Can also state as gravity regression
\begin{equation*}
\ln \tilde{x}_{ij}^{k} = \gamma_{ij} + \gamma_j^k + \theta \ln \tilde{z}_{i}^{k} - \theta \ln d_{ij}^k
\end{equation*}
\end{frame}
% -----------------------------------------
\begin{frame}{CDK's structural answers to specification questions}
Gravity regression, so this isn't a test of Ricardian story vs other stories that deliver a similar gravity equation. 
\begin{equation*}
\ln \left({x}_{ij}^{k}/\pi_{ii}^k \right) = \gamma_{ij} + \gamma_j^k + \theta \ln \tilde{z}_{i}^{k} - \theta \ln d_{ij}^k
\end{equation*}
But it gives structural answers to many questions that lurk in prior empirical work.
\begin{itemize}
	\item What's the appropriate dependent variable?
	\item How do we aggregate across multiple countries/destinations?
	\item What is the meaning of the slope coefficient?
	\item Levels vs logs vs semi-log
	\item What fixed effects are required?
	\item What is in the error term?
\end{itemize}
\end{frame}
% -----------------------------------------
\begin{frame}{Data on productivity $\tilde{z}_i^k$}
\begin{itemize}
	\item Tough to compare productivity across industries and countries (need producer price deflators to get physical quantities)
	\item CDK use International Comparisions of Output and Productivity (ICOP) Industry Database from GGDC (Groningen)
	\item 1997 cross section has 21 OECD countries for 13 manufacturing industries
	\item Since wages are common across industries in a one-factor model,
	relative productivity $\tilde{z}_i^k$ shows up straightforwardly in relative (inverse) producer prices
\end{itemize}
\end{frame}
% -----------------------------------------
\begin{frame}{Estimating equation}
Empirical specification is
\begin{equation*}
\ln \left({x}_{ij}^{k}/\pi_{ii}^k \right) = \gamma_{ij} + \gamma_j^k + \theta \ln \tilde{z}_{i}^{k} + \epsilon_{ij}^k
\end{equation*}
\begin{itemize}
	\item Given fixed effects, log producer price $\ln p_i^k$ is a measure of $-\ln \tilde{z}_{i}^{k}$
	\item Industry-pair specific trade costs $\ln d_{ij}^k$ are sitting in the error term,
	threatening OLS assumption that $\mathbb{E}\left[\ln p_i^k \epsilon_{ij}^k\vert \gamma_{ij},\gamma_j^k\right]=0$
	\item (Classical) measurement error in $\ln p_i^k$ will attenuate $\hat{\theta}$
	\item Exporting and productivity may be simultaneously determined
	\item CDK employ R\&D expenditure as IV for productivity (why isn't R\&D endogenous to trade?)
\end{itemize}
\end{frame}
% -----------------------------------------
\begin{frame}{Estimates of $\theta$}
\includegraphics[height=0.9\textheight]{../images/CostinotDonaldsonKomunjer2012_tab3.pdf}
\end{frame}
% -----------------------------------------
\begin{frame}{CDK's welfare counterfactuals}
\begin{itemize}
	\item ``According to our estimates, the removal of Ricardian comparative advantage at the industry level would only lead, on average, to a 5.3\% decrease in the total gains from trade.''
	\item Some countries actually gain from eliminating comparative advantage
	\item See section 5.3 discussion of heterogeneous trade costs and heterogeneous tastes as potential explanations
	\item The key is that CDK's double-differenced gravity regression did not restrict Cobb-Douglas shares nor $\ln d_{ij}^k$
\end{itemize}
\end{frame}
% -----------------------------------------
\begin{frame}{Caliendo \& Parro (2015): Multi-sector EK with input-output matrix}
Research question is assessing NAFTA
(\href{https://doi.org/10.1016/bs.hesint.2022.02.005}{Antras \& Chor 2022} notation)
\begin{itemize}	
\item EK for each sector $s$: CES aggregation of varieties, Fréchet productivities %($T^{s}_{i},\theta^{s}$)
\item Cobb-Douglas preferences ($\alpha_{j}^s$) and \textit{roundabout} production functions ($\gamma_{j}^{rs}$)
$$
c_{j}^{s}
=
\Gamma_{j}^{s} w_j^{1-\sum_{r=1}^{S} \gamma_{j}^{rs}} \prod_{r=1}^{S} \left(P_{j}^{r}\right)^{\gamma_{j}^{rs}}
$$
\item Gravity equation for sectoral expenditure share:
$$
\pi_{ij}^{s}
=
\frac{T_{i}^{s}\left(c_{i}^{s}\tau_{ij}^{s}\right)^{-\theta^{s}}}
{\sum_{k=1}^{J} T_{k}^{s}\left(c_{k}^{s}\tau_{kj}^{s}\right)^{-\theta^{s}}}
$$
\item Clear markets for each industry in each country
$$
X_{j}^{s} = \sum_{r=1}^{S} \gamma_{j}^{sr}
\sum_{i=1}^{J} X_i^r \pi_{ji}^{r}
+ 
\alpha_{j}^{s}\left(w_j L_j + D_j\right)
$$
\end{itemize}
\end{frame}
% -----------------------------------------
\begin{frame}{CP 2015: Estimation, calibration, counterfactuals}
\begin{itemize}
\item Estimate sectoral trade elasticities $\theta^s$
using a gravity equation assuming non-tariff trade costs are symmetric
(see next week)
\item Compute counterfactual by writing everything in relative changes and initial shares
(extends DEK's ``exact hat algebra'' to multi-sector model)
\item Use 1993 baseline for 30 countries and 40 sectors
\item Compute counterfactual for all tariff changes 1993-2005
and for NAFTA tariff changes 1993-2005
\item Tariff reductions changed Mexican real income +1.3\%, US +0.08\%, and Canada -0.06\%
\item Some trade diversion, but effects on rest of world are small
\item Larger welfare and trade effects than in one-sector and no-input-output-linkages models
\end{itemize}
\end{frame}
% -----------------------------------------
\begin{frame}{Caliendo \& Parro (2015) caveats}
\begin{itemize}
\item ``a staple in the toolkit of international trade economists''
\item Final-use and intermediate sourcing shares are identical (see \href{https://doi.org/10.1016/bs.hesint.2022.02.005}{Antras \& Chor 2022} on extensions that relax)
\item Roundabout production structure:
goods are produced via an endless sequence of steps, with each stage using inputs from prior stages in an infinite loop
\item Confusing: Compute counterfactuals ``without needing to estimate parameters which are difficult to identify in the data, as productivities and iceberg trade costs'' (p.11)
\item \href{https://doi.org/10.1016/bs.hesint.2022.02.005}{Antras \& Chor (2022)}:
``the lack of `external' evidence supporting the out-of-sample performance of these models remains problematic and a clear area with room for improvement in future research''
\end{itemize}
\end{frame}
% -----------------------------------------
\begin{frame}{Wrapping up}
Recap:
\begin{itemize}
\item Quantitative trade models are workhorses for computing trade counterfactuals
\item ``Putting Ricardo to work'' involves defining the question of interest carefully
\item Economic outcomes governed by aggregate elasticities (and therefore isomorphic to CES siblings)
\end{itemize}
Next week:
\begin{itemize}
\item Estimating gravity equations
\item Computing gains from trade in these quantitative models
\end{itemize}
\end{frame}
% -----------------------------------------
\end{document}
