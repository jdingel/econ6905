\documentclass[11pt,notes=hide,aspectratio=169]{beamer}
%Jonathan Dingel; PhD trade course

% PACKAGES
\usepackage{graphics}  % Support for images/figures
\usepackage{graphicx}  % Includes the \resizebox command
\usepackage{url}	   % Includes \urldef and \url commands
\usepackage{soul}      % Includes the underline \ul command
%\usepackage{framed}	   % Includes the \framed command for box around text
\usepackage{booktabs} %\toprule,\bottomrule
%\usepackage{natbib}
\usepackage{bibentry}  % Includes the \nobibliography command
\usepackage{bbm}       %
%\usepackage{pgfpages}  %Supports "notes on second screen" option for beamer
\usepackage{verbatim}  %Supports comments
\usepackage{tikz}		%Supports graphing/drawing
\usepackage{pgfplots} %Supports graphing/drawing
\usepackage{amsfonts}  % Lots of stuff, including \mathbb 
\usepackage{amsmath}   % Standard math package
\usepackage{amsthm}    % Includes the comment functions
\usepackage{physics}

% CUSTOM DEFINITIONS
\def\newblock{} %Get beamer to cooperate with BibTeX
\linespread{1.2}
\hypersetup{backref,pdfpagemode=FullScreen,colorlinks=true,linkcolor=blue,urlcolor=blue}
\newtheorem{proposition}{Proposition}
\newtheorem{assumption}{Assumption}
\newtheorem{condition}{Condition}

% IDENTIFYING INFORMATION
\title{Topics in Trade}
\author{Jonathan I. Dingel}
\date{Fall \the\year}

% BEAMER TEACHING STUFF
\setbeamertemplate{navigation symbols}{}  %Turn off navigation bar

% THEMATIC OPTIONS
\definecolor{columbiablue}{RGB}{185,217,235}  %Columbia blue defined at https://visualidentity.columbia.edu/branding
\definecolor{columbiadarkblue}{RGB}{0,48,135}  %Columbia dark blue defined at https://visualidentity.columbia.edu/branding
\setbeamercovered{transparent=5}
\setbeamercolor{frametitle}{fg=columbiadarkblue}
\setbeamercolor{item}{fg=columbiadarkblue}
\usefonttheme{serif}
\setbeamercolor{button}{bg = white,fg = columbiadarkblue}
\setbeamercolor{button border}{fg = columbiadarkblue}

\setbeamertemplate{footline}{\begin{center}\textcolor{gray}{Dingel -- Topics in Trade -- Week 11 -- \insertframenumber}\end{center}}
\usepackage{listings}
\input{../week00/listings_julia_setup.tex}
\input{../week00/listings-stata.tex}
\newcommand{\beginbackup}{
   \newcounter{framenumbervorappendix}
   \setcounter{framenumbervorappendix}{\value{framenumber}}
}
\newcommand{\backupend}{
  \addtocounter{framenumbervorappendix}{-\value{framenumber}}
  \addtocounter{framenumber}{\value{framenumbervorappendix}} 
}
\begin{document}
% -----------------------------------------
\begin{frame}[plain]
\begin{center}
\large
\textcolor{columbiadarkblue}{ECON G6905\\
Topics in Trade\\ 
Jonathan Dingel\\
Autumn \the\year, Week 11}
\vfill 
\includegraphics[width=0.4\textwidth]{../images/Columbia_logo.png}
\end{center}
\end{frame}
% -----------------------------------------
\begin{frame}{Today: Estimating and simulating discrete-choice models}
My goals for today:
\begin{itemize}
	\item Convey the connections between CES and logit in theory and estimation
	\item Introduce you to estimation and simulation of discrete-choice models in the context of international and urban economics
	\item Share tricks and resources that save lots of (human or computing) time
\end{itemize}
My papers using discrete-choice methods include
\begin{itemize}
	\item How Segregated is Urban Consumption?
	\item Spatial Economics for Granular Settings
\end{itemize}
Blanket recommendations:
\begin{itemize}
\item Ken Train's \textit{Discrete Choice Methods with Simulation} (2009, \href{https://eml.berkeley.edu/books/choice2.html}{PDFs})
\item Simulate your data-generating process to verify your procedures and code
\end{itemize}
\end{frame}
% -----------------------------------------
\begin{frame}{Choosing location(s) is a discrete-choice problem}
Long history of discrete-choice modeling in urban and transportation economics 
\begin{itemize}{\small 
\item See \href{https://www.nobelprize.org/uploads/2018/06/mcfadden-lecture.pdf}{McFadden's Nobel lecture}; e.g., Peter Diamond \& Robert Hall working on transport or multinomial logit vs gravity for BART forecasts
\item The canonical nested-logit and GEV results are in McFadden (1978) ``\href{https://trid.trb.org/view/87722}{Modeling the Choice of Residential Location}''
\item Dennis Carlton on ``\href{https://www.jstor.org/stable/1924189}{The Location and Employment Choices of New Firms}'' in 1983
\item {Estimating logit model by Poisson regression is Guimar{\~a}es, Figueiredo, Woodward ``\href{https://ideas.repec.org/a/tpr/restat/v85y2003i1p201-204.html}{A Tractable Approach To The Firm Location Decision Problem}'' (2003)\par}
}\end{itemize}
Plain-vanilla logit case:
individual $i$ considers choice $j$ (see \href{https://eml.berkeley.edu/books/choice2.html}{Train 2009})
\begin{itemize}
	\item Utility $U_{ij} = V_{ij} + \epsilon_{ij}$
	\item Assume error is iid Gumbel (T1EV): $F\left(\epsilon_{ij}\right)=\exp(-\exp(-\epsilon_{ij}))$
	\item Choice probabilities are
	\begin{equation*}\Pr(U_{ij}>U_{ij'} \ \forall j' \neq j) = \frac{\exp(V_{ij})}{\sum_{j'}\exp(V_{ij'})} \end{equation*}
\end{itemize}
\end{frame}
% -----------------------------------------
\begin{frame}{Import-sourcing decisions are discrete-choice problems}
Trade models feature discrete-choice problems
\begin{itemize}
\item Selecting the least-cost provider of each good/variety is at the heart of neoclassical trade models
\item In the Eaton and Kortum (2002) formulation, this is probabilistic and a discrete-choice problem
\end{itemize}
\smallskip
Rewrite the plain-vanilla logit case from previous slide
to be a cost-minimization problem with multiplicative error term
\begin{itemize}
	\item Cost $\ln c_{ji} = \ln c_j + \ln \tau_{ji} - \epsilon_{j}$
	\item Least-cost probability
	\begin{align*}
	\Pr(-\ln c_{ji}> -\ln c_{j'i} \ \forall j' \neq j) \\
	= \Pr(\ln c_{ji}<\ln c_{j'i} \ \forall j' \neq j) 
	& 
	= \frac{1/(c_j\tau_{ji})}{\sum_{j'}1/(c_{j'}\tau_{j'i})} \\
	\end{align*}
\end{itemize}
\textcolor{gray}{The Frechet distribution is the log-Gumbel distribution.}
\end{frame}
% -----------------------------------------
\begin{frame}{IIA logit demand implies CES market demand}
Constant-elasticity demand functions are closely related.
See \href{https://mitpress.mit.edu/9780262011280/discrete-choice-theory-of-product-differentiation/}{Anderson, de Palma, Thisse (1992) book}
(and \href{https://doi.org/10.1016/0165-1765(87)90239-4}{1987 Economics Letters}
and \href{https://doi.org/10.2307/2526791}{1988 IER})
\begin{align*}
\text{Logit} & & \text{CES} \hfill \qquad
\\
\Pr(U_{ij}>U_{ij'} \ \forall j' \neq j) 
&=
\frac{\exp(V_{ij})}{\sum_{j'}\exp(V_{ij'})}
\qquad
&
X_{ji}
=
\frac{(p_j\tau_{ji} )^{1-\sigma}}{P_i^{1-\sigma}} X_i
\\
\text{let } V_{ij} = (1-\sigma) \ln \left(p_j\tau_{ji}\right)
& \implies
&
\textcolor{gray}{\text{(note subscript order)}}
\\
&=
\frac{(p_j\tau_{ji})^{1-\sigma}}{P_i^{1-\sigma}}
\end{align*}
How to arrange upper-tier preferences to align with aggregate expenditure $X_i$?
If 2\textsuperscript{nd}-stage quantity is $q_{ji}^{*} = X_i / p_{ji}$,
then $V_{ij} = (1-\sigma) \ln \left(p_j\tau_{ji}/X_i\right)$
[see AdPT 1987]
\smallskip
\textcolor{gray}{(This explains the contrasting subscript orders of Ch 3 \& 4 in 2014 \textit{Handbook})}
\end{frame}
% -----------------------------------------
\begin{frame}{Logit MLE coincides with Poisson regression (1/2)}
Estimation of constant-elasticity functions is closely related
(\href{https://doi.org/10.1162/003465303762687811}{Guimaraes, Figueiredo, Woodward 2003}).
Let $p_{ij} \equiv \Pr(U_{ij}>U_{ij'} \ \forall j' \neq j) \propto \exp\left(\boldsymbol{\beta}^{\prime} \mathbf{z}_{ij}\right)$.\\
Simple case: let $p_{ij} = p_j$
$$
\ln \mathcal{L}^{\text{logit}} =\sum_{i=1}^N \sum_{j=1}^J d_{i j} \ln p_{i j}=\sum_{j=1}^J n_j \ln p_j
$$
Consider the Poisson count variable with value $n_j$:
$
\mathbb{E}\left(n_j\right)=\lambda_j=\exp \left(\alpha+\boldsymbol{\beta}^{\prime} \mathbf{z}_j\right) .
$
\begin{align*}
\ln \mathcal{L}^{\text{P}}
=
\sum_{j=1}^J\left(-\lambda_j+n_j \ln \lambda_j-\ln n_{j} !\right) 
=
\sum_{j=1}^J\left[-\exp \left(\alpha+\boldsymbol{\beta}^{\prime} \mathbf{z}_j\right)+n_j\left(\alpha+\boldsymbol{\beta}^{\prime} \mathbf{z}_j\right)\right. \left.-\ln n_{j} !\right]
\\
\frac{\partial \ln \mathcal{L}^{\text{P}}}{\partial \alpha}
=
\sum_{j=1}^J\left[n_j-\exp \left(\alpha+\boldsymbol{\beta}^{\prime} \mathbf{z}_j\right)\right]
=0
\implies
\exp (\alpha)=\frac{N}{\sum_{j=1}^J \exp \left(\boldsymbol{\beta}^{\prime} \mathbf{z}_j\right)}
\end{align*}
Substitute $\alpha$ back in to obtain the concentrated log likelihood\dots
\end{frame}
% -----------------------------------------
\begin{frame}{Logit MLE coincides with Poisson regression (2/2)}
Substitute $\alpha$ back in to obtain the concentrated log likelihood:
\begin{align*}
\ln \mathcal{L}^{P_c}
&=
-N+N \ln N-\sum_{j=1}^J n_j \ln \left(\sum_{j=1}^J \exp \left(\boldsymbol{\beta}^{\prime} \mathbf{z}_j\right)\right)
+
\sum_{j=1}^J n_j \boldsymbol{\beta}^{\prime} \mathbf{z}_j-\sum_{j=1}^J \ln n_{j} ! 
\\
&=
\sum_{j=1}^J n_j \ln p_j-N+N \ln N-\sum_{j=1}^J \ln n_{j}!
\\
&=
\ln \mathcal{L}^{\text{logit}} + \text{constant}
\end{align*}
The logit MLE $\beta$ and the Poisson MLE $\beta$ are the same.
We can therefore use software implementations of PPML with high-dimensional fixed effects to estimate logit models.
\end{frame}
% -----------------------------------------
\begin{frame}{``How Segregated is Urban Consumption?''}
Two questions:
\begin{enumerate}
	\item How segregated is urban consumption?
	\item What are roles of spatial frictions, social frictions, and heterogeneous tastes?
\end{enumerate}
How we answer them:
\begin{itemize}
	\item Gather data on individuals' restaurant visits within New York City
	\item Infer spatial and social frictions from behavior by estimating a discrete-choice model of individuals' visit decisions
	\item Use model-predicted consumer behavior to measure consumption segregation
\end{itemize}
\medskip
Here, I'll focus on the estimation and computational elements
\end{frame}
% -----------------------------------------
\begin{frame}{Observed behavior, spatial frictions, social frictions}
\vspace{-1mm}
\begin{center}
\includegraphics[height=.47\textheight]{../images/week11/DDMM2019_rdfm_timehome_density.pdf}
\includegraphics[height=.47\textheight]{../images/week11/DDMM2019_rdfm_timework_density.pdf}\\
\includegraphics[height=.47\textheight]{../images/week11/DDMM2019_rdfm_eucl_density.pdf}
\end{center}
\end{frame}
% -----------------------------------------
\begin{frame}{Demand estimation in ``How Segregated is Urban Consumption?''}
In a sense, this is ``gravity within the city'', but the discrete-choice formulation is clearer in important respects
\begin{itemize}
\item Model of review-writing behavior necessitated by Yelp data
\item Estimate parameters when data do not identify origin-mode pair $l$ (e.g., from work by car)
\item McFadden (1978) sampling exploits IIA to make computation feasible
\end{itemize}
\end{frame}
\begin{frame}
\frametitle{Behavioral model}   %TODO
\begin{itemize}
	\item Individual $i$ decides at time $t$ whether to visit venue $j$ in choice set $\mathcal{J}$.
	\item Trip may originate from one of six locations $l$, $l \in \mathcal{L} = \left\{\textnormal{car},\textnormal{public transit}\right\} \times \{\textnormal{home, work, commute}\}$
	\begin{equation*}
	U_{ijlt} =\underbrace{\gamma_{l}^{1}X_{ijl}^{1}+\gamma^{2}X_{ij}^{2}+\beta^{1}Z_{j}^{1}+\beta^{2}Z_{ij}^{2}}_{ = V_{ijl}}+\nu_{ijlt}
 	%U_{ijlt}=\beta_{l}^{1}X_{ijl}^{1}+\beta^{2}X_{ij}^{2}+\nu_{ijlt}	
 	\end{equation*}
	\item Dummy $d_{ijlt}=1$ if $i$ visits $j$ from $l$ at $t$
	\item We observe dummy 
	$d_{ijt} = \sum_{l\in\mathcal{L}}d_{ijlt}$
	\item Assume that $\nu_{ijlt}$ have type I extreme value distribution, independent across individuals, restaurants, time periods, and locations:
	\begin{equation*}
	\Pr(d_{ijt}=1|X,Z,\mathcal{J};\gamma,\beta)=\frac{\sum_{l\in\mathcal{L}}\exp(V_{ijl})}{\sum_{j'\in \mathcal{J}}\sum_{l\in\mathcal{L}}\exp(V_{ij'l})}
%	\Pr(d_{ijt}=1|X_{it},J_{it};\beta)=\frac{\sum_{l\in\mathcal{L}}\exp(\beta_{l}^{1}X_{ijl}^{1}+\beta^{2}X_{ij}^{2})}{\sum_{j'\in J_{it}}\sum_{l\in\mathcal{L}}\exp(\beta_{l}^{1}X_{ij'l}^{1}+\beta^{2}X_{ij'}^{2})}
	\end{equation*}
\end{itemize}
\end{frame}
%-------------------------------------------------------------------------------------------
\begin{frame}
\frametitle{Specification}
\begin{equation*}
	V_{ijl} = \gamma_{l}^{1}X_{ijl}^{1}+\gamma^{2}X_{ij}^{2}+\beta^{1}Z_{j}^{1}+\beta^{2}Z_{ij}^{2} 
\end{equation*}
\begin{itemize}
	\item Spatial frictions: $X^{1}_{ijl}$ is log minutes of transit time with mode-origin-specific disutilities $\gamma^{1}_{l}$
	\item Social frictions: $X^{2}_{ij}$ contains demographic differences between user $i$'s home tract (or $i$'s own identity) and the tract of restaurant $j$
	\item Restaurant characteristics: $Z_{j}^{1}$ contains venue rating, 4 price dummies, 9 cuisine dummies, venue tract median household income, and 28 area dummies
	\item User-restaurant characteristics: $Z_{ij}^{2}$ contains rating and price interacted with $i$'s home tract median income, percentage and absolute percentage difference in median household incomes between $i$ and $j$ tracts %, gender-specific coefficients on rating, price, venue tract income, and income differences
\end{itemize}
\end{frame}
%-------------------------------------------------------------------------------------------
\begin{frame}
\frametitle{Identification: Reviews, not visits}
We assume that
\begin{itemize}
	\item users do not review unvisited venues
	\item users only review venues once
	\item probability of writing a review is independent of \emph{ex ante} venue characteristics
\end{itemize}
\begin{align*}
\Pr(d_{ijt}^{r}=1|X,Z,\mathcal{J};\gamma,\beta)	
&=
\Pr(d_{ijt}^{r}=1|d_{ijt}=1,\cdot;\cdot) \times \Pr(d_{ijt}=1|\cdot;\cdot)
\\
&=
w_{it} \times \mathbbm{1}\{j\neq 0,j\notin D_{it}^{r}\} \times \Pr(d_{ijt}=1|\cdot;\cdot)
\end{align*}
\pause
If $\Pr(d_{ijt}^{r}=1|d_{ijt}=1,\cdot;\cdot)$ depends on some restaurant characteristic in $Z$
\begin{itemize}
	\item Coefficient on that characteristic would be biased
	\item Estimates of spatial and social frictions could still be asymptotically unbiased
\end{itemize}
\end{frame}
%-------------------------------------------------------------------------------------------
\begin{frame}
\frametitle{Computation: Large choice sets}
Users choose amongst thousands of NYC restaurants
\begin{itemize}
	\item McFadden (1978): estimate conditional logit model's parameters using a choice set $S_{it}$ that is strict subset of actual choice set $\mathcal{J}$
	\item Construct $S_{it}$ by including $i$'s observed choice at period $t$ plus a random subset of the other alternatives included in the set $J'_i = \mathcal{J} / \left\{D_{i}^{r} \cup \{j=0\}\right\}$
	\item Select unchosen venues of $S_{it}$ with equal probability from $J'_i$ 
		\begin{align*}
		\Pr(d_{ijt}^{r}=1|X,Z,S_{it};\gamma,\beta)	=\frac{\sum_{l\in\mathcal{L}}\exp(V_{ijlt})}{\sum_{j'\in S_{it}}\sum_{l\in\mathcal{L}}\exp(V_{ij'lt})} \\
		L=\sum_{i}\sum_{t}\sum_{j\in S_{it}}\Big\{ d_{ijt}^{r}\log\Big(\Pr\left(d_{ijt}^{r}=1|X,Z,S_{it};\gamma,\beta\right)\Big)\Big\}	
		\end{align*}
	\item We construct $S_{it}$ with 20 elements
\end{itemize}
\hypertarget{choice_set}{}
\end{frame}
%-------------------------------------------------------------------------------------------
\begin{frame}{Details of McFadden (1978) sampling}
Denote by $\pi(S_{it}|d_{ijt}^{*}=1,J'_{it})$ the probability of assigning the set $S_{it}$ to an individual $i$ who reviewed venue $j$ at $t$. Our sampling scheme means
$$
\pi(S_{it}|d_{ijt}^{*}=1,J'_{it})
=
\kappa_{it} \times \mathbf{1}\{j\in S_{it}\}
$$
where $\kappa_{it}\in(0,1)$ is a constant determined by numbers of venues in  $S_{it}$ and $J'_{it}$.
The resulting probability of reviewing restaurant $j$ given sampling is
\begin{align*}
\Pr(d_{ijt}^{*}=1|d_{it}^{*}=1X,Z,\mathcal{J};\gamma,\beta)	
&=
\mathbbm{1}\{j\neq 0,j\notin D_{it}^{r}\} \times \Pr(d_{ijt}=1|\cdot;\cdot)
%=
%&
\\
\implies
\Pr(d_{ijt}^{*}=1|d_{it}^{*}=1,X_{i},Z_{i},S_{it};(\gamma,\beta))
&=
\frac{\mathbbm{1}\{j\in S_{it}\}\sum_{l\in\mathcal{L}}\exp(V_{ijl})}{\sum_{j'\in S_{it}}\sum_{l\in\mathcal{L}}\exp(V_{ij'l})}
\end{align*}
is the probability that $i$ reviews restaurant $j$ at period $t$ conditional on a randomly drawn set $S_{it}$ and that $i$ writes a review at $t$.
McFadden (1978) shows that maximizing this likelihood is a consistent estimator (obviously larger standard errors from not using all observations).
\end{frame}
%-------------------------------------------------------------------------------------------
\begin{frame}{Embracing the independence of irrelevant alternatives}
Logit specification features IIA across venues $j$ and origins $l$
\begin{itemize}
	\item Functional form makes the problem computationally feasible:
	McFadden (1978) sampling works because the ratio of probabilities for any two venues is the same whether or not other venues are available
	\item Estimation with randomly sampled $S_{it}$ yields stable results, consistent with IIA
	\textcolor{gray}{(the other Hausman and McFadden (1984) test)}
	\item See Train (2009) pages 49-50 on tests of IIA
	\item Can go to other extreme: IIA across venues $j$ and identical across origins $l$ $\to$ covariate is $X_{ij}^{1} = \min_{l} X_{ijl}^{1}$
	\item This IIA at individual level, not aggregate market shares
	\item Plausibility of IIA depends on the demand system you model;
	rejection of IIA for country-level trade flows does not mean IIA is a bad description of individuals choosing restaurants with detailed characteristics
%	\item Estimating nested logit, with each venue as nest, yields multinomial as likelihood-maximizing specification
\end{itemize}
\end{frame}
%-------------------------------------------------------------------------------------------
% -----------------------------------------
\begin{frame}{Simulating discrete-choice models}
Some reasons to simulate the data-generating process.
\begin{itemize}
\item Verify that your numerical optimization code behaves correctly when the estimated model is correctly specified
\item Construct confidence intervals for parameter estimates and downstream objects (relevant for finite samples or finite economies)
\begin{itemize}
\item ``Parametric bootstrap'' of estimates (e.g., Davis, Dingel, Monras, Morales 2019)
\item Demonstrating finite-sample bias (e.g., critique of calibrated-shares procedure in Dingel Tintelnot 2023)
\item Building confidence intervals for counterfactual outcomes (e.g., Section 5 of Dingel Tintelnot 2023)
\end{itemize}
\end{itemize}
\textcolor{gray}{(A distinct role for simulation is in estimating models that do not have closed-form expressions for choice probabilities. For example, multinomial probit and, more importantly, mixed (random-coefficients) logit. See Train (2009) starting with Chapter 5.)}
\end{frame}
% -----------------------------------------
\begin{frame}[fragile]{Simulating random variables}
\begin{itemize}
\item The CDF of a random variable has standard uniform distribution $U(0,1)$
\item Any machine can simulate draws of $u \sim U(0,1)$,
so invertible CDF lets you simulate that random variable.
\item Standard Gumbel: $F(\epsilon) = \exp(-\exp(-\epsilon)) \implies \epsilon = -\ln(-\ln(u))$
\item Drawing 100 million standard Gumbel realizations in Julia, Stata, and R
\begin{lstlisting}[language=julia]
	idiosyncratic = -log.(-log.(rand(100_000_000)))
\end{lstlisting}
\begin{lstlisting}[language=stata]
	set obs 100000000
	gen idiosyncratic = -ln(-ln(runiform()))
\end{lstlisting}
\begin{lstlisting}[language=R]
	idiosyncratic = -log(-log(runif(100000000)))
\end{lstlisting}
\end{itemize}
\end{frame}
% -----------------------------------------
\begin{frame}[fragile]{Simulate what? Simulating idiosyncratic preferences vs choices}
Consider two ways to simulate draws when using your estimated model as the data-generating process 
\textcolor{gray}{(e.g., $U_{ij} = V_{ij} + \epsilon_{ij}$ and $F\left(\epsilon_{ij}\right)=\exp(-\exp(-\epsilon_{ij}))$)}
\begin{enumerate}
\item Simulate the logit error terms $\epsilon_{ij}$ and compute utility-maximizing choices
\item Simulate the outcomes given by $\Pr(U_{ij}>U_{ij'} \ \forall j' \neq j) = \frac{\exp(V_{ij})}{\sum_{j'}\exp(V_{ij'})}$
\end{enumerate}
\smallskip
While logically equivalent, these differ tremendously in computational burden
\begin{itemize}
\item Time (in seconds) to produce 100 million Gumbel draws:\\
\qquad $\sim3\text{s}$ in Julia, $\sim5\text{s}$ in Stata, $\sim3.5\text{s}$ in R
\item Taking 100 draws from a multinomial distribution with 1 million possible choices takes $<0.1$ seconds
\end{itemize}
\begin{lstlisting}[language=julia]
using Distributions; mean_util = rand(1_000_000);
@time choices = rand(Multinomial(100,exp.(mean_util) ./ sum(exp.(mean_util))));
\end{lstlisting}
\end{frame}
% -----------------------------------------
\begin{frame}{Parametric bootstrap for confidence intervals in DDMM}
\vspace{-1mm}
\begin{itemize}
\item Draw 500 samples from estimated model (same size as estimation sample)
\item Estimate the model on each generated sample
\only<1-2>{\item Distributions for social frictions and restaurant characteristics look like asymptotic normal distribution}
\only<3-5>{\item Distributions for spatial frictions have fat tails because of collinearity of same-origin modes (\href{https://davegiles.blogspot.com/2011/09/micronumerosity.html}{see Goldberger} on multicollinearity and micronumerosity)}
\end{itemize}
\vspace{-3mm}
\begin{center}
\only<1>{Asian reviewers: Social frictions\\ \includegraphics[height=0.6\textheight]{../images/week11/bootstrap_param_mainspec_social_asian.png} }
\only<2|handout:0>{Black reviewers: Social frictions\\ \includegraphics[height=0.6\textheight]{../images/week11/bootstrap_param_mainspec_social_black.png} }
\only<3>{Asian reviewers: Spatial frictions\\ \includegraphics[height=0.6\textheight]{../images/week11/bootstrap_param_mainspec_spatial_asian.png} }
\only<4|handout:0>{Black reviewers: Spatial frictions\\ \includegraphics[height=0.6\textheight]{../images/week11/bootstrap_param_mainspec_spatial_black.png} }
\only<5>{Spatial frictions in minimum-time specification $\nu_{ijlt} = \nu_{ijt} \ \forall l$\\ \includegraphics[width=1.0\textwidth]{../images/week11/bootstrap_param_mintime_spatial.png} }
\\ \includegraphics[width=0.4\textwidth]{../images/week11/bootstrap_param_legend.png}
\end{center}
\end{frame}
% -----------------------------------------
\begin{frame}{Parametric bootstrap of dissimilarity indices in DDMM}
You can bootstrap any function of the parameters (i.e., any downstream object)
\vspace{-1mm}
\begin{center}
\includegraphics[height=0.86\textheight]{../images/week11/DDMM2019_table6panelA.png}
\end{center}
\end{frame}
% -----------------------------------------
\begin{frame}{Monte Carlo simulations to show finite-sample bias}
Recall Dingel \& Tintelnot (2023) critique of calibrated-shares procedure
\begin{itemize}
\item What parameterization of the baseline equilibrium should be used to compute counterfactual outcomes?
\item Calibrated-shares procedure employs observed shares
\item Covariates-based approach fits more parsimonious spec of commuting costs $\delta_{kn}$
\item An excessively flexible parameterization risks overfitting idiosyncratic noise
\item D \& T use Monte Carlo simulations to demonstrate the overfitting problem
\end{itemize}
\end{frame}
%---------------------------------------------------------------------
\begin{frame}{Monte Carlo in DT (2023): Applying each procedure to finite data}
\begin{itemize}
\item DGP is estimated covariates-based model for NYC in 2010
\item Simulated ``event'': $\uparrow$ productivity of 200 Fifth Ave tract by 9\%
\item Apply calibrated-shares procedure and covariates-based approach \\
\textcolor{gray}{\small (Increase ${A_n}$ to match total employment increase in simulated data)}
\item Does the procedure predict the change in the number of commuters from each residential tract working in the ``treated'' tract?
\item Regress ``true'' changes on predicted changes (2160 obs per simulation)
\item[] \textcolor{gray}{Ideally, want slope = 1 and intercept = 0}
\item Compute forecast errors (MSE for ``true'' vs predicted changes)
\end{itemize}
\end{frame}
%---------------------------------------------------------------------
\begin{frame}[t]{Monte Carlo in DT (2023): Calibrated-shares procedure overfits}
\hypertarget{ctfl:monte_carlo:mainslide}{}
\vspace{-3mm}
\footnotesize{
Apply each procedure to simulated ``2010'' data. % \& ``2012'' data.
100 simulations w/ $I=2,488,905$ \\
Changes in commuter counts ($\ell'_{k\bar{n}}-\ell_{k\bar{n}}$) 
}
\vspace{-2mm}
\begin{center}
\includegraphics[width=.43\textwidth]{../images/week11/slopes_intercepts_densities_continuum_labor_0_2.488905_1.09.eps}
\includegraphics[width=.43\textwidth]{../images/week11/MSE_ratio_histogram_continuum_labor_0_2.488905_1.09.eps}
\\
\resizebox{0.8\textwidth}{!}{
\input{../images/week11/sumstats_continuum_labor_0_1.09.tex}
}
\end{center}
\end{frame}
%---------------------------------------------------------------------
\begin{frame}{Monte Carlo in DT (2023): Calibrated-shares procedure overfits}
\footnotesize{
Apply each procedure to simulated ``2010'' data. % \& ``2012'' data.
100 simulations w/ $I=2,488,905$ \\
Changes in commuter counts ($\ell'_{k\bar{n}}-\ell_{k\bar{n}}$) 
via finite-sample draws from pre- and post- DGPs
}
\begin{center}
\includegraphics[width=.43\textwidth]{../images/week11/slopes_intercepts_densities_iid_labor_0_2.488905_1.09.eps}
\includegraphics[width=.43\textwidth]{../images/week11/MSE_ratio_histogram_iid_labor_0_2.488905_1.09.eps}
\end{center}
\vspace{-3mm}
\begin{columns}
\begin{column}{0.78\textwidth}
\resizebox{\textwidth}{!}{
\input{../images/week11/sumstats_iid_labor_0_1.09.tex}
}
\end{column}
\end{columns}
\end{frame}
%---------------------------------------------------------------------
%---------------------------------------------------------------------
\begin{frame}{A spatial model with a finite number of individuals}
Goal: examine the sensitivity of counterfactual outcomes to the idiosyncratic component of individual decisions
In the limit ($I\to\infty$), the equilibrium of our model with an integer number of individuals
is (almost surely) the equilibrium of the continuum model
Modeling concerns raised by the integer number of individuals:
\begin{itemize}
\item Individuals must have beliefs about equilibrium wages and land prices 
$$
\textcolor{gray}{
\left(\begin{array}{c}I + N^2 -1 \\ N^2 - 1 \end{array}\right) =  \frac{(I + N^2 -1)!}{(N^2 - 1)!I!}
\qquad 
I=10,N=4
\implies
3.27 \times 10^{6}
}
$$
\item There will be a \textit{distribution} of equilibria for each set of parameters $\Upsilon$
\end{itemize}
\end{frame}
%---------------------------------------------------------------------
\begin{frame}{Model: Economic environment}
\begin{itemize}
\item
Each location has productivity $A$ and land endowment $T$
\item
$I$ individuals are endowed with 
$L/I$ units of labor
and hired by 
competitive firms producing freely traded goods differentiated by location of production
\pause
\item
Individuals have Cobb-Douglas preferences over goods and land
\item
Individuals have idiosyncratic tastes for residence-workplace pairs
\item Workers know primitives $\Upsilon \equiv  (L,\{A_n\},\{T_k\},\{\bar{\delta}_{kn}\}, \{\lambda_{kn}\} ,\alpha , \epsilon , \sigma)$
and have 
(common) point-mass beliefs $\tilde{r}_k$ and $\tilde{w}_n$ about land prices and wages
\item Worker $i$ knows own idiosyncratic preferences $\{\nu_{kn}^{i}\}$ but not the full set of  idiosyncratic residence-workplace draws $\boldsymbol \nu^{I}$
\end{itemize}
\only<2>{}
\end{frame}
%---------------------------------------------------------------------
\begin{frame}{Timing: Individuals choose labor allocation, then markets clear}
\begin{enumerate}
	\item Workers choose the $kn$ pair that maximizes
	\begin{equation*}
	\tilde{U}_{kn}^{i} = \epsilon \ln\left(\frac{\tilde{w}_n}{\tilde{P}^{1-\alpha}\tilde{r}_{k}^{\alpha} \delta_{kn}}\right) + \nu_{kn}^i
	\end{equation*}
	given point-mass beliefs $\tilde{r}_k$ and $\tilde{w}_n$
	\item After choosing $kn$ based on their beliefs,
	workers are immobile and cannot relocate
	\item 
	Given the labor allocation $\{\ell_{kn}\}$ and economic primitives $\Upsilon$,
	\textbf{a trade equilibrium} is a set of wages $\{w_n\}$ and land prices $\{r_k\}$
	that clears all markets.
\end{enumerate}
\end{frame}
%---------------------------------------------------------------------
\begin{frame}{Commuting equilibrium with a finite number of individuals}
%=
%{\tilde{w}_{n}^\epsilon \left(\tilde{r}_{k}^\alpha  \delta_{kn}\right)^{-\epsilon}}
%{\sum_{k',n'} \tilde{w}_{n'}^\epsilon \left(\tilde{r}_{k'}^\alpha  \delta_{k'n'}\right)^{-\epsilon}}
%.
Given primitives $\Upsilon$,  idiosyncratic residence-workplace draws $\boldsymbol \nu^{I}$,
and 
point-mass beliefs $\{\tilde{w}_n\},\{\tilde{r}_k\}$,
a \textbf{commuting equilibrium with a finite number of individuals}, $I$, is defined as
a labor allocation $\{\ell_{kn}\}$, wages $\{w_n\}$, and land prices $\{r_k\}$
such that
\begin{itemize}
	%\item $\{\ell_{kn}\}$
	%is the labor allocation resulting from
	%$I$ independent draws from the probability function in equation \eqref{eqn:logitprobabilities};
\item $\ell_{kn} = \frac{L}{I} \sum_{i=1}^{I} \mathbf{1}\{\tilde{U}_{kn}^{i}(\boldsymbol \nu^{I}) > \tilde{U}_{k'n'}^{i}(\boldsymbol \nu^{I}) \ \forall (k',n') \neq (k,n)\}$;	
and
	\item wages $\{w_n\}$ and land prices $\{r_k\}$ are \textit{a trade equilibrium} given the labor allocation $\{\ell_{kn}\}$.
\end{itemize}
\end{frame}
%---------------------------------------------------------------------
\begin{frame}{Comparison with continuum model}
\begin{itemize}
\item Idiosyncratic residence-workplace draws $\boldsymbol \nu^{I}$ $\to$ distributions of equilibrium quantities and prices (for given primitives $\Upsilon$)
\item Mean equilibrium outcomes:
\begin{itemize}
\item Mean commuter counts coincide with those from the continuum model
{\footnotesize $\frac{\ell_{kn}}{L} = \mathbb{E}\left[\Pr(U_{kn}^{i} > U_{k'n'}^{i} \ \forall (k',n') \neq (k,n))\right]$}\\
\item Land prices and wages are solved from a non-linear system of equations 
\end{itemize}
\item Variance of equilibrium outcomes due to idiosyncrasies
\begin{itemize}
\item Confidence interval for residents, workers, wages, and prices
\end{itemize}
\item In counterfactual exercises: Change from $\Upsilon$ to $\Upsilon'$ for given $\boldsymbol \nu^{I}$
\item The set of individuals who change their decisions in response to the change in economic primitives
depends on the particular realized vectors of idiosyncratic preferences
\item The dispersion in this distribution represents uncertainty about counterfactual predictions stemming from individual idiosyncrasies
\end{itemize}
\end{frame}
%---------------------------------------------------------------------
\begin{frame}{Sizable uncertainty about predicted changes from idiosyncrasies}
\begin{center}
\begin{tabular}{cc}
Changes in residents & Changes in workers \\
\includegraphics[width=0.50\linewidth]{../images/week11/fixednu_puncertainty_scatter_res.eps}
&
\includegraphics[width=0.50\linewidth]{../images/week11/fixednu_puncertainty_scatter_emp.eps}
\end{tabular}
\end{center}
\end{frame}
%---------------------------------------------------------------------
\begin{frame}{Sizable uncertainty about predicted changes from idiosyncrasies}
\begin{center}
\begin{tabular}{cc}
{ Changes in rents} & { Changes in wages}\\
\includegraphics[width=0.50\linewidth]{../images/week11/fixednu_puncertainty_scatter_rent.eps}
&
\includegraphics[width=0.50\linewidth]{../images/week11/fixednu_puncertainty_scatter_wage.eps}
\end{tabular}
\end{center}
\end{frame}
%---------------------------------------------------------------------
\begin{frame}{Simulate what? Simulating idiosyncratic preferences vs choices}
Holding $\boldsymbol \nu^{I}$ fixed requires simulating 8 trillion Gumbel draws
\begin{itemize}
\item Change from $\Upsilon$ to $\Upsilon'$ for given $\boldsymbol \nu^{I}$:
$\sim100$ CPU hours per simulation
\item Contrast outcome distributions for $\Upsilon$ and $\Upsilon'$:
$<1$ CPU hour per simulation
\end{itemize}
Do the cheaper simulations first
\begin{itemize}
\item Can you rule out uncertainty attributable to individual idiosyncrasies?
Repeatedly draw multinomial realizations
to compute a distribution of differences between finite-sample outcomes at counterfactual parameters $\Upsilon'$ and the expected outcome at baseline parameters $\Upsilon$.
\item {This distribution of differences is more dispersed than
the distribution of counterfactual changes from the model with a finite number of individuals that fixes the vector of idiosyncratic preferences $\boldsymbol \nu^{I}$.\par}
\end{itemize}
\end{frame}
% -----------------------------------------
\begin{frame}{Summary}
\begin{itemize}
\item CES and logit are close cousins; constant-elasticity siblings, really
\item Reading a couple chapters of Train (2009) goes a long way
\item Smart choices can speed your computations by orders of magnitude
\item Econometrics by simulation is often a good starting point
\end{itemize}
\end{frame}
% -----------------------------------------
\begin{frame}{Next week}
\begin{itemize}
\item Spatial policies
\item Any requests?
\end{itemize}
\end{frame}
% -----------------------------------------
\end{document}
