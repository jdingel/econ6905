\documentclass[11pt,notes=hide,aspectratio=169]{beamer}
%Jonathan Dingel; PhD trade course

% PACKAGES
\usepackage{graphics}  % Support for images/figures
\usepackage{graphicx}  % Includes the \resizebox command
\usepackage{url}	   % Includes \urldef and \url commands
\usepackage{soul}      % Includes the underline \ul command
%\usepackage{framed}	   % Includes the \framed command for box around text
\usepackage{booktabs} %\toprule,\bottomrule
%\usepackage{natbib}
\usepackage{bibentry}  % Includes the \nobibliography command
\usepackage{bbm}       %
%\usepackage{pgfpages}  %Supports "notes on second screen" option for beamer
\usepackage{verbatim}  %Supports comments
\usepackage{tikz}		%Supports graphing/drawing
\usepackage{pgfplots} %Supports graphing/drawing
\usepackage{amsfonts}  % Lots of stuff, including \mathbb 
\usepackage{amsmath}   % Standard math package
\usepackage{amsthm}    % Includes the comment functions
\usepackage{physics}

% CUSTOM DEFINITIONS
\def\newblock{} %Get beamer to cooperate with BibTeX
\linespread{1.2}
\hypersetup{backref,pdfpagemode=FullScreen,colorlinks=true,linkcolor=blue,urlcolor=blue}
\newtheorem{proposition}{Proposition}
\newtheorem{assumption}{Assumption}
\newtheorem{condition}{Condition}

% IDENTIFYING INFORMATION
\title{Topics in Trade}
\author{Jonathan I. Dingel}
\date{Fall \the\year}

% BEAMER TEACHING STUFF
\setbeamertemplate{navigation symbols}{}  %Turn off navigation bar

% THEMATIC OPTIONS
\definecolor{columbiablue}{RGB}{185,217,235}  %Columbia blue defined at https://visualidentity.columbia.edu/branding
\definecolor{columbiadarkblue}{RGB}{0,48,135}  %Columbia dark blue defined at https://visualidentity.columbia.edu/branding
\setbeamercovered{transparent=5}
\setbeamercolor{frametitle}{fg=columbiadarkblue}
\setbeamercolor{item}{fg=columbiadarkblue}
\usefonttheme{serif}
\setbeamercolor{button}{bg = white,fg = columbiadarkblue}
\setbeamercolor{button border}{fg = columbiadarkblue}

\setbeamertemplate{footline}{\begin{center}\textcolor{gray}{Dingel -- Topics in Trade -- \semester -- Week 8 -- \insertframenumber}\end{center}}
\begin{document}
% -----------------------------------------
\begin{frame}[plain]
\begin{center}
\large
\textcolor{columbiadarkblue}{ECON G6905\\
Topics in Trade\\ 
Jonathan Dingel\\
\semester, Week 8}
\vfill 
\includegraphics[width=0.4\textwidth]{../images/Columbia_logo.png}
\end{center}
\end{frame}
% -----------------------------------------
\begin{frame}{Empirical observations about regional economies}
\begin{itemize}
\item Economic activity is geographically concentrated (see last week) 
\item Regions are connected via trade flows: most metropolitan areas spend more on shipments from other metros than on their own output (Allen Arkolakis 2025)
\item The gravity equation for trade flows describes trade between regions well (Allen and Arkolakis 2025 graphs akin to those from Head and Mayer 2014)
\item Trade costs seem essential to understanding the economic geography of some industries (e.g., Holmes and Stevens 2014 on sugar beet processing)
\end{itemize}
\end{frame}
% -----------------------------------------
\begin{frame}{This week: Quantitative regional models}
\begin{itemize}
\item Last week's models (i.e., Rosen-Roback) had no trade or spatial linkages
\item Krugman (1991) introduces two-region model with trade costs and market-size consequences (core-periphery story is applied theory)
\item Modern QRMs are multi-region models designed to be taken to data (quantitative counterfactual scenarios)
\item[] \textcolor{gray}{Hallmarks (like QTMs): many locations, few elasticities, many shifters}
\end{itemize}
Within broader class of ``quantitative spatial models''
\begin{itemize}{\small
\item A general-equilibrium approach:
``locations are not independent observations in a cross-sectional regression but rather are systematically linked to one another through trade, commuting, and migration flows'' (\href{https://doi.org/10.1146/annurev-economics-063016-103713}{Redding \& Rossi-Hansberg 2017})
\item Distinguish quantitative regional economics (goods and labor move between cities-as-points)
from
quantitative urban economics (commuting flows within cities)
\item What is a QSM? \href{https://www.aeaweb.org/articles?id=10.1257/jel.20181414}{Proost and Thisse} (Sec 5.2, 2019) almost answer
\par}\end{itemize}
\end{frame}
% -----------------------------------------
\begin{frame}{Krugman ``Increasing Returns and Economic Geography'' (1991)}
\begin{itemize}
	\item The second of Krugman's pair of Nobel-winning papers
	\item Apply tools from `70s theoretical IO and `80s trade models to look at the geographic concentration of industry
	\item General story about IRS manufacturing vs CRS agriculture rather than industry localization
	\item Circular causation: ``manufactures production will tend to concentrate where there is a large market, but the market will be large where manufactures production is concentrated''
	\item Formalize this unoriginal story
	\item Microfoundations with clear pecuniary externalities
	\item Comparative statics: Agglomeration depends on transport costs, economies of scale, and manufacturing share [exogenous parameters in Krugman's account]
	\item Key restrictions: Immobile peasants and only two locations
\end{itemize}
\end{frame}
% -----------------------------------------
\begin{frame}{Krugman (JPE 1991): ``II. A Two Region Model''}
\begin{columns}
\begin{column}{.5\textwidth}
\begin{small}
\begin{align*}
U 
&=
C_M^{\mu}C_A^{1-\mu}
&
\text{(1)}
\\
C_M 
&=
\left[\sum_{i=1}^{N} c_i^{(\sigma-1)/\sigma}\right]^{\sigma/(\sigma-1)}
&
\text{(2)}
\\
L_1 &+ L_2 
=
\mu
&
\text{(3)}
\\
L_{Mi} 
&= 
\alpha + \beta x_i
&
\text{(4)}
\\
p_1 
&=
\frac{\sigma}{\sigma-1} \beta w_1
&
\text{(5)}
\\
\frac{p_1}{p_2}
&=
\frac{w_1}{w_2}  
&
\text{(6)}
\\
(p_1 &- \beta w_1) x_1 
=
\alpha w_1
&
\text{(7)}
\\
x_1 &= x_2 = \frac{\alpha}{\beta} (\sigma-1)
&
\text{(8)}
\\
\frac{n_1}{n_2} &= \frac{L_1}{L_2}
&
\text{(9)}
\end{align*}
\end{small}
\end{column}
\begin{column}{.48\textwidth}{\small
\begin{itemize}
	\item Immobile peasants, mobile workers, and ``clever'' choice of units in (3)
	\item Usual Dixit-Stiglitz monopolistic competition setup, but $L_i$ endogenous
	\item Freely traded CRS good but still wages $w_1$ and $w_2$
	\item As in Krugman (1980), all action on extensive margin
	\item Iceberg trade costs: Only $\tau<1$ arrives, send $1/\tau$ [inverse of modern notation]
\end{itemize}
}\end{column}
\end{columns}
\end{frame}
% -----------------------------------------
\begin{frame}{III. Short-Run Equilibrium}
$c_{ij}$ is consumption in $i$ of a variety from $j$; 
$z_{1i}$ is relative expenditure in $i$ on varieties from 1;
that's awkward notation;
short-run eqlbm is $w_i,z_{1i}\vert L_i$
\begin{small}
\begin{align*}
\frac{c_{11}}{c_{12}} 
&= \left(\frac{p_1\tau}{p_2}\right)^{-\sigma} 
= \left(\frac{w_1\tau}{w_2}\right)^{-\sigma}
&
\text{(10)}
\\
z_{11} &= \left(\frac{n_1}{n_2}\right)  \left(\frac{p_1\tau}{p_2}\right) \left(\frac{c_{11}}{c_{12}}\right) 
= \left(\frac{L_1}{L_2}\right) \left(\frac{w_1\tau}{w_2}\right)^{-(\sigma-1)}
&
\text{(11)}
\\
z_{12} 
&=
\left(\frac{L_1}{L_2}\right) \left(\frac{w_1}{w_2\tau}\right)^{-(\sigma-1)}
&
\text{(12)}
\\
w_1 L_1 &= 
\mu \left[\left(\frac{z_{11}}{1+z_{11}}\right) Y_1 + \left(\frac{z_{12}}{1+z_{12}}\right) Y_2   \right]
&
\text{(13)}
\\
w_2 L_2 &= 
\mu \left[\left(\frac{1}{1+z_{11}}\right) Y_1 + \left(\frac{1}{1+z_{12}}\right) Y_2   \right]
&
\text{(14)}
\\
Y_1 &= (1-\mu)/2 + w_1 L_1
&
\text{(15)}
\\
Y_2 &= (1-\mu)/2 + w_2 L_2
&
\text{(16)}
\end{align*}
\end{small}
\end{frame}
% -----------------------------------------
\begin{frame}{III. Long-Run Equilibrium}
\begin{itemize}
\item $L_i$ is endogenous, not fixed. Let $f \equiv L_1/\mu$.
\item Workers care about real wages $\omega_i$, not nominal wages $w_i$.
\end{itemize}
\begin{align*}
P_1 &= \left[fw_1^{-(\sigma-1)} + (1-f) \left(\frac{w_2}{\tau}\right)^{-(\sigma-1)} \right]^{-1/(\sigma-1)}
&
\text{(17)}
\\
P_2 &= \left[f\left(\frac{w_1}{\tau}\right)^{-(\sigma-1)} + (1-f) \left(w_2\right)^{-(\sigma-1)} \right]^{-1/(\sigma-1)}
&
\text{(18)}
\\
\omega_1 &= w_1 P_1^{-\mu}
&
\text{(19)}
\\
\omega_2 &= w_2 P_2^{-\mu}
&
\text{(20)}
\end{align*}
\vspace{-6mm}
\begin{itemize}
	\item Given $w_i$, greater $f$ lowers $P_1/P_2$ and therefore raises $\omega_1/\omega_2$.
	\item A race between home market effect and price index effect (convergence) and competition for sales to peasants (divergence)
	\item See Figure 1 for numerical example varying $\tau$
\end{itemize}
\end{frame}
% -----------------------------------------
\begin{frame}{IV. Necessary Conditions for Mfg Concentration}
Suppose there are $n$ manufacturing firms and all are in region 1.
Their value of sales is $V_1$ and a potential defector's value of sales is $V_2$. \\
Defecting firm must pay workers wage premium to compensate for cost of living.
\begin{align*}
\frac{Y_2}{Y_1} 
&=
\frac{1-\mu}{1+\mu}
& \text{(21)} \\
V_1 
&=
\left(\frac{\mu}{n}\right) \left(Y_1+Y_2\right)
& \text{(22)} \\
\frac{w_2}{w_1} 
&=
\left(\frac{1}{\tau}\right)^{\mu}
& \text{(23)} \\
V_2 
&=
\left(\frac{\mu}{n}\right)\left[\left(\frac{w_2}{w_1 \tau}\right)^{-(\sigma-1)}Y_1 + \left(\frac{w_2 \tau}{w_1}\right)^{-(\sigma-1)}Y_2 \right]
& \text{(24)} \\
\frac{V_2}{V_1} 
&=
\frac{1}{2} \tau^{\mu(\sigma-1)}\left[\left(1+\mu\right)\tau^{\sigma-1} + (1-\mu)\tau^{-(\sigma-1)}\right]
& \text{(25)}
\end{align*}
Defection profitable if $V_2/V_1 > w_2 / w_1 = \tau^{-\mu}$.
\end{frame}
% -----------------------------------------
\begin{frame}{IV. Necessary Conditions for Mfg Concentration}
Defection profitable if $V_2/V_1 > \tau^{-\mu} \iff \nu <1$.
\begin{align*}
\nu 
&=
\frac{1}{2} \tau^{\mu\sigma}\left[\left(1+\mu\right)\tau^{\sigma-1} + (1-\mu)\tau^{-(\sigma-1)}\right]
& \text{(26)} \\
\frac{\partial\nu}{\partial\mu} 
&=
\nu\sigma(\ln\tau) + \frac{1}{2}\tau^{\sigma\mu} \left[\tau^{\sigma-1} - \tau^{-(\sigma-1)}\right]
< 0
& \text{(27)} \\
\frac{\partial\nu}{\partial\tau} 
&=
\frac{\mu\sigma\nu}{\tau} + \frac{\tau^{\sigma\mu}(\sigma-1)\left[\left(1+\mu\right)\tau^{\sigma-1} - (1-\mu)\tau^{-(\sigma-1)}\right]}{2\tau}
& \text{(28)} \\
\frac{\partial\nu}{\partial\sigma} 
&=
\ln(\tau) \left\{\mu\sigma + \frac{1}{2}\tau^{\mu\sigma}\left[\left(1+\mu\right)\tau^{\sigma-1} - (1-\mu)\tau^{-(\sigma-1)}\right]\right\}
\\
&= \ln(\tau) \left(\frac{\tau}{\sigma}\right) \left(\frac{\partial \nu}{\partial \tau}\right)
& \text{(29)} \\
\end{align*}
Note $\tau=1\implies\nu=1$.
\end{frame}
% -----------------------------------------
\begin{frame}{The bifurcated ``tomahawk'' diagram}
\begin{columns}
\begin{column}{.45\textwidth}
\includegraphics[width=1.15\textwidth]{../images/Neary2001_fig2.pdf}
\begin{center}
{\footnotesize Figure from \href{https://www.aeaweb.org/articles?id=10.1257/jel.39.2.536}{Neary (2001)}}
\end{center}
\end{column}
\begin{column}{.53\textwidth}
\begin{itemize}
	\item {\small Notation: $1/\tau \to T$, $f \to \lambda$}
	\item Krugman (1991) established a ``sustain point'' $\nu$ necessary for concentration
	\item There is also a ``break point'' that is sufficient for concentration to be the (stable) equilibrium outcome (Puga 1999)
	\item Higher trade costs can prevent regional divergence; sufficiently low trade costs rule out symmetric outcomes
\end{itemize}
\vfill
\end{column}
\end{columns}
\end{frame}
% -----------------------------------------
\begin{frame}{More on that diagram and stability}
\begin{itemize}
	\item If workers ``move faster'' than firms, same conclusions about break and sustain points (Puga 1999; see Figure 1 below; $\tau \geq 1$)
	\item There are at most two interior asymmetric steady states. If they exist, they're unstable. (\href{https://ideas.repec.org/a/oup/jecgeo/v5y2005i2p201-234.html}{Robert-Nicoud 2005})
\end{itemize}
\includegraphics[width=.32\textwidth]{../images/Puga1999_fig1a.pdf}
\includegraphics[width=.32\textwidth]{../images/Puga1999_fig1b.pdf}
\includegraphics[width=.32\textwidth]{../images/Puga1999_fig1c.pdf}
\end{frame}
% -----------------------------------------
% -----------------------------------------
\begin{frame}{``Of Hype and Hyperbolas''}
\href{https://www.aeaweb.org/articles?id=10.1257/jel.39.2.536}{Neary (\textit{JEL} 2001)} on Fujita, Krugman, Venables monograph:
\begin{itemize}
{\small
\item ``the model used throughout the book has a number of special features that make it less suitable for addressing some issues''
\item ``As the authors disarmingly admit, the book `sometimes looks as if it should be entitled \textit{Games You Can Play with CES Functions}!'''
\item ``Though $\sigma$ starts as a taste parameter, it ends up as an index of returns to scale
\item ``the Dixit-Stiglitz model has almost nothing to say about individual firms''
\item ``while costs may be fixed they are never sunk, so firms, industries, and even cities are always free to move''
\item The ``iceberg'' assumption: ``Of all industries, it seems to be characterized by very high ratios of fixed to variable costs''
\item \href{https://www.jstor.org/stable/116870}{Davis (1998)}: This doesn't work with comparable agricultural trade costs
\item ``The book deals in turn with regions, cities and countries, but there is nothing intrinsic to the models which conclusively identifies these units.''
}
\end{itemize}
\end{frame}
% -----------------------------------------
\begin{frame}{Helpman (1998): Overview}
\begin{itemize}
	\item Vocab: centripetal = agglomeration, centrifugal  = dispersion
	\item Replace freely traded agriculture with non-traded fixed factor (housing) $\Rightarrow$ Cobb-Douglas preferences over housing and varieties
	\item Replace location-bound peasant income with assumption that housing is owned equally by all individuals (regardless of location) who spend income where they live
	\item These two deviations flip the comparative statics for trade costs in Krugman (1991)! ``While in Krugman's model low transport costs lead to agglomeration and high transport costs lead to dispersion, in my model, low transport costs lead to dispersion and high transport costs lead to agglomeration.''
	\item The first deviation (freely traded ag) is the key
	\item Question: What's the difference in equilibria with freely traded manufactures?
	\item Also evaluates welfare efficiency of market equilibrium
\end{itemize}
\end{frame}
% -----------------------------------------
\begin{frame}{Helpman (1998): Setup}
\begin{itemize}
	\item $\beta$ is Cobb-Douglas expenditure share on housing
	\item $\epsilon$ is elasticity of substitution across differentiated varieties
	\item $t>1$ is the iceberg trade cost
	\item $f = N_1 / N$ is the population share of region 1
	\item $v = u_1 / u_2 $ is the relative utility level of 1
\end{itemize}
\end{frame}
% -----------------------------------------
\begin{frame}{Helpman (1998)}
\begin{itemize}
{\small
	\item When $t \to 1$, only housing prices matter: Go to less populous place
	\item When $\beta\epsilon>1$, housing prices relatively more important than traded prices
	\item When $\beta\epsilon<1$, differentiated products are poor substitutes and demand for housing is low. In this case, trade costs matter.
}
\end{itemize}
\includegraphics[height=.60\textheight]{../images/Helpman1995_fig1.pdf}
\includegraphics[height=.60\textheight]{../images/Helpman1995_fig2.pdf}
\end{frame}
% -----------------------------------------
\begin{frame}{Helpman (1998): The ``tomahawk'' reverses}
\includegraphics[width=.19\textwidth]{../images/Helpman1995_fig1.pdf}
\includegraphics[width=.19\textwidth]{../images/Helpman1995_fig2.pdf}
\includegraphics[width=.29\textwidth]{../images/Helpman1995_fig3.pdf}
\includegraphics[width=.29\textwidth]{../images/Helpman1995_fig4.pdf}
\end{frame}
% -----------------------------------------
\begin{frame}{Redding and Sturm (2008)}
\begin{itemize}
{\small
	\item ``We exploit the division of Germany after the Second World War and the reunification of East and West Germany in 1990 as a source of exogenous variation to provide evidence for the \textcolor{red}{causal} importance of market access for economic development.''
	\item ``The key idea behind our empirical approach is that West German cities close to the new border experienced a disproportionate loss of market access relative to other West German cities.''
	\item Extend Helpman (1998) to a multi-region version (assume that $\beta\epsilon>1 \Rightarrow$ unique equilibrium)
	\item Difference-in-differences design compares proximate vs distant West German cities before and after division
	\item Triple difference: ``the greater dependence of small cities on markets in other cities implies that this effect will be particularly pronounced for small cities.''
	\item[] \textcolor{gray}{Note the shift from qualitative applied theory question to quantitative empirical application}
\par}
\end{itemize}
\end{frame}
% -----------------------------------------
\begin{frame}{Redding and Sturm (2008): Difference in differences}
\begin{center}
\includegraphics[height=.75\textheight]{../images/ReddingSturm2008_fig3.pdf}
\end{center}
{\small Table 2 reports estimate of -0.75 (se 0.18) for the coefficient on the interaction of ``within 75km of border'' dummy and 1950--1988 dummy\par}
\end{frame}
% -----------------------------------------
\begin{frame}{Redding and Sturm (2008): Quantitative model fit}
{\small Model can explain ``the quantitative magnitude of the relative decline of small and large cities along the East-West German border relative to other West German cities''\par}
\begin{columns}
\begin{column}{0.45\textwidth}
	\begin{itemize}
	{\small
	\item Choose parameter values to minimize the distance between small-large third difference in the simulation and data
	\item Finding ``plausible parameter values'' is ``further evidence that their relative decline is indeed due to a loss of market access''
	\item Contrast this fit check with non-falsifiable calibrations in later literature
	\par}
\end{itemize}
\end{column}
\begin{column}{0.53\textwidth}
\includegraphics[height=.70\textheight]{../images/ReddingSturm2008_fig6.pdf}
\end{column}
\end{columns}
\end{frame}
% -----------------------------------------
\begin{frame}
\frametitle{Allen and Arkolakis: ``Trade and Topography of the Spatial Economy''}
\begin{itemize}
\item Stylized geographies (e.g., line or circle) are awkward for empirical application
\item Authors develop a quantitative framework (more general than Redding-Sturm) that extends new economic geography to much broader class
\item Derive sufficient conditions for existence and uniqueness of equilibrium in spatial geography models (with a continuum of locations)
\item RR17: ``a major contribution of this quantitative economic geography literature has been to preserve sufficient analytical tractability to provide conditions under which there exists a unique spatial equilibrium distribution of economic activity and to permit some analytical comparative statics''
\item Illustrative quantitative exercise: What was the welfare effect of the interstate highway system?
\end{itemize}
\end{frame}
% -----------------------------------------
\begin{frame}{AA `14: Model -- Geography}
\begin{itemize}
\item Continuum of locations, $i \in S$ ($S$ is a closed and bounded set of a finite dimensional Euclidean space) 
\item Location $i \in S$ with population $L(i)$:
\begin{itemize}
\item Endowed with differentiated variety (Armington assumption)
\item Productivity: $A(i) = \bar{A}(i) L(i)^\alpha$ [$\bar{A}(i)$ is exogenous]
\item Amenity: $u(i) = \bar{u}\left(i\right) L(i)^\beta $ [$\bar{u}(i)$ is exogenous part]
\end{itemize}
\item Spillovers are local (only affect $i$). Presumably $\alpha \geq 0$, $\beta \leq 0$.
\item For all $i,j \in S$, symmetric iceberg bilateral trade cost $T\left(i,j\right)$
\item Together, $\bar{A}$, $\bar{u}$, and $T$ comprise the \textit{geography} of $S$
\item A geography is \textit{regular} if $\bar{A}$, $\bar{u}$, and $T$ are continuous and bounded above and below by strictly positive numbers.
\end{itemize}
\end{frame}
% -----------------------------------------
\begin{frame}{AA `14: Model -- Workers}
\begin{itemize}
\item Can choose to live/work in any location (static model with spatial equilibrium)
\item Receive wage $w\left(i\right)$ for their inelastically supplied unit of labor
\item CES preferences over locations' varieties with elasticity of substitution $\sigma>1$
\item Welfare in location $i$ is
\begin{equation*}
W\left(i\right)
=
\left(\int_{s\in S}q\left(s,i\right)^{\frac{\sigma-1}{\sigma}}ds\right)^{\frac{\sigma}{\sigma-1}}u\left(i\right)
=
\frac{w(i)}{P(i)}u(i)
\end{equation*}
 where $q\left(s,i\right)$ is the per capita quantity consumed in location $i$ of the good produced in location $s$ and $u\left(i\right)$ is the local amenity.
\end{itemize}
\end{frame}
% -----------------------------------------
\begin{frame}{AA `14: Model -- Production}
\begin{itemize}
\item Labor is the only factor of production, $L(i)$  is the density of workers.
\item Productivity of worker in location $i$ is $A(i)$
\item Price of good from $i$ is $\frac{w\left(i\right)}{A\left(i\right)}T\left(i,j\right)$  in location $j$
\item Trade flows: $X(i,j) = \left (\frac{T(i,j)w(i)}{A(i)P(j)}\right)^{1-\sigma} w(j)L(j)$
\item Price index: $P(j)^{1-\sigma} = \int_{S}  T(s,j)^{1-\sigma} A(s)^{\sigma-1} w(s)^{1-\sigma} ds$
\end{itemize}
\end{frame}
% -----------------------------------------
\begin{frame}{AA `14: Model -- Equilibrium}
\begin{itemize}
\item A spatial equilibrium is a distribution of economic activity (functions $w$ and $L$) such that:
\begin{itemize}
\item Markets clear, i.e. $w\left(i\right)L\left(i\right)=\int_{S}X\left(i,s\right)ds$, 
\item Welfare is equalized, i.e. $W\in\mathbb{R}_{++}$  such that for all $ i\in S$ , $W\left(i\right)\leq W$, with equality if $L\left(i\right)>0$, 
\item The aggregate labor market clears, i.e. $\int_{S}L\left(s\right)ds=\bar{L}$ . 
\end{itemize}
\medskip
\item A spatial equilibrium is \textit{regular} if $L$ and $w$ are continuous and strictly positive (i.e. every location is inhabited).\medskip
\item A spatial equilibrium is point-wise locally stable if $\frac{dW\left(i\right)}{dL\left(i\right)}<0$ for all $  i\in S$ (i.e. no small number of workers can increase welfare by moving to another location; similar to Henderson 1974).
\end{itemize}
\end{frame}
% -----------------------------------------
\begin{frame}
\frametitle{AA `14: Solving for equilibrium}
Plugging the expression for trade flows and indirect utility function into the goods market clearing condition yields:
\begin{equation*}
L(i)w(i)^{\sigma}=\int_{S} W(s)^{1-\sigma} T(i,s)^{1-\sigma}A(i)^{\sigma-1}u(s)^{\sigma-1}L(s)w(s)^{\sigma}ds 
\end{equation*}
Combining the price index with the indirect utility function yields:
\begin{equation*}
w(i)^{1-\sigma}=\int_{S} W(s)^{1-\sigma} T(s,i)^{1-\sigma}u\left(i\right)^{\sigma-1}A(s)^{\sigma-1}w(s)^{1-\sigma}ds 
\end{equation*}
We are looking for functions $w(i)$ and $ L(i)$ that solve these equations and will look for equilibria in which every location is inhabited (regular equilibrium); hence we consider $W(s) = W \ \forall s$
\end{frame}
% -----------------------------------------
\begin{frame}{AA `14: Solving for equilibrium without spillovers}
With no productivity nor amenity spillovers ($\alpha=\beta=0$) and welfare equalized across space, the (discrete analogues of the) previous equations can be written as:
\begin{align*}
g &= \lambda K g \\
f &= \lambda K' f
\end{align*}
with eigenfunctions $g(i) = L(i) w(i)^{\sigma}$ and $f(i) = w(i)^{1-\sigma}$
and eigenvalues $\lambda=W^{1-\sigma}$.
Let $K(i,j) = T(i,j)^{1-\sigma} \bar{A}(i)^{\sigma-1} \bar{u}(j)^{\sigma-1}  > 0 \ \forall i,j$.
The two kernels $K$ and $K'$ are transposes of each other.
``extensions of standard results in linear algebra guarantee the existence and uniqueness of a positive solution''
Solution by function iteration: 
$$
f_{n+1}(i) = \frac{\int_{S}K(i,s)f_n(s)\textrm{d}s}{\int_{S}\int_{S}K(i,s)f_n(s)\textrm{d}s\textrm{d}i}
$$
\end{frame}
% -----------------------------------------
\begin{frame}{Connecting no-spillovers case to Roback (1982)}
With free trade ($T(i,s) = 1 \forall i,s$) and equal welfare:
\begin{align*}
L(i)w(i)^{\sigma}
&=
W^{1-\sigma} \int_{S} \bar{A}(i)^{\sigma-1}\bar{u}(s)^{\sigma-1}L(s)w(s)^{\sigma}\textrm{d}s 
\\
w(i)^{1-\sigma}
&=
W^{1-\sigma} \int_{S} \bar{u}(i)^{\sigma-1}\bar{A}(s)^{\sigma-1}w(s)^{1-\sigma}\textrm{d}s 
\end{align*}
Labor demand is $L(s) = A(s)^{\sigma-1} w(s)^{-\sigma} Y/P$
and labor supply is $W = u(i)w(i)/P$.
If $\bar{A}(i)^{\sigma-1}w(i)^{1-\sigma} = \phi L(i)w(i)^{\sigma}\bar{u}(i)^{\sigma-1}$ with $\phi>0$ then
two equations reduce to
\begin{align*}
L(i)^{\tilde{\sigma}} 
&=
\bar{u}(i)^{(1-\tilde{\sigma})(\sigma-1)} \bar{A}(i)^{\tilde{\sigma}(\sigma-1)} W^{1-\sigma}
\int_{S} \bar{A}(s)^{(1-\tilde{\sigma})(\sigma-1)} \bar{u}(s)^{\tilde{\sigma}(\sigma-1)} L(s)^{\tilde{\sigma}} \textrm{d} s
\end{align*}
where $\tilde{\sigma} \equiv \frac{\sigma-1}{2\sigma-1} <1$.
Solve for $L(i)$ and $W$.
\textcolor{gray}{(Draw $\ln w(i)$ vs $\ln L(i)$ diagram)}
\end{frame}
% -----------------------------------------
\begin{frame}{AA `14: Solving for equilibrium with spillovers}
When there are productivity or amenity spillovers and welfare is equalized across space, the previous equations yield:
\begin{align*}
L(i)^{1-\alpha(\sigma-1)}w(i)^{\sigma}&=W^{1-\sigma}\int_{S}T(i,s)^{1-\sigma}\bar{A}(i)^{\sigma-1}\bar{u}(s)^{\sigma-1}L(s)^{1+\beta(\sigma-1)}w(s)^{\sigma}ds \\ 
w(i)^{1-\sigma}L(i)^{\beta(1-\sigma)}&=W^{1-\sigma}\int_{S}T(s,i)^{1-\sigma}\bar{A}(s){}^{\sigma-1}\bar{u}(i){}^{\sigma-1}w(s)^{1-\sigma}L(s)^{\alpha(\sigma-1)}ds
\end{align*}
If trade costs are symmetric, it turns out the system can be written as
\begin{align}
A(i)^{\sigma-1}w(i)^{1-\sigma} =\phi L(i)w(i)^{\sigma}u(i)^{\sigma-1} \label{eq:bla} \\
L(i)^{\tilde{\sigma}\gamma_{1}} =K_{1}(i)W^{1-\sigma}\int_{S}T\left(s,i\right)^{1-\sigma}K_{2}(s)\left(L(s)^{\tilde{\sigma}\gamma_{1}}\right)^{\frac{\gamma_{2}}{\gamma_{1}}}ds, 
\end{align}
where $K_{1}(i)$  and $K_{2}(i)$ are functions of $\bar{A}(i)$ and $\bar{u}(i)$, $\gamma_{1}$, $\gamma_{2}$, and $\tilde{\sigma}$ are functions of $\alpha$, $\beta$, and $\sigma$. 
\\
The last equation is a Hammerstein non-linear integral equation 
\end{frame}
% -----------------------------------------
\begin{frame}
\frametitle{AA `14: Existence and uniqueness (with spillovers)}
\textbf{Theorem 2}: Consider any regular geography with endogenous productivity and amenities with $T$ symmetric. Define $\gamma_{1} = 1-\alpha\left(\sigma-1\right)-\beta\sigma$, and $\gamma_{2} = 1+\alpha\sigma+\left(\sigma-1\right)\beta$. If $\gamma_{1}\neq0$, then:
\begin{enumerate}
\item There exists a regular equilibrium. 
\item If $\gamma_{1}<0$, no regular equilibria are point-wise locally stable.
\item If $\gamma_{1}>0$, all equilibria are regular and point-wise locally stable.\medskip
\item If $\frac{\gamma_{2}}{\gamma_{1}}\in(-1,1]$, the equilibrium is unique and can be computed iteratively. 
\end{enumerate}
Note that
\begin{equation*}
W(i) = \frac{\left( \int_{S} T(i,s)^{1-\sigma}P(s)^{\sigma-1} w(s) L(s) ds  \right)^{\frac{1}{\sigma}}}{P(i)} \bar{A}(i)^{\frac{\sigma-1}{\sigma}}\bar{u}(i)L(i)^{-\frac{\gamma_{1}}{\sigma}}
\end{equation*}
hence parts 2 and 3 follow from $\text{sign}\left(\frac{dW(i)}{dL(i)}\right)=-\text{sign}(\gamma_1)$
\\ 
Sufficient conditions for uniqueness satisfied only if no net spillovers, i.e. $\alpha + \beta \leq 0$
\end{frame}
% -----------------------------------------
\begin{frame}
\frametitle{AA `14: Existence and uniqueness (with spillovers)}
\begin{center}
\includegraphics[height=.92\textheight]{../images/AllenArkolakis2014_fig1.pdf}
\end{center}
\end{frame}
% -----------------------------------------
\begin{frame}{AA `14: Geographic component $T$}
\begin{itemize}
	\item Apart from everything above, Allen and Arkolakis (2014) introduce the ``fast marching method'' into spatial economics
	\item Suppose $S$ is a compact surface (e.g., line, plane, cow) 
	\item Let $\tau: S\to\mathbb{R}_{+}$ be a continuous function where $\tau(i)$ is the instantaneous cost of traveling over location $i$
	\item Trade cost $T(i,j) = f(t(i,j)), f'>0, f(0)=1$ is the total iceberg trade cost incurred along least-cost route from $i$ to $j$
	$$t(i,j) = \min_{g\in\Gamma(i,j)} \int_{0}^{1} \tau(g(t)) \left\Vert \frac{dg(t)}{dt} \right\Vert dt$$
	where $g:[0,1]\to S$ is a path and $\Gamma$ is set of possible continuous once-differentiable paths
\end{itemize}
\end{frame}
% -----------------------------------------
\begin{frame}{AA `14: Eikonal equations and FMM}
\begin{itemize}
	\item Previous equation is oft-studied in physics (wave propagation)
	\item A necessary condition for its solution is the following eikonal partial differential equation
	$$\left\Vert \nabla t(i,j) \right\Vert =\tau(j) $$
	where the gradient is taken with respect to the destination $j$.
	\item One solution algorithm for this is the fast marching method
	\item ``FMM can be interpreted as a generalization of Dijkstra to continuous spaces''
	\item \href{https://sites.google.com/site/treballen/research}{Treb's website} has an example of implementing FMM in Matlab
	\item In R: \href{https://cran.r-project.org/web/packages/fastmaRching/index.html}{fastmaRching}
	\item Julia is popular with people who solve PDEs: \href{https://github.com/SciML/DifferentialEquations.jl}{DifferentialEquations.jl}
	\item I have not yet tried \href{https://github.com/JuliaInv/EikonalInv.jl}{EikonalInv.jl}, but I hope to soon
\end{itemize}
\end{frame}
%-----------------------------------------
\begin{frame}{Application to the US economy}
\begin{itemize}
\item Estimate bilateral trade costs
\item Given trade costs, identify (composite) productivities and amenities 
\item Quantify the importance of geographic location
\item Perform counterfactual exercise: remove the Interstate Highway System.
\item Note: Cannot identify $\alpha, \beta, \sigma$; they do analysis for a large variety of ($\alpha, \beta$) while assuming $\sigma$ = 9.
\end{itemize}
\end{frame}
% -----------------------------------------
\begin{frame}
\frametitle{AA `14: Estimating trade costs}
Goal: Find trade costs that best rationalize the bilateral trade flows observed in 2007 Commodity Flow Survey (CFS).
Three-step process:
\begin{enumerate}
\item Using fast marching method and observed transportation network, calculate the (normalized) distance between every CFS area for each major mode of travel (road, rail, air, and water).
\item Using a discrete-choice framework and observed mode-specific bilateral trade shares, estimate the relative cost of each mode of travel.
(``the discrete choice framework is entirely distinct from the economic geography model developed in Section II and is used only as a tool to estimate trade costs based on mode-specific trade shares'')
\item Using a gravity model and observed total bilateral trade flows, pin down normalization (and incorporate non-geographic trade costs).
\end{enumerate}
\end{frame}
% -----------------------------------------
\begin{frame}{Practicalities}
\begin{itemize}
\item The Commodity Flow Survey only covers agriculture, manufacturing, mining, and wholesaling.
Where do tradable services show up in this model?
\item The market-clearing condition $w\left(i\right)L\left(i\right)=\int_{S}X\left(i,s\right)ds$ is a balanced-trade condition.
\item CFS areas are not counties
\item Gravity does not aggregate simply: sum of log-linear equations is not a log-linear equation
\item ``each CFS area (in the estimation of trade costs) and each county (in the estimation of overall productivities and amenities) are distinct locations''
\end{itemize}
\end{frame}
% -----------------------------------------
\begin{frame}
\frametitle{AA `14: Estimating trade costs}
\begin{itemize}
\item For any $i,j\in S$, assume $\exists$ traders $t$ choosing mode $m\in\{1,...,M\}$ of transit where cost is: $\exp\left(\tau_{m}d_{m}\left(i,j\right)+f_{m}+\nu_{tm}\right)$
 
\item Then mode-specific bilateral trade shares are:
\begin{equation} \label{eq:share}
\pi_{m}\left(i,j\right)=\frac{\exp\left(-a_{m}d_{m}\left(i,j\right)-b_{m}\right)}{\sum_{k}\left(\exp\left(-a_{k}d_{k}\left(i,j\right)-b_{k}\right)\right)},
\end{equation}
  where $a_{m} = \theta\tau_{m}$ and $b_{m} = \theta f_{m}$. 
  
\item Combined with model, yields gravity equation:
\begin{equation} \label{eq:flows}
\ln X_{ij}=\frac{\sigma-1}{\theta}\ln\sum_{m}\left(\exp\left(-a_{m}d_{mij}-b_{m}\right)\right)+\left(1-\sigma\right)\beta'\ln\mathbf{C}_{ij}+\delta_{i}+\delta_{j}
 \end{equation}
\item Estimate $a_{m}$ and $b_{m}$ using bilateral trade share eq (\ref{eq:share}), $\theta$ using gravity eq  (\ref{eq:flows}). Assume $\sigma = 9$.
\end{itemize}
\end{frame}
% -----------------------------------------
\begin{frame}
\frametitle{AA `14: Trade cost estimates}
\begin{figure}[htbp] \centering
\includegraphics[scale = .5, clip]{../images/AllenArkolakis2014_tab2.pdf}
\end{figure}
\end{frame}
% -----------------------------------------
\begin{frame}
\frametitle{AA `14: Recovering productivities and amenities}
\begin{itemize}
\item Given data on wages and population across space, productivities $A$ and amenities $u$ can be recovered. 
\item To see this, plug (\ref{eq:bla}) into the indirect utility function (after substituing the price index into it). This yields:
\begin{equation*}
u(i)^{1-\sigma} = \frac{W^{1-\sigma}}{\phi} \int_{S} T(s,i)^{1-\sigma} w(i)^{\sigma-1} w(s)^{\sigma}L(s)u(s)^{\sigma-1}ds 
\end{equation*}
\item Again, this is a Hammerstein non-linear integral equation, which can be uniquely solved for $u(i)$. 
\item Then A(i) can be recovered from (\ref{eq:bla}).
\item Note: $\bar{A}$   and $\bar{u}$ cannot be identified without knowledge of $\alpha$ and $\beta$.
\end{itemize}
\end{frame}
% -----------------------------------------
\begin{frame}
\frametitle{AA `14: Population and wages -- data}
\begin{figure}[htbp] \centering
\includegraphics[height=.92\textheight]{../images/AllenArkolakis2014_fig12.jpg}
\end{figure}
\end{frame}
% -----------------------------------------
\begin{frame}{AA `14: Productivities and amenities}
\includegraphics[height=.92\textheight]{../images/AllenArkolakis2014_fig13.jpg}
\includegraphics[height=.92\textheight]{../images/AllenArkolakis2014_fig14.jpg}
\end{frame}
% -----------------------------------------
\begin{frame}
\frametitle{AA `14: Price index}
\begin{figure}[htbp] \centering
\includegraphics[height=.88\textheight]{../images/AllenArkolakis2014_fig15.jpg}
\end{figure}
\end{frame}
% -----------------------------------------
\begin{frame}{AA `14: Counterfactual scenario: Remove interstate highways}
\begin{columns}
\begin{column}{0.49\textwidth}
\includegraphics[width=\textwidth]{../images/AllenArkolakis2014_fig17.jpg}
\end{column}
\begin{column}{0.49\textwidth}
\includegraphics[width=0.90\textwidth]{../images/AllenArkolakis2014_fig18.jpg}
\end{column}
\end{columns}
\end{frame}
% -----------------------------------------
\begin{frame}{Bartelme (2018): Market access slows the shift to the Sun Belt}
\begin{itemize}
\item South's share of US population rose from 24\% in 1950 to 30\% in 2000 (\href{https://doi.org/10.1002/j.2325-8012.2008.tb00856.x}{Glaeser and Tobio 2008})
\item Bartelme (2018) ``assess[es] the quantitative contribution of trade costs to US economic geography''
\item Preferences a la Helpman (1998) with Eaton-Kortum production structure and agglomeration elasticity
\item Compact system of equations with two key elasticities:
response of wages to market access and response of population to market access
\item Proposes shift-share instrument for market access based on neighbor's industrial composition
\item Counterfactual: ``reducing trade costs would result in large population shifts from the Northeast towards the South and West, along with a flattening of the city size distribution''
\end{itemize}
\end{frame}
% -----------------------------------------
\begin{frame}{Bartelme (2018): System of equations}
\begin{align*}
w_n &= \Phi_n^{\epsilon_w} \chi_n^w
\\
L_n &= \Phi_n^{\epsilon_l} \chi_n^l
\\
\Phi_n &= \sum_{i} \frac{w_i L_i}{\Phi_i} \tau_{in}^{-\theta}
\end{align*}
How to get exogenous variation in market access $\Phi_n$ over time?
\begin{enumerate}
\item 
Exogenous shifts in trade costs $\tau_{in}$
(Redding and Sturm 2008; Donaldson and Hornbeck 2016)
\item 
Shocks to other regions' fundamentals that shift $w_i L_i$ in other regions
\end{enumerate}
\end{frame}
% -----------------------------------------
\begin{frame}{Bartelme (2018): Shift-share design}
\begin{itemize}
\item Shift-share design requires a multi-industry model that still delivers aggregate gravity:
``assume a common cost function $w_i$ (up to a productivity shifter $T_{nk}$), common trade costs, and equality between the upper and lower tier elasticities, $\sigma - 1 = \theta$''
\item Construct predicted change in market access by summing over predicted partner's growth using national industry growth times local employment share
\item I expect that industrial employment shares are spatially correlated
\item See Adao, Kolesar, Morales (2019) on concerns for shift-share designs
\item I am not aware of an existence-and-uniqueness theorem for multi-sector models with varying parameters
\end{itemize}
\end{frame}
% -----------------------------------------
\begin{frame}{Caliendo, Parro, Rossi-Hansberg, Sarte (2018): Multi-sector model}
\begin{itemize}
\item Two factors: labor and equipped land
\item Labor moves to equalize welfare across regions $U = I_n / P_n$
\item Multi-sector EK/CP production with intermediates
\item Imbalanced trade with regional transfers
\item EK and equipped land are dispersion forces; no agglomeration mechanism; equilibrium is presumably unique
\item Calibrated shares: ``we do not need direct information on transport costs since all the relevant information is embedded in the observed trade flows''
\item Computing counterfactuals: Function iteration on relative regional factor prices $\hat{\omega}$
(Appendix A.3)
\item {``the effects of disaggregated productivity changes depend in complex ways on the details of which sectors and regions are affected, and how these are linked through input–output and trade relationships to other sectors and regions''\par}
\end{itemize}
\end{frame}
% -----------------------------------------
\begin{frame}{What's missing/next?}
Allen and Arkolakis (2025) handbook chapter on quantitative regional economics:
\begin{itemize}
\item Multiple sectors
\item Central place theory
\item Migration costs
\item Inter-regional knowledge spillovers
\item Heterogeneous people
\item Granular firms
\item Market power
\end{itemize}
\end{frame}
% -----------------------------------------
\begin{frame}{Next week}
Quantitative urban models featuring commuting
\end{frame}
% -----------------------------------------
\end{document}
