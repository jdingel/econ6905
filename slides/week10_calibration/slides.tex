\documentclass[11pt,notes=hide,aspectratio=169]{beamer}
%Jonathan Dingel; PhD trade course

% PACKAGES
\usepackage{graphics}  % Support for images/figures
\usepackage{graphicx}  % Includes the \resizebox command
\usepackage{url}	   % Includes \urldef and \url commands
\usepackage{soul}      % Includes the underline \ul command
%\usepackage{framed}	   % Includes the \framed command for box around text
\usepackage{booktabs} %\toprule,\bottomrule
%\usepackage{natbib}
\usepackage{bibentry}  % Includes the \nobibliography command
\usepackage{bbm}       %
%\usepackage{pgfpages}  %Supports "notes on second screen" option for beamer
\usepackage{verbatim}  %Supports comments
\usepackage{tikz}		%Supports graphing/drawing
\usepackage{pgfplots} %Supports graphing/drawing
\usepackage{amsfonts}  % Lots of stuff, including \mathbb 
\usepackage{amsmath}   % Standard math package
\usepackage{amsthm}    % Includes the comment functions
\usepackage{physics}

% CUSTOM DEFINITIONS
\def\newblock{} %Get beamer to cooperate with BibTeX
\linespread{1.2}
\hypersetup{backref,pdfpagemode=FullScreen,colorlinks=true,linkcolor=blue,urlcolor=blue}
\newtheorem{proposition}{Proposition}
\newtheorem{assumption}{Assumption}
\newtheorem{condition}{Condition}

% IDENTIFYING INFORMATION
\title{Topics in Trade}
\author{Jonathan I. Dingel}
\date{Fall \the\year}

% BEAMER TEACHING STUFF
\setbeamertemplate{navigation symbols}{}  %Turn off navigation bar

% THEMATIC OPTIONS
\definecolor{columbiablue}{RGB}{185,217,235}  %Columbia blue defined at https://visualidentity.columbia.edu/branding
\definecolor{columbiadarkblue}{RGB}{0,48,135}  %Columbia dark blue defined at https://visualidentity.columbia.edu/branding
\setbeamercovered{transparent=5}
\setbeamercolor{frametitle}{fg=columbiadarkblue}
\setbeamercolor{item}{fg=columbiadarkblue}
\usefonttheme{serif}
\setbeamercolor{button}{bg = white,fg = columbiadarkblue}
\setbeamercolor{button border}{fg = columbiadarkblue}

\setbeamertemplate{footline}{\begin{center}\textcolor{gray}{Dingel -- Topics in Trade -- \semester -- Week 10 -- \insertframenumber}\end{center}}
\begin{document}
% -----------------------------------------
\begin{frame}[plain]
\begin{center}
\large
\textcolor{columbiadarkblue}{ECON G6905\\
Topics in Trade\\ 
Jonathan Dingel\\
\semester, Week 10}
\vfill 
\includegraphics[width=0.4\textwidth]{../images/Columbia_logo.png}
\end{center}
\end{frame}
% -----------------------------------------
\begin{frame}{This week: Identification, calibration, and exact hat algebra}
An introduction to econometric issues in applied general equilibrium models
using the CES Armington model of trade as our running example:
\begin{itemize}
\item Identification \textcolor{gray}{is a model property and precedes estimation}
\item Comparative statics \textcolor{gray}{are counterfactual scenarios}
\item Calibrating parameters to match observed shares \textcolor{gray}{is estimation}
\item Specification checks \textcolor{gray}{still apply to saturated models}
\item Finite-sample performance \textcolor{gray}{is poor with one parameter per observation}
\end{itemize}
\end{frame}
% -----------------------------------------
\begin{frame}{Identification, in general: A refresher}
\begin{itemize}
\item ``Econometric identification really means just one thing: model parameters or features being uniquely determined from the observable population that generates the data.'' (\href{https://www.aeaweb.org/articles?id=10.1257/jel.20181361}{Lewbel 2019})
\item For a parameter to be identified, alternative values of that parameter must imply different distributions of the observable data.
\item Identification presumes structure: Identification of a structural feature (i.e., a parameter) is only defined in the context of a class defined by a maintained hypothesis (a ``model'').
\item ``The study of identification logically precedes estimation, inference, and testing.'' (\href{https://www.aeaweb.org/articles?id=10.1257/jel.20181361}{Lewbel 2019}) 
\end{itemize}
{\footnotesize \textcolor{gray}{Asides: Name your errors terms (are those supply shocks or demand shocks in the residual?). ``Consideration of underlying structure is needed to convincingly argue that covariates included in the model as controls will actually function as they are intended.''}\par}
\end{frame}
% -----------------------------------------
\begin{frame}{Are CES Armington model parameters identified by trade flows?}
{\small
A statement equivalent to
``one can calibrate the model to observed shares without estimating the full structure of the model''
is
``the model's parameters are not identified by observed trade shares''
\begin{itemize}
\item
If $\{T_i,L_i\}$, $\{\tau_{ij}\}$, $\epsilon$, and $\{w_i\}$ satisfy
$$
w_i L_i
=
\sum_j \frac{T_i \left(w_i \tau_{ij}\right)^{-\epsilon}}{\sum_{\ell} T_{\ell} \left(w_{\ell} \tau_{\ell j}\right)^{-\epsilon}}
w_j L_j
$$
then
$\{\delta_i^{\epsilon}T_i,L_i\}$, $\{\delta_i\tau_{ij}\}$, $\epsilon$, and $\{w_i\}$
also satisfy this system with the same $\{X_{ij}\}$
$\forall \delta_{i} > 0$.
Iceberg trade costs: productivity and export-cost shifter are indistinguishable
\item
If $\{T_i,L_i\}$, $\{\tau_{ij}\}$, $\epsilon$, and $\{w_i\}$ satisfy this system,
then
$\{T_i,L_i\}$, $\{\xi_j \tau_{ij}\}$, $\epsilon$, and $\{w_i\}$ 
also satisfy this system
for any $\xi_j > 0$.
Expenditure shares reflect relative prices.
\item
Similarly,
$\{\mu_{i}^{\epsilon}T_i,\mu_{i}^{-1} L_i\}$, $\{\tau_{ij}\}$, $\epsilon$, and $\{\mu_{i}w_i\}$
also satisfy this system
for any $\mu_{i} > 0$.
\textcolor{gray}{(Isomorphic endowment economy has $Q_i = T_i^{1/\epsilon} L_i$)}
%(\href{https://twitter.com/borusyak/status/1550633634427801605}{Question remains}: Is $\tau_{ij}^{\epsilon}$ then identified given observed $X_{ij}$?)
\end{itemize}
}
\end{frame}
% -----------------------------------------
\begin{frame}{Are  Armington model parameters identified by trade flows?}
Suppose you want to identify \textit{relative} trade costs, $\tau_{ij} / \tau_{jj}$.
Normalize $\tau_{jj} = 1 \ \forall j$.
\medskip
Using two prior results,
if $\{T_i,L_i\}$, $\{\tau_{ij}\}$, $\epsilon$, and $\{w_i\}$ satisfy this system,
then
$\{\delta_i^{\epsilon} T_i,L_i\}$, $\{\delta_i \xi_j \tau_{ij}\}$, $\epsilon$, and $\{w_i\}$
also satisfy this system
for any $\delta_{i} > 0$ and $\xi_j > 0$.
\begin{align*}
Y_i
&=
\sum_j \frac{T_i \left(w_i \tau_{ij}\right)^{-\epsilon}}{\sum_{\ell} T_{\ell} \left(w_{\ell} \tau_{\ell j}\right)^{-\epsilon}}
Y_j
=
\sum_j \frac{X_{ij}}{X_{jj}} 
X_{jj}
=
\sum_j \frac{T_i \left(w_i \tau_{ij}\right)^{-\epsilon}}{T_j \left(w_j \tau_{jj}\right)^{-\epsilon}}
X_{jj}
\\
&=
\sum_j \frac{\delta_i^{\epsilon}T_i \left(w_i \delta_i \xi_j\tau_{ij}\right)^{-\epsilon}}{\delta_j^{\epsilon}T_j \left(w_j \delta_j \xi_j \tau_{jj}\right)^{-\epsilon}}
X_{jj}
\end{align*}
If one chooses $\xi_j = \delta^{-1}_j$, then this preserves $\tau_{jj} = 1$.
\medskip
We cannot recover productivities and \textit{directed} trade costs from directed trade flows.
\end{frame}
% -----------------------------------------
\begin{frame}{We can normalize $T_i = L_i = 1 \ \forall i$ and $\tau_{ii} = 1 \ \forall i$}
Using three prior results:
If $\{T_i,L_i\}$, $\{\tau_{ij}\}$, $\epsilon$, and $\{w_i\}$ satisfy this system,
then
$\{\mu_{i}^{\epsilon}\delta_i^{\epsilon} T_i,\mu_{i}^{-1} L_i\}$, $\{\delta_i \xi_j \tau_{ij}\}$, $\epsilon$, and $\{\mu_{i}w_i\}$
also satisfy this system
for any $\mu_{i} > 0$, $\delta_{i} > 0$, and $\xi_j > 0$.
\medskip
Choose
$\mu_{i} =  L_i$,
$\delta_{i}  = \mu_{i}^{-1} T_i^{-1/\epsilon}$,
and
$\xi_j = \delta^{-1}_j$.
This defines a trade-cost matrix $\tau$ and wage vector $w$ satisfying the system of equations 
given $T_i = L_i = 1 \ \forall i$ and $\epsilon$.
\medskip
By $L_i = 1$, $w_i = Y_i = \sum_{j} X_{ij}$.
That's observed.
We can thus identify relative trade costs from relative expenditures and relative producer prices:
$$
\frac{X_{ij}}{X_{jj}} = \left(\frac{Y_i}{Y_j}\right)^{-\epsilon} \left(\frac{\tau_{ij}}{\tau_{jj}}\right)^{-\epsilon}
\implies
\frac{\tau_{ij}}{\tau_{jj}} = \left(\frac{X_{ij}}{X_{jj}}\right)^{-1/\epsilon}\left(\frac{Y_i}{Y_j}\right)^{-1}
$$
\smallskip
In sum, imposing $\tau_{ii} = T_i = L_i = 1 \ \forall i$ determines a unique solution for $\tau_{ij}$ given $\epsilon$:
$$
\tau_{ij} = \left(X_{ij} / X_{jj}\right)^{-1/\epsilon} Y_j / Y_i
$$
\end{frame}
% -----------------------------------------
\begin{frame}{Trade flows alone are insufficient}
\begin{itemize}
\item 
Even after normalizing $\{T_i,L_i,\tau_{ii}\}_{i=1}^{N}$,
the union of trade costs $\{\tau_{ij}\}$ and the trade elasticity $\epsilon$
are not identified from trade flows $\{X_{ij}\}$ alone.
\item
Recall from week 2 that trade-cost proxies like distance are insufficient to identify $\epsilon$:
one needs trade costs to estimate trade elasticity \textcolor{gray}{(versus distance elasticity)}
\item
Researchers often use import tariffs and freight charges, which provide needed ad valorem trade costs
\textcolor{gray}{assuming that they are appropriately orthogonal to other trade costs in the error term}
\end{itemize}
\end{frame}
% -----------------------------------------
\begin{frame}{Factory-gate prices do not discipline the Armington model much}
\begin{itemize}
\item
Recall that bilateral trade costs $\tau_{ij}$ are isomorphic to bilateral preference shifters $\beta_{ij}$ in the Armington model (week 1).
\item
Conditional on $Y_i = p_i Q_i$, any observed $p_i$ can be rationalized by introducing preference shifters.
\item
Similarly, in Melitz (2003), firm-level productivity $\varphi$ might shift TFPR by lowering costs (TFPQ) or raising demand (quality).
Given this isomorphism, firm-level prices can do little to discipline the model.
\item
If you've committed to the Eaton and Kortum (2002) model with symmetric preferences,
then the price index may be more informative.
But an isomorphic ``quality'' interpretation of $T_i$ seems straightforward.
\end{itemize}
\end{frame}
% -----------------------------------------
\begin{frame}{Counterfactual outcomes by exact hat algebra in trade model}
Recall exact hat algebra for Armington CES model from week 1:
\begin{equation*}
\hat{w}_i \hat{L}_i
=
\hat{Y}_{i}
=
\sum_{j=1}^{N} \gamma_{ij} \hat{\lambda}_{ij} \hat{Y}_{j}
= 
\sum_{j=1}^{N}
\frac{\gamma_{ij} \hat{T}_{i} \hat{w}_{i}^{-\epsilon}\hat{\tau}_{ij}^{-\epsilon}}
{\sum_{l=1}^{N} \lambda_{lj} \hat{T}_{l} \hat{w}_l^{-\epsilon}\hat{\tau}_{lj}^{-\epsilon}}
\hat{w}_{j} \hat{L}_{j}
\end{equation*}
\vspace{-3mm}
\begin{itemize}
\item \href{https://tradediversion.net/2018/05/07/on-hat-algebra/}{The name} refers to Jones (1965) ``hat algebra'' for comparative statics
\item It's ``exact'' because it's global given whole demand and supply system
\end{itemize}
\vspace{3mm}
A system of equations characterizing counterfactual endogenous variables $\hat{w}_i$
in terms of 
initial equilibrium shares $\lambda_{ij} \equiv \frac{X_{ij}}{X_j}$ and $\gamma_{ij} \equiv \frac{X_{ij}}{Y_i}$,
the trade elasticity $\epsilon$,
and
counterfactual exogenous changes $\hat{L}_{i}, \hat{T}_i, \hat{\tau}_{ij}$
\end{frame}
% -----------------------------------------
\begin{frame}{Exact hat algebra: Sufficient statistics for comparative statics}
\vspace{-3mm}
\begin{equation*}
\hat{w}_i \hat{L}_i
= 
\sum_{j=1}^{N}
\frac{\gamma_{ij} \hat{T}_{i} \hat{w}_{i}^{-\epsilon}\hat{\tau}_{ij}^{-\epsilon}}
{\sum_{l=1}^{N} \lambda_{lj} \hat{T}_{l} \hat{w}_l^{-\epsilon}\hat{\tau}_{lj}^{-\epsilon}}
\hat{w}_j \hat{L}_j
\end{equation*}
Exact hat algebra concerns comparative statics, not calibration (\href{https://tradediversion.net/2023/10/02/exact-hat-algebra-concerns-comparative-statics-not-calibration/}{blog post})
\begin{itemize}
\item The system of equations defines counterfactual outcomes for any model of the baseline shares $\lambda_{ij}$ and $\gamma_{ij}$ and the trade elasticity $\epsilon$
\item Model parameters $\{T_i,L_i\}$ and $\{\tau_{ij}\}$ matter (only) because they determine the baseline shares %and the trade elasticity
\end{itemize}
Identifying comparative statics may not require identifying all model parameters
\begin{itemize}
\item $\{T_i,L_i\}$ and $\{\tau_{ij}\}$ matter only because they determine the baseline shares
\item Even if you did distinguish between $T_i$ and $L_i$, it would be of no consequence for counterfactual changes if they delivered the same $\lambda_{ij}$ and $\gamma_{ij}$
\item \textcolor{gray}{Though one might find it strange to know $\hat{T}_{i} \equiv T'_i / T_i$ when you do not know $T_i$}
\end{itemize}
\end{frame}
% -----------------------------------------
\begin{frame}{Counterfactual scenarios and required parameter values}
Some counterfactual scenarios require (knowledge of) more parameters than
just the information encoded in 
initial equilibrium shares $\lambda_{ij}$ and $\gamma_{ij}$ and
the trade elasticity $\epsilon$.
\href{https://doi.org/10.1111/1468-0262.00352}{Eaton and Kortum (2002)} conduct two thought experiments:
\begin{itemize}
\item Autarky: $\tau_{ij} \to \infty$ for $i \neq j$. $\hat{\tau}_{ij} = \infty$.
\item ``Zero gravity'': $\tau_{ij} \to \tau_{jj}$ for $i \neq j$. $\hat{\tau}_{ij} = \frac{\tau_{jj}}{\tau_{ij}}$.
\end{itemize}
Knowing model parameters up to $\lambda_{ij}$, $\gamma_{ij}$, $\epsilon$
is \textit{insufficient} to compute the zero-gravity scenario
because that requires knowing $\hat{\tau}_{ij} = \frac{\tau_{jj}}{\tau_{ij}}$.
\end{frame}
% -----------------------------------------
\begin{frame}{The sufficient statistics are observed shares}
\begin{itemize}
\item
Given trade flows data \textcolor{gray}{(including the diagonal)},
you possess the sufficient statistics
$\lambda_{ij} \equiv \frac{X_{ij}}{X_j}$ and $\gamma_{ij} \equiv \frac{X_{ij}}{Y_i}$
\item
Given trade flows $X_{ij}$, the trade elasticity $\epsilon$, and
the assumption $T_i = L_i = \tau_{ii} = 1 \ \forall i$,
there exists a (unique) solution for $\tau_{ij}$
\item
Therefore,
there exist parameter values consistent with the observed sufficient statistics
for \textit{any} (non-negative) trade matrix
\\ \textcolor{gray}{$\tau_{ii} = 1$ normalization does presume $X_{ii} > 0$ \\ obviously the parameter value is not unique}
\item
Ergo, feel free to use the observed baseline shares $\lambda_{ij}$ and $\gamma_{ij}$
without estimating $T_i$, $L_i$, and $\tau_{ij}$
\end{itemize}
\end{frame}
% -----------------------------------------
\begin{frame}{Prior descriptions of exact hat algebra and calibration}
\href{https://doi.org/10.1016/B978-0-444-54314-1.00004-5}{Costinot and Rodriguez-Clare (\textit{Handbook} 2014)}:\\
{\small
``We have described how to use gravity models to perform welfare and counterfactual analysis. An appealing feature of this approach, which we have referred to as the exact hat algebra, is that the impact of various counterfactual scenarios can be computed without estimating the full structure of the model. All the relevant information about preferences, technology, and trade costs can be inferred directly from the cross-section of bilateral trade flows and estimates of the trade elasticity.''
\par}
\vspace{3mm}
\href{https://doi.org/10.1093/restud/rdu035}{Caliendo and Parro (2015)}:\\
{\small
``we can solve for the general equilibrium of the model without needing to estimate parameters which are difficult to identify in the data, [such] as productivities and iceberg trade costs.''
(\textcolor{red}{c.f.} \href{https://www.aeaweb.org/articles?id=10.1257/jel.20181361}{Lewbel 2019})
\par}
\end{frame}
% -----------------------------------------
\begin{frame}{Using observed shares is calibrating (combinations of) parameters}
Finding (products of) parameters to match observed shares is calibrating them.
\medskip
For purposes of computing counterfactual outcomes, using observed shares typically leverages two properties of the model and the counterfactual scenario:
\begin{itemize}
\item
The model has enough degrees of freedom that there exist parameters that can rationalize any observed pattern of trade flows.
\textcolor{gray}{(``Saturated'' model)}
\item
The baseline shares are sufficient to calibrate the combinations of parameters needed to compute counterfactual outcomes.
\end{itemize}
\medskip
Corollaries to these two features when using observed shares:
\begin{itemize}
\item
No pattern of baseline shares can reject the model specification.\\
\textcolor{gray}{(Models with fewer degrees of freedom admit tests.)}
\item
The set of admissible counterfactual scenarios is restricted.
\textcolor{gray}{(e.g., zero gravity)}
\end{itemize}
\end{frame}
% -----------------------------------------
\begin{frame}{Admissible counterfactuals and model specification}
{\href{https://doi.org/10.1016/S1573-4412(07)06070-9}{Heckman \& Vytlacil (2007)}: ``for many decisions (policy problems), only combinations of explicit economic parameters are required—no single economic parameter need be identified''\par}
\begin{quote}{\small
	Following Marschak's Maxim, we postulate specific economic questions that are interesting to address and ask what combinations of underlying economic parameters or functionals are required to answer them. Answering one question well usually requires fewer assumptions, and places less demands on the data, than answering a wide array of questions -- the original goal of structural econometrics.\par
}\end{quote}
Gene Grossman (\href{https://economics.princeton.edu/news/alumni-conversations-gene-grossman-and-mits-david-atkin}{2022 transcript}):
\begin{quote}{\small
When you force a simple model to fit a complicated world and you don't get standard errors, you don't know what damage you're doing. And then you immediately run to counterfactuals, and what am I to make of those numbers that it generates?\par}\end{quote}
\textcolor{gray}{(Using sufficient statistics does not require you to use an unfalsifiable model)}
\end{frame}
% -----------------------------------------
\begin{frame}{A saturated model may still be falsifiable}
\begin{itemize}
\item We cannot use baseline shares (nor prices) to evaluate the model
\item The CES model of trade flows is a demand system and we know how to test demand systems
\textcolor{gray}{(at least, the one-elasticity straightjacket of CES/IIA)}
\item One approach: Estimate an encompassing specification and test the parameter restrictions imposed by CES
\textcolor{gray}{(e.g., Hausman and McFadden (1984) test IIA logit by estimating a nested-logit specification and testing whether $\lambda = 1$)}
\item Other tests specifically leverage IIA:
Estimate on subsets of countries and see if elasticities are common
\textcolor{gray}{(the other Hausman and McFadden (1984) test)}
\item One can also test the complete passthrough from tariffs to prices (\href{https://direct.mit.edu/rest/article/doi/10.1162/rest_a_01369/117708/Poor-Substitutes-Counterfactual-Methods-in-IO-and}{Head, Mayer 2023})
\item One could test the triangle inequality on trade costs (cf. \href{https://doi.org/10.1016/j.jinteco.2024.104018}{Foellmi, Hepenstrick, Torun 2024})
\end{itemize}
\end{frame}
% -----------------------------------------
\begin{frame}{The CES/IIA model with one elasticity does not fit the data}
{\small
\href{https://www.aeaweb.org/articles?id=10.1257/aer.20150956}{Adao, Costinot, and Donaldson (2017)} study reduced exchange economies:
\begin{itemize}
\item ``the mixed CES demand system allows data to speak to whether this independence of irrelevant alternatives embodied in CES holds empirically''
\item ``the deviations from IIA\dots are a systematic feature of the data\dots related to the similarity of competitors in terms of per capita GDP''
\end{itemize}
\href{https://www.aeaweb.org/articles?id=10.1257/aer.20190781}{Lind, Ramondo (2023)} develop a Ricardian model with richer productivity patterns:
\begin{itemize}
\item ``By relaxing the independence assumptions used in the literature, the model generates import demand systems spanning the entire generalized extreme value (GEV) class (McFadden 1978, 1981).''
\item ``significant sharing of technologies across countries and sectors\dots manifests in considerable heterogeneity in correlation in productivity, which, in turn, changes the answers to standard counterfactuals''
\item ``our [latent factor model] estimates capture quite accurately departures from IIA within and across sectors''
\end{itemize}
}\end{frame}
% -----------------------------------------
\begin{frame}{What's in those baseline shares?}
Gene Grossman (\href{https://economics.princeton.edu/news/alumni-conversations-gene-grossman-and-mits-david-atkin}{2022 transcript}):
\begin{quote}{
So the models often fit the data by introducing things we can't measure like amenities and productivity. In my generation, we used to call that the error term, but now it's part of the model. Then when you do counterfactuals you don't know what the hell they are, but you have to hold them constant. So why would we think these things would remain constant once we start changing the environment?
\par}\end{quote}
\end{frame}
% -----------------------------------------
\begin{frame}{Overfitting}
\begin{itemize}
\item
One overfits a model by using a more flexible parameterization that improves in-sample fit but worsens out-of-sample performance.
\item
Adding more covariates always improves the $R^2$ of OLS, but ``use as many variables as possible'' is not the preferred specification of forecasters 
\item[]
\textcolor{gray}{(Perhaps more relevant: Consider 2SLS with $N$ observations and $N$ instrumental variables. The first-stage $R^2$ of $1.0$ is not good news: $\beta^{\text{OLS}} = \beta^{\text{2SLS}}$. Now imagine using $N$ white-noise draws as IVs.)}
\item
\href{https://hastie.su.domains/ElemStatLearn/}{Hastie, Tibshirani, and Friedman (2009)}:
``Unfortunately training error is not a good estimate of the test error, as seen in Figure 7.1. Training error consistently decreases with model complexity, typically dropping to zero if we increase the model complexity enough. However, a model with zero training error is overfit to the training data and will typically generalize poorly.''
\end{itemize}
\end{frame}
% -----------------------------------------
\begin{frame}{A saturated model is overfit if there's noise}
\href{https://www.aeaweb.org/articles?id=10.1257/jep.28.2.29}{Belloni, Chernozhukov, Hansen (2014)} come close to describing the calibrated-shares procedure:\\
{\small
The key concept underlying the analysis of high-dimensional data is that dimension reduction or ``regularization'' is necessary to draw meaningful conclusions. The need for regularization can easily be seen when one considers an example where there are exactly as many variables (plus a constant) as there are observations. In this case, the ordinary least squares estimator will fit the data perfectly, returning an $R^2$ of one. However, using the estimated model is likely to result in very poor forecasting properties out-of-sample because the model estimated by least squares is overfit: the least-squares fit captures not only the signal about how predictor variables may be used to forecast the outcome, but also fits the noise that is present in the given sample, and is not useful for forming out-of-sample predictions. Producing a useful forecasting model in this simple case requires regularization; that is, the estimates must be constrained so that overfitting is avoided and useful out-of-sample forecasts can be obtained.
\par}
\end{frame}
% -----------------------------------------
\begin{frame}{A more critical perspective on calibrated shares}
\href{https://doi.org/10.1016/bs.hesint.2022.02.005}{Antras and Chor (Handbook, 2022)}:\\
{\small
``
practitioners of this approach often praise how parsimonious it is relative to CGE models, which involve the estimation of thousands of parameters. An often glossed-over fact, however, is that the hat-algebra approach requires the model to fit the data \textit{exactly}, which amounts to calibrating all parameters of the model (or combinations of them) to values that ensure this exact fit
\dots
although quantitative work often requires strong assumptions on functional forms, calibrating thousands of parameters to fit the data exactly can be problematic for the validity or reliability of the counterfactual predictions of those models. The problem is similar to overfitting in regression analysis leading to poor out-of-sample performance. As recently shown in Dingel and Tintelnot (2020), this is a particularly severe problem in spatial environments in which the data the model is fitted to contains a significant number of zeros. Note that even in the WIOD -- a WIOT focusing on relatively rich countries -- the share of zeroes is 13.7\% in the matrix of input-use coefficients and 46.8\% in the matrix of final-use column vectors.%
''\par}
\end{frame}
% -----------------------------------------
\begin{frame}{Dingel and Tintelnot (2023)}
The calibrated-shares procedure can perform poorly in high-dimensional settings because matching (noisy) observed shares amount to overfitting the model parameters.
Go to Dingel and Tintelnot (2023) slidedeck.
First half of paper concerns this overfitting problem.
Second half of paper will be covered later in course.
\end{frame}
% -----------------------------------------
\end{document}
