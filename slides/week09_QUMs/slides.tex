\documentclass[11pt,notes=hide,aspectratio=169]{beamer}
%Jonathan Dingel; PhD trade course

% PACKAGES
\usepackage{graphics}  % Support for images/figures
\usepackage{graphicx}  % Includes the \resizebox command
\usepackage{url}	   % Includes \urldef and \url commands
\usepackage{soul}      % Includes the underline \ul command
%\usepackage{framed}	   % Includes the \framed command for box around text
\usepackage{booktabs} %\toprule,\bottomrule
%\usepackage{natbib}
\usepackage{bibentry}  % Includes the \nobibliography command
\usepackage{bbm}       %
%\usepackage{pgfpages}  %Supports "notes on second screen" option for beamer
\usepackage{verbatim}  %Supports comments
\usepackage{tikz}		%Supports graphing/drawing
\usepackage{pgfplots} %Supports graphing/drawing
\usepackage{amsfonts}  % Lots of stuff, including \mathbb 
\usepackage{amsmath}   % Standard math package
\usepackage{amsthm}    % Includes the comment functions
\usepackage{physics}

% CUSTOM DEFINITIONS
\def\newblock{} %Get beamer to cooperate with BibTeX
\linespread{1.2}
\hypersetup{backref,pdfpagemode=FullScreen,colorlinks=true,linkcolor=blue,urlcolor=blue}
\newtheorem{proposition}{Proposition}
\newtheorem{assumption}{Assumption}
\newtheorem{condition}{Condition}

% IDENTIFYING INFORMATION
\title{Topics in Trade}
\author{Jonathan I. Dingel}
\date{Fall \the\year}

% BEAMER TEACHING STUFF
\setbeamertemplate{navigation symbols}{}  %Turn off navigation bar

% THEMATIC OPTIONS
\definecolor{columbiablue}{RGB}{185,217,235}  %Columbia blue defined at https://visualidentity.columbia.edu/branding
\definecolor{columbiadarkblue}{RGB}{0,48,135}  %Columbia dark blue defined at https://visualidentity.columbia.edu/branding
\setbeamercovered{transparent=5}
\setbeamercolor{frametitle}{fg=columbiadarkblue}
\setbeamercolor{item}{fg=columbiadarkblue}
\usefonttheme{serif}
\setbeamercolor{button}{bg = white,fg = columbiadarkblue}
\setbeamercolor{button border}{fg = columbiadarkblue}

\setbeamertemplate{footline}{\begin{center}\textcolor{gray}{Dingel -- Topics in Trade -- \semester -- Week 9 -- \insertframenumber}\end{center}}
\begin{document}
% -----------------------------------------
\begin{frame}[plain]
\begin{center}
\large
\textcolor{columbiadarkblue}{ECON G6905\\
Topics in Trade\\ 
Jonathan Dingel\\
\semester, Week 9}
\vfill 
\includegraphics[width=0.4\textwidth]{../images/Columbia_logo.png}
\end{center}
\end{frame}
% -----------------------------------------
\begin{frame}{This week: Quantitative urban models}
Subset of QSMs featuring commuting (choose residence-workplace pair)
\begin{itemize}
\item QUM's gravity equation features cross-hauling of homogeneous labor
\item Contrast with the canonical monocentric-city model
\item[] \textcolor{gray}{Link to Davis (1995) on intra-industry trade}
\end{itemize}
\smallskip
While GE machinery developed recently, gravity is a staple of urban economics:
\begin{itemize}{\small
\item Gerald A. P. Carrothers (1956) ``\href{https://doi.org/10.1080/01944365608979229}{An Historical Review of the Gravity and Potential Concepts of Human Interaction}'', \textit{Journal of the American Institute of Planners}, 22:2, 94-102.
\item Waldo Tobler (1975), ``\href{https://pure.iiasa.ac.at/id/eprint/241/1/RR-75-019.pdf}{Spatial Interaction Patterns}''. ``\textcolor{gray}{A table of county-to-county interactions in the United States, for example, would yield nearly $10^7$ numbers, an incomprehensible amount.}''
\item Colwell, P.F. (1982), ``\href{https://doi.org/10.1111/j.1467-9787.1982.tb00775.x}{Central Place Theory and The Simple Economic Foundations of the Gravity Model}'' \textit{Journal of Regional Science}, 22: 541-546.
}\end{itemize}
\end{frame}
% -----------------------------------------
\begin{frame}{Motivation: Why use a GE gravity model of commuting flows?}
\begin{itemize}
\item Commuting frees us to separate choices of home and work locations
\item A monocentric city with one employment area may be too stylized for many applications
\end{itemize}
{\small\href{https://www.nber.org/papers/w33130}{Redding (2024)}:
{``Perhaps the most common empirical application is to transport infrastructure improvements (e.g., railroads, roads, public transit), but they also provide frameworks to evaluate zoning and land use regulations, other place-based policy interventions, and the implications of new communication technologies.''\par}}
\begin{itemize}{\small
\item Freeway disamenities (Brinkman and Lin 2024)
\item Los Angeles subway system (Chris Severen)
\item Bus rapid transit in Bogota (Nick Tsivanidis)
\item Commuting and formality in Mexico City (Roman David Zarate)
\item Steam railway in 19\textsuperscript{th} century London (Heblich Redding Sturm 2020)
\item Revitalizing empty blocks in Detroit (Owens et al 2020)
\item Transportation infrastructure with congestion (Allen Arkolakis 2022)
}\end{itemize}
\end{frame}
%---------------------------------------------------------------------
\begin{frame}{Baseline model: Economic environment}
\textcolor{gray}{This is Dingel and Tintelnot (2025) with ARSW (2015) notation for $\beta$, $z$, $d$, $ij$}
\begin{itemize}
\item
Each location has productivity $A$ and land endowment $T$
\item 
Measure $L$ individuals w/ one unit of labor
and hired by 
competitive firms producing freely traded goods differentiated by location of production
\item
Individuals have Cobb-Douglas preferences over goods ($\beta$) and land ($1-\beta$) 
\item
Commuting costs: $d_{ij} = \underbrace{\bar{d}_{ij}}_{\text{time}} \times \underbrace{\lambda_{ij}}_{\text{disutility}}$
\item
Individuals have idiosyncratic tastes for pairs of residential and workplace locations,
such that
$o$'s utility from living in $i$ and working in $j$ is
\begin{equation*}
U_{ij}^{o} = \epsilon \ln\left(\frac{w_j}{r_i^{1-\beta} P^\beta d_{ij}}\right) + z_{ij}^{o}
\qquad
z_{ij}^{o} \stackrel{\text{iid}}{\sim} \text{T1EV}
\end{equation*}
\item Labor-supply elasticity equals commuting elasticity $\epsilon$
\item $\epsilon \to \infty$ is the Rosen-Roback case
\end{itemize}
\end{frame}
%---------------------------------------------------------------------
\begin{frame}{Baseline model: Equilibrium}
Given economic primitives ($\beta$, $\epsilon$, $\sigma$,$L$,$\{A_j\}$,$\{T_i\}$,$\{d_{ij}\}$),
an equilibrium is a set of wages $\{w_j\}$, rents $\{r_i\}$, and labor allocation $\{\ell_{ij}\}$
such that
\begin{align}
\text{labor allocation (gravity):}&&
\frac{\ell_{ij}}{L}
&=
\frac
{w_{j}^\epsilon \left(r_{i}^{1-\beta} d_{ij}\right)^{-\epsilon}}
{\sum_{i',j'}w_{j'}^\epsilon \left(r_{i'}^{1-\beta}  d_{i'j'}\right)^{-\epsilon}}
\label{eqn:cont_laborallocation}
&\forall i,j
\\
\text{goods markets:}&&
A_j \sum_i \frac{\ell_{ij}}{\bar{d}_{ij}} 
&=
\frac{\left({w_j}/{A_j}\right)^{-\sigma}}{P^{1-\sigma}} Y % \sum_{i',j'} \frac{\ell_{i'j'}}{d_{i'j'}} w_{j'} 
&\forall j
\label{eqn:tradeeqlbm:goodsmarketclearing}
\\
\text{land markets:}&&
T_i
&=
\frac{1-\beta}{r_i} \sum_j \underbrace{\frac{\ell_{ij}}{\bar{d}_{ij}} w_j}_{y_{ij}}
&\forall i,j
\label{eqn:tradeeqlbm:landsmarketclearing} 
\end{align}
\textcolor{gray}{Uniqueness: Sufficiently elastic labor demand or sufficiently small land share
$\left(\frac{1+\epsilon}{\sigma + \epsilon}\right) \left(\frac{(1-\beta) \epsilon}{1+(1-\beta) \epsilon}\right) \leq \frac{1}{2} \implies$ unique equilibrium (Allen Arkolakis Li 2023)}
\end{frame}
% -------------------------------------------------------------------------------
\begin{frame}{Baseline model: Counterfactual outcomes by exact hat algebra}
\vspace{-1cm}
\begin{align}
\hat{w}_{j}
&=
\textcolor{red}{\hat{A}_j}
\left(\sum_{i} \hat{y}_{ij}\textcolor{blue}{\frac{y_{ij}}{\sum_{i'} y_{i'j}}}\right)^{\frac{1}{1-\sigma}}
\left(\sum_{j'} \left(\frac{\hat{w}_{j'}}{\textcolor{red}{\hat{A}_{j'}}}\right)^{1-\sigma} \sum_{i} \textcolor{blue}{\frac{y_{ij'}}{Y}} \right)^{\frac{1}{1-\sigma}}
\hat{Y}^{\frac{1}{\sigma-1}}\\
\hat{r}_{i}
&=
\textcolor{red}{\hat{T}_{i}}^{-1} \sum_{j}\hat{y}_{ij}\textcolor{blue}{\frac{y_{ij}}{\sum_{j'}y_{ij'}}}
\\
\hat{\ell}_{ij} 
&=
\left\{
\begin{aligned}
&1, \; \text{if} \; \ell_{ij}=0 \\
&\frac{\hat{w}_{j}^{\varepsilon}\left(\hat{r}_{i}^{1-\beta} \textcolor{red}{\hat{\bar{d}}_{ij}\hat{ \lambda}_{ij}} \right)^{-\varepsilon}}{\sum_{i',j'} \hat{w}_{j'}^{\varepsilon}\left(\hat{r}_{i'}^{1-\beta} \textcolor{red}{\hat{\bar{d}}_{i'j'}\hat{ \lambda}_{i'j'}} \right)^{-\varepsilon} \textcolor{blue}{\frac{\ell_{i'j'}}{L}}} \qquad \text{ if } \ell_{ij} > 0 \\
\end{aligned}
\right.
\end{align}
Compute $\hat{w}_j$, $\hat{r}_i$, $\hat{\ell}_{ij}$, and 
$\hat{y}_{ij} = \hat{\ell}_{ij} \hat{w}_{j} / \hat{\bar{d}}_{ij}$
given 
elasticities $\beta$, $\sigma$ and $\epsilon$, 
baseline shares \textcolor{blue}{$\frac{\ell_{ij}}{L}$}
and \textcolor{blue}{$\frac{y_{ij}}{Y}$},
and
relative exogenous parameters
\textcolor{red}{$\hat{A}_{j}$}, \textcolor{red}{$\hat{T}_{i}$}, \textcolor{red}{$\hat{\bar{d}}_{ij}$} and \textcolor{red}{$\hat{ \lambda}_{ij}$}.
\end{frame}
%---------------------------------------------------------------------
\begin{frame}{Commuting and the productivity elasticity of employment (1/3)}
What is the productivity elasticity of employment? ($\hat{A}_{j^{*}} > 1$ is only shock) \\
To start, consider an economy of no commuting ($d_{ij} = \infty \text{ and } \ell_{ij} = 0 \text{ if } i \neq j$) and perfectly elastic labor demand ($\sigma = \infty$)
\begin{equation*}
\hat{w}_{j}
=
\hat{A}_j
\qquad
\hat{r}_{i}
=
\hat{y}_{ii}
=
\hat{w}_{ii} \hat{\ell}_{ii}
\qquad
\hat{\ell}_{ii} 
=
\frac{\hat{w}_{i}^{\varepsilon}\hat{r}_{i}^{-\varepsilon(1-\beta)}}{\sum_{i'} \hat{w}_{i'}^{\varepsilon} \hat{r}_{i'}^{-\varepsilon(1-\beta)} \frac{\ell_{i'i'}}{L}}
\end{equation*}
Compare log changes to eliminate the denominator:
\begin{equation*}
\frac{\hat{\ell}_{j^{*}j^{*}}}{\hat{\ell}_{jj}}
=
\frac{\hat{w}_{j^{*}}^{\varepsilon}\left(\hat{r}_{j^{*}}\right)^{-\varepsilon(1-\beta)}}%
{\hat{r}_{j}^{-\varepsilon(1-\beta)}}
=
\frac{\hat{A}_{j^{*}}^{\beta\varepsilon}\left(\hat{\ell}_{j^{*}j^{*}}\right)^{-\varepsilon(1-\beta)}}%
{\hat{\ell}_{jj}^{-\varepsilon(1-\beta)}}
=
\hat{A}_{j^{*}}^{\frac{\beta\varepsilon}{1+\varepsilon(1-\beta)}}
\end{equation*}
With no commuting, \textit{residential} land supply $T_{j^*}$ is a brake on \textit{employment} expansion in $j^*$.
Lower land expenditure share (higher $\beta$) raises employment elasticity.
\textcolor{gray}{($\beta = 1$ makes $T_{j^*}$ irrelevant)}
\end{frame}
% -----------------------------------------
\begin{frame}{Commuting and the productivity elasticity of employment (2/3)}
Now consider an economy with costless commuting ($d_{ij} = 1 \ \forall ij$) and perfectly elastic labor demand ($\sigma = \infty$)
\begin{equation*}
\hat{w}_{j}
=
\hat{A}_j
\qquad
\hat{r}_{i}
=
\sum_{j}\hat{y}_{ij}\frac{y_{ij}}{\sum_{j'}y_{ij'}}
%=
\qquad
\hat{\ell}_{ij} 
=
\frac{\hat{w}_{j}^{\varepsilon}\hat{r}_{i}^{-\varepsilon(1-\beta)}}{\sum_{i',j'} \hat{w}_{j'}^{\varepsilon}\hat{r}_{i'}^{-\varepsilon(1-\beta)} \frac{\ell_{i'j'}}{L}}
\end{equation*}
Because commuting is costless,
the residence $i$ share of workplace $j$ employees
$\frac{\ell_{ij}}{\sum_{i'} \ell_{i'j}} = \frac{r_{i}^{-\varepsilon(1-\beta)}}{\sum_{i'} r_{i'}^{-\varepsilon(1-\beta)}}$
is independent of $j$.
%$\frac{\ell_{ij}}{\sum_{j'} \ell_{ij'}} = \frac{w_j^{\varepsilon}}{\sum_{j'} w_{j'}^{\varepsilon}}$,
%$\frac{y_{ij}}{\sum_{j'} y_{ij'}} = \frac{w_j^{\varepsilon+1}}{\sum_{j'} w_{j'}^{\varepsilon+1}}$
What happens to employment?
$$
\frac{\hat{L}_{j^{*}}}{\hat{L}_{j}}
=
\frac{\sum_{i}\hat{\ell}_{ij^{*}} \frac{\ell_{ij^{*}}}{L_j^{*}}}{\sum_{i}\hat{\ell}_{ij} \frac{\ell_{ij}}{L_j}}
=
\frac{\hat{w}_{j^{*}}^{\varepsilon} \sum_{i} \hat{r}_{i}^{-\varepsilon(1-\beta)} \frac{\ell_{ij^{*}}}{L_j^{*}}}{\hat{w}_{j}^{\varepsilon} \sum_{i} \hat{r}_{i}^{-\varepsilon(1-\beta)} \frac{\ell_{ij}}{L_j}}
=
\frac{\hat{A}_{j^{*}}^{\varepsilon}}{\hat{A}_{j}^{\varepsilon}}
=
\hat{A}_{j^{*}}^{\varepsilon}
$$
With costless commuting, the productivity elasticity of employment is independent of the land expenditure share $\beta$.
Choices of workplace and residences are wholly separated (everyone is a teleworker).
\end{frame}
% -----------------------------------------
\begin{frame}{Commuting and the productivity elasticity of employment (3/3)}
\vspace{2mm}
\begin{columns}
\begin{column}{0.62\textwidth}
{\href{https://www.aeaweb.org/articles?id=10.1257/aer.20151507}{Monte, Redding, Rossi-Hansberg (2018)} study how the local employment elasticity varies with commuting ties between US counties\par}
\begin{itemize}{\small
\item Commuting complicates studies of ``local labor markets''
\textcolor{gray}{(Also, counties vary greatly in size)}
\item In a model of counties with commuting, productivity elasticity of county employment varies
\item Share of residents who work where they live predicts this variation in elasticities
\item Million-dollar plants elicit larger employment increases in more open (lower $\frac{\ell_{ii}}{\sum_{j} \ell_{ij}}$) counties
\item {Reduced commuting costs could substitute for relaxing housing-supply restrictions\par}
}\end{itemize}
\end{column}
\begin{column}{0.36\textwidth}
\includegraphics[width=1.0\textwidth]{../images/MRR2018_fig1.pdf}\\
\includegraphics[width=1.0\textwidth]{../images/MRR2018_fig2.pdf}
\end{column}
\end{columns}
\end{frame}
% -----------------------------------------
\begin{frame}{Ahlfeldt, Redding, Sturm, Wolf (Ecma, 2015)}
ARSW is often referred to as ``the Berlin Wall paper'':
\begin{itemize}
\item Develop a quantitative model of the city to identify intra-city agglomeration and dispersion forces
\item Estimate using 1936, 1986 and 2006 data for thousands of city blocks on land prices, workplace employment, and employment by residence
\item Use the division of Berlin after WWII and its reunification in 1989 as sources of exogenous variation in the surrounding concentration of economic activity
\end{itemize}
Won the \href{https://www.econometricsociety.org/society/awards}{2018 Frisch Medal} for best applied paper in \textit{Econometrica}:
{\small ``The paper provides an outstanding example of how to credibly and transparently use a quasi-experimental approach to structurally estimate model parameters that can serve as critical inputs for counterfactual policy analyses.''
(\textcolor{red}{cf.} \href{https://www.dropbox.com/s/8kwtwn30dyac18s/intro.pdf?dl=0}{Haile})\par}
\end{frame}
% -----------------------------------------
\begin{frame}{Dividing Berlin}
\begin{itemize}
{\small
\item A protocol signed during WWII organized Germany into American, British, French, and Soviet occupation zones
\item Although 200km inside the Soviet zone, Berlin was to be jointly occupied and organized into four sectors (initially three roughly equal-sized, then British sector was split between French and British)
\item Protocol envisioned a joint city administration (``Kommandatura''), but Cold War:
	\begin{itemize}
	\item East and West Germany founded as separate states and separate city governments created in East and West Berlin in 1949
	\item The adoption of Soviet-style policies of command and control in East Berlin limited economic interactions with West Berlin
	\item To stop civilians leaving for West Germany, the East German authorities constructed the Berlin Wall in 1961
	\end{itemize}
\item Germany reunified in 1989
}
\end{itemize}
\end{frame}
%%%%%%%%%%%%%%%%%%%%%%%%%%%%%%%%%%%%%%%%%
\begin{frame}{ARSW model: Overview}
\begin{itemize}
\item The city consists of a set of discrete blocks indexed by $i$ 
\item Single, freely traded (numeraire) final good \textcolor{gray}{$(\sigma = \infty)$}
\item Floor space can be used for residential or commercial use
\item Firms choose a block of production and inputs of labor and floor space 
\item Workers choose block of residence, block of employment, and consumption of the final good 
\item Reservation level of utility $(\bar{U})$ for living outside the city
\item[] {\small Individuals who choose Berlin and realize utility below the city-wide average cannot leave.\par}
\end{itemize}
\end{frame}
%%%%%%%%%%%%%%%%%%%%%%%%%%%%%%%%%%%%%%%
\begin{frame}{ARSW model: Preferences }
\begin{itemize}
\item Utility for worker $o$ residing in block $i$ and working in block $j$:
\begin{equation*}
C_{ijo} = \frac{B_{i} z_{ijo}}{d_{ij}}
\left( \frac{c_{ij}}{\beta} \right)^{\beta } \left( \frac{\ell
_{ij }}{1-\beta} \right)^{1-\beta },\qquad 0<\beta <1
\end{equation*}
\vspace{-5mm}
\begin{columns}
\begin{column}{0.50\textwidth}
\begin{itemize}
\item Consumption of numeraire final good $c_{ij}$
\item Residential floor space $\ell_{ij}$ at price $Q_{i}$ \textcolor{gray}{(sorry)}
\end{itemize}
\end{column}
\begin{column}{0.40\textwidth}
\begin{itemize}
\item Residential amenity $B_{i}$
\item Commuting costs $d_{ij}$
\item Idiosyncratic shock $z_{ijo}$ 
\end{itemize}
\end{column}
\end{columns}
\item Indirect utility given wage $w_j$ at workplace $j$
\begin{equation*}
U_{ijo} = z_{ijo} B_{i} w_{j} Q_{i}^{\beta-1} / d_{ij},
\end{equation*}
\item Idiosyncratic part of worker productivity is Fr\'{e}chet distributed:
\begin{equation*}
F(z_{ijo})=e^{-T_{i} E_{j} z_{ijo}^{-\epsilon }},\qquad T_{i}, E_{j} >0,\ \epsilon >1
\end{equation*}
\item[] \textcolor{gray}{If $x$ has a Fr\'{e}chet distribution, then $\ln(x)$ has a Gumbel distribution}
\end{itemize}
\end{frame}
%%%%%%%%%%%%%%%%%%%%%%%%%%%%%%%%%%%%
\begin{frame}{ARSW model: Location decisions within Berlin}
\begin{itemize}{\footnotesize
\item Probability worker chooses to live in $i$ and work in $j$ is
\begin{equation*} \label{eq:com_prob}
\pi _{ij}= \frac{T_{i} E_{j} \left( d_{ij}
Q_{i}^{1-\beta} \right)^{-\epsilon} \left( B_{i} w_{j} \right)^{\epsilon}}{%
\sum_{r=1}^{S} \sum_{s=1}^{S} T_{r} E_{s} \left( d_{rs} Q_{r}^{1-\beta}
\right)^{-\epsilon} \left( B_{r} w_{s} \right)^{\epsilon}} \equiv \frac{\Phi_{ij}}{\Phi} .
\end{equation*}
\item People choose Berlin (or not) before idiosyncratic shocks  $z_{ijo}$ are realized.
Indifference equates expected utility in Berlin to reservation utility:
\begin{equation*} \label{eq:pm}
\mathbb{E} \left[\max_{ij} U_{ij}^{o} \right]  = \gamma \left[ \sum_{r=1}^{S} \sum_{s=1}^{S} T_{r} E_{s} \left( d_{rs}
Q_{r}^{1-\beta} \right)^{-\epsilon} \left( B_{r} w_{s} \right)^{\epsilon} %
\right]^{1/\epsilon} = \bar{U},  
\end{equation*}
\item Residential and workplace choice probabilities
\begin{equation*} \label{eq:ch_prob}
\pi_{Ri} = \sum_{j=1}^{S} \pi_{ij} = \frac{1}{\Phi} \sum_{j=1}^{S} \Phi_{ij}, \qquad  \pi_{Mj} = \sum_{i=1}^{S} \pi_{ij} = \frac{1}{\Phi}\sum_{i=1}^{S} \Phi_{ij}.
\end{equation*}
\item Conditional on living in block $i$, the probability that a worker commutes to block $j$ %follows a gravity equation:
\begin{equation*}
\pi _{ij|i}=\frac{ E_{j} \left( w_{j} / d_{ij} \right)^{\epsilon}}{
\sum_{s=1}^{S} E_{s} \left( w_{s} / d_{is} \right)^{\epsilon}},
\end{equation*}
}\end{itemize}
\end{frame}
%%%%%%%%%%%%%%%%%%%%%%%%%%%%%%%%%%%%%%%%%%%%%%%%%%%%%%%%%%%%%%%%%%%%%%%
\begin{frame}{ARSW model: Commuting }
\begin{itemize}
\item Employment in block $j$ equals the sum across all blocks $i$ of people living in residence times the probability of commuting from $i$ to $j$:
\begin{equation*} \label{eq:wbo}
H_{Mj}=\sum_{i=1}^{S} \frac{E_{j}\left(
w_{j}/d_{ij}\right) ^{\epsilon }}{\sum_{s=1}^{S}E_{s}\left( w_{s}/d_{is}\right)
^{\epsilon }} H_{Ri}, \qquad d_{ij} = e^{\kappa \tau_{ij}}.
\end{equation*} \smallskip
\item This is the labor-supply curve for workplace $j$
\end{itemize}
\end{frame}
%%%%%%%%%%%%%%%%%%%%%%%%%%%%%%%%%%%%%%%%%%%%%%%%%%%%%%%%
\begin{frame}{ARSW model: Production}
\begin{itemize}
\item A single final good (numeraire) is produced by competitive firms with constant returns to scale (and no trade costs):
\[
y_{j} = A_{j} \left( H_{Mj} \right)^{\alpha} \left( L_{Mj} \right)^{1-\alpha},\qquad 0<\alpha <1
\]
\item $H_{Mj}$ is workplace employment 
\item $L_{Mj}$ is measure of floor space used commercially
\item Isocost curve in commercial floor space $q_j$ and labor $w_j$ with shifter $A_j$:
\begin{equation*} \label{eq:zp}
q_{j}=(1-\alpha )\left( \frac{\alpha }{w_{j}}\right) ^{\frac{\alpha }{%
1-\alpha }}A_{j}^{\frac{1}{1-\alpha }}
\end{equation*}
\end{itemize}
\end{frame}
%%%%%%%%%%%%%%%%%%%%%%%%%%%%%%%%%%%%%%%%%%%%%%
\begin{frame}{ARSW model: Floor space prices}
\begin{itemize}
\item The share of floor space used commercially:
\begin{equation*} \label{eq_na}
\theta_i = 1 \quad \text{if} \qquad q_i> \xi_i Q_i,
\end{equation*}
\[
\theta_i \in [0,1] \quad  \text{if} \qquad q_i = \xi_i Q_i,
\]
\[
\theta_i = 0 \quad  \text{if} \qquad q_i < \xi_i Q_i.
\]
\item $\xi_i \geq 1$ represents the tax equivalent of regulations restricting commercial land use
\item Assume observed land price is maximum of commercial and residential price: $\mathbb{Q}_{i} = \max\{q_i,Q_i\}$
\end{itemize}
\end{frame}
%%%%%%%%%%%%%%%%%%%%%%%%%%%%%%%%%%%%%%%%%%%%%%%%%%%%%%%%%%%%%
\begin{frame}{ARSW model: Floor-space- and land-market clearing}
\begin{itemize}
\item Floor space $L$ is supplied competitively using Cobb-Douglas combination of local land $K_i$ at price $\mathbb{R}_{i}$ and freely traded capital input $M$
\begin{equation*}
L_{i} = M_{i}^{\mu} K_{i}^{1-\mu}
\implies
\mathbb{Q}_{i} = \chi \mathbb{R}_{i}^{1-\mu}
\end{equation*}
\item Cobb-Douglas demand for residential floor space:
\begin{equation*} \label{eq:res_mc}
(1-\theta_{i}) L_{i} = \frac{ (1-\beta) \mathbb{E}(w \mid i) }{Q_i} H_{Ri}
\end{equation*}
\item Cobb-Douglas demand for commercial floor space:
\begin{equation*} \label{eq:com_mc}
\theta_{i} L_{i} = \left( \frac{(1-\alpha) A_{i}}{q_{i}} \right)^{\frac{1}{\alpha}} H_{Mi}
\end{equation*}
\item Density of development $(M_{i}^{\mu})$ from land-market clearing:
\[
M_{i}^{\mu} = \frac{L_{i}}{K_{i}^{1-\mu}} = \frac{(1-\theta_{i}) L_{i} + \theta_{i} L_{i}}{K_{i}^{1-\mu}}
\]
\end{itemize}
\end{frame}
%%%%%%%%%%%%%%%%%%%%%%%%%%%%%%%%%%
\begin{frame}{Equilibrium with exogenous locational characteristics}
\begin{itemize}
\item \textbf{Proposition 1:} Given the model's parameters $[\alpha,\beta,\mu,\epsilon,\kappa]$, the reservation utility $\bar{U}$, and vectors of exogenous location characteristics $[T,E,A,B,\phi,K,\xi,\tau]$, there exists a unique general equilibrium vector $[\pi_{M},\pi_{R},H,Q,q,w,\theta]$, where $H$ denotes total city population. 
\item These seven components are determined by the system of seven equations: 
commercial land market clearing, %\eqref{eq:com_mc},
residential land market clearing, %\eqref{eq:res_mc},
zero profits, %\eqref{eq:zp},
no arbitrage between alternative uses of land, %\eqref{eq_na},
residential choice probability $\pi_{Ri}$, % in \eqref{eq:ch_prob}),
workplace choice probability $\pi_{Mi}$, % in \eqref{eq:ch_prob})),
and indifference with reservation utility. %\eqref{eq:pm}. 
\item Notice that $B_i$ and $A_j$ are exogenous in this proposition
\end{itemize}
\end{frame}
%%%%%%%%%%%%%%%%%%%%%%%%%%%%%%%%%%%%%%%%%%
\begin{frame}{Amenity and productivity spillovers}
\begin{itemize}
\item Productivity ($A_j$) depends on fundamentals ($a_j$) and spillovers ($\Upsilon_{j}$):
\[
A_{j} = a_{j} \Upsilon_{j}^{\lambda}, \qquad \Upsilon_{j} \equiv \left[ \sum\limits_{s=1}^{S} e^{-\delta \tau_{js}} \left( \frac{H_{Ms}}{K_s} \right) \right],
\]
\item $\delta$ is the rate of decay of spillovers
\item $\lambda$ captures the relative importance of spillovers
\textcolor{gray}{(Pardon the clash with $\lambda_{ij}$)}
\item Residential amenities are influenced by both fundamentals ($b_{i}$) and spillovers ($\Omega_{i}$)
\[
 B_{i} = b_{i} \Omega_{i}^{\eta}, \qquad \Omega_{i} \equiv \left[ \sum\limits_{s=1}^{S} e^{-\rho \tau_{is}} \left( \frac{H_{Rs}}{K_s} \right) \right].
\]
\item These local externalities are the heart of the paper
\item[] {\footnotesize \textcolor{gray}{Subsequent research allows explicit trips to enjoy amenities (Hausman, Samuels, Cohen, Sasson  2025) or consume non-traded services (Miyauchi, Nakajima, Redding 2025)}\par}
\end{itemize}
\end{frame}
%%%%%%%%%%%%%%%%%%%%%%%%%%%%%%%%%%%%%%%%%%%%%%%%%%%%%%%%%%
\begin{frame}{Overview of remainder of paper}
\begin{itemize}
\item Relate land prices to distance to pre-war CBD
\item Estimate the model without any agglomeration effects. In counterfactuals, this model ``is unable to account quantitatively for the observed impact of division and reunification on the pattern of economic activity within West Berlin.''
\item Estimate model with local production and residential amenity externalities. These are interesting in their own right and improve model fit (they also create possibility of multiple equilibria, which must be handled carefully)
\end{itemize}
\end{frame}
%%%%%%%%%%%%%%%%%%%%%%%%%%%%%%%%%%%%%%
\begin{frame}{Data}
\begin{itemize}
\item Data on land prices, workplace employment, residence employment and bilateral travel times
\item Data for Greater Berlin in 1936 and 2006 and data for West Berlin in 1986 by
	\begin{itemize}
	\item Pre-war districts (``Bezirke''), 20 in Greater Berlin, 12 in West Berlin
	\item Statistical areas (``Gebiete''), around 90 in West Berlin
	\item Statistical blocks, around 9,000 in West Berlin
	\end{itemize}
\item Land prices: official assessed land value of a representative undeveloped property or the fair market value of a developed property if not developed
\item Geographical Information Systems (GIS) data on\\
{\small land area, land use, building density, proximity to U-Bahn (underground) and S-Bahn (suburban) stations, schools, parks, lakes, canals and rivers, Second World War destruction, location of government buildings and urban regeneration programs\par}
\end{itemize}
\end{frame}
%%%%%%%%%%%%%%%%%%%%%%%%%%%%%%%%%%%%%%%%%%%%%%%%%%%%%%%
\begin{frame}{Land prices in Berlin in 1936}
Land prices are normalized to have a mean of 1 in each year.
\begin{figure}
\centering
\includegraphics[height=.92\textheight]{../images/ARSW_Map3d1.pdf}
\end{figure}
\end{frame}
% -----------------------------------------
\begin{frame}{Land prices in West Berlin in 1936}
\begin{figure}
\centering
\includegraphics[height=.95\textheight]{../images/ARSW_Map3d2.pdf}
\end{figure}
\end{frame}
%%%%%%%%%%%%%%%%%%%%%%%%%%%%%%%%%%%
\begin{frame}{Land prices in West Berlin in 1986}
\begin{figure}
\centering
\includegraphics[height=.95\textheight]{../images/ARSW_Map3d3.pdf}
\end{figure}
\end{frame}
% -----------------------------------------
\begin{frame}{Land prices in Berlin in 2006}
\begin{figure}
\centering
\includegraphics[height=.95\textheight]{../images/ARSW_Map3d4.pdf}
\end{figure}
\end{frame}
% -----------------------------------------
\begin{frame}{Land prices in West Berlin in 2006}
\begin{figure}
\centering
\includegraphics[height=.95\textheight]{../images/ARSW_Map3d5.pdf}
\end{figure}
\end{frame}
% -----------------------------------------
\begin{frame}{Diff-in-diffs specification}
\begin{itemize}
\item Estimate difference-in-differences specification for division and reunification separately (for areas in West Berlin):
\begin{equation*}
\Delta \ln Q_{i} = \psi + \sum_{k=1}^{K} \mathbf{1}_{ik} \beta_{k} + \ln X_{i} \zeta + \chi_{i},\label{emprent3}
\end{equation*}
\item $\mathbf{1}_{ik}$ is a $(0,1)$ dummy which equals one if block $i$ lies within distance grid cell $k$ from the pre-war CBD and zero otherwise
\item Observable block characteristics ($X_{i}$): Land area, land use, distance to nearest U-Bahn station, S-Bahn station, school, lake, river or canal, and park, war destruction, government buildings and urban regeneration programs
\end{itemize}
\end{frame}
%%%%%%%%%%%%%%%%%%%%%%%%%%%%%%%%%%%%%%%%%%%%
\begin{frame}{Division and Pre-War CBD}
\begin{figure}
\centering
\includegraphics[height=.95\textheight]{../images/ARSW_centraldivision.pdf}
\end{figure}
\end{frame}
%%%%%%%%%%%%%%%%%%%%%%%%%%%%%%%%%%%%%%%%%%%%%%%%%%
\begin{frame}{Diff-in-diffs West Berlin 1936-86}
\begin{figure}
\centering
\includegraphics[width=1\textwidth]{../images/ARSW_Table1.pdf}
\end{figure}
\end{frame}
\begin{frame}{Diff-in-diffs West Berlin 1986-2006}
\begin{figure}
\centering
\includegraphics[width=1\textwidth]{../images/ARSW_Table2.pdf}
\end{figure}
\end{frame}
%%%%%%%%%%%%%%%%%%%%%%%%%%%%%%%%%%%%%%%%%%%%%%%%
\begin{frame}{Land prices over time}
1928--1936 is placebo test.
Most of 1936--1986 change occurs by 1966.
\begin{center}
\includegraphics[height=.87\textheight]{../images/ARSW_Figure3.pdf}
\end{center}
\end{frame}
%%%%%%%%%%%%%%%%%%%%%%%%%%%%%%%%%%%%%%%%%%%%%%%%%%
\begin{frame}{Gravity regression for commuting between districts}
\vspace{-2mm}
Log commuters from residence $I$ to workplace $J$ (Jensen's inequality):
\begin{equation*}
\ln \pi_{IJ} = - \nu \tau_{IJ}  + \vartheta_{I} + \varsigma_{J} + e_{IJ},
\end{equation*}
where $\tau_{IJ}$ is transit minutes, $\nu = \epsilon \kappa$, and 
$\vartheta_{I}$ and  $\varsigma_{J}$ are fixed effects
\begin{center}
\includegraphics[height=.7\textheight]{../images/ARSW_Table3.pdf}
\end{center}
\end{frame}
%%%%%%%%%%%%%%%%%%%%%%%%%%%%%%%%%%%%%%%%%%%%%%%%%
\begin{frame}{Backing out productivities}
\begin{itemize}
\item Composites:
$\tilde{A}_{j} \equiv A_j E_j^{\alpha/\epsilon}$,
$\tilde{a}_{j} \equiv a_j E_j^{\alpha/\epsilon}$
$\omega_{j} \equiv \tilde{w}_{j}^{\epsilon} = E_j w_{j}^{\epsilon}$
\item Using estimated $\nu$, and data on residence and workplace employment, one can solve for transformed wages $E_j w_{j}^{\epsilon}$ from commuting equation (summed over origins)
\item Recover overall productivity $A_{j}$ from zero-profit equation:
\begin{equation*}
\ln\left(\frac{\tilde{A}_{jt}}{\bar{\tilde{A}}_{t}} \right) = (1-\alpha) \ln \left(\frac{\mathbb{Q}_{jt}}{\bar{\mathbb{Q}}_t} \right) + \frac{\alpha}{\epsilon} \ln \left(\frac{\omega_{jt}}{\bar{\omega}_t} \right)
\end{equation*}
where $\bar{\tilde{A}}_t = \exp(1/S \sum_{s=1}^{S} \ln \tilde{A}_{st})$ (geometric mean)
\item[] \textcolor{gray}{Zero-profit condition of Rosen-Roback model}
\item High floor prices and wages require high final good productivity for zero profits to be satisfied 
\end{itemize}
\end{frame}
%%%%%%%%%%%%%%%%%%%%%%%%%%%%%%%%%%%%%%%%%%%%%%%%%
%%%%%%%%%%%%%%%%%%%%%%%%%%%%%%%%%%%%%%%%%%%%%%%%%
\begin{frame}{Backing out amenities}
\begin{itemize}
\item Composites:
$\tilde{B}_{i} \equiv b_i T_i^{1/\epsilon}\zeta_{Ri}^{1-\beta}$,
$\tilde{b}_{i} \equiv b_i T_i^{1/\epsilon}\zeta_{Ri}^{1-\beta}$,
where
$\zeta_{Ri} = 1$ or $\zeta_{Ri} = \xi_{i}$ based on land use
\item Recover amenities $B_{i}$ from residential choice probabilities:
\begin{equation*}
\ln\left(\frac{\tilde{B}_{it}}{\bar{\tilde{B}}_{t}} \right) = \frac{1}{\epsilon} \ln \left(\frac{H_{Rit}}{\bar{H}_{Rt}} \right) +  (1-\beta) \ln \left(\frac{\mathbb{Q}_{it}}{\bar{\mathbb{Q}}_t} \right)
- \frac{1}{\epsilon} \ln \left(\frac{W_{it}}{\bar{W}_t} \right)
\end{equation*}
where $W_{it} = \sum_{s=1}^{S} \omega_{st} / e^{\nu \tau_{ist}}$  and variables with an upper bar denote that variable's geometric mean.
\item[] \textcolor{gray}{If $\epsilon = \infty$, this is Rosen-Roback indifference condition}
\item High floor prices and high residence employment must be explained either by high wage commuting access or high amenities
\item (So far not making assumptions about the relative importance of production and residential externalities versus fundamentals) 
\end{itemize}
\end{frame}
%%%%%%%%%%%%%%%%%%%%%%%%%%%%%%%%%%%%%%%%%%%%%%%%%%%%%%%
%%%%%%%%%%%%%%%%%%%%%%%%%%%%%%%%%%%%%%%%%%%%%%%%%
\begin{frame}{Backing out amenities and productivities }
\begin{itemize}
\item Set $\alpha, \beta, \mu$ ``to central estimates from the existing empirical literature''
\item Use estimate of $\nu = \epsilon \kappa = 0.07$ from gravity regression
\item ``To calibrate the value of the Fréchet shape parameter ($\epsilon$), we use our data on the dispersion of log wages by workplace across the districts of West Berlin for 1986.''
\item \textcolor{red}{``}This value of $\epsilon = 6.83$ for commuting decisions is broadly in line with the range of estimates for the Fréchet shape parameter for international trade flows.\textcolor{red}{''}
\item Table IV columns 1-4 show changes in amenities and productivities over time.
\item Table IV columns 5-6 show the impact of the division and reunification on West Berlin land-price gradient, holding productivity and amenities constant at their 1936 values. Poor fit ($-.408 \neq -.800$; $-.010 \neq +.398$)
\end{itemize}
\end{frame}
% -----------------------------------------
\begin{frame}{Changes in fundamentals and counterfactuals with exogenous $A$, $B$}
\begin{figure}
\centering
\includegraphics[height=.95\textheight]{../images/ARSW_Table4.pdf}
\end{figure}
\end{frame}
% -----------------------------------------
\begin{frame}{ARSW (2015) Section 7: Estimation of structural model}
\begin{center}{\small
\begin{tabular}{llll} \toprule
Assumed Parameter & & Source & Value \\ \midrule
Residential land & $ 1-\beta $ & Davis \& Ortalo-Magne (2011) & 0.25 \\
Commercial land & $1-\alpha$ & Valentinyi-Herrendorf (2008) & 0.20 \\
Fr\'{e}chet Scale & $T$ & (normalization) & 1 \\
Expected Utility & $\bar{u}$ & (normalization) & 1000 \\ \bottomrule
\end{tabular}
\begin{tabular}{ll} \toprule
Estimated Parameter \\ \midrule
Production externalities elasticity & $\lambda$ \\
Production externalities decay & $\delta$ \\
Residential externalities elasticity & $\eta$ \\
Residential externalities decay & $\rho$ \\
Commuting semi-elasticity & $\nu=\epsilon \kappa$ \\
Commuting heterogeneity & $\epsilon$ \\ \bottomrule
\end{tabular}
}\end{center}
\end{frame}
\begin{frame}{GMM estimation procedure}
\begin{itemize}
\item Use exogenous variation from Berlin's division and reunification \textcolor{red}{``}to structurally estimate\textcolor{red}{''} the agglomeration parameters $\{\lambda,\delta,\eta,\rho\}$.
\begin{align*}
\Delta \ln\left(\frac{a_{it}}{\bar{a}_{t}} \right) 
&=
(1-\alpha) \Delta \ln \left(\frac{\mathbb{Q}_{it}}{\bar{\mathbb{Q}}_t} \right) + \frac{\alpha}{\epsilon} \Delta \ln \left(\frac{\omega_{it}}{\bar{\omega}_t} \right) - \lambda \Delta \ln \left( \frac{\Upsilon_{it}}{\bar{\Upsilon}_{t}} \right)
\\
\Delta \ln\left(\frac{b_{it}}{\bar{b}_{t}} \right) 
&=
\frac{1}{\epsilon} \Delta \ln \left(\frac{H_{Rit}}{\bar{H}_{Rt}} \right) +  (1-\beta) \Delta \ln \left(\frac{\mathbb{Q}_{it}}{\bar{\mathbb{Q}}_t} \right) \\
&\quad
+ \frac{1}{\epsilon} \Delta \ln \left(\frac{W_{it}}{\bar{W}_t} \right)   - \eta \Delta \ln \left( \frac{\Omega_{it}}{\bar{\Omega}_{t}} \right)
\end{align*}
\item Production externalities $\Upsilon_{it}$ depend on travel-time weighted sum of observed workplace employment densities
\item Residential externalities $\Omega_{it}$ depend on travel-time weighted sum of observed residence employment densities
\item Adjusted fundamentals relative to geometric mean are ``structural residuals''
\end{itemize}
\end{frame}
%%%%%%%%%%%%%%%%%%%%%%%%%%%%%%%%%%%%%%%%%%%%
\begin{frame}{Moment conditions}
\begin{itemize}
\item Changes in adjusted fundamentals uncorrelated with exogenous change in surrounding economic activity from division/reunification
\begin{equation*}
\mathbb{E} \left[ \mathbb{I}_{k} \times \Delta \ln \left( a_{it} / \bar{a}_{t} \right) \right] = 0, \qquad k \in \{1, \dots, K_{\mathbb{I}}\},
\end{equation*}
\begin{equation*}
\mathbb{E} \left[ \mathbb{I}_{k} \times \Delta \ln \left( b_{it} / \bar{b}_{t} \right) \right] = 0, \qquad k \in \{1, \dots, K_{\mathbb{I}}\}.
\end{equation*}
where $ \mathbb{I}_{k}$ are indicators for distance grid cells from pre-war CBD
\item Other moments are fraction of workers that commute less than 30 minutes and wage dispersion
\begin{equation*}
\mathbb{E} \left[ \psi H_{Mj} - \sum_{i \in \aleph_{j}}^{S} \frac{ \omega_{j} / e^{\nu \tau_{ij}} }
{\sum_{s=1}^{S} \omega_{s} / e^{\nu \tau_{is}} } H_{Ri} \right] = 0,
\end{equation*}
\begin{equation*}
\mathbb{E} \left[ \left( 1/\epsilon \right)^{2} \ln \left( \omega_{j} \right)^{2} - \sigma_{\ln w_{i}}^{2} \right] = 0,
\end{equation*}
\end{itemize}
\end{frame}
\begin{frame}{Why estimate $\nu = \epsilon \kappa$ twice?}
Section 6 uses bilateral flows in 2008 between 12 districts:
\begin{itemize}
\item Table III: $\nu = \epsilon \kappa = 0.07$
\item ``minimize the squared difference between the variances across districts of log adjusted wages in the model and log wages in the data'' $\implies \epsilon = 6.83$
\item ``While the model uses measures of bilateral travel times that we construct based on the transport network, the micro survey data include self-reported travel times for each commuter\dots assume that this measurement error is uncorrelated with self-reported travel times.''
\end{itemize}
Section 7 uses fraction of Berlin commutes under 30 minutes during division:
\begin{itemize}
\item $\nu$: ``none of the other parameters $\{\epsilon,\lambda,\delta,\eta,\rho\}$ affect the commuting moment condition'' $\nu = \epsilon \kappa = 0.10$ 
\item $\epsilon$: ``none of the other parameters $\{\lambda,\delta,\eta,\rho\}$ affect the wage moment condition'' $\epsilon = 6.69$
\end{itemize}
\end{frame}
% -----------------------------------------
\begin{frame}{Severen (2023) on workplace amenities}
\begin{equation*}
\mathbb{E} \left[ \left( 1/\epsilon \right)^{2} \ln \left( \omega_{j} \right)^{2} - \sigma_{\ln w_{i}}^{2} \right] = 0
\qquad
\omega_{j} = E_j w_{j}^{\epsilon}
\end{equation*}
\begin{columns}
\begin{column}{0.49\textwidth}
\includegraphics[width=\textwidth]{../images/Severen2023_page1083.png}
\end{column}
\begin{column}{0.49\textwidth}
\includegraphics[width=\textwidth]{../images/Severen2023_fn25.png}\\
{\footnotesize ``Supplementary results in appendix D indicate that wages explain only a relatively small amount of the variation in workplace fixed effects $\omega$.''\par}
\end{column}
\end{columns}
\end{frame}
% -----------------------------------------
\begin{frame}{ARSW: Estimated parameters}
\begin{figure}
\centering
\includegraphics[height=.93\textheight]{../images/ARSW_Table5.pdf}
\end{figure}
\end{frame}
\begin{frame}{Very localized externalities}
\begin{center}
\includegraphics[height=.90\textheight]{../images/ARSW_Table6.pdf}
\end{center}
\end{frame}
\begin{frame}{Counterfactuals}
\includegraphics[width=1.0\textwidth]{../images/ARSW_Table7.pdf}
\end{frame}
% -----------------------------------------
\begin{frame}{Two primary contributions of ARSW}
\begin{itemize}
\item Estimates of very localized externalities (``The economics of density'')
\item[] e.g., Stuart S. Rosenthal and William C. Strange ``\href{https://www.aeaweb.org/articles?id=10.1257/jep.34.3.27}{How Close Is Close? The Spatial Reach of Agglomeration Economies}'' \textit{JEP} 2020
\item Canonical model of commuting within a city
\item[] i.e., the baseline model described in Dingel and Tintelnot (2025) at start of this class
\end{itemize}
\end{frame}
% -----------------------------------------
\begin{frame}{How these models work: Normative properties}
Davis, Gregory (2021) - ``\href{https://www.nber.org/papers/w29045}{Place-Based Redistribution in Simple Location-Choice Models}''
\begin{itemize}
\item
\href{https://academic.oup.com/qje/article-abstract/135/2/959/5697213}{Fajgelbaum and Gaubert (2020)} and subsequent papers
prescribe fiscal transfers (trade deficits) to high-MU-of-tradables places
(cf. \href{https://www.brookings.edu/wp-content/uploads/2008/03/2008a_bpea_glaeser.pdf}{Glaeser and Gottlieb 2008})
\item
Idiosyncratic errors terms are uninsured and affect location choices, the level of utility, and marginal utility
\item 
Frechet vs Weibull:
marginal utilities can differ even when allocations and elasticities for marginal households coincide,
thus making the planning problems differ even when positive predictions coincide
\item
I conjecture that if you use GEV errors the model is not identified
\item
I am skeptical that their proposed computational approach to the planning problem will scale to empirical applications
\end{itemize}
\end{frame}
% -----------------------------------------
\begin{frame}{Next week}
Next week: Identification, calibration, and exact hat algebra
\end{frame}
% -----------------------------------------
\end{document}
