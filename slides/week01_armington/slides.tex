\documentclass[10pt,notes=hide]{beamer}
%Jonathan Dingel; PhD trade course

% PACKAGES
\usepackage{graphics}  % Support for images/figures
\usepackage{graphicx}  % Includes the \resizebox command
\usepackage{url}	   % Includes \urldef and \url commands
\usepackage{soul}      % Includes the underline \ul command
%\usepackage{framed}	   % Includes the \framed command for box around text
\usepackage{booktabs} %\toprule,\bottomrule
%\usepackage{natbib}
\usepackage{bibentry}  % Includes the \nobibliography command
\usepackage{bbm}       %
%\usepackage{pgfpages}  %Supports "notes on second screen" option for beamer
\usepackage{verbatim}  %Supports comments
\usepackage{tikz}		%Supports graphing/drawing
\usepackage{pgfplots} %Supports graphing/drawing
\usepackage{amsfonts}  % Lots of stuff, including \mathbb 
\usepackage{amsmath}   % Standard math package
\usepackage{amsthm}    % Includes the comment functions
\usepackage{physics}

% CUSTOM DEFINITIONS
\def\newblock{} %Get beamer to cooperate with BibTeX
\linespread{1.2}
\hypersetup{backref,pdfpagemode=FullScreen,colorlinks=true,linkcolor=blue,urlcolor=blue}
\newtheorem{proposition}{Proposition}
\newtheorem{assumption}{Assumption}
\newtheorem{condition}{Condition}

% IDENTIFYING INFORMATION
\title{Topics in Trade}
\author{Jonathan I. Dingel}
\date{Fall \the\year}

% BEAMER TEACHING STUFF
\setbeamertemplate{navigation symbols}{}  %Turn off navigation bar

% THEMATIC OPTIONS
\definecolor{columbiablue}{RGB}{185,217,235}  %Columbia blue defined at https://visualidentity.columbia.edu/branding
\definecolor{columbiadarkblue}{RGB}{0,48,135}  %Columbia dark blue defined at https://visualidentity.columbia.edu/branding
\setbeamercovered{transparent=5}
\setbeamercolor{frametitle}{fg=columbiadarkblue}
\setbeamercolor{item}{fg=columbiadarkblue}
\usefonttheme{serif}
\setbeamercolor{button}{bg = white,fg = columbiadarkblue}
\setbeamercolor{button border}{fg = columbiadarkblue}

\setbeamertemplate{footline}{\begin{center}\textcolor{gray}{Dingel -- Topics in Trade -- \semester -- Week 1 -- \insertframenumber}\end{center}}
\begin{document}
% -----------------------------------------
\begin{frame}[plain]
\begin{center}
\large
\textcolor{columbiadarkblue}{ECON G6905\\
Topics in Trade\\ 
Jonathan Dingel\\
\semester, Week 1}
\vfill 
\includegraphics[width=0.5\textwidth]{../images/Columbia_logo.png}
\end{center}
\end{frame}
% -----------------------------------------
\begin{frame}{Outline of today}
\begin{itemize}
	\item Introduction + logistics
	\item Overview of the course
	\item Brief introduction to trade theory
	\item The CES Armington model of international trade
\end{itemize}
\end{frame}
% -----------------------------------------
\begin{frame}{Logistics}
This class
\begin{itemize}
\item Wednesdays, 8:10-10:00, IAB 1101
\item Jonathan Dingel
\begin{itemize}
	\item Email: \href{mailto:jid2106@columbia.edu}{jid2106@columbia.edu}
	\item Office: IAB 1126B
	\item Office hours: By appointment, please email
\end{itemize}
\item Course materials:
\href{http://github.com/jdingel/econ6905}{github.com/jdingel/econ6905}
and \href{https://courseworks2.columbia.edu}{courseworks2.columbia.edu}
\end{itemize}
Broader context:
\begin{itemize}
\item This class is the economic-geography bridge between\\ Weinstein's trade class and Davis's urban class
\item I will emphasize computational aspects
\item You should attend the Trade and Spatial Colloquium (Wednesdays, 12--1, IAB 1101)
\end{itemize}
\end{frame}
% -----------------------------------------
\begin{frame}{Assessment}
My goal is to introduce some concepts and tools in international trade and economic geography so you can tackle relevant research questions
\begin{itemize}
\item Grades based on assignments (70\%) and a final exam (30\%)
\item Three types of assignments
\begin{itemize}
\item Economics: Derive a theoretical result or survey an empirical literature.
\item Programming: Write a function that solves for equilibrium or estimates a parameter.
\item Referee report: Assess a recent working paper.
\end{itemize}
\item Final exam at end of semester
\end{itemize}
Grab assignments from GitHub.
Submit your work via Courseworks.
\end{frame}
% -----------------------------------------
\begin{frame}{Coding}
Submit transparent, self-contained code:
\begin{itemize}
\item Your code must reproduce your work in the ``just press play'' sense of the \href{https://www.aeaweb.org/journals/data/faq\#run}{AEA Data Editor}
\item You may use Julia or Matlab. \href{https://tradediversion.net/2018/09/17/why-i-encourage-econ-phd-students-to-learn-julia/}{Use Julia}.
\end{itemize}
See my \href{http://www.jdingel.com/teaching/advice.html}{recommended resources} webpage for suggestions.
\begin{itemize}
\item Grant McDermott - \href{https://github.com/uo-ec607/lectures}{Data science for economists}
\item Ivan Rudik - \href{https://github.com/AEM7130/class-repo}{AEM 7130 Dynamic Optimization}
\item Paul Schrimpf and Jesse Perla - \href{https://github.com/ubcecon/ECON622/}{Computational Economics with Data Science Applications}
\item Jesus Fernandez-Villaverde - \href{https://www.sas.upenn.edu/~jesusfv/teaching.html}{Computational Methods for Economists}
\item Perla, Sargent, Stachurski - \href{https://julia.quantecon.org/intro.html}{Quantitative Economics}
\end{itemize}
How many have used: Matlab? Julia? Git? \href{https://tradediversion.net/2019/11/06/why-your-research-project-needs-build-automation/}{Build automation}?
\end{frame}
% -----------------------------------------
\begin{frame}{Objectives}
\begin{itemize}
\item My goal is to prepare students to tackle research questions in trade, spatial, and urban economics
\item Writing papers is about matching skills with opportunities
\item In my experience, spotting opportunities is a hard-to-teach combination of insight and luck
\item This class will aim to equip you with skills so your technical quiver is full when you spot a target
\end{itemize}
\end{frame}
% -----------------------------------------
\begin{frame}{Topics}
\begin{enumerate}
\item The CES Armington model
\item Gains from trade and comparative advantage
\item Quantitative Ricardian trade models
\item Gravity regressions
\item Multiple factors of production
\item Increasing returns and home-market effects
\item Agglomeration economies
\item Quantitative spatial models
\item Quantitative urban models
\item Exact hat algebra and calibration
\item Spatial sorting of skills and sectors
\item Discrete choice estimation and simulations
\item Spatial environmental economics
\end{enumerate}
See my comments on ``\href{https://tradediversion.net/2017/09/17/linkages-between-international-trade-and-urban-economics/}{Linkages between international trade and urban economics}''
\end{frame}
% -----------------------------------------
\begin{frame}{Why trade and spatial are interesting}
\begin{itemize}
	\item International trade has long intellectual history (Smith, Ricardo) and is hot policy topic today (Brexit, Trump)
	\item Healthy balance of theory and empirics (cf. theory-dominated from 1817 to 1990s) in which each informs the other
	\item Trade has tools and insights relevant for topics ranging from intracity commuting to national TFP growth
	\item I used to say trade economists sometimes have a data advantage because governments track cross-border transactions
	\item Spatial economics is a small but rapidly growing field (e.g., \href{https://tradediversion.net/2019/11/11/the-rapid-rise-of-spatial-economics-among-jmcs/}{The rapid rise of spatial economics among JMCs}, \href{https://urbaneconomics.org/about/history.html}{UEA history})
\end{itemize}
\vspace{4mm}
Why are you interested in trade/spatial/urban?
\pause \\ \vspace{2mm}
This week, we start with international trade
\end{frame}
% -----------------------------------------
\begin{frame}{Trade's interplay between theory and empirics}
Descriptive facts motivate theoretical work
\begin{itemize}
	\item Observed intra-industry trade motivated ``new trade theory'' (e.g., Krugman 1980)
	\item Observed firm-level heterogeneity motivated ``new new trade theory'' (e.g., Melitz 2003)
\end{itemize}
Empirical evidence comes from wide range of methods
\begin{itemize}
	\item Descriptive statistics
	\item Estimated/calibrated quantitative models
	\item Applications employing sufficient statistics
	\item Quasi-natural experiments (rare, but see Japanese autarky, Suez Canal, the telegraph, etc)
\end{itemize}
Testing is tricky: See \href{https://www.nber.org/papers/w8675}{Harrigan (2001)} and \href{https://www.nber.org/papers/w31321}{Adao et al (2023)}
\begin{itemize}
\item Is it a ``test''? Is there a clearly specified alternative hypothesis?
\item How does the test isolate the distinctive GE prediction?
\item Today, many have ``abandon[ed] testing altogether''
\end{itemize}
\end{frame}
% -----------------------------------------
\begin{frame}{International trade theory}
\begin{itemize}
	\item A dominant view is that international trade is an applied branch of general-equilibrium theory
	\item Any GE model has preferences + technology + equilibrium
	\item International trade theory focuses on locations, such that preferences (rarely) and technology (typically) are location-specific
	\item Trade theory traditionally has ``international'' goods markets and ``domestic'' factor markets
	\item Consumers have preferences over goods; factors are employed to produce goods
	\item Questions: How does international integration affect the goods market, the factor market, and welfare?
	\item One flavor of spatial economics is trade in goods plus mobile factors.
\end{itemize}
\end{frame}
% -----------------------------------------
\begin{frame}{Variants of trade models}
One view: ``positive trade theory uses a variety of models, each one suited to a limited but still important range of questions''
{\small (Jones and Neary 1980)}
\bigskip
\resizebox{\textwidth}{!}{
\begin{tabular}{l lll}
\toprule
& Demand & Supply & Market structure \\
\midrule
Goods & General; & Constant returns  & Perfect competition; \\
markets& CES preferences; & to scale; & Monopolistic competition; \\
& Translog, NHCES, etc & Increasing returns &  Oligopoly \\
\midrule
Factor & Demand derived  & Often perfectly  & Almost always \\
markets & from supply of goods & inelastic & competitive \\
\bottomrule
\end{tabular}
}
\bigskip
If you stop at the goods market, it's partial-equilibrium.
\end{frame}
% -----------------------------------------
\begin{frame}{Neoclassical trade models}
\begin{itemize}
\item ``Neoclassical trade models" are characterized by three key
assumptions:
\begin{itemize}
	\item perfect competition
	\item constant returns to scale
	\item no distortions
\end{itemize}
\item Can accommodate decreasing returns to scale (DRS) using ``hidden'' factors in fixed supply;
IRS is ``new trade theory''
\item Given the generality of these assumptions, there is not a wealth of results, but one can obtain two canonical insights: 
\begin{itemize}
	\item gains from trade (Samuelson 1939)
	\item law of comparative advantage (Deardorff 1980)
\end{itemize}
\item By contrast, we are going to dive deeply into one very specific neoclassical model
\end{itemize}
\end{frame}
% -----------------------------------------
\begin{frame}{The CES Armington model}
Features:
\begin{itemize}
\item Concise: A one-elasticity model
\item Relevant: Same macro-level predictions as other, important gravity-based models
\end{itemize}
Shortcomings:
\begin{itemize}
\item Supply side (endowment economy) is wholly uninteresting 
\item Preferences (national differentiation with IIA) are ad hoc
\end{itemize}
We will discuss
\begin{itemize}
\item Primitives
\item Existence and uniqueness of equilibrium
\item Solving for equilibrium
\item Computing counterfactual outcomes
\end{itemize}
\end{frame}
% -----------------------------------------
\begin{frame}{Armington model with CES prefences}
\begin{itemize}
	\item Each country has its own ``signature'' good (others have zero productivity in this good; maximal absolute advantage)
	\item Consumers in each country have identical CES preferences over the $N$ goods with elasticity $\sigma$
	(see \href{http://www.columbia.edu/~jid2106/td/dixitstiglitzbasics.pdf}{Dingel 2009} for CES refresher)
	\item Bilateral trade costs of the \href{https://tradediversion.net/2019/10/28/whats-an-iceberg-commuting-cost/}{iceberg} form $\tau_{ij}$
	\item Demand: Consumer in $j$ with total expenditure $X_j$ spends $X_{ij}$ on good from $i$
	\begin{equation*}
	X_{ij}	= \frac{(p_i\tau_{ij} )^{1-\sigma}}{\sum_{\ell} (p_\ell\tau_{\ell j})^{1-\sigma}} X_j
			= \frac{(p_i\tau_{ij} )^{1-\sigma}}{P_j^{1-\sigma}} X_j
	\end{equation*}
	\item Economy $i$ endowed with $Q_i$ units so GDP is $Y_i = p_i Q_i$
	\begin{equation*}
	X_{ij}	= \frac{Y_i^{1-\sigma}}{Q_i^{1-\sigma}} \frac{X_j}{P_j^{1-\sigma}}\tau_{ij}^{1-\sigma}
	\end{equation*}
	\item Balanced-trade equilibrium is $\{Y_i\}_{i=1}^{N}$ such that 
	\begin{equation*}
	X_i = Y_i = \sum_j X_{ij}
	\end{equation*}
\end{itemize}
\end{frame}
% -----------------------------------------
\begin{frame}{Equilibrium system of equations}
Combine the last two equations to get $N$ equations in $N$ unknowns:
\begin{align*}
Y_i 
&=
\sum_j X_{ij}
\\
&=
\sum_j \frac{Y_i^{1-\sigma}}{Q_i^{1-\sigma}} \frac{Y_j}{P_j^{1-\sigma}}\tau_{ij}^{1-\sigma}
\\
&=
\sum_j \frac{Y_i^{1-\sigma}}{Q_i^{1-\sigma}} \frac{Y_j}{\sum_{\ell} \left(\tau_{\ell j}Y_{\ell}/Q_{\ell}\right)^{1-\sigma}}\tau_{ij}^{1-\sigma}
\end{align*}
The $N$ unknowns can be $\{Y_i\}_{i=1}^{N}$ or $\{p_i\}_{i=1}^{N}$:
$$
p_i Q_i
=
\sum_j p_i^{1-\sigma} \frac{p_j Q_j}{\sum_{\ell} \left(p_{\ell} \tau_{\ell j}\right)^{1-\sigma}}\tau_{ij}^{1-\sigma}
$$
Denote ``trade elasticity'' by $\epsilon \equiv \sigma - 1$ and expenditure shares by $\lambda_{ij}$
$$
p_i Q_i
=
\sum_j \underbrace{\frac{p_i^{-\epsilon} \tau_{ij}^{-\epsilon}}{\sum_{\ell} \left(p_{\ell} \tau_{\ell j}\right)^{-\epsilon}}}_{\equiv \lambda_{ij}} p_j Q_j
$$
\end{frame}
% -----------------------------------------
\begin{frame}{Existence and uniqueness}
\begin{itemize}
\item We want an equilibrium to exist:
a model without an equilibrium leaves us little to analyze
\item Should we want the equilibrium to be unique?
\begin{itemize}
\item Certainly relevant for computing outcomes
\item May be relevant to identification (\href{https://www.aeaweb.org/articles?id=10.1257/jel.20181361}{Lewbel 2019}),
but point identification concerns uniqueness of parameters given observable outcomes,
not uniqueness of outcomes
\item May be relevant for counterfactual scenarios,
but we can report sets of counterfactual equilibria
(multiplicity seems more a threat to forecasting than counterfactual scenarios)
\end{itemize}
\end{itemize}
\href{https://doi.org/10.1086/704385}{Allen, Arkolakis, Takahashi (2020)} show
\begin{itemize}
\item $\sigma \neq 0$: an interior equilibrium exists
\item $\sigma \geq 0$: all equilibria are interior
\item $\sigma \geq 1$: interior eqlbm is unique (aggregate demand slopes down)
\end{itemize}
\vspace{2mm}
{\small Recent related lit:
\href{https://dx.doi.org/10.2139/ssrn.4699361}{Ouazad (2024)} and
\href{https://economics.mit.edu/people/phd-students/tishara-garg}{Garg (2025)}
on enumerating all equilibria via polynomial roots;
maybe check out \href{https://www.juliahomotopycontinuation.org/}{HomotopyContinuation.jl}\par}
\end{frame}
% -----------------------------------------
\begin{frame}{Solving for equilibrium numerically (1/2)}
You want to find a fixed point $\{Y_i\}_{i=1}^{N}$ that satisfies
$$
Y_i
=
\sum_j \frac{\left(Y_i/Q_i\right)^{-\epsilon} \tau_{ij}^{-\epsilon}}{\sum_{\ell} \left(Y_{\ell} \tau_{\ell j} / Q_{\ell}\right)^{-\epsilon}} Y_j
.$$
Choose a numeraire to pin this down.
You might define a differentiable objective function and find its minimum (the fixed point where it is zero).
This can be slow.
$$
\min_{\{Y_{i}\}_{i=1}^{N}}
\left(Y_i
-
\sum_j \frac{\left(Y_i/Q_i\right)^{-\epsilon} \tau_{ij}^{-\epsilon}}{\sum_{\ell} \left(Y_{\ell} \tau_{\ell j} / Q_{\ell}\right)^{-\epsilon}} Y_j
\right)^2
$$
An iterative approach can be quite fast. \href{https://raw.githack.com/AEM7130/class-repo/master/lecture-notes/04-optimization/04-optimization.html\#40}{Function iteration} means
\begin{itemize}
\item Guess $\{Y_{i}^{s}\}_{i=1}^{N}$ starting with $s=0$.
\item Compute implied LHS when using $\{Y_{i}^{s}\}_{i=1}^{N}$ in RHS
\item Update $\mathbf{Y}^{s+1}$ based on convex combination of $\mathbf{Y}^{s}$ and implied $\mathbf{Y}$
\item Iterate until $\mathbf{Y}^{s+1} = \mathbf{Y}^{s}$
\end{itemize}
\end{frame}
% -----------------------------------------
\begin{frame}{Solving for equilibrium numerically (2/2)}
As in Alvarez and Lucas (2007), define the excess demand function
$$
f_i(\mathbf{p}) 
=
\frac{1}{p_i} \sum_{j} \lambda_{ij} p_j Q_j - Q_i
=
\frac{1}{p_i} \sum_{j}
\frac{\left(p_i\tau_{ij}\right)^{-\epsilon}}{\sum_{\ell} \left(p_{\ell} \tau_{\ell j}\right)^{-\epsilon}}
p_j Q_j - Q_i
$$
Compute equilibrium by defining mapping with damper $\kappa \in (0,1]$:
$$M_i(\mathbf{p}) = p_i \left[1 + \kappa f_i(\mathbf{p}) / Q_{i}\right]$$
If we start with prices such that $\sum_{i=1}^{N} p_i Q_i = 1$, then
\begin{align*}
\sum_{i} M_i(\mathbf{p}) Q_i
&=
\sum_{i} p_i Q_i + \sum_{i} p_i \kappa f_i(\mathbf{p})
=
1 + \kappa \sum_{i} p_i \left[\frac{1}{p_i} \sum_{j} \lambda_{ij} p_j Q_j - Q_i\right]
\\
&=
1 + \kappa \sum_{i} \sum_{j} \lambda_{ij} p_j Q_j - \kappa \sum_{i} p_i  Q_i
=
1
\end{align*}
This maps the set $\left\{\mathbf{p} \in \mathbb{R}^{N}_{+}: \sum_{i} p_i Q_i = 1\right\}$ to itself.
Iteration converges to $M_{i}(\mathbf{p}) = p_i$ (see Alvarez and Lucas 2007).
\end{frame}
% -----------------------------------------
\begin{frame}{Introducing a production function}
Switch from an endowment economy to a simple production function
\begin{itemize}
\item 
One factor of production in fixed supply: $L_i$
\item
Constant returns to scale: $Q_i = A_i L_i$
\item
Perfect competition: $p_{i} = w_i / A_i$ and $Y_i = w_i L_i$
\item
(Choose units to define $T_i \equiv A_i^{\epsilon}$)
\end{itemize}
Our equilibrium system of equations is now
$$
w_i L_i
=
\sum_j \frac{T_i \left(w_i \tau_{ij}\right)^{-\epsilon}}{\sum_{\ell} T_{\ell} \left(w_{\ell} \tau_{\ell j}\right)^{-\epsilon}} w_j L_j
$$
\end{frame}
% -----------------------------------------
\begin{frame}{Introducing asymmetric preferences}
Consider an Armington model with asymmetric preferences:
\begin{align*}
	U_j &= \left(\sum_{i} \beta_{ij} q_{ij}^{(\sigma-1)/\sigma}\right)^{\sigma/(\sigma-1)}
	\\
	\Rightarrow
	\frac{X_{ij}}{X_j} &= \beta_{ij} \left(\frac{p_{ij}}{P_j}\right)^{1-\sigma} 
	=
	\frac{w_i^{1-\sigma}}{P_j^{1-\sigma}}\beta_{ij}\tau_{ij}^{1-\sigma}
\end{align*}
Bilateral trade costs and bilateral preferences are observationally equivalent.
\textcolor{gray}{(The CES price index $P_j$ on this slide differs from previous $P_j$.)}
\end{frame}
% -----------------------------------------
\begin{frame}{Welfare}
\begin{itemize}
\item There is only one factor of production and it is inelastically supplied
\item If we know the CES price index, we can study the real wage $w_i / P_i$, real income $w_i L_i / P_i$, and so forth for each country
\item Real wage in country $j$ with symmetric preferences:
$$
\frac{w_j}{P_{j}}
=
\frac{w_j}{\left(\sum_{i=1}^{N} \left(p_{i}\tau_{ij}\right)^{1-\sigma}\right)^{\frac{1}{1-\sigma}}}
=
\frac{w_j}{\left(\sum_{i=1}^{N} \left(w_i \tau_{ij} / A_i\right)^{1-\sigma}\right)^{\frac{1}{1-\sigma}}}
$$
\end{itemize}
\end{frame}
% -----------------------------------------
\begin{frame}{Counterfactual outcomes}
Counterfactual scenarios:
\begin{itemize}
\item If our model has parameters $\{T_i,L_i,\tau_{ij},\epsilon\}$,
a counterfactual scenario is an alternative parameter vector
$\{T'_i,L'_i,\tau'_{ij},\epsilon'\}$.
\item The model's baseline equilibrium outcomes are $\{w_{i}\}$
and the counterfactual outcomes by primes are $\{w'_{i}\}$
\item[] (Be careful with $\epsilon \to \epsilon'$ exercises)
\end{itemize}
We can address many counterfactuals even in this simple model. Examples:
\begin{itemize}
\item How large are the gains from trade relative to autarky?
\item How much would countries gain from frictionless trade?
\item Which countries gain from Chinese productivity growth?
\item When is productivity growth immiserizing?
\end{itemize}
\end{frame}
% -----------------------------------------
\begin{frame}{Counterfactual outcomes by exact hat algebra}
One way of stating counterfactual outcomes is ``exact hat algebra''
(\href{https://doi.org/10.1016/B978-0-444-54314-1.00004-5}{Costinot and Rodriguez-Clare 2014})
\begin{itemize}
\item A counterfactual equilibrium can be expressed in terms of
counterfactual endogenous outcomes relative to baseline endogenous outcomes,
counterfactual exogenous parameters relative to baseline exogenous parameters,
elasticities,
and
baseline equilibrium shares.
\item \href{https://tradediversion.net/2018/05/07/on-hat-algebra/}{The name} refers to the ``hat algebra'' of Jones (1965):
obtaining comparative statics by totally differentiating a model in logarithms
\item It's ``exact'' because it's global (not only small changes) thanks to knowing the whole demand and supply system
\item We will discuss the use (and misuse) of this technique (and its name) more later in the course
\end{itemize}
\end{frame}
% -----------------------------------------
\begin{frame}{Counterfactual Armington outcomes by EHA (1/2)}
Start from the market-clearing condition and the gravity equation:
\begin{equation*}
w_i L_i
=
\sum_{j=1}^{N} \lambda_{ij} w_j L_j 
\quad \quad
\lambda_{ij} 
=
\frac{T_{i} \left(\tau_{ij}w_i\right)^{-\epsilon}}{\sum_{l=1}^{N} T_{l}  \left(\tau_{lj}w_l\right)^{-\epsilon}}
\end{equation*}
We consider a shock to $\hat{T}_{i} \equiv \frac{T'_{i}}{T_{i}}$.
By assumption, $\hat{\tau}=1$ and $\hat{L}=1$.
We want to solve for the endogenous variables $\hat{\lambda}_{ij}, \hat{X}_{ij}$ and $\hat{w}_{i}$.
In the following derivation, 
define ``sales shares'' by
$\gamma_{ij}\equiv\frac{X_{ij}}{Y_{i}}$.
\begin{align}
w_i L_i
&=
\sum_{j=1}^{N} \lambda_{ij} w_j L_j,
\quad
w'_i L'_i
=
\sum_{j=1}^{N} \lambda'_{ij} w'_j L_j
=
\sum_{j=1}^{N} X'_{ij}
\nonumber \\
\hat{w}_i \hat{L}_i
&= 
\sum_{j=1}^{N} \frac{X'_{ij}}{w_i L_i}
=
\sum_{j=1}^{N} \frac{X_{ij}}{w_i L_i} \hat{X}_{ij}
\equiv
\sum_{j=1}^{N} \gamma_{ij} \hat{X}_{ij}  \label{eqn:20180428:1sector:hatincome}
\end{align}
\end{frame}
% -----------------------------------------
\begin{frame}{Counterfactual Armington outcomes by EHA (2/2)}
\begin{align}
\lambda_{ij} 
&=
\frac{T_{i} \left(\tau_{ij}w_i\right)^{-\epsilon}}{\sum_{l=1}^{N} T_{l}  \left(\tau_{lj}w_l\right)^{-\epsilon}},
\quad
\lambda'_{ij} 
=
\frac{T'_{i} \left(\tau_{ij}w'_i\right)^{-\epsilon}}{\sum_{l=1}^{N} T'_{l}  \left(\tau_{lj}w'_l\right)^{-\epsilon}}
\nonumber \\
\hat{\lambda}_{ij}
\equiv 
\frac{\lambda'_{ij}}{\lambda_{ij}}
&=
\hat{T}_{i} \hat{w}_{i}^{-\epsilon} \hat{\tau}_{ij}^{-\epsilon}
\frac{\sum_{l=1}^{N} T_{l}  \left(\tau_{lj}w_l \right)^{-\epsilon}}
	 {\sum_{l=1}^{N} T'_{l} \left(\tau_{lj}w'_l\right)^{-\epsilon}}
=
\frac{\hat{T}_{i} \hat{w}_{i}^{-\epsilon}\hat{\tau}_{ij}^{-\epsilon}}{\sum_{l=1}^{N} \lambda_{lj} \hat{T}_{l} \hat{w}_l^{-\epsilon} \hat{\tau}_{lj}^{-\epsilon}}
\label{eqn:20180428:1sector:hatgravity}
\end{align}
Combining equations \eqref{eqn:20180428:1sector:hatincome} and \eqref{eqn:20180428:1sector:hatgravity}
under the assumptions that $\hat{Y}_{i}=\hat{X}_{i}$ and $\hat{\tau}=\hat{L}=1$, we
obtain a system of equation characterizing an equilibrium $\hat{w}_i$ as a
function of shocks $\hat{T}_i$, initial equilibrium shares $\lambda_{ij}$ and
$\gamma_{ij}$, and the trade elasticity $\epsilon$:
\begin{equation}
\hat{w}_i \hat{L}_i
= 
\sum_{j=1}^{N} \gamma_{ij} \hat{X}_{ij}
=
\sum_{j=1}^{N} \gamma_{ij} \hat{\lambda}_{ij} \hat{w}_{j}
\nonumber 
\Rightarrow
\hat{w}_i
= 
\sum_{j=1}^{N}
\frac{\gamma_{ij} \hat{T}_{i} \hat{w}_{i}^{-\epsilon} \hat{w}_{j} }
{\sum_{l=1}^{N} \lambda_{lj} \hat{T}_{l} \hat{w}_l^{-\epsilon}}
\label{eqn:20180428:1sector:hatequilibriumequations}
\end{equation}
Given a model parameterization that defines $\epsilon, \lambda_{ij}$, and $\gamma_{ij}$,
we can choose arbitrary productivity shocks $\{\hat{T}_{i}\}_{i=1}^{N}$ and solve for $\{\hat{w}\}_{i=1}^{N}$.
\textcolor{gray}{(This generalizes to arbitrary $\hat{\tau},\hat{L}$.)}
\end{frame}
% -----------------------------------------
\begin{frame}{Counterfactual outcomes: Autarky and free trade}
Autarky
\begin{itemize}
\item The autarky counterfactual scenario is the alternative parameter vector
in which $\tau_{ij} = \infty \ i \neq j$
($\{T_i,L_i,\{\tau_{ij}^{-1}\}=I_N,\epsilon\}$)
\item Can compute by exact hat algebra: $\hat{\tau}_{ij} = \infty \text{ for } \ i \neq j$
\end{itemize}
Free trade
\begin{itemize}
\item Given $\{T_i,L_i,\tau_{ij},\epsilon\}$ where $\tau_{ii} = 1 \ \forall i$,
the free-trade counterfactual scenario is the alternative parameter vector
in which $\tau_{ij} = 1 \ \forall ij$
($\{T_i,L_i,\mathbf{1}_{N \times N},\epsilon\}$)
\item Cannot compute using only shares. Need level of $\tau_{ij}$.
\end{itemize}
\end{frame}
% -----------------------------------------
\begin{frame}{Special case: Symmetric trade costs}
When $\tau_{ij} = \tau_{ji} \ \forall i,j$,
we can rewrite the system in terms of market access $\Phi_i \equiv P_j^{1-\sigma}$
(see Appendix A.1.3 of \href{http://www.jdingel.com/research/DingelMengHsiang.pdf}{Dingel, Meng, Hsiang}):
\begin{align*}
Y_i = w_i L_i  	&= \sum_j \left(\frac{w_i}{A_i}\right)^{-\epsilon}\tau_{ij}^{-\epsilon} \frac{w_j L_j}{\Phi_j}
			= \left(\frac{w_i}{A_i}\right)^{-\epsilon} \Omega_i \\
\Rightarrow \frac{w_i}{A_i}	&= \left(\frac{\Omega_i}{A_i L_i}\right)^{\frac{1}{\epsilon+1}} \\
\Rightarrow \Phi_i 	&= \sum_j \tau_{ji}^{-\epsilon} \left(\frac{w_j}{A_j}\right)^{-\epsilon} 
					= \sum_j \tau_{ji}^{-\epsilon} \left(A_j L_j / \Omega_j\right)^{\frac{\epsilon}{\epsilon+1}}\\
					&= \sum_j \tau_{ji}^{-\epsilon} \left(A_j L_j / \Phi_j\right)^{\frac{\epsilon}{\epsilon+1}}
\end{align*}
The last equality exploits the fact that 
we can normalize incomes such that $\Phi_i = \Omega_i$
when trade is balanced and $\tau_{ij}^{-\epsilon}$ is symmetric
(Anderson and van Wincoop 2003; Head and Mayer 2014).
\end{frame}
% -----------------------------------------
\begin{frame}{Multi-sector Armington model}
\begin{itemize}{\small
\item \textbf{Preferences}.
Cobb-Douglas over sectors $s = 1, \dots, S$ and CES within:
\begin{equation*}
P_i = \prod_{s=1}^S P_{is}^{\alpha_{is}} \text{ and } P_{is} = \left(\sum_{i=1}^{N} p_i(\omega_s)^{1-\sigma_s} \right)^{1/(1-\sigma_s)}
\end{equation*}
\item \textbf{Production}.
Sector-specific productivities $A_{is}$ and trade costs $\tau_{ijs}$.
\item \textbf{Gravity equation}.
Denote sales from $i$ to $j$ in sector $s$ by $X_{ijs}$ 
and $j$'s total expenditure by $X_j \equiv \sum_{i=1}^{N} \sum_{s=1}^{S} X_{ijs}$.
\begin{equation*}
\lambda_{ijs} = \frac{X_{ijs}}{X_{js}}
= \frac{T_{is} \left(\tau_{ijs}w_i\right)^{-\epsilon_s}}{\sum_{l=1}^{N} T_{ls}  \left(\tau_{ljs}w_l\right)^{-\epsilon_s}}
=\frac{T_{is} \left(\tau_{ijs}w_i\right)^{-\epsilon_s}}{\Phi_{js}}.
\end{equation*}
\item \textbf{Equilibrium}.
Labor-market clearing, goods-market clearing, and budget constraints mean
total income $Y_i = w_i L_i$ and sectoral income $Y_{is} = w_i L_{is}$ satisfy
$Y_{is} = \sum_{j=1}^{N} X_{ijs}$, $Y_i = \sum_{s=1}^{S} Y_{is}$, and $X_{is} = \alpha_{is} Y_i$ for all countries.
\begin{equation*}
	Y_{is} = \sum_{j=1}^{N} \lambda_{ijs} \alpha_{js} \sum_{s'=1}^{S} Y_{js'} .
\end{equation*}
}\end{itemize}
\textcolor{gray}{(Recall $T_{is} = A_{is}^{\epsilon_s} = A_{is}^{\sigma_1 - 1}$)}
\end{frame}
% -----------------------------------------
\begin{frame}{Wrapping up}
Next week: Gains from trade and comparative advantage
\vspace{2cm}
Extra time? Discuss assignment 4
\end{frame}
% -----------------------------------------
\end{document}
