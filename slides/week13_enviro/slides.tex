\documentclass[11pt,notes=hide,aspectratio=169]{beamer}
%Jonathan Dingel; PhD trade course

% PACKAGES
\usepackage{graphics}  % Support for images/figures
\usepackage{graphicx}  % Includes the \resizebox command
\usepackage{url}	   % Includes \urldef and \url commands
\usepackage{soul}      % Includes the underline \ul command
%\usepackage{framed}	   % Includes the \framed command for box around text
\usepackage{booktabs} %\toprule,\bottomrule
%\usepackage{natbib}
\usepackage{bibentry}  % Includes the \nobibliography command
\usepackage{bbm}       %
%\usepackage{pgfpages}  %Supports "notes on second screen" option for beamer
\usepackage{verbatim}  %Supports comments
\usepackage{tikz}		%Supports graphing/drawing
\usepackage{pgfplots} %Supports graphing/drawing
\usepackage{amsfonts}  % Lots of stuff, including \mathbb 
\usepackage{amsmath}   % Standard math package
\usepackage{amsthm}    % Includes the comment functions
\usepackage{physics}

% CUSTOM DEFINITIONS
\def\newblock{} %Get beamer to cooperate with BibTeX
\linespread{1.2}
\hypersetup{backref,pdfpagemode=FullScreen,colorlinks=true,linkcolor=blue,urlcolor=blue}
\newtheorem{proposition}{Proposition}
\newtheorem{assumption}{Assumption}
\newtheorem{condition}{Condition}

% IDENTIFYING INFORMATION
\title{Topics in Trade}
\author{Jonathan I. Dingel}
\date{Fall \the\year}

% BEAMER TEACHING STUFF
\setbeamertemplate{navigation symbols}{}  %Turn off navigation bar

% THEMATIC OPTIONS
\definecolor{columbiablue}{RGB}{185,217,235}  %Columbia blue defined at https://visualidentity.columbia.edu/branding
\definecolor{columbiadarkblue}{RGB}{0,48,135}  %Columbia dark blue defined at https://visualidentity.columbia.edu/branding
\setbeamercovered{transparent=5}
\setbeamercolor{frametitle}{fg=columbiadarkblue}
\setbeamercolor{item}{fg=columbiadarkblue}
\usefonttheme{serif}
\setbeamercolor{button}{bg = white,fg = columbiadarkblue}
\setbeamercolor{button border}{fg = columbiadarkblue}

\setbeamertemplate{footline}{\begin{center}\textcolor{gray}{Dingel -- Topics in Trade -- Week 13 -- \insertframenumber}\end{center}}
\usepackage{listings}

\lstdefinelanguage{julia}
{
  keywordsprefix=\@,
  morekeywords={
    exit,whos,edit,load,is,isa,isequal,typeof,tuple,ntuple,uid,hash,finalizer,convert,promote,
    subtype,typemin,typemax,realmin,realmax,sizeof,eps,promote_type,method_exists,applicable,
    invoke,dlopen,dlsym,system,error,throw,assert,new,Inf,Nan,pi,im,begin,while,for,in,return,
    break,continue,macro,quote,let,if,elseif,else,try,catch,end,bitstype,ccall,do,using,module,
    import,export,importall,baremodule,immutable,local,global,const,Bool,Int,Int8,Int16,Int32,
    Int64,Uint,Uint8,Uint16,Uint32,Uint64,Float32,Float64,Complex64,Complex128,Any,Nothing,None,
    function,type,typealias,abstract
  },
  sensitive=true,
  morecomment=[l]{\#},
  morestring=[b]',
  morestring=[b]" 
}

\lstset{language=julia,
  frame=tb,
  aboveskip=3mm,
  belowskip=3mm,
  showstringspaces=false,
  columns=flexible,
  basicstyle={\small\ttfamily},
  numbers=none,
  numberstyle=\tiny\color{gray},
  keywordstyle=\color{blue},
  commentstyle=\color{dkgreen},
  stringstyle=\color{mauve},
  breaklines=true,
  breakatwhitespace=false,
  tabsize=3,
  upquote=true
}

\lstset{language=matlab,upquote=true}
% ---------------------------------------------------------------------
% Program: listings-stata.tex
% Author:  github.com/mcaceresb
% Purpose: Stata language definition for LaTeX listings package
% Usage:   Add % ---------------------------------------------------------------------
% Program: listings-stata.tex
% Author:  github.com/mcaceresb
% Purpose: Stata language definition for LaTeX listings package
% Usage:   Add % ---------------------------------------------------------------------
% Program: listings-stata.tex
% Author:  github.com/mcaceresb
% Purpose: Stata language definition for LaTeX listings package
% Usage:   Add \input{listings-stata.tex} to your preamble

% Syntax from
% - https://github.com/isagalaev/highlight.js/blob/master/src/languages/stata.js
% - https://github.com/jpitblado/vim-stata/blob/master/syntax/stata.vim
% - http://fmwww.bc.edu/RePEc/bocode/s/synlightlist.ado

\RequirePackage{listings}
\RequirePackage{color}
%\RequirePackage[svgnames]{xcolor}
%\definecolor{spRed}{HTML}{BE646C}

% ---------------------------------------------------------------------
% Stata language definition

\lstdefinelanguage{stata}{
  sensitive=true,
  %
  % Macros, global and local
  alsoletter={\{\}0123456789},
  keywordsprefix=\$,
  morecomment=[n][keywordstyle9]{`}{'},
  morekeywords={},
  %
  % Comments
  morecomment=[f][\color{Green}\slshape][0]*,
  morecomment=[l]{//},
  morecomment=[s]{/*}{*/},
  %
  % Strings
  morecomment=[n][\color{Maroon}]{`"}{"'},
  morestring=[b]",
  %
  % Add-ons and system Commands
  morekeywords=[2]{
    if ,else ,in ,foreach ,for ,forv ,forva ,forval ,forvalu ,forvalue
    ,forvalues ,by ,bys ,bysort ,xi ,quietly ,qui ,capture ,about
    ,ac ,ac_7 ,acprplot ,acprplot_7 adjust ,ado ,adopath ,adoupdate
    ,alpha ,ameans ,an ,ano ,anov ,anova ,anova_estat ,anova_terms
    ,anovadef ,aorder ,ap ,app ,appe ,appen ,append ,arch ,arch_dr
    ,arch_estat ,arch_p ,archlm ,areg ,areg_p ,args ,arima ,arima_dr
    ,arima_estat ,arima_p ,as ,asmprobit ,asmprobit_estat ,asmprobit_lf
    ,asmprobit_mfx__dlg ,asmprobit_p ,ass ,asse ,asser ,assert ,avplot
    ,avplot_7 ,avplots ,avplots_7 bcskew0 ,bgodfrey ,binreg ,bip0_lf
    ,biplot ,bipp_lf ,bipr_lf ,bipr_p ,biprobit ,bitest ,bitesti
    ,bitowt ,blogit ,bmemsize ,boot ,bootsamp ,bootstrap ,bootstrap_8
    ,boxco_l ,boxco_p ,boxcox ,boxcox_6 ,boxcox_p ,bprobit ,br ,break
    ,brier ,bro ,brow ,brows ,browse ,brr ,brrstat ,bs ,bs_7 ,bsampl_w
    ,bsample ,bsample_7 ,bsqreg ,bstat ,bstat_7 ,bstat_8 ,bstrap
    ,bstrap_7 ,ca ,ca_estat ,ca_p ,cabiplot ,camat ,canon ,canon_8
    ,canon_8_p ,canon_estat ,canon_p ,cap ,caprojection ,capt ,captu
    ,captur ,capture ,cat ,cc ,cchart ,cchart_7 ,cci ,cd ,censobs_table
    ,centile ,cf ,char ,chdir ,checkdlgfiles ,checkestimationsample
    ,checkhlpfiles ,checksum ,chelp ,ci ,cii ,cl ,class ,classutil
    ,clear ,cli ,clis ,clist ,clo ,clog ,clog_lf ,clog_p ,clogi
    ,clogi_sw ,clogit ,clogit_lf ,clogit_p ,clogitp ,clogl_sw ,cloglog
    ,clonevar ,clslistarray ,cluster ,cluster_measures ,cluster_stop
    ,cluster_tree ,cluster_tree_8 ,clustermat ,cmdlog ,cnr ,cnre
    ,cnreg ,cnreg_p ,cnreg_sw ,cnsreg ,codebook ,collaps4 ,collapse
    ,colormult_nb ,colormult_nw ,compare ,compress ,conf ,confi
    ,confir ,confirm ,conren ,cons ,const ,constr ,constra ,constrai
    ,constrain ,constraint ,continue ,contract ,copy ,copyright
    ,copysource ,cor ,corc ,corr ,corr2data ,corr_anti ,corr_kmo
    ,corr_smc ,corre ,correl ,correla ,correlat ,correlate ,corrgram
    ,cou ,coun ,count ,cox ,cox_p ,cox_sw ,coxbase ,coxhaz ,coxvar
    ,cprplot ,cprplot_7 ,crc ,cret ,cretu ,cretur ,creturn ,cross ,cs
    ,cscript ,cscript_log ,csi ,ct ,ct_is ,ctset ,ctst_5 ,ctst_st
    ,cttost ,cumsp ,cumsp_7 ,cumul ,cusum ,cusum_7 ,cutil ,d ,datasig
    ,datasign ,datasigna ,datasignat ,datasignatu ,datasignatur
    ,datasignature ,datetof ,db ,dbeta ,de ,dec ,deco ,decod ,decode
    ,deff ,des ,desc ,descr ,descri ,describ ,describe ,destring
    ,dfbeta ,dfgls ,dfuller ,di ,di_g ,dir ,dirstats ,dis ,discard
    ,disp ,disp_res ,disp_s ,displ ,displa ,display ,distinct ,do
    ,doe ,doed ,doedi ,doedit ,dotplot ,dotplot_7 ,dprobit ,drawnorm
    ,drop ,ds ,ds_util ,dstdize ,duplicates ,durbina ,dwstat ,dydx ,e
    ,ed ,edi ,edit ,egen ,eivreg ,emdef ,en ,enc ,enco ,encod ,encode
    ,eq ,erase ,ereg ,ereg_lf ,ereg_p ,ereg_sw ,ereghet ,ereghet_glf
    ,ereghet_glf_sh ,ereghet_gp ,ereghet_ilf ,ereghet_ilf_sh ,ereghet_ip
    ,eret ,eretu ,eretur ,ereturn ,err ,erro ,error ,est ,est_cfexist
    ,est_cfname ,est_clickable ,est_expand ,est_hold ,est_table
    ,est_unhold ,est_unholdok ,estat ,estat_default ,estat_summ
    ,estat_vce_only ,esti ,estimates ,etodow ,etof ,etomdy ,ex ,exi
    ,exit ,expand ,expandcl ,fac ,fact ,facto ,factor ,factor_estat
    ,factor_p ,factor_pca_rotated ,factor_rotate ,factormat ,fcast
    ,fcast_compute ,fcast_graph ,fdades ,fdadesc ,fdadescr ,fdadescri
    ,fdadescrib ,fdadescribe ,fdasav ,fdasave ,fdause ,fh_st ,file
    ,open ,file ,read ,file ,close ,file ,filefilter ,fillin
    ,find_hlp_file ,findfile ,findit ,findit_7 ,fit ,fl ,fli ,flis
    ,flist ,for5_0 ,form ,forma ,format ,fpredict ,frac_154 ,frac_adj
    ,frac_chk ,frac_cox ,frac_ddp ,frac_dis ,frac_dv ,frac_in ,frac_mun
    ,frac_pp ,frac_pq ,frac_pv ,frac_wgt ,frac_xo ,fracgen ,fracplot
    ,fracplot_7 ,fracpoly ,fracpred ,fron_ex ,fron_hn ,fron_p ,fron_tn
    ,fron_tn2 ,frontier ,ftodate ,ftoe ,ftomdy ,ftowdate ,g ,gamhet_glf
    ,gamhet_gp ,gamhet_ilf ,gamhet_ip ,gamma ,gamma_d2 ,gamma_p
    ,gamma_sw ,gammahet ,gdi_hexagon ,gdi_spokes ,ge ,gen ,gene ,gener
    ,genera ,generat ,generate ,genrank ,genstd ,genvmean ,gettoken
    ,gl ,gladder ,gladder_7 ,glim_l01 ,glim_l02 glim_l03 ,glim_l04
    ,glim_l05 ,glim_l06 ,glim_l07 ,glim_l08 ,glim_l09 ,glim_l10 glim_l11
    ,glim_l12 ,glim_lf ,glim_mu ,glim_nw1 ,glim_nw2 ,glim_nw3 ,glim_p
    ,glim_v1 ,glim_v2 ,glim_v3 ,glim_v4 ,glim_v5 ,glim_v6 ,glim_v7 ,glm
    ,glm_6 glm_p ,glm_sw ,glmpred ,glo ,glob ,globa ,global ,glogit
    ,glogit_8 ,glogit_p ,gmeans ,gnbre_lf ,gnbreg ,gnbreg_5 ,gnbreg_p
    ,gomp_lf ,gompe_sw ,gomper_p ,gompertz ,gompertzhet ,gomphet_glf
    ,gomphet_glf_sh ,gomphet_gp ,gomphet_ilf ,gomphet_ilf_sh ,gomphet_ip
    ,gphdot ,gphpen ,gphprint ,gprefs ,gprobi_p ,gprobit ,gprobit_8
    ,gr ,gr7 ,gr_copy ,gr_current ,gr_db ,gr_describe ,gr_dir ,gr_draw
    ,gr_draw_replay ,gr_drop ,gr_edit ,gr_editviewopts ,gr_example
    ,gr_example2 gr_export ,gr_print ,gr_qscheme ,gr_query ,gr_read
    ,gr_rename ,gr_replay ,gr_save ,gr_set ,gr_setscheme ,gr_table
    ,gr_undo ,gr_use ,graph ,graph7 grebar ,greigen ,greigen_7
    ,greigen_8 ,grmeanby ,grmeanby_7 ,gs_fileinfo ,gs_filetype
    ,gs_graphinfo ,gs_stat ,gsort ,gwood ,h ,hadimvo ,hareg ,hausman
    ,haver ,he ,heck_d2 ,heckma_p ,heckman ,heckp_lf ,heckpr_p ,heckprob
    ,hel ,help ,hereg ,hetpr_lf ,hetpr_p ,hetprob ,hettest ,hexdump
    ,hilite ,hist ,hist_7 histogram ,hlogit ,hlu ,hmeans ,hotel
    ,hotelling ,hprobit ,hreg ,hsearch ,icd9 ,icd9_ff ,icd9p ,iis
    ,impute ,imtest ,inbase ,include ,inf ,infi ,infil ,infile ,infix
    ,inp ,inpu ,input ,ins ,insheet ,insp ,inspe ,inspec ,inspect ,integ
    ,inten ,intreg ,intreg_7 ,intreg_p ,intrg2_ll ,intrg_ll ,intrg_ll2
    ,ipolate ,iqreg ,ir ,irf ,irf_create ,irfm ,iri ,is_svy ,is_svysum
    ,isid ,istdize ,ivprob_1_lf ,ivprob_lf ,ivprobit ,ivprobit_p ,ivreg
    ,ivreg_footnote ,ivtob_1_lf ,ivtob_lf ,ivtobit ,ivtobit_p ,jackknife
    ,jacknife ,jknife ,jknife_6 ,jknife_8 ,jkstat ,joinby ,kalarma1
    ,kap ,kap_3 ,kapmeier ,kappa ,kapwgt ,kdensity ,kdensity_7 keep
    ,ksm ,ksmirnov ,ktau ,kwallis ,l ,la ,lab ,labe ,label ,labelbook
    ,ladder ,levels ,levelsof ,leverage ,lfit ,lfit_p ,li ,lincom ,line
    ,linktest ,lis ,list ,lloghet_glf ,lloghet_glf_sh ,lloghet_gp
    ,lloghet_ilf ,lloghet_ilf_sh ,lloghet_ip ,llogi_sw ,llogis_p
    ,llogist ,llogistic ,llogistichet ,lnorm_lf ,lnorm_sw ,lnorma_p
    ,lnormal ,lnormalhet ,lnormhet_glf ,lnormhet_glf_sh ,lnormhet_gp
    ,lnormhet_ilf ,lnormhet_ilf_sh ,lnormhet_ip ,lnskew0 ,loadingplot
    ,loc ,loca ,local ,log ,logi ,logis_lf ,logistic ,logistic_p
    ,logit ,logit_estat ,logit_p ,loglogs ,logrank ,loneway ,lookfor
    ,lookup ,lowess ,lowess_7 ,lpredict ,lrecomp ,lroc ,lroc_7 ,lrtest
    ,ls ,lsens ,lsens_7 ,lsens_x ,lstat ,ltable ,ltable_7 ,ltriang
    ,lv ,lvr2plot ,lvr2plot_7 ,m ,ma ,mac ,macr ,macro ,makecns ,man
    ,manova ,manova_estat ,manova_p ,manovatest ,mantel ,mark ,markin
    ,markout ,marksample ,mat ,mat_capp ,mat_order ,mat_put_rr ,mat_rapp
    ,mata ,mata_clear ,mata_describe ,mata_drop ,mata_matdescribe
    ,mata_matsave ,mata_matuse ,mata_memory ,mata_mlib ,mata_mosave
    ,mata_rename ,mata_which ,matalabel ,matcproc ,matlist ,matname
    ,matr ,matri ,matrix ,matrix_input__dlg ,matstrik ,mcc ,mcci ,md0_
    ,md1_ ,md1debug_ ,md2_ ,md2debug_ ,mds ,mds_estat ,mds_p ,mdsconfig
    ,mdslong ,mdsmat ,mdsshepard ,mdytoe ,mdytof ,me_derd ,mean ,means
    ,median ,memory ,memsize ,meqparse ,mer ,merg ,merge ,mfp ,mfx
    ,mhelp ,mhodds ,minbound ,mixed_ll ,mixed_ll_reparm ,mkassert
    ,mkdir ,mkmat ,mkspline ,ml ,ml_5 ml_adjs ,ml_bhhhs ,ml_c_d
    ,ml_check ,ml_clear ,ml_cnt ,ml_debug ,ml_defd ,ml_e0 ml_e0_bfgs
    ,ml_e0_cycle ,ml_e0_dfp ,ml_e0i ,ml_e1 ,ml_e1_bfgs ,ml_e1_bhhh
    ,ml_e1_cycle ,ml_e1_dfp ,ml_e2 ,ml_e2_cycle ,ml_ebfg0 ,ml_ebfr0
    ,ml_ebfr1 ml_ebh0q ,ml_ebhh0 ,ml_ebhr0 ,ml_ebr0i ,ml_ecr0i ,ml_edfp0
    ,ml_edfr0 ,ml_edfr1 ml_edr0i ,ml_eds ,ml_eer0i ,ml_egr0i ,ml_elf
    ,ml_elf_bfgs ,ml_elf_bhhh ,ml_elf_cycle ,ml_elf_dfp ,ml_elfi
    ,ml_elfs ,ml_enr0i ,ml_enrr0 ,ml_erdu0 ml_erdu0_bfgs ,ml_erdu0_bhhh
    ,ml_erdu0_bhhhq ,ml_erdu0_cycle ,ml_erdu0_dfp ,ml_erdu0_nrbfgs
    ,ml_exde ,ml_footnote ,ml_geqnr ,ml_grad0 ,ml_graph ,ml_hbhhh
    ,ml_hd0 ,ml_hold ,ml_init ,ml_inv ,ml_log ,ml_max ,ml_mlout
    ,ml_mlout_8 ,ml_model ,ml_nb0 ,ml_opt ,ml_p ,ml_plot ,ml_query
    ,ml_rdgrd ,ml_repor ,ml_s_e ,ml_score ,ml_searc ,ml_technique
    ,ml_unhold ,mleval ,mlf_ ,mlmatbysum ,mlmatsum ,mlog ,mlogi ,mlogit
    ,mlogit_footnote ,mlogit_p ,mlopts ,mlsum ,mlvecsum ,mnl0_ ,mor
    ,more ,mov ,move ,mprobit ,mprobit_lf ,mprobit_p ,mrdu0_ ,mrdu1_
    ,mvdecode ,mvencode ,mvreg ,mvreg_estat ,n ,nbreg ,nbreg_al
    ,nbreg_lf ,nbreg_p ,nbreg_sw ,nestreg ,net ,newey ,newey_7 ,newey_p
    ,news ,nl ,nl_7 ,nl_9 ,nl_9_p ,nl_p ,nl_p_7 nlcom ,nlcom_p ,nlexp2
    ,nlexp2_7 ,nlexp2a ,nlexp2a_7 ,nlexp3 ,nlexp3_7 ,nlgom3 nlgom3_7
    ,nlgom4 ,nlgom4_7 ,nlinit ,nllog3 ,nllog3_7 ,nllog4 ,nllog4_7
    ,nlog_rd ,nlogit ,nlogit_p ,nlogitgen ,nlogittree ,nlpred ,no
    ,nobreak ,noi ,nois ,noisi ,noisil ,noisily ,note ,notes ,notes_dlg
    ,nptrend ,numlabel ,numlist ,odbc ,old_ver ,olo ,olog ,ologi
    ,ologi_sw ,ologit ,ologit_p ,ologitp ,on ,one ,onew ,onewa ,oneway
    ,op_colnm ,op_comp ,op_diff ,op_inv ,op_str ,opr ,opro ,oprob
    ,oprob_sw ,oprobi ,oprobi_p ,oprobit ,oprobitp ,opts_exclusive
    ,order ,orthog ,orthpoly ,ou ,out ,outf ,outfi ,outfil ,outfile
    ,outs ,outsh ,outshe ,outshee ,outsheet ,ovtest ,pac ,pac_7 ,palette
    ,parse ,parse_dissim ,pause ,pca ,pca_8 pca_display ,pca_estat
    ,pca_p ,pca_rotate ,pcamat ,pchart ,pchart_7 ,pchi ,pchi_7 ,pcorr
    ,pctile ,pentium ,pergram ,pergram_7 ,permute ,permute_8 ,personal
    ,peto_st ,pkcollapse ,pkcross ,pkequiv ,pkexamine ,pkexamine_7
    ,pkshape ,pksumm ,pksumm_7 ,pl ,plo ,plot ,plugin ,pnorm ,pnorm_7
    ,poisgof ,poiss_lf ,poiss_sw ,poisso_p ,poisson ,poisson_estat
    ,post ,postclose ,postfile ,postutil ,pperron ,pr ,prais ,prais_e
    ,prais_e2 ,prais_p ,predict ,predictnl ,preserve ,print ,pro ,prob
    ,probi ,probit ,probit_estat ,probit_p ,proc_time ,procoverlay
    ,procrustes ,procrustes_estat ,procrustes_p ,profiler ,prog ,progr
    ,progra ,program ,prop ,proportion ,prtest ,prtesti ,pwcorr ,pwd
    ,q ,s ,qby ,qbys ,qchi ,qchi_7 ,qladder ,qladder_7 ,qnorm ,qnorm_7
    ,qqplot ,qqplot_7 ,qreg ,qreg_c ,qreg_p ,qreg_sw ,qu ,quadchk
    ,quantile ,quantile_7 ,que ,quer ,query ,range ,ranksum ,ratio
    ,rchart ,rchart_7 ,rcof ,recast ,reclink ,recode ,reg ,reg3
    ,reg3_p ,regdw ,regr ,regre ,regre_p2 ,regres ,regres_p ,regress
    ,regress_estat ,regriv_p ,remap ,ren ,rena ,renam ,rename ,renpfix
    ,repeat ,replace ,report ,reshape ,restore ,ret ,retu ,retur ,return
    ,rm ,rmdir ,robvar ,roccomp ,roccomp_7 ,roccomp_8 ,rocf_lf ,rocfit
    ,rocfit_8 ,rocgold ,rocplot ,rocplot_7 ,roctab ,roctab_7 ,rolling
    ,rologit ,rologit_p ,rot ,rota ,rotat ,rotate ,rotatemat ,rreg
    ,rreg_p ,ru ,run ,runtest ,rvfplot ,rvfplot_7 ,rvpplot ,rvpplot_7
    ,sa ,safesum ,sample ,sampsi ,sav ,save ,savedresults ,saveold ,sc
    ,sca ,scal ,scala ,scalar ,scatter ,scm_mine ,sco ,scob_lf ,scob_p
    ,scobi_sw ,scobit ,scor ,score ,scoreplot ,scoreplot_help ,scree
    ,screeplot ,screeplot_help ,sdtest ,sdtesti ,se ,search ,separate
    ,seperate ,serrbar ,serrbar_7 ,serset ,set ,set_defaults ,sfrancia
    ,sh ,she ,shel ,shell ,shewhart ,shewhart_7 ,signestimationsample
    ,signrank ,signtest ,simul ,simul_7 simulate ,simulate_8 ,sktest
    ,sleep ,slogit ,slogit_d2 ,slogit_p ,smooth ,snapspan ,so ,sor
    ,sort ,spearman ,spikeplot ,spikeplot_7 ,spikeplt ,spline_x ,split
    ,sqreg ,sqreg_p ,sret ,sretu ,sretur ,sreturn ,ssc ,st ,st_ct ,st_hc
    ,st_hcd ,st_hcd_sh ,st_is ,st_issys ,st_note ,st_promo ,st_set
    ,st_show ,st_smpl ,st_subid ,stack ,statsby ,statsby_8 ,stbase
    ,stci ,stci_7 ,stcox ,stcox_estat ,stcox_fr ,stcox_fr_ll ,stcox_p
    ,stcox_sw ,stcoxkm ,stcoxkm_7 ,stcstat ,stcurv ,stcurve ,stcurve_7
    ,stdes ,stem ,stepwise ,stereg ,stfill ,stgen ,stir ,stjoin ,stmc
    ,stmh ,stphplot ,stphplot_7 ,stphtest ,stphtest_7 ,stptime ,strate
    ,strate_7 ,streg ,streg_sw ,streset ,sts ,sts_7 ,stset ,stsplit
    ,stsum ,sttocc ,sttoct ,stvary ,stweib ,su ,suest ,suest_8 ,sum
    ,summ ,summa ,summar ,summari ,summariz ,summarize ,sunflower
    ,sureg ,survcurv ,survsum ,svar ,svar_p ,svmat ,svy ,svy_disp
    ,svy_dreg ,svy_est ,svy_est_7 ,svy_estat ,svy_get ,svy_gnbreg_p
    ,svy_head ,svy_header ,svy_heckman_p ,svy_heckprob_p ,svy_intreg_p
    ,svy_ivreg_p ,svy_logistic_p ,svy_logit_p ,svy_mlogit_p ,svy_nbreg_p
    ,svy_ologit_p ,svy_oprobit_p ,svy_poisson_p ,svy_probit_p
    ,svy_regress_p ,svy_sub ,svy_sub_7 ,svy_x ,svy_x_7 ,svy_x_p ,svydes
    ,svydes_8 ,svygen ,svygnbreg ,svyheckman ,svyheckprob ,svyintreg
    ,svyintreg_7 ,svyintrg ,svyivreg ,svylc ,svylog_p ,svylogit
    ,svymarkout ,svymarkout_8 ,svymean ,svymlog ,svymlogit ,svynbreg
    ,svyolog ,svyologit ,svyoprob ,svyoprobit ,svyopts ,svypois
    ,svypois_7 svypoisson ,svyprobit ,svyprobt ,svyprop ,svyprop_7
    ,svyratio ,svyreg ,svyreg_p ,svyregress ,svyset ,svyset_7 ,svyset_8
    ,svytab ,svytab_7 ,svytest ,svytotal ,sw ,sw_8 ,swcnreg ,swcox
    ,swereg ,swilk ,swlogis ,swlogit ,swologit ,swoprbt ,swpois
    ,swprobit ,swqreg ,swtobit ,swweib ,symmetry ,symmi ,symplot
    ,symplot_7 syntax ,sysdescribe ,sysdir ,sysuse ,szroeter ,ta ,tab
    ,tab1 ,tab2 ,tab_or ,tabd ,tabdi ,tabdis ,tabdisp ,tabi ,table
    ,tabodds ,tabodds_7 ,tabstat ,tabu ,tabul ,tabula ,tabulat ,tabulate
    ,te ,tempfile ,tempname ,tempvar ,tes ,test ,testnl ,testparm
    ,teststd ,tetrachoric ,time_it ,timer ,tis ,tob ,tobi ,tobit
    ,tobit_p ,tobit_sw ,token ,tokeni ,tokeniz ,tokenize ,tostring
    ,total ,translate ,translator ,transmap ,treat_ll ,treatr_p
    ,treatreg ,trim ,trnb_cons ,trnb_mean ,trpoiss_d2 ,trunc_ll
    ,truncr_p ,truncreg ,tsappend ,tset ,tsfill ,tsline ,tsline_ex
    ,tsreport ,tsrevar ,tsrline ,tsset ,tssmooth ,tsunab ,ttest
    ,ttesti ,tut_chk ,tut_wait ,tutorial ,tw ,tware_st ,two ,twoway
    ,twoway__fpfit_serset ,twoway__function_gen ,twoway__histogram_gen
    ,twoway__ipoint_serset ,twoway__ipoints_serset ,twoway__kdensity_gen
    ,twoway__lfit_serset ,twoway__normgen_gen ,twoway__pci_serset
    ,twoway__qfit_serset ,twoway__scatteri_serset ,twoway__sunflower_gen
    ,twoway_ksm_serset ,ty ,typ ,type ,typeof ,u ,unab ,unabbrev
    ,unabcmd ,update ,us ,use ,uselabel ,var ,var_mkcompanion
    ,var_p ,varbasic ,varfcast ,vargranger ,varirf ,varirf_add
    ,varirf_cgraph ,varirf_create ,varirf_ctable ,varirf_describe
    ,varirf_dir ,varirf_drop ,varirf_erase ,varirf_graph ,varirf_ograph
    ,varirf_rename ,varirf_set ,varirf_table ,varlist ,varlmar
    ,varnorm ,varsoc ,varstable ,varstable_w ,varstable_w2 ,varwle
    ,vce ,vec ,vec_fevd ,vec_mkphi ,vec_p ,vec_p_w ,vecirf_create
    ,veclmar ,veclmar_w ,vecnorm ,vecnorm_w ,vecrank ,vecstable
    ,verinst ,vers ,versi ,versio ,version ,view ,viewsource ,vif
    ,vwls ,wdatetof ,webdescribe ,webseek ,webuse ,weib1_lf ,weib2_lf
    ,weib_lf ,weib_lf0 weibhet_glf ,weibhet_glf_sh ,weibhet_glfa
    ,weibhet_glfa_sh ,weibhet_gp ,weibhet_ilf ,weibhet_ilf_sh
    ,weibhet_ilfa ,weibhet_ilfa_sh ,weibhet_ip ,weibu_sw ,weibul_p
    ,weibull ,weibull_c ,weibull_s ,weibullhet ,wh ,whelp ,whi ,which
    ,whil ,while ,wilc_st ,wilcoxon ,win ,wind ,windo ,window ,winexec
    ,wntestb ,wntestb_7 ,wntestq ,xchart ,xchart_7 ,xcorr ,xcorr_7 ,xi
    ,xi_6 ,xmlsav ,xmlsave ,xmluse ,xpose ,xsh ,xshe ,xshel ,xshell
    ,xt_iis ,xt_tis ,xtab_p ,xtabond ,xtbin_p ,xtclog ,xtcloglog
    ,xtcloglog_8 ,xtcloglog_d2 ,xtcloglog_pa_p ,xtcloglog_re_p ,xtcnt_p
    ,xtcorr ,xtdata ,xtdes ,xtfront_p ,xtfrontier ,xtgee ,xtgee_elink
    ,xtgee_estat ,xtgee_makeivar ,xtgee_p ,xtgee_plink ,xtgls ,xtgls_p
    ,xthaus ,xthausman ,xtht_p ,xthtaylor ,xtile ,xtint_p ,xtintreg
    ,xtintreg_8 ,xtintreg_d2 xtintreg_p ,xtivp_1 ,xtivp_2 ,xtivreg
    ,xtline ,xtline_ex ,xtlogit ,xtlogit_8 xtlogit_d2 ,xtlogit_fe_p
    ,xtlogit_pa_p ,xtlogit_re_p ,xtmixed ,xtmixed_estat ,xtmixed_p
    ,xtnb_fe ,xtnb_lf ,xtnbreg ,xtnbreg_pa_p ,xtnbreg_refe_p ,xtpcse
    ,xtpcse_p ,xtpois ,xtpoisson ,xtpoisson_d2 ,xtpoisson_pa_p
    ,xtpoisson_refe_p ,xtpred ,xtprobit ,xtprobit_8 ,xtprobit_d2
    ,xtprobit_re_p ,xtps_fe ,xtps_lf ,xtps_ren ,xtps_ren_8 ,xtrar_p
    ,xtrc ,xtrc_p ,xtrchh ,xtrefe_p ,xtreg ,xtreg_be ,xtreg_fe
    ,xtreg_ml ,xtreg_pa_p ,xtreg_re ,xtregar ,xtrere_p ,xtset
    ,xtsf_ll ,xtsf_llti ,xtsum ,xttab ,xttest0 ,xttobit ,xttobit_8
    ,xttobit_p ,xttrans ,yx ,yxview__barlike_draw ,yxview_area_draw
    ,yxview_bar_draw ,yxview_dot_draw ,yxview_dropline_draw
    ,yxview_function_draw ,yxview_iarrow_draw ,yxview_ilabels_draw
    ,yxview_normal_draw ,yxview_pcarrow_draw ,yxview_pcbarrow_draw
    ,yxview_pccapsym_draw ,yxview_pcscatter_draw ,yxview_pcspike_draw
    ,yxview_rarea_draw ,yxview_rbar_draw ,yxview_rbarm_draw
    ,yxview_rcap_draw ,yxview_rcapsym_draw ,yxview_rconnected_draw
    ,yxview_rline_draw ,yxview_rscatter_draw ,yxview_rspike_draw
    ,yxview_spike_draw ,yxview_sunflower_draw ,zap_s ,zinb ,zinb_llf
    ,zinb_plf ,zip ,zip_llf ,zip_p ,zip_plf ,zt_ct_5 ,zt_hc_5 ,zt_hcd_5
    ,zt_is_5 ,zt_iss_5 ,zt_sho_5 zt_smp_5 ,ztbase_5 ,ztcox_5 ,ztdes_5
    ,ztereg_5 ,ztfill_5 ,ztgen_5 ,ztir_5 ztjoin_5 ,ztnb ,ztnb_p ,ztp
    ,ztp_p ,zts_5 ,ztset_5 ,ztspli_5 ,ztsum_5 ,zttoct_5 ztvary_5
    ,ztweib_5
  },
  %
  % Built-in functions
  morekeywords=[3]{
    Cdhms ,Chms ,Clock ,Cmdyhms ,Cofc ,Cofd ,F ,Fden ,Ftail ,I ,J
    ,_caller ,abbrev ,abs ,acos ,acosh ,asin ,asinh ,atan ,atan2
    ,atanh ,autocode ,betaden ,binomial ,binomialp ,binomialtail
    ,binormal ,bofd ,byteorder ,c ,ceil ,char ,chi2 ,chi2den ,chi2tail
    ,cholesky ,chop ,clip ,clock ,cloglog ,cofC ,cofd ,colnumb ,colsof
    ,comb ,cond ,corr ,cos ,cosh ,d ,daily ,date ,day ,det ,dgammapda
    ,dgammapdada ,dgammapdadx ,dgammapdx ,dgammapdxdx ,dhms ,diag
    ,diag0cnt ,digamma ,dofC ,dofb ,dofc ,dofh ,dofm ,dofq ,dofw ,dofy
    ,dow ,doy ,dunnettprob ,e ,el ,epsdouble ,epsfloat ,exp ,fileexists
    ,fileread ,filereaderror ,filewrite ,float ,floor ,fmtwidth
    ,gammaden ,gammap ,gammaptail ,get ,group ,h ,hadamard ,halfyear
    ,halfyearly ,has_eprop ,hh ,hhC ,hms ,hofd ,hours ,hypergeometric
    ,hypergeometricp ,ibeta ,ibetatail ,index ,indexnot ,inlist
    ,inrange ,int ,inv ,invF ,invFtail ,invbinomial ,invbinomialtail
    ,invchi2 ,invchi2tail ,invcloglog ,invdunnettprob ,invgammap
    ,invgammaptail ,invibeta ,invibetatail ,invlogit ,invnFtail
    ,invnbinomial ,invnbinomialtail ,invnchi2 ,invnchi2tail ,invnibeta
    ,invnorm ,invnormal ,invnttail ,invpoisson ,invpoissontail ,invsym
    ,invt ,invttail ,invtukeyprob ,irecode ,issym ,issymmetric ,itrim
    ,length ,ln ,lnfact ,lnfactorial ,lngamma ,lnnormal ,lnnormalden
    ,log ,log10 ,logit ,lower ,ltrim ,m ,match ,matmissing ,matrix
    ,matuniform ,max ,maxbyte ,maxdouble ,maxfloat ,maxint ,maxlong ,mdy
    ,mdyhms ,mi ,mi ,min ,minbyte ,mindouble ,minfloat ,minint ,minlong
    ,minutes ,missing ,mm ,mmC ,mod ,mofd ,month ,monthly ,mreldif
    ,msofhours ,msofminutes ,msofseconds ,nF ,nFden ,nFtail ,nbetaden
    ,nbinomial ,nbinomialp ,nbinomialtail ,nchi2 ,nchi2den ,nchi2tail
    ,nibeta ,norm ,normal ,normalden ,normd ,npnF ,npnchi2 ,npnt ,nt
    ,ntden ,nttail ,nullmat ,plural ,poisson ,poissonp ,poissontail
    ,proper ,q ,qofd ,quarter ,quarterly ,r ,rbeta ,rbinomial ,rchi2
    real ,recode ,regexm ,regexr ,regexs ,reldif ,replay ,return
    ,reverse ,rgamma ,rhypergeometric ,rnbinomial ,rnormal ,round
    ,rownumb ,rowsof ,rpoisson ,rt ,rtrim ,runiform ,s ,scalar ,seconds
    ,sign ,sin ,sinh ,smallestdouble ,soundex ,soundex_nara ,sqrt ,ss
    ,ssC ,strcat ,strdup ,string ,strlen ,strlower ,strltrim ,strmatch
    ,strofreal ,strpos ,strproper ,strreverse ,strrtrim ,strtoname
    ,strtrim ,strupper ,subinstr ,subinword ,substr ,sum ,sweep ,syminv
    ,t ,tC ,tan ,tanh ,tc ,td ,tden ,th ,tin ,tm ,tq ,trace ,trigamma
    ,trim ,trunc ,ttail ,tukeyprob ,tw ,twithin ,uniform ,upper ,vec
    ,vecdiag ,w ,week ,weekly ,wofd ,word ,wordcount ,year ,yearly
    ,yh ,ym ,yofd ,yq ,yw
  },
  %
  % Numbers
  morekeywords=[4]{
    0 ,1 ,2 ,3 ,4 ,5 ,6 ,7 ,8 ,9
  },
}

% ---------------------------------------------------------------------
% Stata editor style

\providecommand{\textcolordummy}[2]{#2}
\lstalias{Stata}{stata}
\lstdefinestyle{stata-editor}{
    language=stata,
    %
    % Global variables
    keywordstyle={\bfseries\color{spRed}},
    %
    % Add-ons system commands
    keywordstyle=[2]{\bfseries\color{NavyBlue}},
    %
    % Built-in functions
    keywordstyle=[3]{\color{blue}},
    %
    % Numbers
    keywordstyle=[4]{\color{blue}},
    %
    % User macros (variables)
    keywordstyle = [9]{\bfseries\color{LightSteelBlue}\let\textcolor\textcolordummy},
    %
    % Strings and comments
    stringstyle  = \color{Maroon},
    commentstyle = \color{Green}\slshape,
}

% ---------------------------------------------------------------------
% Suggested settings

% \lstset{
%   basicstyle        = \setmonofont{DejaVu Sans Mono}\footnotesize\ttfamily,
%   tabsize           = 4,      % Tab size
%   showstringspaces  = false,  % Don't underline spaces in strings
%   showspaces        = false,  % Don't underline spaces
%   breaklines        = true,   % Automatic line breaking
%   breakatwhitespace = true,   % Breaks only at white space.
%   lineskip          = 1.5pt,  % Sparing between lines of code
%   commentstyle      = \color{black!50}\itshape \let\textcolor\textcolordummy,
% } to your preamble

% Syntax from
% - https://github.com/isagalaev/highlight.js/blob/master/src/languages/stata.js
% - https://github.com/jpitblado/vim-stata/blob/master/syntax/stata.vim
% - http://fmwww.bc.edu/RePEc/bocode/s/synlightlist.ado

\RequirePackage{listings}
\RequirePackage{color}
%\RequirePackage[svgnames]{xcolor}
%\definecolor{spRed}{HTML}{BE646C}

% ---------------------------------------------------------------------
% Stata language definition

\lstdefinelanguage{stata}{
  sensitive=true,
  %
  % Macros, global and local
  alsoletter={\{\}0123456789},
  keywordsprefix=\$,
  morecomment=[n][keywordstyle9]{`}{'},
  morekeywords={},
  %
  % Comments
  morecomment=[f][\color{Green}\slshape][0]*,
  morecomment=[l]{//},
  morecomment=[s]{/*}{*/},
  %
  % Strings
  morecomment=[n][\color{Maroon}]{`"}{"'},
  morestring=[b]",
  %
  % Add-ons and system Commands
  morekeywords=[2]{
    if ,else ,in ,foreach ,for ,forv ,forva ,forval ,forvalu ,forvalue
    ,forvalues ,by ,bys ,bysort ,xi ,quietly ,qui ,capture ,about
    ,ac ,ac_7 ,acprplot ,acprplot_7 adjust ,ado ,adopath ,adoupdate
    ,alpha ,ameans ,an ,ano ,anov ,anova ,anova_estat ,anova_terms
    ,anovadef ,aorder ,ap ,app ,appe ,appen ,append ,arch ,arch_dr
    ,arch_estat ,arch_p ,archlm ,areg ,areg_p ,args ,arima ,arima_dr
    ,arima_estat ,arima_p ,as ,asmprobit ,asmprobit_estat ,asmprobit_lf
    ,asmprobit_mfx__dlg ,asmprobit_p ,ass ,asse ,asser ,assert ,avplot
    ,avplot_7 ,avplots ,avplots_7 bcskew0 ,bgodfrey ,binreg ,bip0_lf
    ,biplot ,bipp_lf ,bipr_lf ,bipr_p ,biprobit ,bitest ,bitesti
    ,bitowt ,blogit ,bmemsize ,boot ,bootsamp ,bootstrap ,bootstrap_8
    ,boxco_l ,boxco_p ,boxcox ,boxcox_6 ,boxcox_p ,bprobit ,br ,break
    ,brier ,bro ,brow ,brows ,browse ,brr ,brrstat ,bs ,bs_7 ,bsampl_w
    ,bsample ,bsample_7 ,bsqreg ,bstat ,bstat_7 ,bstat_8 ,bstrap
    ,bstrap_7 ,ca ,ca_estat ,ca_p ,cabiplot ,camat ,canon ,canon_8
    ,canon_8_p ,canon_estat ,canon_p ,cap ,caprojection ,capt ,captu
    ,captur ,capture ,cat ,cc ,cchart ,cchart_7 ,cci ,cd ,censobs_table
    ,centile ,cf ,char ,chdir ,checkdlgfiles ,checkestimationsample
    ,checkhlpfiles ,checksum ,chelp ,ci ,cii ,cl ,class ,classutil
    ,clear ,cli ,clis ,clist ,clo ,clog ,clog_lf ,clog_p ,clogi
    ,clogi_sw ,clogit ,clogit_lf ,clogit_p ,clogitp ,clogl_sw ,cloglog
    ,clonevar ,clslistarray ,cluster ,cluster_measures ,cluster_stop
    ,cluster_tree ,cluster_tree_8 ,clustermat ,cmdlog ,cnr ,cnre
    ,cnreg ,cnreg_p ,cnreg_sw ,cnsreg ,codebook ,collaps4 ,collapse
    ,colormult_nb ,colormult_nw ,compare ,compress ,conf ,confi
    ,confir ,confirm ,conren ,cons ,const ,constr ,constra ,constrai
    ,constrain ,constraint ,continue ,contract ,copy ,copyright
    ,copysource ,cor ,corc ,corr ,corr2data ,corr_anti ,corr_kmo
    ,corr_smc ,corre ,correl ,correla ,correlat ,correlate ,corrgram
    ,cou ,coun ,count ,cox ,cox_p ,cox_sw ,coxbase ,coxhaz ,coxvar
    ,cprplot ,cprplot_7 ,crc ,cret ,cretu ,cretur ,creturn ,cross ,cs
    ,cscript ,cscript_log ,csi ,ct ,ct_is ,ctset ,ctst_5 ,ctst_st
    ,cttost ,cumsp ,cumsp_7 ,cumul ,cusum ,cusum_7 ,cutil ,d ,datasig
    ,datasign ,datasigna ,datasignat ,datasignatu ,datasignatur
    ,datasignature ,datetof ,db ,dbeta ,de ,dec ,deco ,decod ,decode
    ,deff ,des ,desc ,descr ,descri ,describ ,describe ,destring
    ,dfbeta ,dfgls ,dfuller ,di ,di_g ,dir ,dirstats ,dis ,discard
    ,disp ,disp_res ,disp_s ,displ ,displa ,display ,distinct ,do
    ,doe ,doed ,doedi ,doedit ,dotplot ,dotplot_7 ,dprobit ,drawnorm
    ,drop ,ds ,ds_util ,dstdize ,duplicates ,durbina ,dwstat ,dydx ,e
    ,ed ,edi ,edit ,egen ,eivreg ,emdef ,en ,enc ,enco ,encod ,encode
    ,eq ,erase ,ereg ,ereg_lf ,ereg_p ,ereg_sw ,ereghet ,ereghet_glf
    ,ereghet_glf_sh ,ereghet_gp ,ereghet_ilf ,ereghet_ilf_sh ,ereghet_ip
    ,eret ,eretu ,eretur ,ereturn ,err ,erro ,error ,est ,est_cfexist
    ,est_cfname ,est_clickable ,est_expand ,est_hold ,est_table
    ,est_unhold ,est_unholdok ,estat ,estat_default ,estat_summ
    ,estat_vce_only ,esti ,estimates ,etodow ,etof ,etomdy ,ex ,exi
    ,exit ,expand ,expandcl ,fac ,fact ,facto ,factor ,factor_estat
    ,factor_p ,factor_pca_rotated ,factor_rotate ,factormat ,fcast
    ,fcast_compute ,fcast_graph ,fdades ,fdadesc ,fdadescr ,fdadescri
    ,fdadescrib ,fdadescribe ,fdasav ,fdasave ,fdause ,fh_st ,file
    ,open ,file ,read ,file ,close ,file ,filefilter ,fillin
    ,find_hlp_file ,findfile ,findit ,findit_7 ,fit ,fl ,fli ,flis
    ,flist ,for5_0 ,form ,forma ,format ,fpredict ,frac_154 ,frac_adj
    ,frac_chk ,frac_cox ,frac_ddp ,frac_dis ,frac_dv ,frac_in ,frac_mun
    ,frac_pp ,frac_pq ,frac_pv ,frac_wgt ,frac_xo ,fracgen ,fracplot
    ,fracplot_7 ,fracpoly ,fracpred ,fron_ex ,fron_hn ,fron_p ,fron_tn
    ,fron_tn2 ,frontier ,ftodate ,ftoe ,ftomdy ,ftowdate ,g ,gamhet_glf
    ,gamhet_gp ,gamhet_ilf ,gamhet_ip ,gamma ,gamma_d2 ,gamma_p
    ,gamma_sw ,gammahet ,gdi_hexagon ,gdi_spokes ,ge ,gen ,gene ,gener
    ,genera ,generat ,generate ,genrank ,genstd ,genvmean ,gettoken
    ,gl ,gladder ,gladder_7 ,glim_l01 ,glim_l02 glim_l03 ,glim_l04
    ,glim_l05 ,glim_l06 ,glim_l07 ,glim_l08 ,glim_l09 ,glim_l10 glim_l11
    ,glim_l12 ,glim_lf ,glim_mu ,glim_nw1 ,glim_nw2 ,glim_nw3 ,glim_p
    ,glim_v1 ,glim_v2 ,glim_v3 ,glim_v4 ,glim_v5 ,glim_v6 ,glim_v7 ,glm
    ,glm_6 glm_p ,glm_sw ,glmpred ,glo ,glob ,globa ,global ,glogit
    ,glogit_8 ,glogit_p ,gmeans ,gnbre_lf ,gnbreg ,gnbreg_5 ,gnbreg_p
    ,gomp_lf ,gompe_sw ,gomper_p ,gompertz ,gompertzhet ,gomphet_glf
    ,gomphet_glf_sh ,gomphet_gp ,gomphet_ilf ,gomphet_ilf_sh ,gomphet_ip
    ,gphdot ,gphpen ,gphprint ,gprefs ,gprobi_p ,gprobit ,gprobit_8
    ,gr ,gr7 ,gr_copy ,gr_current ,gr_db ,gr_describe ,gr_dir ,gr_draw
    ,gr_draw_replay ,gr_drop ,gr_edit ,gr_editviewopts ,gr_example
    ,gr_example2 gr_export ,gr_print ,gr_qscheme ,gr_query ,gr_read
    ,gr_rename ,gr_replay ,gr_save ,gr_set ,gr_setscheme ,gr_table
    ,gr_undo ,gr_use ,graph ,graph7 grebar ,greigen ,greigen_7
    ,greigen_8 ,grmeanby ,grmeanby_7 ,gs_fileinfo ,gs_filetype
    ,gs_graphinfo ,gs_stat ,gsort ,gwood ,h ,hadimvo ,hareg ,hausman
    ,haver ,he ,heck_d2 ,heckma_p ,heckman ,heckp_lf ,heckpr_p ,heckprob
    ,hel ,help ,hereg ,hetpr_lf ,hetpr_p ,hetprob ,hettest ,hexdump
    ,hilite ,hist ,hist_7 histogram ,hlogit ,hlu ,hmeans ,hotel
    ,hotelling ,hprobit ,hreg ,hsearch ,icd9 ,icd9_ff ,icd9p ,iis
    ,impute ,imtest ,inbase ,include ,inf ,infi ,infil ,infile ,infix
    ,inp ,inpu ,input ,ins ,insheet ,insp ,inspe ,inspec ,inspect ,integ
    ,inten ,intreg ,intreg_7 ,intreg_p ,intrg2_ll ,intrg_ll ,intrg_ll2
    ,ipolate ,iqreg ,ir ,irf ,irf_create ,irfm ,iri ,is_svy ,is_svysum
    ,isid ,istdize ,ivprob_1_lf ,ivprob_lf ,ivprobit ,ivprobit_p ,ivreg
    ,ivreg_footnote ,ivtob_1_lf ,ivtob_lf ,ivtobit ,ivtobit_p ,jackknife
    ,jacknife ,jknife ,jknife_6 ,jknife_8 ,jkstat ,joinby ,kalarma1
    ,kap ,kap_3 ,kapmeier ,kappa ,kapwgt ,kdensity ,kdensity_7 keep
    ,ksm ,ksmirnov ,ktau ,kwallis ,l ,la ,lab ,labe ,label ,labelbook
    ,ladder ,levels ,levelsof ,leverage ,lfit ,lfit_p ,li ,lincom ,line
    ,linktest ,lis ,list ,lloghet_glf ,lloghet_glf_sh ,lloghet_gp
    ,lloghet_ilf ,lloghet_ilf_sh ,lloghet_ip ,llogi_sw ,llogis_p
    ,llogist ,llogistic ,llogistichet ,lnorm_lf ,lnorm_sw ,lnorma_p
    ,lnormal ,lnormalhet ,lnormhet_glf ,lnormhet_glf_sh ,lnormhet_gp
    ,lnormhet_ilf ,lnormhet_ilf_sh ,lnormhet_ip ,lnskew0 ,loadingplot
    ,loc ,loca ,local ,log ,logi ,logis_lf ,logistic ,logistic_p
    ,logit ,logit_estat ,logit_p ,loglogs ,logrank ,loneway ,lookfor
    ,lookup ,lowess ,lowess_7 ,lpredict ,lrecomp ,lroc ,lroc_7 ,lrtest
    ,ls ,lsens ,lsens_7 ,lsens_x ,lstat ,ltable ,ltable_7 ,ltriang
    ,lv ,lvr2plot ,lvr2plot_7 ,m ,ma ,mac ,macr ,macro ,makecns ,man
    ,manova ,manova_estat ,manova_p ,manovatest ,mantel ,mark ,markin
    ,markout ,marksample ,mat ,mat_capp ,mat_order ,mat_put_rr ,mat_rapp
    ,mata ,mata_clear ,mata_describe ,mata_drop ,mata_matdescribe
    ,mata_matsave ,mata_matuse ,mata_memory ,mata_mlib ,mata_mosave
    ,mata_rename ,mata_which ,matalabel ,matcproc ,matlist ,matname
    ,matr ,matri ,matrix ,matrix_input__dlg ,matstrik ,mcc ,mcci ,md0_
    ,md1_ ,md1debug_ ,md2_ ,md2debug_ ,mds ,mds_estat ,mds_p ,mdsconfig
    ,mdslong ,mdsmat ,mdsshepard ,mdytoe ,mdytof ,me_derd ,mean ,means
    ,median ,memory ,memsize ,meqparse ,mer ,merg ,merge ,mfp ,mfx
    ,mhelp ,mhodds ,minbound ,mixed_ll ,mixed_ll_reparm ,mkassert
    ,mkdir ,mkmat ,mkspline ,ml ,ml_5 ml_adjs ,ml_bhhhs ,ml_c_d
    ,ml_check ,ml_clear ,ml_cnt ,ml_debug ,ml_defd ,ml_e0 ml_e0_bfgs
    ,ml_e0_cycle ,ml_e0_dfp ,ml_e0i ,ml_e1 ,ml_e1_bfgs ,ml_e1_bhhh
    ,ml_e1_cycle ,ml_e1_dfp ,ml_e2 ,ml_e2_cycle ,ml_ebfg0 ,ml_ebfr0
    ,ml_ebfr1 ml_ebh0q ,ml_ebhh0 ,ml_ebhr0 ,ml_ebr0i ,ml_ecr0i ,ml_edfp0
    ,ml_edfr0 ,ml_edfr1 ml_edr0i ,ml_eds ,ml_eer0i ,ml_egr0i ,ml_elf
    ,ml_elf_bfgs ,ml_elf_bhhh ,ml_elf_cycle ,ml_elf_dfp ,ml_elfi
    ,ml_elfs ,ml_enr0i ,ml_enrr0 ,ml_erdu0 ml_erdu0_bfgs ,ml_erdu0_bhhh
    ,ml_erdu0_bhhhq ,ml_erdu0_cycle ,ml_erdu0_dfp ,ml_erdu0_nrbfgs
    ,ml_exde ,ml_footnote ,ml_geqnr ,ml_grad0 ,ml_graph ,ml_hbhhh
    ,ml_hd0 ,ml_hold ,ml_init ,ml_inv ,ml_log ,ml_max ,ml_mlout
    ,ml_mlout_8 ,ml_model ,ml_nb0 ,ml_opt ,ml_p ,ml_plot ,ml_query
    ,ml_rdgrd ,ml_repor ,ml_s_e ,ml_score ,ml_searc ,ml_technique
    ,ml_unhold ,mleval ,mlf_ ,mlmatbysum ,mlmatsum ,mlog ,mlogi ,mlogit
    ,mlogit_footnote ,mlogit_p ,mlopts ,mlsum ,mlvecsum ,mnl0_ ,mor
    ,more ,mov ,move ,mprobit ,mprobit_lf ,mprobit_p ,mrdu0_ ,mrdu1_
    ,mvdecode ,mvencode ,mvreg ,mvreg_estat ,n ,nbreg ,nbreg_al
    ,nbreg_lf ,nbreg_p ,nbreg_sw ,nestreg ,net ,newey ,newey_7 ,newey_p
    ,news ,nl ,nl_7 ,nl_9 ,nl_9_p ,nl_p ,nl_p_7 nlcom ,nlcom_p ,nlexp2
    ,nlexp2_7 ,nlexp2a ,nlexp2a_7 ,nlexp3 ,nlexp3_7 ,nlgom3 nlgom3_7
    ,nlgom4 ,nlgom4_7 ,nlinit ,nllog3 ,nllog3_7 ,nllog4 ,nllog4_7
    ,nlog_rd ,nlogit ,nlogit_p ,nlogitgen ,nlogittree ,nlpred ,no
    ,nobreak ,noi ,nois ,noisi ,noisil ,noisily ,note ,notes ,notes_dlg
    ,nptrend ,numlabel ,numlist ,odbc ,old_ver ,olo ,olog ,ologi
    ,ologi_sw ,ologit ,ologit_p ,ologitp ,on ,one ,onew ,onewa ,oneway
    ,op_colnm ,op_comp ,op_diff ,op_inv ,op_str ,opr ,opro ,oprob
    ,oprob_sw ,oprobi ,oprobi_p ,oprobit ,oprobitp ,opts_exclusive
    ,order ,orthog ,orthpoly ,ou ,out ,outf ,outfi ,outfil ,outfile
    ,outs ,outsh ,outshe ,outshee ,outsheet ,ovtest ,pac ,pac_7 ,palette
    ,parse ,parse_dissim ,pause ,pca ,pca_8 pca_display ,pca_estat
    ,pca_p ,pca_rotate ,pcamat ,pchart ,pchart_7 ,pchi ,pchi_7 ,pcorr
    ,pctile ,pentium ,pergram ,pergram_7 ,permute ,permute_8 ,personal
    ,peto_st ,pkcollapse ,pkcross ,pkequiv ,pkexamine ,pkexamine_7
    ,pkshape ,pksumm ,pksumm_7 ,pl ,plo ,plot ,plugin ,pnorm ,pnorm_7
    ,poisgof ,poiss_lf ,poiss_sw ,poisso_p ,poisson ,poisson_estat
    ,post ,postclose ,postfile ,postutil ,pperron ,pr ,prais ,prais_e
    ,prais_e2 ,prais_p ,predict ,predictnl ,preserve ,print ,pro ,prob
    ,probi ,probit ,probit_estat ,probit_p ,proc_time ,procoverlay
    ,procrustes ,procrustes_estat ,procrustes_p ,profiler ,prog ,progr
    ,progra ,program ,prop ,proportion ,prtest ,prtesti ,pwcorr ,pwd
    ,q ,s ,qby ,qbys ,qchi ,qchi_7 ,qladder ,qladder_7 ,qnorm ,qnorm_7
    ,qqplot ,qqplot_7 ,qreg ,qreg_c ,qreg_p ,qreg_sw ,qu ,quadchk
    ,quantile ,quantile_7 ,que ,quer ,query ,range ,ranksum ,ratio
    ,rchart ,rchart_7 ,rcof ,recast ,reclink ,recode ,reg ,reg3
    ,reg3_p ,regdw ,regr ,regre ,regre_p2 ,regres ,regres_p ,regress
    ,regress_estat ,regriv_p ,remap ,ren ,rena ,renam ,rename ,renpfix
    ,repeat ,replace ,report ,reshape ,restore ,ret ,retu ,retur ,return
    ,rm ,rmdir ,robvar ,roccomp ,roccomp_7 ,roccomp_8 ,rocf_lf ,rocfit
    ,rocfit_8 ,rocgold ,rocplot ,rocplot_7 ,roctab ,roctab_7 ,rolling
    ,rologit ,rologit_p ,rot ,rota ,rotat ,rotate ,rotatemat ,rreg
    ,rreg_p ,ru ,run ,runtest ,rvfplot ,rvfplot_7 ,rvpplot ,rvpplot_7
    ,sa ,safesum ,sample ,sampsi ,sav ,save ,savedresults ,saveold ,sc
    ,sca ,scal ,scala ,scalar ,scatter ,scm_mine ,sco ,scob_lf ,scob_p
    ,scobi_sw ,scobit ,scor ,score ,scoreplot ,scoreplot_help ,scree
    ,screeplot ,screeplot_help ,sdtest ,sdtesti ,se ,search ,separate
    ,seperate ,serrbar ,serrbar_7 ,serset ,set ,set_defaults ,sfrancia
    ,sh ,she ,shel ,shell ,shewhart ,shewhart_7 ,signestimationsample
    ,signrank ,signtest ,simul ,simul_7 simulate ,simulate_8 ,sktest
    ,sleep ,slogit ,slogit_d2 ,slogit_p ,smooth ,snapspan ,so ,sor
    ,sort ,spearman ,spikeplot ,spikeplot_7 ,spikeplt ,spline_x ,split
    ,sqreg ,sqreg_p ,sret ,sretu ,sretur ,sreturn ,ssc ,st ,st_ct ,st_hc
    ,st_hcd ,st_hcd_sh ,st_is ,st_issys ,st_note ,st_promo ,st_set
    ,st_show ,st_smpl ,st_subid ,stack ,statsby ,statsby_8 ,stbase
    ,stci ,stci_7 ,stcox ,stcox_estat ,stcox_fr ,stcox_fr_ll ,stcox_p
    ,stcox_sw ,stcoxkm ,stcoxkm_7 ,stcstat ,stcurv ,stcurve ,stcurve_7
    ,stdes ,stem ,stepwise ,stereg ,stfill ,stgen ,stir ,stjoin ,stmc
    ,stmh ,stphplot ,stphplot_7 ,stphtest ,stphtest_7 ,stptime ,strate
    ,strate_7 ,streg ,streg_sw ,streset ,sts ,sts_7 ,stset ,stsplit
    ,stsum ,sttocc ,sttoct ,stvary ,stweib ,su ,suest ,suest_8 ,sum
    ,summ ,summa ,summar ,summari ,summariz ,summarize ,sunflower
    ,sureg ,survcurv ,survsum ,svar ,svar_p ,svmat ,svy ,svy_disp
    ,svy_dreg ,svy_est ,svy_est_7 ,svy_estat ,svy_get ,svy_gnbreg_p
    ,svy_head ,svy_header ,svy_heckman_p ,svy_heckprob_p ,svy_intreg_p
    ,svy_ivreg_p ,svy_logistic_p ,svy_logit_p ,svy_mlogit_p ,svy_nbreg_p
    ,svy_ologit_p ,svy_oprobit_p ,svy_poisson_p ,svy_probit_p
    ,svy_regress_p ,svy_sub ,svy_sub_7 ,svy_x ,svy_x_7 ,svy_x_p ,svydes
    ,svydes_8 ,svygen ,svygnbreg ,svyheckman ,svyheckprob ,svyintreg
    ,svyintreg_7 ,svyintrg ,svyivreg ,svylc ,svylog_p ,svylogit
    ,svymarkout ,svymarkout_8 ,svymean ,svymlog ,svymlogit ,svynbreg
    ,svyolog ,svyologit ,svyoprob ,svyoprobit ,svyopts ,svypois
    ,svypois_7 svypoisson ,svyprobit ,svyprobt ,svyprop ,svyprop_7
    ,svyratio ,svyreg ,svyreg_p ,svyregress ,svyset ,svyset_7 ,svyset_8
    ,svytab ,svytab_7 ,svytest ,svytotal ,sw ,sw_8 ,swcnreg ,swcox
    ,swereg ,swilk ,swlogis ,swlogit ,swologit ,swoprbt ,swpois
    ,swprobit ,swqreg ,swtobit ,swweib ,symmetry ,symmi ,symplot
    ,symplot_7 syntax ,sysdescribe ,sysdir ,sysuse ,szroeter ,ta ,tab
    ,tab1 ,tab2 ,tab_or ,tabd ,tabdi ,tabdis ,tabdisp ,tabi ,table
    ,tabodds ,tabodds_7 ,tabstat ,tabu ,tabul ,tabula ,tabulat ,tabulate
    ,te ,tempfile ,tempname ,tempvar ,tes ,test ,testnl ,testparm
    ,teststd ,tetrachoric ,time_it ,timer ,tis ,tob ,tobi ,tobit
    ,tobit_p ,tobit_sw ,token ,tokeni ,tokeniz ,tokenize ,tostring
    ,total ,translate ,translator ,transmap ,treat_ll ,treatr_p
    ,treatreg ,trim ,trnb_cons ,trnb_mean ,trpoiss_d2 ,trunc_ll
    ,truncr_p ,truncreg ,tsappend ,tset ,tsfill ,tsline ,tsline_ex
    ,tsreport ,tsrevar ,tsrline ,tsset ,tssmooth ,tsunab ,ttest
    ,ttesti ,tut_chk ,tut_wait ,tutorial ,tw ,tware_st ,two ,twoway
    ,twoway__fpfit_serset ,twoway__function_gen ,twoway__histogram_gen
    ,twoway__ipoint_serset ,twoway__ipoints_serset ,twoway__kdensity_gen
    ,twoway__lfit_serset ,twoway__normgen_gen ,twoway__pci_serset
    ,twoway__qfit_serset ,twoway__scatteri_serset ,twoway__sunflower_gen
    ,twoway_ksm_serset ,ty ,typ ,type ,typeof ,u ,unab ,unabbrev
    ,unabcmd ,update ,us ,use ,uselabel ,var ,var_mkcompanion
    ,var_p ,varbasic ,varfcast ,vargranger ,varirf ,varirf_add
    ,varirf_cgraph ,varirf_create ,varirf_ctable ,varirf_describe
    ,varirf_dir ,varirf_drop ,varirf_erase ,varirf_graph ,varirf_ograph
    ,varirf_rename ,varirf_set ,varirf_table ,varlist ,varlmar
    ,varnorm ,varsoc ,varstable ,varstable_w ,varstable_w2 ,varwle
    ,vce ,vec ,vec_fevd ,vec_mkphi ,vec_p ,vec_p_w ,vecirf_create
    ,veclmar ,veclmar_w ,vecnorm ,vecnorm_w ,vecrank ,vecstable
    ,verinst ,vers ,versi ,versio ,version ,view ,viewsource ,vif
    ,vwls ,wdatetof ,webdescribe ,webseek ,webuse ,weib1_lf ,weib2_lf
    ,weib_lf ,weib_lf0 weibhet_glf ,weibhet_glf_sh ,weibhet_glfa
    ,weibhet_glfa_sh ,weibhet_gp ,weibhet_ilf ,weibhet_ilf_sh
    ,weibhet_ilfa ,weibhet_ilfa_sh ,weibhet_ip ,weibu_sw ,weibul_p
    ,weibull ,weibull_c ,weibull_s ,weibullhet ,wh ,whelp ,whi ,which
    ,whil ,while ,wilc_st ,wilcoxon ,win ,wind ,windo ,window ,winexec
    ,wntestb ,wntestb_7 ,wntestq ,xchart ,xchart_7 ,xcorr ,xcorr_7 ,xi
    ,xi_6 ,xmlsav ,xmlsave ,xmluse ,xpose ,xsh ,xshe ,xshel ,xshell
    ,xt_iis ,xt_tis ,xtab_p ,xtabond ,xtbin_p ,xtclog ,xtcloglog
    ,xtcloglog_8 ,xtcloglog_d2 ,xtcloglog_pa_p ,xtcloglog_re_p ,xtcnt_p
    ,xtcorr ,xtdata ,xtdes ,xtfront_p ,xtfrontier ,xtgee ,xtgee_elink
    ,xtgee_estat ,xtgee_makeivar ,xtgee_p ,xtgee_plink ,xtgls ,xtgls_p
    ,xthaus ,xthausman ,xtht_p ,xthtaylor ,xtile ,xtint_p ,xtintreg
    ,xtintreg_8 ,xtintreg_d2 xtintreg_p ,xtivp_1 ,xtivp_2 ,xtivreg
    ,xtline ,xtline_ex ,xtlogit ,xtlogit_8 xtlogit_d2 ,xtlogit_fe_p
    ,xtlogit_pa_p ,xtlogit_re_p ,xtmixed ,xtmixed_estat ,xtmixed_p
    ,xtnb_fe ,xtnb_lf ,xtnbreg ,xtnbreg_pa_p ,xtnbreg_refe_p ,xtpcse
    ,xtpcse_p ,xtpois ,xtpoisson ,xtpoisson_d2 ,xtpoisson_pa_p
    ,xtpoisson_refe_p ,xtpred ,xtprobit ,xtprobit_8 ,xtprobit_d2
    ,xtprobit_re_p ,xtps_fe ,xtps_lf ,xtps_ren ,xtps_ren_8 ,xtrar_p
    ,xtrc ,xtrc_p ,xtrchh ,xtrefe_p ,xtreg ,xtreg_be ,xtreg_fe
    ,xtreg_ml ,xtreg_pa_p ,xtreg_re ,xtregar ,xtrere_p ,xtset
    ,xtsf_ll ,xtsf_llti ,xtsum ,xttab ,xttest0 ,xttobit ,xttobit_8
    ,xttobit_p ,xttrans ,yx ,yxview__barlike_draw ,yxview_area_draw
    ,yxview_bar_draw ,yxview_dot_draw ,yxview_dropline_draw
    ,yxview_function_draw ,yxview_iarrow_draw ,yxview_ilabels_draw
    ,yxview_normal_draw ,yxview_pcarrow_draw ,yxview_pcbarrow_draw
    ,yxview_pccapsym_draw ,yxview_pcscatter_draw ,yxview_pcspike_draw
    ,yxview_rarea_draw ,yxview_rbar_draw ,yxview_rbarm_draw
    ,yxview_rcap_draw ,yxview_rcapsym_draw ,yxview_rconnected_draw
    ,yxview_rline_draw ,yxview_rscatter_draw ,yxview_rspike_draw
    ,yxview_spike_draw ,yxview_sunflower_draw ,zap_s ,zinb ,zinb_llf
    ,zinb_plf ,zip ,zip_llf ,zip_p ,zip_plf ,zt_ct_5 ,zt_hc_5 ,zt_hcd_5
    ,zt_is_5 ,zt_iss_5 ,zt_sho_5 zt_smp_5 ,ztbase_5 ,ztcox_5 ,ztdes_5
    ,ztereg_5 ,ztfill_5 ,ztgen_5 ,ztir_5 ztjoin_5 ,ztnb ,ztnb_p ,ztp
    ,ztp_p ,zts_5 ,ztset_5 ,ztspli_5 ,ztsum_5 ,zttoct_5 ztvary_5
    ,ztweib_5
  },
  %
  % Built-in functions
  morekeywords=[3]{
    Cdhms ,Chms ,Clock ,Cmdyhms ,Cofc ,Cofd ,F ,Fden ,Ftail ,I ,J
    ,_caller ,abbrev ,abs ,acos ,acosh ,asin ,asinh ,atan ,atan2
    ,atanh ,autocode ,betaden ,binomial ,binomialp ,binomialtail
    ,binormal ,bofd ,byteorder ,c ,ceil ,char ,chi2 ,chi2den ,chi2tail
    ,cholesky ,chop ,clip ,clock ,cloglog ,cofC ,cofd ,colnumb ,colsof
    ,comb ,cond ,corr ,cos ,cosh ,d ,daily ,date ,day ,det ,dgammapda
    ,dgammapdada ,dgammapdadx ,dgammapdx ,dgammapdxdx ,dhms ,diag
    ,diag0cnt ,digamma ,dofC ,dofb ,dofc ,dofh ,dofm ,dofq ,dofw ,dofy
    ,dow ,doy ,dunnettprob ,e ,el ,epsdouble ,epsfloat ,exp ,fileexists
    ,fileread ,filereaderror ,filewrite ,float ,floor ,fmtwidth
    ,gammaden ,gammap ,gammaptail ,get ,group ,h ,hadamard ,halfyear
    ,halfyearly ,has_eprop ,hh ,hhC ,hms ,hofd ,hours ,hypergeometric
    ,hypergeometricp ,ibeta ,ibetatail ,index ,indexnot ,inlist
    ,inrange ,int ,inv ,invF ,invFtail ,invbinomial ,invbinomialtail
    ,invchi2 ,invchi2tail ,invcloglog ,invdunnettprob ,invgammap
    ,invgammaptail ,invibeta ,invibetatail ,invlogit ,invnFtail
    ,invnbinomial ,invnbinomialtail ,invnchi2 ,invnchi2tail ,invnibeta
    ,invnorm ,invnormal ,invnttail ,invpoisson ,invpoissontail ,invsym
    ,invt ,invttail ,invtukeyprob ,irecode ,issym ,issymmetric ,itrim
    ,length ,ln ,lnfact ,lnfactorial ,lngamma ,lnnormal ,lnnormalden
    ,log ,log10 ,logit ,lower ,ltrim ,m ,match ,matmissing ,matrix
    ,matuniform ,max ,maxbyte ,maxdouble ,maxfloat ,maxint ,maxlong ,mdy
    ,mdyhms ,mi ,mi ,min ,minbyte ,mindouble ,minfloat ,minint ,minlong
    ,minutes ,missing ,mm ,mmC ,mod ,mofd ,month ,monthly ,mreldif
    ,msofhours ,msofminutes ,msofseconds ,nF ,nFden ,nFtail ,nbetaden
    ,nbinomial ,nbinomialp ,nbinomialtail ,nchi2 ,nchi2den ,nchi2tail
    ,nibeta ,norm ,normal ,normalden ,normd ,npnF ,npnchi2 ,npnt ,nt
    ,ntden ,nttail ,nullmat ,plural ,poisson ,poissonp ,poissontail
    ,proper ,q ,qofd ,quarter ,quarterly ,r ,rbeta ,rbinomial ,rchi2
    real ,recode ,regexm ,regexr ,regexs ,reldif ,replay ,return
    ,reverse ,rgamma ,rhypergeometric ,rnbinomial ,rnormal ,round
    ,rownumb ,rowsof ,rpoisson ,rt ,rtrim ,runiform ,s ,scalar ,seconds
    ,sign ,sin ,sinh ,smallestdouble ,soundex ,soundex_nara ,sqrt ,ss
    ,ssC ,strcat ,strdup ,string ,strlen ,strlower ,strltrim ,strmatch
    ,strofreal ,strpos ,strproper ,strreverse ,strrtrim ,strtoname
    ,strtrim ,strupper ,subinstr ,subinword ,substr ,sum ,sweep ,syminv
    ,t ,tC ,tan ,tanh ,tc ,td ,tden ,th ,tin ,tm ,tq ,trace ,trigamma
    ,trim ,trunc ,ttail ,tukeyprob ,tw ,twithin ,uniform ,upper ,vec
    ,vecdiag ,w ,week ,weekly ,wofd ,word ,wordcount ,year ,yearly
    ,yh ,ym ,yofd ,yq ,yw
  },
  %
  % Numbers
  morekeywords=[4]{
    0 ,1 ,2 ,3 ,4 ,5 ,6 ,7 ,8 ,9
  },
}

% ---------------------------------------------------------------------
% Stata editor style

\providecommand{\textcolordummy}[2]{#2}
\lstalias{Stata}{stata}
\lstdefinestyle{stata-editor}{
    language=stata,
    %
    % Global variables
    keywordstyle={\bfseries\color{spRed}},
    %
    % Add-ons system commands
    keywordstyle=[2]{\bfseries\color{NavyBlue}},
    %
    % Built-in functions
    keywordstyle=[3]{\color{blue}},
    %
    % Numbers
    keywordstyle=[4]{\color{blue}},
    %
    % User macros (variables)
    keywordstyle = [9]{\bfseries\color{LightSteelBlue}\let\textcolor\textcolordummy},
    %
    % Strings and comments
    stringstyle  = \color{Maroon},
    commentstyle = \color{Green}\slshape,
}

% ---------------------------------------------------------------------
% Suggested settings

% \lstset{
%   basicstyle        = \setmonofont{DejaVu Sans Mono}\footnotesize\ttfamily,
%   tabsize           = 4,      % Tab size
%   showstringspaces  = false,  % Don't underline spaces in strings
%   showspaces        = false,  % Don't underline spaces
%   breaklines        = true,   % Automatic line breaking
%   breakatwhitespace = true,   % Breaks only at white space.
%   lineskip          = 1.5pt,  % Sparing between lines of code
%   commentstyle      = \color{black!50}\itshape \let\textcolor\textcolordummy,
% } to your preamble

% Syntax from
% - https://github.com/isagalaev/highlight.js/blob/master/src/languages/stata.js
% - https://github.com/jpitblado/vim-stata/blob/master/syntax/stata.vim
% - http://fmwww.bc.edu/RePEc/bocode/s/synlightlist.ado

\RequirePackage{listings}
\RequirePackage{color}
%\RequirePackage[svgnames]{xcolor}
%\definecolor{spRed}{HTML}{BE646C}

% ---------------------------------------------------------------------
% Stata language definition

\lstdefinelanguage{stata}{
  sensitive=true,
  %
  % Macros, global and local
  alsoletter={\{\}0123456789},
  keywordsprefix=\$,
  morecomment=[n][keywordstyle9]{`}{'},
  morekeywords={},
  %
  % Comments
  morecomment=[f][\color{Green}\slshape][0]*,
  morecomment=[l]{//},
  morecomment=[s]{/*}{*/},
  %
  % Strings
  morecomment=[n][\color{Maroon}]{`"}{"'},
  morestring=[b]",
  %
  % Add-ons and system Commands
  morekeywords=[2]{
    if ,else ,in ,foreach ,for ,forv ,forva ,forval ,forvalu ,forvalue
    ,forvalues ,by ,bys ,bysort ,xi ,quietly ,qui ,capture ,about
    ,ac ,ac_7 ,acprplot ,acprplot_7 adjust ,ado ,adopath ,adoupdate
    ,alpha ,ameans ,an ,ano ,anov ,anova ,anova_estat ,anova_terms
    ,anovadef ,aorder ,ap ,app ,appe ,appen ,append ,arch ,arch_dr
    ,arch_estat ,arch_p ,archlm ,areg ,areg_p ,args ,arima ,arima_dr
    ,arima_estat ,arima_p ,as ,asmprobit ,asmprobit_estat ,asmprobit_lf
    ,asmprobit_mfx__dlg ,asmprobit_p ,ass ,asse ,asser ,assert ,avplot
    ,avplot_7 ,avplots ,avplots_7 bcskew0 ,bgodfrey ,binreg ,bip0_lf
    ,biplot ,bipp_lf ,bipr_lf ,bipr_p ,biprobit ,bitest ,bitesti
    ,bitowt ,blogit ,bmemsize ,boot ,bootsamp ,bootstrap ,bootstrap_8
    ,boxco_l ,boxco_p ,boxcox ,boxcox_6 ,boxcox_p ,bprobit ,br ,break
    ,brier ,bro ,brow ,brows ,browse ,brr ,brrstat ,bs ,bs_7 ,bsampl_w
    ,bsample ,bsample_7 ,bsqreg ,bstat ,bstat_7 ,bstat_8 ,bstrap
    ,bstrap_7 ,ca ,ca_estat ,ca_p ,cabiplot ,camat ,canon ,canon_8
    ,canon_8_p ,canon_estat ,canon_p ,cap ,caprojection ,capt ,captu
    ,captur ,capture ,cat ,cc ,cchart ,cchart_7 ,cci ,cd ,censobs_table
    ,centile ,cf ,char ,chdir ,checkdlgfiles ,checkestimationsample
    ,checkhlpfiles ,checksum ,chelp ,ci ,cii ,cl ,class ,classutil
    ,clear ,cli ,clis ,clist ,clo ,clog ,clog_lf ,clog_p ,clogi
    ,clogi_sw ,clogit ,clogit_lf ,clogit_p ,clogitp ,clogl_sw ,cloglog
    ,clonevar ,clslistarray ,cluster ,cluster_measures ,cluster_stop
    ,cluster_tree ,cluster_tree_8 ,clustermat ,cmdlog ,cnr ,cnre
    ,cnreg ,cnreg_p ,cnreg_sw ,cnsreg ,codebook ,collaps4 ,collapse
    ,colormult_nb ,colormult_nw ,compare ,compress ,conf ,confi
    ,confir ,confirm ,conren ,cons ,const ,constr ,constra ,constrai
    ,constrain ,constraint ,continue ,contract ,copy ,copyright
    ,copysource ,cor ,corc ,corr ,corr2data ,corr_anti ,corr_kmo
    ,corr_smc ,corre ,correl ,correla ,correlat ,correlate ,corrgram
    ,cou ,coun ,count ,cox ,cox_p ,cox_sw ,coxbase ,coxhaz ,coxvar
    ,cprplot ,cprplot_7 ,crc ,cret ,cretu ,cretur ,creturn ,cross ,cs
    ,cscript ,cscript_log ,csi ,ct ,ct_is ,ctset ,ctst_5 ,ctst_st
    ,cttost ,cumsp ,cumsp_7 ,cumul ,cusum ,cusum_7 ,cutil ,d ,datasig
    ,datasign ,datasigna ,datasignat ,datasignatu ,datasignatur
    ,datasignature ,datetof ,db ,dbeta ,de ,dec ,deco ,decod ,decode
    ,deff ,des ,desc ,descr ,descri ,describ ,describe ,destring
    ,dfbeta ,dfgls ,dfuller ,di ,di_g ,dir ,dirstats ,dis ,discard
    ,disp ,disp_res ,disp_s ,displ ,displa ,display ,distinct ,do
    ,doe ,doed ,doedi ,doedit ,dotplot ,dotplot_7 ,dprobit ,drawnorm
    ,drop ,ds ,ds_util ,dstdize ,duplicates ,durbina ,dwstat ,dydx ,e
    ,ed ,edi ,edit ,egen ,eivreg ,emdef ,en ,enc ,enco ,encod ,encode
    ,eq ,erase ,ereg ,ereg_lf ,ereg_p ,ereg_sw ,ereghet ,ereghet_glf
    ,ereghet_glf_sh ,ereghet_gp ,ereghet_ilf ,ereghet_ilf_sh ,ereghet_ip
    ,eret ,eretu ,eretur ,ereturn ,err ,erro ,error ,est ,est_cfexist
    ,est_cfname ,est_clickable ,est_expand ,est_hold ,est_table
    ,est_unhold ,est_unholdok ,estat ,estat_default ,estat_summ
    ,estat_vce_only ,esti ,estimates ,etodow ,etof ,etomdy ,ex ,exi
    ,exit ,expand ,expandcl ,fac ,fact ,facto ,factor ,factor_estat
    ,factor_p ,factor_pca_rotated ,factor_rotate ,factormat ,fcast
    ,fcast_compute ,fcast_graph ,fdades ,fdadesc ,fdadescr ,fdadescri
    ,fdadescrib ,fdadescribe ,fdasav ,fdasave ,fdause ,fh_st ,file
    ,open ,file ,read ,file ,close ,file ,filefilter ,fillin
    ,find_hlp_file ,findfile ,findit ,findit_7 ,fit ,fl ,fli ,flis
    ,flist ,for5_0 ,form ,forma ,format ,fpredict ,frac_154 ,frac_adj
    ,frac_chk ,frac_cox ,frac_ddp ,frac_dis ,frac_dv ,frac_in ,frac_mun
    ,frac_pp ,frac_pq ,frac_pv ,frac_wgt ,frac_xo ,fracgen ,fracplot
    ,fracplot_7 ,fracpoly ,fracpred ,fron_ex ,fron_hn ,fron_p ,fron_tn
    ,fron_tn2 ,frontier ,ftodate ,ftoe ,ftomdy ,ftowdate ,g ,gamhet_glf
    ,gamhet_gp ,gamhet_ilf ,gamhet_ip ,gamma ,gamma_d2 ,gamma_p
    ,gamma_sw ,gammahet ,gdi_hexagon ,gdi_spokes ,ge ,gen ,gene ,gener
    ,genera ,generat ,generate ,genrank ,genstd ,genvmean ,gettoken
    ,gl ,gladder ,gladder_7 ,glim_l01 ,glim_l02 glim_l03 ,glim_l04
    ,glim_l05 ,glim_l06 ,glim_l07 ,glim_l08 ,glim_l09 ,glim_l10 glim_l11
    ,glim_l12 ,glim_lf ,glim_mu ,glim_nw1 ,glim_nw2 ,glim_nw3 ,glim_p
    ,glim_v1 ,glim_v2 ,glim_v3 ,glim_v4 ,glim_v5 ,glim_v6 ,glim_v7 ,glm
    ,glm_6 glm_p ,glm_sw ,glmpred ,glo ,glob ,globa ,global ,glogit
    ,glogit_8 ,glogit_p ,gmeans ,gnbre_lf ,gnbreg ,gnbreg_5 ,gnbreg_p
    ,gomp_lf ,gompe_sw ,gomper_p ,gompertz ,gompertzhet ,gomphet_glf
    ,gomphet_glf_sh ,gomphet_gp ,gomphet_ilf ,gomphet_ilf_sh ,gomphet_ip
    ,gphdot ,gphpen ,gphprint ,gprefs ,gprobi_p ,gprobit ,gprobit_8
    ,gr ,gr7 ,gr_copy ,gr_current ,gr_db ,gr_describe ,gr_dir ,gr_draw
    ,gr_draw_replay ,gr_drop ,gr_edit ,gr_editviewopts ,gr_example
    ,gr_example2 gr_export ,gr_print ,gr_qscheme ,gr_query ,gr_read
    ,gr_rename ,gr_replay ,gr_save ,gr_set ,gr_setscheme ,gr_table
    ,gr_undo ,gr_use ,graph ,graph7 grebar ,greigen ,greigen_7
    ,greigen_8 ,grmeanby ,grmeanby_7 ,gs_fileinfo ,gs_filetype
    ,gs_graphinfo ,gs_stat ,gsort ,gwood ,h ,hadimvo ,hareg ,hausman
    ,haver ,he ,heck_d2 ,heckma_p ,heckman ,heckp_lf ,heckpr_p ,heckprob
    ,hel ,help ,hereg ,hetpr_lf ,hetpr_p ,hetprob ,hettest ,hexdump
    ,hilite ,hist ,hist_7 histogram ,hlogit ,hlu ,hmeans ,hotel
    ,hotelling ,hprobit ,hreg ,hsearch ,icd9 ,icd9_ff ,icd9p ,iis
    ,impute ,imtest ,inbase ,include ,inf ,infi ,infil ,infile ,infix
    ,inp ,inpu ,input ,ins ,insheet ,insp ,inspe ,inspec ,inspect ,integ
    ,inten ,intreg ,intreg_7 ,intreg_p ,intrg2_ll ,intrg_ll ,intrg_ll2
    ,ipolate ,iqreg ,ir ,irf ,irf_create ,irfm ,iri ,is_svy ,is_svysum
    ,isid ,istdize ,ivprob_1_lf ,ivprob_lf ,ivprobit ,ivprobit_p ,ivreg
    ,ivreg_footnote ,ivtob_1_lf ,ivtob_lf ,ivtobit ,ivtobit_p ,jackknife
    ,jacknife ,jknife ,jknife_6 ,jknife_8 ,jkstat ,joinby ,kalarma1
    ,kap ,kap_3 ,kapmeier ,kappa ,kapwgt ,kdensity ,kdensity_7 keep
    ,ksm ,ksmirnov ,ktau ,kwallis ,l ,la ,lab ,labe ,label ,labelbook
    ,ladder ,levels ,levelsof ,leverage ,lfit ,lfit_p ,li ,lincom ,line
    ,linktest ,lis ,list ,lloghet_glf ,lloghet_glf_sh ,lloghet_gp
    ,lloghet_ilf ,lloghet_ilf_sh ,lloghet_ip ,llogi_sw ,llogis_p
    ,llogist ,llogistic ,llogistichet ,lnorm_lf ,lnorm_sw ,lnorma_p
    ,lnormal ,lnormalhet ,lnormhet_glf ,lnormhet_glf_sh ,lnormhet_gp
    ,lnormhet_ilf ,lnormhet_ilf_sh ,lnormhet_ip ,lnskew0 ,loadingplot
    ,loc ,loca ,local ,log ,logi ,logis_lf ,logistic ,logistic_p
    ,logit ,logit_estat ,logit_p ,loglogs ,logrank ,loneway ,lookfor
    ,lookup ,lowess ,lowess_7 ,lpredict ,lrecomp ,lroc ,lroc_7 ,lrtest
    ,ls ,lsens ,lsens_7 ,lsens_x ,lstat ,ltable ,ltable_7 ,ltriang
    ,lv ,lvr2plot ,lvr2plot_7 ,m ,ma ,mac ,macr ,macro ,makecns ,man
    ,manova ,manova_estat ,manova_p ,manovatest ,mantel ,mark ,markin
    ,markout ,marksample ,mat ,mat_capp ,mat_order ,mat_put_rr ,mat_rapp
    ,mata ,mata_clear ,mata_describe ,mata_drop ,mata_matdescribe
    ,mata_matsave ,mata_matuse ,mata_memory ,mata_mlib ,mata_mosave
    ,mata_rename ,mata_which ,matalabel ,matcproc ,matlist ,matname
    ,matr ,matri ,matrix ,matrix_input__dlg ,matstrik ,mcc ,mcci ,md0_
    ,md1_ ,md1debug_ ,md2_ ,md2debug_ ,mds ,mds_estat ,mds_p ,mdsconfig
    ,mdslong ,mdsmat ,mdsshepard ,mdytoe ,mdytof ,me_derd ,mean ,means
    ,median ,memory ,memsize ,meqparse ,mer ,merg ,merge ,mfp ,mfx
    ,mhelp ,mhodds ,minbound ,mixed_ll ,mixed_ll_reparm ,mkassert
    ,mkdir ,mkmat ,mkspline ,ml ,ml_5 ml_adjs ,ml_bhhhs ,ml_c_d
    ,ml_check ,ml_clear ,ml_cnt ,ml_debug ,ml_defd ,ml_e0 ml_e0_bfgs
    ,ml_e0_cycle ,ml_e0_dfp ,ml_e0i ,ml_e1 ,ml_e1_bfgs ,ml_e1_bhhh
    ,ml_e1_cycle ,ml_e1_dfp ,ml_e2 ,ml_e2_cycle ,ml_ebfg0 ,ml_ebfr0
    ,ml_ebfr1 ml_ebh0q ,ml_ebhh0 ,ml_ebhr0 ,ml_ebr0i ,ml_ecr0i ,ml_edfp0
    ,ml_edfr0 ,ml_edfr1 ml_edr0i ,ml_eds ,ml_eer0i ,ml_egr0i ,ml_elf
    ,ml_elf_bfgs ,ml_elf_bhhh ,ml_elf_cycle ,ml_elf_dfp ,ml_elfi
    ,ml_elfs ,ml_enr0i ,ml_enrr0 ,ml_erdu0 ml_erdu0_bfgs ,ml_erdu0_bhhh
    ,ml_erdu0_bhhhq ,ml_erdu0_cycle ,ml_erdu0_dfp ,ml_erdu0_nrbfgs
    ,ml_exde ,ml_footnote ,ml_geqnr ,ml_grad0 ,ml_graph ,ml_hbhhh
    ,ml_hd0 ,ml_hold ,ml_init ,ml_inv ,ml_log ,ml_max ,ml_mlout
    ,ml_mlout_8 ,ml_model ,ml_nb0 ,ml_opt ,ml_p ,ml_plot ,ml_query
    ,ml_rdgrd ,ml_repor ,ml_s_e ,ml_score ,ml_searc ,ml_technique
    ,ml_unhold ,mleval ,mlf_ ,mlmatbysum ,mlmatsum ,mlog ,mlogi ,mlogit
    ,mlogit_footnote ,mlogit_p ,mlopts ,mlsum ,mlvecsum ,mnl0_ ,mor
    ,more ,mov ,move ,mprobit ,mprobit_lf ,mprobit_p ,mrdu0_ ,mrdu1_
    ,mvdecode ,mvencode ,mvreg ,mvreg_estat ,n ,nbreg ,nbreg_al
    ,nbreg_lf ,nbreg_p ,nbreg_sw ,nestreg ,net ,newey ,newey_7 ,newey_p
    ,news ,nl ,nl_7 ,nl_9 ,nl_9_p ,nl_p ,nl_p_7 nlcom ,nlcom_p ,nlexp2
    ,nlexp2_7 ,nlexp2a ,nlexp2a_7 ,nlexp3 ,nlexp3_7 ,nlgom3 nlgom3_7
    ,nlgom4 ,nlgom4_7 ,nlinit ,nllog3 ,nllog3_7 ,nllog4 ,nllog4_7
    ,nlog_rd ,nlogit ,nlogit_p ,nlogitgen ,nlogittree ,nlpred ,no
    ,nobreak ,noi ,nois ,noisi ,noisil ,noisily ,note ,notes ,notes_dlg
    ,nptrend ,numlabel ,numlist ,odbc ,old_ver ,olo ,olog ,ologi
    ,ologi_sw ,ologit ,ologit_p ,ologitp ,on ,one ,onew ,onewa ,oneway
    ,op_colnm ,op_comp ,op_diff ,op_inv ,op_str ,opr ,opro ,oprob
    ,oprob_sw ,oprobi ,oprobi_p ,oprobit ,oprobitp ,opts_exclusive
    ,order ,orthog ,orthpoly ,ou ,out ,outf ,outfi ,outfil ,outfile
    ,outs ,outsh ,outshe ,outshee ,outsheet ,ovtest ,pac ,pac_7 ,palette
    ,parse ,parse_dissim ,pause ,pca ,pca_8 pca_display ,pca_estat
    ,pca_p ,pca_rotate ,pcamat ,pchart ,pchart_7 ,pchi ,pchi_7 ,pcorr
    ,pctile ,pentium ,pergram ,pergram_7 ,permute ,permute_8 ,personal
    ,peto_st ,pkcollapse ,pkcross ,pkequiv ,pkexamine ,pkexamine_7
    ,pkshape ,pksumm ,pksumm_7 ,pl ,plo ,plot ,plugin ,pnorm ,pnorm_7
    ,poisgof ,poiss_lf ,poiss_sw ,poisso_p ,poisson ,poisson_estat
    ,post ,postclose ,postfile ,postutil ,pperron ,pr ,prais ,prais_e
    ,prais_e2 ,prais_p ,predict ,predictnl ,preserve ,print ,pro ,prob
    ,probi ,probit ,probit_estat ,probit_p ,proc_time ,procoverlay
    ,procrustes ,procrustes_estat ,procrustes_p ,profiler ,prog ,progr
    ,progra ,program ,prop ,proportion ,prtest ,prtesti ,pwcorr ,pwd
    ,q ,s ,qby ,qbys ,qchi ,qchi_7 ,qladder ,qladder_7 ,qnorm ,qnorm_7
    ,qqplot ,qqplot_7 ,qreg ,qreg_c ,qreg_p ,qreg_sw ,qu ,quadchk
    ,quantile ,quantile_7 ,que ,quer ,query ,range ,ranksum ,ratio
    ,rchart ,rchart_7 ,rcof ,recast ,reclink ,recode ,reg ,reg3
    ,reg3_p ,regdw ,regr ,regre ,regre_p2 ,regres ,regres_p ,regress
    ,regress_estat ,regriv_p ,remap ,ren ,rena ,renam ,rename ,renpfix
    ,repeat ,replace ,report ,reshape ,restore ,ret ,retu ,retur ,return
    ,rm ,rmdir ,robvar ,roccomp ,roccomp_7 ,roccomp_8 ,rocf_lf ,rocfit
    ,rocfit_8 ,rocgold ,rocplot ,rocplot_7 ,roctab ,roctab_7 ,rolling
    ,rologit ,rologit_p ,rot ,rota ,rotat ,rotate ,rotatemat ,rreg
    ,rreg_p ,ru ,run ,runtest ,rvfplot ,rvfplot_7 ,rvpplot ,rvpplot_7
    ,sa ,safesum ,sample ,sampsi ,sav ,save ,savedresults ,saveold ,sc
    ,sca ,scal ,scala ,scalar ,scatter ,scm_mine ,sco ,scob_lf ,scob_p
    ,scobi_sw ,scobit ,scor ,score ,scoreplot ,scoreplot_help ,scree
    ,screeplot ,screeplot_help ,sdtest ,sdtesti ,se ,search ,separate
    ,seperate ,serrbar ,serrbar_7 ,serset ,set ,set_defaults ,sfrancia
    ,sh ,she ,shel ,shell ,shewhart ,shewhart_7 ,signestimationsample
    ,signrank ,signtest ,simul ,simul_7 simulate ,simulate_8 ,sktest
    ,sleep ,slogit ,slogit_d2 ,slogit_p ,smooth ,snapspan ,so ,sor
    ,sort ,spearman ,spikeplot ,spikeplot_7 ,spikeplt ,spline_x ,split
    ,sqreg ,sqreg_p ,sret ,sretu ,sretur ,sreturn ,ssc ,st ,st_ct ,st_hc
    ,st_hcd ,st_hcd_sh ,st_is ,st_issys ,st_note ,st_promo ,st_set
    ,st_show ,st_smpl ,st_subid ,stack ,statsby ,statsby_8 ,stbase
    ,stci ,stci_7 ,stcox ,stcox_estat ,stcox_fr ,stcox_fr_ll ,stcox_p
    ,stcox_sw ,stcoxkm ,stcoxkm_7 ,stcstat ,stcurv ,stcurve ,stcurve_7
    ,stdes ,stem ,stepwise ,stereg ,stfill ,stgen ,stir ,stjoin ,stmc
    ,stmh ,stphplot ,stphplot_7 ,stphtest ,stphtest_7 ,stptime ,strate
    ,strate_7 ,streg ,streg_sw ,streset ,sts ,sts_7 ,stset ,stsplit
    ,stsum ,sttocc ,sttoct ,stvary ,stweib ,su ,suest ,suest_8 ,sum
    ,summ ,summa ,summar ,summari ,summariz ,summarize ,sunflower
    ,sureg ,survcurv ,survsum ,svar ,svar_p ,svmat ,svy ,svy_disp
    ,svy_dreg ,svy_est ,svy_est_7 ,svy_estat ,svy_get ,svy_gnbreg_p
    ,svy_head ,svy_header ,svy_heckman_p ,svy_heckprob_p ,svy_intreg_p
    ,svy_ivreg_p ,svy_logistic_p ,svy_logit_p ,svy_mlogit_p ,svy_nbreg_p
    ,svy_ologit_p ,svy_oprobit_p ,svy_poisson_p ,svy_probit_p
    ,svy_regress_p ,svy_sub ,svy_sub_7 ,svy_x ,svy_x_7 ,svy_x_p ,svydes
    ,svydes_8 ,svygen ,svygnbreg ,svyheckman ,svyheckprob ,svyintreg
    ,svyintreg_7 ,svyintrg ,svyivreg ,svylc ,svylog_p ,svylogit
    ,svymarkout ,svymarkout_8 ,svymean ,svymlog ,svymlogit ,svynbreg
    ,svyolog ,svyologit ,svyoprob ,svyoprobit ,svyopts ,svypois
    ,svypois_7 svypoisson ,svyprobit ,svyprobt ,svyprop ,svyprop_7
    ,svyratio ,svyreg ,svyreg_p ,svyregress ,svyset ,svyset_7 ,svyset_8
    ,svytab ,svytab_7 ,svytest ,svytotal ,sw ,sw_8 ,swcnreg ,swcox
    ,swereg ,swilk ,swlogis ,swlogit ,swologit ,swoprbt ,swpois
    ,swprobit ,swqreg ,swtobit ,swweib ,symmetry ,symmi ,symplot
    ,symplot_7 syntax ,sysdescribe ,sysdir ,sysuse ,szroeter ,ta ,tab
    ,tab1 ,tab2 ,tab_or ,tabd ,tabdi ,tabdis ,tabdisp ,tabi ,table
    ,tabodds ,tabodds_7 ,tabstat ,tabu ,tabul ,tabula ,tabulat ,tabulate
    ,te ,tempfile ,tempname ,tempvar ,tes ,test ,testnl ,testparm
    ,teststd ,tetrachoric ,time_it ,timer ,tis ,tob ,tobi ,tobit
    ,tobit_p ,tobit_sw ,token ,tokeni ,tokeniz ,tokenize ,tostring
    ,total ,translate ,translator ,transmap ,treat_ll ,treatr_p
    ,treatreg ,trim ,trnb_cons ,trnb_mean ,trpoiss_d2 ,trunc_ll
    ,truncr_p ,truncreg ,tsappend ,tset ,tsfill ,tsline ,tsline_ex
    ,tsreport ,tsrevar ,tsrline ,tsset ,tssmooth ,tsunab ,ttest
    ,ttesti ,tut_chk ,tut_wait ,tutorial ,tw ,tware_st ,two ,twoway
    ,twoway__fpfit_serset ,twoway__function_gen ,twoway__histogram_gen
    ,twoway__ipoint_serset ,twoway__ipoints_serset ,twoway__kdensity_gen
    ,twoway__lfit_serset ,twoway__normgen_gen ,twoway__pci_serset
    ,twoway__qfit_serset ,twoway__scatteri_serset ,twoway__sunflower_gen
    ,twoway_ksm_serset ,ty ,typ ,type ,typeof ,u ,unab ,unabbrev
    ,unabcmd ,update ,us ,use ,uselabel ,var ,var_mkcompanion
    ,var_p ,varbasic ,varfcast ,vargranger ,varirf ,varirf_add
    ,varirf_cgraph ,varirf_create ,varirf_ctable ,varirf_describe
    ,varirf_dir ,varirf_drop ,varirf_erase ,varirf_graph ,varirf_ograph
    ,varirf_rename ,varirf_set ,varirf_table ,varlist ,varlmar
    ,varnorm ,varsoc ,varstable ,varstable_w ,varstable_w2 ,varwle
    ,vce ,vec ,vec_fevd ,vec_mkphi ,vec_p ,vec_p_w ,vecirf_create
    ,veclmar ,veclmar_w ,vecnorm ,vecnorm_w ,vecrank ,vecstable
    ,verinst ,vers ,versi ,versio ,version ,view ,viewsource ,vif
    ,vwls ,wdatetof ,webdescribe ,webseek ,webuse ,weib1_lf ,weib2_lf
    ,weib_lf ,weib_lf0 weibhet_glf ,weibhet_glf_sh ,weibhet_glfa
    ,weibhet_glfa_sh ,weibhet_gp ,weibhet_ilf ,weibhet_ilf_sh
    ,weibhet_ilfa ,weibhet_ilfa_sh ,weibhet_ip ,weibu_sw ,weibul_p
    ,weibull ,weibull_c ,weibull_s ,weibullhet ,wh ,whelp ,whi ,which
    ,whil ,while ,wilc_st ,wilcoxon ,win ,wind ,windo ,window ,winexec
    ,wntestb ,wntestb_7 ,wntestq ,xchart ,xchart_7 ,xcorr ,xcorr_7 ,xi
    ,xi_6 ,xmlsav ,xmlsave ,xmluse ,xpose ,xsh ,xshe ,xshel ,xshell
    ,xt_iis ,xt_tis ,xtab_p ,xtabond ,xtbin_p ,xtclog ,xtcloglog
    ,xtcloglog_8 ,xtcloglog_d2 ,xtcloglog_pa_p ,xtcloglog_re_p ,xtcnt_p
    ,xtcorr ,xtdata ,xtdes ,xtfront_p ,xtfrontier ,xtgee ,xtgee_elink
    ,xtgee_estat ,xtgee_makeivar ,xtgee_p ,xtgee_plink ,xtgls ,xtgls_p
    ,xthaus ,xthausman ,xtht_p ,xthtaylor ,xtile ,xtint_p ,xtintreg
    ,xtintreg_8 ,xtintreg_d2 xtintreg_p ,xtivp_1 ,xtivp_2 ,xtivreg
    ,xtline ,xtline_ex ,xtlogit ,xtlogit_8 xtlogit_d2 ,xtlogit_fe_p
    ,xtlogit_pa_p ,xtlogit_re_p ,xtmixed ,xtmixed_estat ,xtmixed_p
    ,xtnb_fe ,xtnb_lf ,xtnbreg ,xtnbreg_pa_p ,xtnbreg_refe_p ,xtpcse
    ,xtpcse_p ,xtpois ,xtpoisson ,xtpoisson_d2 ,xtpoisson_pa_p
    ,xtpoisson_refe_p ,xtpred ,xtprobit ,xtprobit_8 ,xtprobit_d2
    ,xtprobit_re_p ,xtps_fe ,xtps_lf ,xtps_ren ,xtps_ren_8 ,xtrar_p
    ,xtrc ,xtrc_p ,xtrchh ,xtrefe_p ,xtreg ,xtreg_be ,xtreg_fe
    ,xtreg_ml ,xtreg_pa_p ,xtreg_re ,xtregar ,xtrere_p ,xtset
    ,xtsf_ll ,xtsf_llti ,xtsum ,xttab ,xttest0 ,xttobit ,xttobit_8
    ,xttobit_p ,xttrans ,yx ,yxview__barlike_draw ,yxview_area_draw
    ,yxview_bar_draw ,yxview_dot_draw ,yxview_dropline_draw
    ,yxview_function_draw ,yxview_iarrow_draw ,yxview_ilabels_draw
    ,yxview_normal_draw ,yxview_pcarrow_draw ,yxview_pcbarrow_draw
    ,yxview_pccapsym_draw ,yxview_pcscatter_draw ,yxview_pcspike_draw
    ,yxview_rarea_draw ,yxview_rbar_draw ,yxview_rbarm_draw
    ,yxview_rcap_draw ,yxview_rcapsym_draw ,yxview_rconnected_draw
    ,yxview_rline_draw ,yxview_rscatter_draw ,yxview_rspike_draw
    ,yxview_spike_draw ,yxview_sunflower_draw ,zap_s ,zinb ,zinb_llf
    ,zinb_plf ,zip ,zip_llf ,zip_p ,zip_plf ,zt_ct_5 ,zt_hc_5 ,zt_hcd_5
    ,zt_is_5 ,zt_iss_5 ,zt_sho_5 zt_smp_5 ,ztbase_5 ,ztcox_5 ,ztdes_5
    ,ztereg_5 ,ztfill_5 ,ztgen_5 ,ztir_5 ztjoin_5 ,ztnb ,ztnb_p ,ztp
    ,ztp_p ,zts_5 ,ztset_5 ,ztspli_5 ,ztsum_5 ,zttoct_5 ztvary_5
    ,ztweib_5
  },
  %
  % Built-in functions
  morekeywords=[3]{
    Cdhms ,Chms ,Clock ,Cmdyhms ,Cofc ,Cofd ,F ,Fden ,Ftail ,I ,J
    ,_caller ,abbrev ,abs ,acos ,acosh ,asin ,asinh ,atan ,atan2
    ,atanh ,autocode ,betaden ,binomial ,binomialp ,binomialtail
    ,binormal ,bofd ,byteorder ,c ,ceil ,char ,chi2 ,chi2den ,chi2tail
    ,cholesky ,chop ,clip ,clock ,cloglog ,cofC ,cofd ,colnumb ,colsof
    ,comb ,cond ,corr ,cos ,cosh ,d ,daily ,date ,day ,det ,dgammapda
    ,dgammapdada ,dgammapdadx ,dgammapdx ,dgammapdxdx ,dhms ,diag
    ,diag0cnt ,digamma ,dofC ,dofb ,dofc ,dofh ,dofm ,dofq ,dofw ,dofy
    ,dow ,doy ,dunnettprob ,e ,el ,epsdouble ,epsfloat ,exp ,fileexists
    ,fileread ,filereaderror ,filewrite ,float ,floor ,fmtwidth
    ,gammaden ,gammap ,gammaptail ,get ,group ,h ,hadamard ,halfyear
    ,halfyearly ,has_eprop ,hh ,hhC ,hms ,hofd ,hours ,hypergeometric
    ,hypergeometricp ,ibeta ,ibetatail ,index ,indexnot ,inlist
    ,inrange ,int ,inv ,invF ,invFtail ,invbinomial ,invbinomialtail
    ,invchi2 ,invchi2tail ,invcloglog ,invdunnettprob ,invgammap
    ,invgammaptail ,invibeta ,invibetatail ,invlogit ,invnFtail
    ,invnbinomial ,invnbinomialtail ,invnchi2 ,invnchi2tail ,invnibeta
    ,invnorm ,invnormal ,invnttail ,invpoisson ,invpoissontail ,invsym
    ,invt ,invttail ,invtukeyprob ,irecode ,issym ,issymmetric ,itrim
    ,length ,ln ,lnfact ,lnfactorial ,lngamma ,lnnormal ,lnnormalden
    ,log ,log10 ,logit ,lower ,ltrim ,m ,match ,matmissing ,matrix
    ,matuniform ,max ,maxbyte ,maxdouble ,maxfloat ,maxint ,maxlong ,mdy
    ,mdyhms ,mi ,mi ,min ,minbyte ,mindouble ,minfloat ,minint ,minlong
    ,minutes ,missing ,mm ,mmC ,mod ,mofd ,month ,monthly ,mreldif
    ,msofhours ,msofminutes ,msofseconds ,nF ,nFden ,nFtail ,nbetaden
    ,nbinomial ,nbinomialp ,nbinomialtail ,nchi2 ,nchi2den ,nchi2tail
    ,nibeta ,norm ,normal ,normalden ,normd ,npnF ,npnchi2 ,npnt ,nt
    ,ntden ,nttail ,nullmat ,plural ,poisson ,poissonp ,poissontail
    ,proper ,q ,qofd ,quarter ,quarterly ,r ,rbeta ,rbinomial ,rchi2
    real ,recode ,regexm ,regexr ,regexs ,reldif ,replay ,return
    ,reverse ,rgamma ,rhypergeometric ,rnbinomial ,rnormal ,round
    ,rownumb ,rowsof ,rpoisson ,rt ,rtrim ,runiform ,s ,scalar ,seconds
    ,sign ,sin ,sinh ,smallestdouble ,soundex ,soundex_nara ,sqrt ,ss
    ,ssC ,strcat ,strdup ,string ,strlen ,strlower ,strltrim ,strmatch
    ,strofreal ,strpos ,strproper ,strreverse ,strrtrim ,strtoname
    ,strtrim ,strupper ,subinstr ,subinword ,substr ,sum ,sweep ,syminv
    ,t ,tC ,tan ,tanh ,tc ,td ,tden ,th ,tin ,tm ,tq ,trace ,trigamma
    ,trim ,trunc ,ttail ,tukeyprob ,tw ,twithin ,uniform ,upper ,vec
    ,vecdiag ,w ,week ,weekly ,wofd ,word ,wordcount ,year ,yearly
    ,yh ,ym ,yofd ,yq ,yw
  },
  %
  % Numbers
  morekeywords=[4]{
    0 ,1 ,2 ,3 ,4 ,5 ,6 ,7 ,8 ,9
  },
}

% ---------------------------------------------------------------------
% Stata editor style

\providecommand{\textcolordummy}[2]{#2}
\lstalias{Stata}{stata}
\lstdefinestyle{stata-editor}{
    language=stata,
    %
    % Global variables
    keywordstyle={\bfseries\color{spRed}},
    %
    % Add-ons system commands
    keywordstyle=[2]{\bfseries\color{NavyBlue}},
    %
    % Built-in functions
    keywordstyle=[3]{\color{blue}},
    %
    % Numbers
    keywordstyle=[4]{\color{blue}},
    %
    % User macros (variables)
    keywordstyle = [9]{\bfseries\color{LightSteelBlue}\let\textcolor\textcolordummy},
    %
    % Strings and comments
    stringstyle  = \color{Maroon},
    commentstyle = \color{Green}\slshape,
}

% ---------------------------------------------------------------------
% Suggested settings

% \lstset{
%   basicstyle        = \setmonofont{DejaVu Sans Mono}\footnotesize\ttfamily,
%   tabsize           = 4,      % Tab size
%   showstringspaces  = false,  % Don't underline spaces in strings
%   showspaces        = false,  % Don't underline spaces
%   breaklines        = true,   % Automatic line breaking
%   breakatwhitespace = true,   % Breaks only at white space.
%   lineskip          = 1.5pt,  % Sparing between lines of code
%   commentstyle      = \color{black!50}\itshape \let\textcolor\textcolordummy,
% }
\begin{document}
% -----------------------------------------
\begin{frame}[plain]
\begin{center}
\large
\textcolor{columbiadarkblue}{ECON G6905\\
Topics in Trade\\ 
Jonathan Dingel\\
Spring \the\year, Week 12}
\vfill 
\includegraphics[width=0.4\textwidth]{../images/Columbia_logo.png}
\end{center}
\end{frame}
% -----------------------------------------
\begin{frame}{Today: Spatial environmental economics}
We've already seen environmental features play various roles in this course.
For example, 
first-nature ``locational fundamentals'' are often environmental features:
deep-water harbors lower trade costs,
and
waterfront views are amenities that anchor neighborhood sorting by income.
\vspace{1cm}
Three themes for today's whirlwind tour of spatial environmental economics
\begin{itemize}
	\item Valuing environmental amenities
	\item Environmental policy and comparative advantage
	\item Spatial margins of adjustment 
\end{itemize}
\end{frame}
% -----------------------------------------
\begin{frame}{Hedonic pricing of air pollution}
\begin{itemize}
\item Classic application of spatial equilibrium: value environmental amenities
\item Does the US government regulate air too much/little? Cost-benefit analysis requires estimates of each, and there isn't a (direct) market for clean air
\item \href{https://doi.org/10.1086/427462}{Chay and Greenstone (2005)} study the capitalization of total suspended particulates (TSPs) air pollution into housing values
\item Price of house $i$ depends on its characteristics, $P_i = P(Z)$. What's $\frac{\partial P_i(Z)}{\partial z_n}$?
\item If some elements of $Z$ are unobserved and covary with $z_n = \text{TSPs}$, difficult to estimate the hedonic price schedule
\item E.g., areas with higher levels of TSPs tend to be more urbanized and have higher per capita incomes, population densities, and crime rates
\item \href{https://doi.org/10.1086/261981}{Smith and Huang (1995)}: 1/4 of estimates have perverse \textit{positive} correlation between housing prices and pollution levels
\item I'll focus on the OVB problem; see article for heterogeneous WTP
\end{itemize}
\end{frame}
% -----------------------------------------
\begin{frame}{Clean Air Act regulation of TSPs}
\begin{itemize}
\item Clean Air Act Amendments of 1970 requires the Environmental Protection Agency to annually assign each county to either nonattainment or attainment status for each of five pollutants if above either:
\begin{itemize}
\item Threshold 1: the annual geometric mean concentration exceeds 75 $\mu$ g/m$^{3}$
\item Threshold 2: the second-highest daily concentration exceeds 260 $\mu$ g/m$^{3}$
\end{itemize}
\item In their nonattainment counties, states are required to develop plant-specific regulations for every major source of pollution
\item Local rules mandate investments in state-of-the-art pollution abatement equipment
\item Chay and Greenstone (2005) focus on TSPs because limited data for other regulated pollutants in early 1970s
\end{itemize}
\end{frame}
% -----------------------------------------
\begin{frame}{CAAAs reduced TSPs quite a bit}
\includegraphics[height=0.9\textheight]{../images/week12/ChayGreenstone2005_fig2.pdf}
\end{frame}
% -----------------------------------------
\begin{frame}{Nonattainment counties have TSP decreases and house-price increases}
\only<1>{\includegraphics[width=0.49\textwidth]{../images/week12/ChayGreenstone2005_fig4.pdf}}
\only<1>{\includegraphics[width=0.49\textwidth]{../images/week12/ChayGreenstone2005_fig5.pdf}}
\end{frame}
% -----------------------------------------
\begin{frame}{Cross-sectional and first-difference regressions suffer OVB}
\begin{center}
\includegraphics[height=0.9\textheight]{../images/week12/ChayGreenstone2005_tab3.pdf}
\end{center}
\end{frame}
% -----------------------------------------
\begin{frame}{2SLS estimates using nonattainment IVs}
\begin{center}
\includegraphics[height=0.86\textheight]{../images/week12/ChayGreenstone2005_tab5.pdf}
\end{center}
\vspace{-4mm}
{\footnotesize \textcolor{gray}{See Chay and Greenstone (2005) Table 2 column 3 on Panel C IV problems}}
{  Data cover 988 counties
First, in the absence of arbitrary assumptions about which counties constitute separate markets, it is necessary to assume that there is a national housing market.
the hedonic approach as originally conceived is an individual-level model, and aggregation to the county level may induce some biases
}
\end{frame}
% -----------------------------------------
\begin{frame}{Valuations}
\begin{itemize}
\item A welfare analysis of the non-marginal reductions in TSPs induced by the mid-decade TSPs regulations requires estimation of the MWTP \textit{function}
\item ``A popular, but likely invalid, assumption is that preferences are homogeneous and linear with respect to air quality, so that the MWTP for clean air is constant''
\item Mean housing values increased by roughly \$2,400 in nonattainment counties. With $\sim$19 million houses in these counties, the WTP is $\sim$\$45 billion.
\item Per Rosen-Roback model, cannot only look at housing prices: WTP for lower TSPs shows up in housing-price differences \textit{and} wage differences
\item 2SLS regression with log income as outcome yields precise zero
\end{itemize}
\end{frame}
% -----------------------------------------
\begin{frame}{``East-Side Story'': Air pollution and persistent neighborhood sorting}
\href{https://doi.org/10.1086/713101}{Heblich, Trew, Zylberberg (2021)} on 70 largest metropolitan areas in England
\begin{itemize}
\item ``The heavy reliance on coal 1850--1960 generated unprecedented concentrations of sulphur dioxide in the atmosphere, which scarred cities''
\item ``Pollution from historical factories accounted for about 15\% of the variation in neighborhood composition in 1881.''
\item ``There is no evidence of excess deprivation in neighborhoods downwind from industrial chimneys before the rise of industrial coal in 1817.''
\item ``Industrial coal pollution effectively stopped in the 1970s, but the path dependence in neighborhood sorting is still felt today.''
\item ``The persistence of neighborhood sorting is tied not only to relocation frictions but also to its interaction with preferences for neighborhood composition.''
\end{itemize}
This is akin to Lee and Lin (2018) with a temporary disamenity.
Also relates to debates about poor people's higher exposure to environmental disamenities
\end{frame}
% -----------------------------------------
\begin{frame}{Rising coal consumption amidst urbanization}
\begin{center}
\includegraphics[height=0.9\textheight]{../images/week12/HeblichTrewZylberberg2021_fig1.pdf}
\end{center}
\end{frame}
% -----------------------------------------
\begin{frame}{Locating chimneys and their coal usage}
\includegraphics[height=0.55\textheight]{../images/week12/HeblichTrewZylberberg2021_fig2.pdf}
\begin{itemize}{\small
\item
Ordnance Survey maps show roads, railway, rivers, canals, the outline of each building and their use, and chimneys
\item 
Georeference $\geq 5,000$ chimneys and use associated building description, assigned to one of 11 industrial categories by textual analysis
\item 
Compute coal use per chimney from coal use per worker, total employment, and count of chimneys\par
}\end{itemize}
\end{frame}
% -----------------------------------------
\begin{frame}{Modeling local air pollution}
\begin{columns}[t]
\begin{column}{0.48\textwidth}
\begin{itemize}{\small
\item
Atmospheric dispersion modeling system (ADMS) incorporates the impact of temperature and humidity and accounts for complex terrain and changes in surface roughness.
\item
Winds blow mostly from the west/southwest (less predictable in Northern England)
\item
Residential sources: assume domestic chimneys are uniformly distributed within city at a very low altitude\par
}\end{itemize}
\end{column}
\begin{column}{0.50\textwidth}
{\small Estimated pollution correlates with 1915 deposits\par}
\includegraphics[height=0.60\textheight]{../images/week12/HeblichTrewZylberberg2021_fig5.pdf}\\
{\small \textcolor{gray}{More polluted places have more dark moths}\par}
\end{column}
\end{columns}
\end{frame}
% -----------------------------------------
\begin{frame}{Polluted neighborhoods have less skilled residents}
Regress skill share $Y_{it}$ of LSOA $i$ in parish $p$ in city $c$ on historical pollution $P_i$
\begin{equation*}
Y_{it} = \alpha + \beta P_{i} + \gamma X_{i} + \nu Y_p + \delta_{c} + \varepsilon_{it}
\end{equation*}
%$P_i$ is historical pollution, and
$X_i$ contains topographic, geographic, and disamenity controls
\begin{center}
\includegraphics[width=0.9\textwidth]{../images/week12/HeblichTrewZylberberg2021_tab2.pdf}
\end{center}
\end{frame}
% -----------------------------------------
\begin{frame}{Predicting air pollution using 1827 waterways}
Steam engines need water for cooling
\begin{center}
\includegraphics[height=0.8\textheight]{../images/week12/HeblichTrewZylberberg2021_tab3.pdf}
\end{center}
\end{frame}
% -----------------------------------------
\begin{frame}{Historical air pollution, modern neighborhood composition}
\begin{center}
\includegraphics[height=0.8\textheight]{../images/week12/HeblichTrewZylberberg2021_tab5.pdf}
\end{center}
\end{frame}
% -----------------------------------------
\begin{frame}{Pre-coal placebo test and modern outcomes}
\begin{center}
\includegraphics[height=0.85\textheight]{../images/week12/HeblichTrewZylberberg2021_fig10.pdf}
\end{center}
\end{frame}
% -----------------------------------------
\begin{frame}{Model of dynamics}
\begin{itemize}
\item ``To interpret the persistence of past pollution, we develop a dynamic model of neighborhood choice with two types of households--low-skilled and high-skilled--that differ only in their income
\item Each period, $1-\theta$ of households may move: migration reveals preferences over neighborhood characteristics
\item If $\theta = 0$, it's equivalent to a static model
\item With $\theta > 0$, demand depends on the past and future expected allocation of households
\item Authors use historical pollution as an instrument for current neighborhood composition and its subsequent evolution
\end{itemize}
\end{frame}
% -----------------------------------------
\begin{frame}{Scale, composition, and technique}
Consider environmental policy and comparative advantage.
Since \href{https://www.nber.org/papers/w3914}{Grossman and Krueger (1993)}, economists have often decomposed variation in industrial pollution into
\begin{itemize}
\item the \textit{scale} of output
\item the \textit{composition} of output across industries
\item the \textit{techniques} for producing a given good within an industry
\end{itemize}
This decomposition can be applied to changes over time or variation across places
\end{frame}
% -----------------------------------------
\begin{frame}{Denser counties: greater scale, dirtier industries, cleaner techniques}
\begin{center}
\only<1>{\includegraphics[width=\textwidth]{../images/week12/BalboniShapiro2025_fig3ab.pdf}}
\only<2>{\includegraphics[width=\textwidth]{../images/week12/BalboniShapiro2025_fig3cd.pdf}}
\\{\scriptsize \href{https://www.nber.org/papers/w33377}{Balboni and Shapiro (2025)}, Figure 3}
\end{center}
\begin{itemize}
\only<1>{\item Denser counties have higher GDP per capita and more people}
\only<1>{\item Denser counties have manufacturing and utilities rather than agriculture}
\only<2>{\item Denser counties use cleaner production techniques}
\only<2>{\item Denser counties are subject to more stringent regulation}
\end{itemize}
\end{frame}
% -----------------------------------------
\begin{frame}{Are cities greener? Smaller environmental footprint per person?}
\href{https://doi.org/10.1016/j.ecolecon.2015.05.012}{Holian and Kahn (2015)}:
``a standardized household creates less greenhouse gas emissions from transportation in those metropolitan areas with a larger downtown share of college graduates''
and
``public transit ridership has increased more in those metropolitan areas where a larger share of downtown adult residents are college graduates''
\smallskip
\href{https://doi.org/10.1016/B978-0-444-59517-1.00007-6}{Kahn and Walsh (Handbook 2015)}:
``the social costs of suburbanization literature is really focused on a second-best world in which households and firms are spreading out without facing the social costs of their actions''
\smallskip
\href{https://www.journals.uchicago.edu/doi/abs/10.1086/680663}{Holian and Kahn (2015)}:
``homeowner communities in suburban areas are more likely to vote against such regulation, while homeowners in the center city area are more likely to favor carbon pricing.''
\end{frame}
% -----------------------------------------
\begin{frame}{Shapiro (2025): institutional comparative advantage in clean industries}
Strong institutions provide comparative advantage in clean industries, improving a country's environmental quality
\begin{itemize}
\item industries that depend on institutions are clean
\item strong institutions increase relative exports in clean industries
\item industry's complexity explains clean goods' dependence on institutions
\item cross-country differences in industry composition explain meaningful share of variation in emissions
\item quantitative GE model says improving one country's  institutions reduces its pollution through cross-country industrial reallocation
\end{itemize}
\end{frame}
% -----------------------------------------
\begin{frame}{Some context: Institutional sources of comparative advantage}
\hypertarget{institutionalsourcesofCA}{}
\href{http://dx.doi.org/10.3982/ECTA7636}{Costinot (2009)} \textcolor{gray}{[Recall log-supermodularity from earlier class \hyperlink{appendix:LSM}{\beamergotobutton{Recall}}]}:
\begin{itemize}
\item Ricardian model has $q(\omega,\sigma,\gamma) = h(\omega) a(\sigma,\gamma)$ 
\item If $a(\sigma,\gamma)$ is LSM, then aggregate output $Q(\sigma,\gamma)$ is LSM
\item {\footnotesize
``This first result %has played an important, albeit implicit, role in many applications and extensions of the Ricardian model. It 
\dots
is at the heart, for example, of the recent literature on institutions and trade; see, for example, Acemoglu, Antras, and Helpman (2007), Costinot (2007), Cuñat and Melitz (2006), Levchenko (2007), Matsuyama (2005), Nunn (2007), and Vogel (2007). At a formal level, these papers all share the same fundamental objective: providing microtheoretical foundations for the log-supermodularity of factor productivity with respect to countries’ ``institutional quality'' and sectors’ ``institutional dependence,'' whatever those characteristics may be.''\par}
\end{itemize}
\href{http://dx.doi.org/10.1016/j.jinteco.2010.07.004}{Chor (\textit{JIE} 2010)}:
{\small ``I find strong evidence for the importance of factor endowments, financial development, legal institutions, and labor market regimes as sources of comparative advantage, even when all interaction terms are run in one regression.''\par}
\end{frame}
% -----------------------------------------
\begin{frame}{Measuring institutions, institutional dependence, and cleanliness}
\begin{itemize}
\item Institutions: Financial depth (private credit/GDP), rule of law (World Bank index), labor market freedom (Heritage Foundation index)
\item Institutional dependence: asset intangibility (Rajan \& Zingales 1998, Manova 2013), differentiated inputs (Rauch 1999, Nunn 2007), std dev within-firm sales growth
\item Factor endowments: capital per worker and human capital index (PWT)
\item Factor intensities: capital and labor shares
\item Country's environmental regulation: first PCA of 8 measures
\item Industry's clean index: one minus PCA of tons of air/water emissions per dollar output (EPA + Census of Manufactures)
\item Tariffs: CEPII's Market Access Map
\end{itemize}
\textcolor{gray}{All variables are normalized, so a unit change is one standard deviation}
\end{frame}
% -----------------------------------------
\begin{frame}{Motivating correlation: Cleaner countries have better institutions}
\only<1>{\includegraphics[width=\textwidth]{../images/week12/Shapiro2025_fig1a.pdf}}
\only<2>{\includegraphics[width=\textwidth]{../images/week12/Shapiro2025_fig1c.pdf}}
\only<3>{\includegraphics[width=\textwidth]{../images/week12/Shapiro2025_fig1e.pdf}}
\only<3>{}
\end{frame}
% -----------------------------------------
\begin{frame}{Industries that depend more on institutions are cleaner}
\includegraphics[width=\textwidth]{../images/week12/Shapiro2025_fig2ab.pdf}
\end{frame}
% -----------------------------------------
\begin{frame}{Countries with strong institutions export more in clean industries}
\begin{center}
\includegraphics[height=0.9\textheight]{../images/week12/Shapiro2025_fig3b.pdf}
\end{center}
\end{frame}
% -----------------------------------------
\begin{frame}{Gravity regression: Sources of comparative advantage}
Standard gravity model from a multi-sector Ricardian model:
$$
X_{ijs} = \xi \frac{T_{is}\left(c_{is}\phi_{ijs}\right)^{-\theta_s}}{P_{js}^{-\theta_s}} X_{js}
$$
Estimating equation for a cross section of trade flows:
$$
\ln X_{ijs} = \alpha \underbrace{E_i \times I_s}_{\text{institutions}} + \sum_{f} \beta_f \underbrace{E_i^f \times I_s^f}_{\text{factor inputs}} 
+ \pi \underbrace{R_i \times Z_s}_{\text{regulation}} + \gamma \ln \underbrace{(1 + t_{ijs})}_{\text{a.v. tariffs}}
+ \zeta_{js} + \eta_{ij} + \epsilon_{ijs}
$$
Replace institutional dependence $I_s$ by industry's clean index $Z_s$
$$
\ln X_{ijs} = \alpha^C {E_i Z_s} + \sum_{f} \beta_f^C {E_i^f I_s^f} 
+ \pi^C {R_i Z_s} + \gamma^C \ln {(1 + t_{ijs})}
+ \zeta_{js}^C + \eta_{ij}^C + \epsilon_{ijs}^C
$$
\end{frame}
% -----------------------------------------
\begin{frame}{Stronger environmental regulations reduce dirty exports}
\includegraphics[height=0.93\textheight]{../images/week12/Shapiro2025_tab2a.pdf}
\end{frame}
% -----------------------------------------
\begin{frame}{Stronger institutions increase relative exports in cleaner industries}
\includegraphics[height=0.93\textheight]{../images/week12/Shapiro2025_tab2b.pdf}
\end{frame}
% -----------------------------------------
\begin{frame}{Improving institutions and faster export growth in clean industries}
\begin{center}
\includegraphics[height=0.93\textheight]{../images/week12/Shapiro2025_fig3c.pdf}
\end{center}
\end{frame}
% -----------------------------------------
\begin{frame}{Industry's complexity explains clean goods' dependence on institutions}
{\small ``Polluting industries, often described as `heavy industry', use large, long-lived, tangible assets like machines and boilers to process and convert dense raw materials into finished products.''\par}
\begin{itemize}{\small
\item Large, expensive machines are tangible collateral for loans
\item Dirty industries disproportionately use raw materials that are relatively homogeneous and traded through open markets (e.g., fewer patent lawsuits) \par
}\end{itemize}
\includegraphics[width=0.9\textwidth]{../images/week12/Shapiro2025_tab3.pdf}
\end{frame}
% -----------------------------------------
\begin{frame}{Industry composition explains some of the variation in emissions}
{\footnotesize Mean country has total pollution emissions 72\% lower than US; gross output 89\% lower}
\begin{center}
\includegraphics[height=0.89\textheight]{../images/week12/Shapiro2025_tab4.pdf}
\end{center}
\end{frame}
% -----------------------------------------
\begin{frame}{Climate change: Changing locational fundamentals and reallocation}
\begin{itemize}
\item Climate change can alter productivity, marginal utility of consumption/leisure, disutility of work, capital depreciation rates, hazard rate of destructive disasters, etc
\item Many studies use variation in weather to estimate climate damage functions
\item Spatial pattern of climate changes is complicated: ``The models predict greater temperature change in the polar regions than near the equator. This change in gradient can drive changes in circulation. The results may be warmer in some places and colder in others, wetter in some places and drier in others\dots'' (\href{https://www.jstor.org/stable/2117599}{Schelling 1992})
\item Even a uniform rise in temperatures would have heterogeneous consequences:
a warmer Yukon Territory is likely more attractive
\item I will emphasize spatial margins of adjustments to climate change
\end{itemize}
\end{frame}
% -----------------------------------------
\begin{frame}{Temperature damages to US crops (Schlenker and Roberts 2009)}
\begin{center}
\only<1>{\scriptsize
	\includegraphics[width=\textwidth]{../images/week12/SchlenkerRoberts2009_fig1.pdf}\\
	``Our data are comprised of new fine-scale weather outcomes merged together with a large panel of crop yields that spans most U.S. counties from 1950 to 2005. The new weather data include the length of time each crop is exposed to each one-degree Celsisus temperature interval in each day, summed across all days of the growing season, all estimated for the specific locations within each county where crops are grown. The new fine-scale weather data facilitate estimation of a flexible model that can detect nonlinearities and breakpoints in the effect of temperature on yield.''\par
}
\only<2>{
\begin{columns}
\begin{column}{0.45\textwidth}
\includegraphics[height=0.9\textheight]{../images/week12/SchlenkerRoberts2009_fig2.pdf}
\end{column}
\begin{column}{0.54\textwidth}{\small
``Holding Current Growing Regions Fixed, Area-Weighted Average Yields Are Predicted to Decrease by 30--46\% Before the End of the Century Under the Slowest Hadley III Warming Scenario (B1), and Decline by 63--82\% Under the Most Rapid Warming Scenario (A1FI).''
\par}\end{column}
\end{columns}
}
\end{center}
\only<2>{}
\end{frame}
% -----------------------------------------
\begin{frame}{Statistical extrapolation is not a sensible economic forecast}
\begin{itemize}
\item
The statistical model of crop $c$ in place $i$ at time $t$
$$\text{yields}_{cit} = f_c(X_{it},T_{it})$$
is not an economic model;
it is closer to production-function estimation with omitted inputs
($X_{it}$ contains county FEs, time trends, and precipitation)
\item
A naive extrapolation like
$\text{yields}^{\text{forecast}}_{ci,2099} = f_c(X_{i\bar{t}},T^{\text{forecast}}_{i,2099})$
that feeds in 2099 temperatures
and fixes endogenous/control variables $X$ at their time $\bar{t}$ values
is not a counterfactual scenario in an economic model
\item Output is produced by decision makers who can
\begin{itemize}
\item invest in heat-resistant crop varieties
\item switch to planting other crops
\item move to other places
\end{itemize}
\end{itemize}
\end{frame}
% -----------------------------------------
\begin{frame}{Spatial margins of adjustment to climate change}
{\footnotesize
\href{https://crossmark.crossref.org/dialog/?doi=10.1016/j.jue.2015.04.004}{Desmet, Rossi-Hansberg (2015)}:\\
{``Temperature varies by parallel from $0^{\circ}$ Celsius in the North Pole to $28^{\circ}$ Celsius in the Equator\dots
most land in the world is essentially economically unused and empty\dots
any substantial cost of climate change must be associated with the frictions involved in moving production and people from their current sites to the regions that will be suitable for production in the future''
\par}
\vspace{3mm}
\href{http://www.nber.org/papers/w20716}{Kahn (2014)}:\\
{``The menu of choices offered by a system of cities reduces our exposure to climate change risk\dots
Nations featuring multiple cities with low migration barriers between these cities are uniquely suited to adapt to most of the emerging challenges associated with climate change\dots
The themes of competition, choice, innovation and experimentation are the basis for my optimism concerning our collective ability as urbanites to withstand this emerging threat.''
\par}
\vspace{3mm}
\href{https://www.jstor.org/stable/2117599}{Schelling (AER 1992)}:\\
{``Almost everybody who attends this lecture in New Orleans will have undergone a greater change in the past few days than is expected to occur in any fixed locality during the coming century.''
\par}
}
\end{frame}
% -----------------------------------------
\begin{frame}{Costinot, Donaldson, Smith (2016)}
\begin{itemize}
\item
If climate-change damages are spatially uneven, how might global agricultural trade respond?
\item
Food and Agriculture Organization's Global Agro-Ecological Zones (GAEZ) data set uses agronomic models of yields and high-resolution data on soil, topography, and climatic conditions to predict crop-by-crop yields for 1.7 million grid cells.
\item
Competitive model of trade between countries with heterogeneous ``fields''
\item 
Two elasticities of substitution and one productivity dispersion parameter drive responses
\item
Counterfactual scenarios show crop switching damps climate-change damages by about two-thirds, while adjustments to trade flows have little consequence
\end{itemize}
\textcolor{gray}{These exercises rely heavily on the GAEZ productivities for un-produced outputs}
\end{frame}
% -----------------------------------------
\begin{frame}{Climate damages could raise gains from trade (relative to autarky)}
Simple example:
\begin{itemize}
\item Imagine symmetric $2\times2\times2$ economy: North and South islands have East and West fields with identical yields for wheat and rice
\item Equilibrium: relative price is one, West field grows wheat, East field grows rice, no net trade
\item Climate change: South's wheat yields fall, North's rice yields fall
\item In autarky, large losses; with free trade, no loss at all because damage created comparative advantage and countries switch to fully specialized production
\end{itemize}
\end{frame}
% -----------------------------------------
% Preferences Slide
\begin{frame}{Preferences}
The representative agent in country $i$ has quasi-linear preferences over outside good and a composite of all crops $C_i$:
% Equation 1: Utility function
\begin{equation*}
U_i = C_i^0 + \beta_i \ln C_i
\end{equation*}
$C_i$ aggregates crops indexed by $k$ (elasticity $\kappa$) and crop varieties indexed by origin $j$ (elasticity $\sigma$)
% Equation 2: Aggregate crop consumption
% Equation 3: Consumption of each crop
\begin{align*}
C_i &= \left[ \sum_{k \in \mathcal{K}} (\beta_i^k)^{1/\kappa} (C_i^k)^{(\kappa-1)/\kappa} \right]^{\kappa/(\kappa-1)}
\\
C_i^k &= \left[ \sum_{j \in \mathcal{I}} (\beta_{ji}^k)^{1/\sigma} (C_{ji}^k)^{(\sigma-1)/\sigma} \right]^{\sigma/(\sigma-1)}
\end{align*}
\end{frame}
% -----------------------------------------
% Production Slide
\begin{frame}{Production technologies}
\begin{itemize}
\item $A_i^0$ is labor productivity in freely traded outside good
\item Crop $k$ is Leontief combination of $L$ hectares of land and $N$ laborers in parcel $\omega$ of field $f$
with TFP $A_i^{fk}(\omega)$ and labor intensity $\nu_i^f(\omega)$
% Equation 4: Production function
\begin{equation*}
Q_i^{fk}(\omega) = A_i^{fk}(\omega) \min\{L_i^{fk}(\omega), N_i^{fk}(\omega)/\nu_i^f(\omega)\}
\end{equation*}
$A_i^{fk}(\omega)$ and $\nu_i^f(\omega)$ are iid Frechet distributed with location parameters $A_i^{fk}$ (from GAEZ data) and shape parameter $\theta$
\item Competitive markets with iceberg trade costs
% Equation 5: Nonarbitrage constraint
\begin{equation*}
p_{ij}^k = \tau_{ij}^k p_i^k
\end{equation*}
\end{itemize}
\end{frame}
% -----------------------------------------
% Competitive Equilibrium Slide
\begin{frame}{Competitive equilibrium}
No surprises here. Utility maximization:
% Equation 6: Demand from utility maximization
\begin{equation*}
C_{ji}^k = \beta_i  \frac{\beta_i^k (P_i^k)^{1-\kappa}}{\sum_{l \in \mathcal{K}} \beta_i^l (P_i^l)^{1-\kappa}} \frac{\beta_{ji}^k (\tau_{ji}^k p_j^k)^{-\sigma}}{\sum_{n \in \mathcal{I}} \beta_{ni}^k (\tau_{ni}^k p_n^k)^{1-\sigma}} \quad \text{for all } i,j \in \mathcal{I}, k \in \mathcal{K}
\end{equation*}
% Equation 7: Land allocation probability
Probability parcel $\omega$ of field $f$ in country $i$ allocated to crop $k$:
\begin{equation*}
\pi_i^{fk} = \frac{(p_i^k A_i^{fk})^{\theta}}{(\alpha_i)^{\theta} + \sum_{l \in \mathcal{K}} (p_i^l A_i^{fl})^{\theta}} \quad \text{for all } f \in \mathcal{F}_i, i \in \mathcal{I}, \text{ and } k \in \mathcal{K}
\end{equation*}
% Equation 8: Supply of crop
Output is sum across fields (of area $s_i^f$) using selection-corrected productivities
\begin{equation*}
Q_i^k = \sum_{f \in \mathcal{F}_i} s_i^f A_i^{fk} \left[ \frac{(p_i^k A_i^{fk})^{\theta}}{(\alpha_i)^{\theta} + \sum_{l \in \mathcal{K}} (p_i^l A_i^{fl})^{\theta}} \right]^{(\theta-1)/\theta} \quad \text{for all } i \in \mathcal{I} \text{ and } k \in \mathcal{K}
\end{equation*}
% Equation 9: Market clearing condition
Markets clear:
\begin{equation*}
Q_i^k = \sum_{j \in \mathcal{I}} \tau_{ij}^k C_{ij}^k \quad \text{for all } i \in \mathcal{I} \text{ and } k \in \mathcal{K}
\end{equation*}
\end{frame}
% -----------------------------------------
\begin{frame}{Data}
\begin{itemize}
\item 50 countries (90\% of world crop value) and 10 crops (71\%)
\begin{center}
\begin{tabular}{lrlr}
Rice&17.3\%    &White potato&5.6\% \\
Maize&11.4\%   &Sugarcane&4.7\% \\
Wheat&10.7\%   &Banana&3.4\% \\
Tomato&5.9\%   &Cotton&3.2\% \\
Soybean&5.6\%  &Oil palm&3.2\% \\
\end{tabular}
\end{center}
\item GAEZ provides pre- (1961-1990) and post-CC (2071-2100) field-level productivities (scenario: high inputs, rain-fed water supply)
\item FAO: Output quantity, land use, and prices in 2009 by crop-country ($ki$) pair
\item Comtrade: Bilateral crop-level trade values $X_{ij}^k$ in 2009
\end{itemize}
\end{frame}
% -----------------------------------------
\begin{frame}{Estimation: Demand system}
\begin{itemize}
\item Estimate $\sigma = 5.4$ using average GAEZ yield $Z_i^k$ as IV for $p_i^k$  
$$
\ln(X_{ij}^k/X_j^k) = M_j^k + (1-\sigma) \ln p_i^k + \underbrace{\ln [\beta_{ij}^k (\tau_{ij}^k)^{1-\sigma}]}_{\equiv \ \epsilon_{ij}^{k}}
$$
\textcolor{gray}{(no $X_{ij}^k = 0$ observations; no $\tau_{ij}$ covariates; cannot reject that $\sigma^k = \sigma \ \forall k$)}
\item Estimate $\kappa = 2.8$ using CES demand and $Z_j^k$ as IV for $P_j^k$
%$P_j^k = \left[\sum_{i \in \mathcal{I}} \beta_{ij}^k \left(\tau_{ij}^k p_i^k\right)^{1-\sigma}\right]^{1/(1-\sigma)}$
$$
\ln(X_{j}^k/X_j) = M_j + (1-\kappa) \ln \underbrace{\left[\sum_{i \in \mathcal{I}} \beta_{ij}^k \left(\tau_{ij}^k p_i^k\right)^{1-\sigma}\right]^{\frac{1}{1-\sigma}}}_{= P_j^k} + \epsilon_{j}^{k}
$$
\item Quasi-linear preferences make demand shifter $\beta_j = X_j$
\end{itemize}
\end{frame}
% -----------------------------------------
\begin{frame}{Estimation: Productivity heterogeneity ($\theta,\alpha_i$)}
\begin{itemize}
\item Given $\theta$, choose labor cost shifter $\alpha_i$ to match land allocated to crops
$$
L_i(\theta,\alpha_i) = \sum_{k \in \mathcal{K}} \sum_{f \in \mathcal{F}_i} s_i^f A_i^{fk} \left[\pi_{i}^{fk}\right]^{(\theta-1)/\theta}
$$
\item Estimate $\theta$ to minimize distance between $Q_i^k$ in model and data
$$
Q_i^k(\theta,\alpha_i) = \sum_{f \in \mathcal{F}_i} s_i^f A_i^{fk} {\underbrace{\left[ \frac{(p_i^k A_i^{fk})^{\theta}}{(\alpha_i)^{\theta} + \sum_{l \in \mathcal{K}} (p_i^l A_i^{fl})^{\theta}}\right]}_{= \pi_{i}^{fk}}}^{(\theta-1)/\theta}
$$
\item Estimate $\alpha_i$ and $\theta = 2.46$ by constrained non-linear least squares
$$
\min_{\theta,\{\alpha_i\}} \sum_{i \in \mathcal{I}} \sum_{k \in \mathcal{K}} \left[\ln Q_i^k(\theta,\alpha_i) - \ln Q_i^k\right]^2
\text{ s.t. } L_i(\theta,\alpha_i) = L_i \ \forall i \in \mathcal{I}
$$
\item This assumes measurement error in outputs, not productivities
\item See paper on model fit
\end{itemize}
\end{frame}
% -----------------------------------------
\begin{frame}{Counterfactual scenarios: Productivity change}
First-order approximation of productivity changes: $\Delta A_i = \sum_{k \in K} \sum_{f \in \mathcal{F}_i} \rho_i^{fk} \Delta A_i^{fk}$
\begin{itemize}
\item The approximation ignores terms of trade effects
\item
$\rho_i^{fk} \equiv p_i^k Q_i^{fk} / Y_i$  denotes the initial $fk$ share of country $i$'s GDP
\item
$\rho_i^{fk} = \rho_i^k \times \phi_i^{fk}$ with revenue $\rho_i^k \equiv p_i^k Q_i^k / Y_i $ and output $\phi_i^{fk} \equiv Q_i^{fk} / Q_i^k$ shares
\item Logit math yields
$\phi_i^{fk} = \frac{(A_i^{fk})^\theta[\alpha_i^\theta + \sum_{l\in K} (p_i^l A_i^{fl})^\theta]^{(1-\theta)/\theta}}{\sum_{g\in\mathcal{F}_i} (A_i^{gk})^\theta[\alpha_i^\theta + \sum_{l\in K} (p_i^l A_i^{gl})^\theta]^{(1-\theta)/\theta}},$
which can be computed using GAEZ productivity, price data, and estimated $\alpha_i$ and $\theta$
\item Indonesia, the median country, has $\Delta A_i = - 1.3\%$
\item For world, climate change is $-0.34\%$ decrease in GDP
\end{itemize}
Full model using $A'_i$ instead of $A_i$: 
\begin{itemize}
\item welfare loss of 0.26\% of world GDP
\end{itemize}
\end{frame}
% -----------------------------------------
\begin{frame}{Counterfactual scenarios: Trade versus reallocation}
No production adjustment: fix $\pi_i^{fk}$ at baseline shares
\begin{itemize}
\item $(Q_i^k)^{'} = \sum_{f \in \mathcal{F}_i} s_i^f (A_i^{fk})^{'} \left[ \pi_i^{fk} \right]^{(\theta-1)/\theta}$
\item World welfare loss of $-0.78\%$ of GDP is triple the baseline scenario
\item Farmers' ability to substitute crops when comparative advantage shifts may substantially mitigate the ill effects of climate change.
\end{itemize}
No trade adjustment: fix crop export shares at baseline
\begin{itemize}
\item Require exports of crop $k$ by country $i$ to stay constant fraction of output
$$\sum_{j\neq i} \tau_{ij}^k (C_{ij}^k)^{'} / (Q_i^k)^{'} = \sum_{j\neq i} \tau_{ij}^k C_{ij}^k/ Q_i^k  \ \forall ik$$
\item Impose quantity restrictions by equivalent wedges $\delta_i^k$ (trade taxes)
$$(p_{ij}^k)^{'} = \delta_i^k \tau_{ij}^k (p_i^k)^{'} \ \forall i \neq j \in \mathcal{I}, \forall k \in \mathcal{K}$$
\item World welfare loss of 0.27\% is virtually same as in case of full adjustment
\end{itemize}
\end{frame}
% -----------------------------------------
\begin{frame}{Summary}
\begin{itemize}
\item Spatial-equilibrium logic lets us value local environmental benefits
\item Local air pollution interacts with spatial sorting by skill
\item Comparative advantage is important for environmental policy and climate adaptation
\item Spatial margins are one important dimension of adaptation to climate change 
\end{itemize}
\vspace{1cm}
\end{frame}
% -----------------------------------------
\appendix
% -----------------------------------------
\begin{frame}{Recall: Log-supermodularity}
\hypertarget{appendix:LSM}{}
A function $g:\mathbb{R}^n\to\mathbb{R}^{+}$ is \emph{log-supermodular} if $\forall x,x'\in\mathbb{R}^n$
\begin{align*}
g\left(\max\left(x,x'\right)\right)\cdot g\left(\min\left(x,x'\right)\right)\geq g(x)\cdot g(x')
\end{align*}
where $\max$ and $\min$ are component-wise operators.
\begin{itemize}
%	\item Example: $A: \Sigma\times\mathbb{C}\to\mathbb{R}^{+}$, where $\Sigma\subseteq\mathbb{R}$ and $\mathbb{C}\subseteq\mathbb{R}$, with $\sigma>\sigma'$ and $c>c'$
%		\begin{align*}
%		A(\sigma,c)A(\sigma',c')\geq A(\sigma',c)A(\sigma,c')
%		\end{align*}
	\item $g(x)$ is LSM in $(x_i,x_j)$ if $g(x_i,x_j;x_{-i,-j})$ is LSM 
	\item $g(x)$ is LSM $\iff g(x)$ is LSM in $(x_i,x_j)$  $\forall i,j$
	\item $g>0$ and $g$ is $C^2$ $\Rightarrow \frac{\partial^2 \ln g}{\partial x_i \partial x_j}\geq 0 \iff g(x)$ is LSM in $(x_i,x_j)$
\end{itemize}
Some handy properties:
\begin{itemize}
	\item LSM is preserved by multiplication and integration
	\item An argmax of a LSM objective is increasing in other variables
\end{itemize}
\hyperlink{institutionalsourcesofCA}{\beamerreturnbutton}
\end{frame}
% -----------------------------------------
\end{document}
