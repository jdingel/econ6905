\documentclass[11pt,notes=hide,aspectratio=169]{beamer}
%Jonathan Dingel; PhD trade course

% PACKAGES
\usepackage{graphics}  % Support for images/figures
\usepackage{graphicx}  % Includes the \resizebox command
\usepackage{url}	   % Includes \urldef and \url commands
\usepackage{soul}      % Includes the underline \ul command
%\usepackage{framed}	   % Includes the \framed command for box around text
\usepackage{booktabs} %\toprule,\bottomrule
%\usepackage{natbib}
\usepackage{bibentry}  % Includes the \nobibliography command
\usepackage{bbm}       %
%\usepackage{pgfpages}  %Supports "notes on second screen" option for beamer
\usepackage{verbatim}  %Supports comments
\usepackage{tikz}		%Supports graphing/drawing
\usepackage{pgfplots} %Supports graphing/drawing
\usepackage{amsfonts}  % Lots of stuff, including \mathbb 
\usepackage{amsmath}   % Standard math package
\usepackage{amsthm}    % Includes the comment functions
\usepackage{physics}

% CUSTOM DEFINITIONS
\def\newblock{} %Get beamer to cooperate with BibTeX
\linespread{1.2}
\hypersetup{backref,pdfpagemode=FullScreen,colorlinks=true,linkcolor=blue,urlcolor=blue}
\newtheorem{proposition}{Proposition}
\newtheorem{assumption}{Assumption}
\newtheorem{condition}{Condition}

% IDENTIFYING INFORMATION
\title{Topics in Trade}
\author{Jonathan I. Dingel}
\date{Fall \the\year}

% BEAMER TEACHING STUFF
\setbeamertemplate{navigation symbols}{}  %Turn off navigation bar

% THEMATIC OPTIONS
\definecolor{columbiablue}{RGB}{185,217,235}  %Columbia blue defined at https://visualidentity.columbia.edu/branding
\definecolor{columbiadarkblue}{RGB}{0,48,135}  %Columbia dark blue defined at https://visualidentity.columbia.edu/branding
\setbeamercovered{transparent=5}
\setbeamercolor{frametitle}{fg=columbiadarkblue}
\setbeamercolor{item}{fg=columbiadarkblue}
\usefonttheme{serif}
\setbeamercolor{button}{bg = white,fg = columbiadarkblue}
\setbeamercolor{button border}{fg = columbiadarkblue}

\setbeamertemplate{footline}{\begin{center}\textcolor{gray}{Dingel -- Topics in Trade -- \semester -- Week 7 -- \insertframenumber}\end{center}}
\begin{document}
% -----------------------------------------
\begin{frame}[plain]
\begin{center}
\large
\textcolor{columbiadarkblue}{ECON G6905\\
Topics in Trade\\ 
Jonathan Dingel\\
\semester, Week 7}
\vfill 
\includegraphics[width=0.4\textwidth]{../images/Columbia_logo.png}
\end{center}
\end{frame}
% -----------------------------------------
\begin{frame}[plain]
\begin{center}
\includegraphics[height=0.8\textheight]{../images/noaa_nightlights_4096.jpg} \\
{\small
Image from \href{ftp://public.sos.noaa.gov/land/earth_night/nightlights/4096.jpg}{NOAA} \\
(Defense Meteorological Program Operational Linescan System)\\
Donaldson \& Storeygard, ``\href{https://www.aeaweb.org/articles?id=10.1257/jep.30.4.171}{The View from Above: \\ Applications of Satellite Data in Economics}'', \textit{JEP}, 2016
\par}
\end{center}
\end{frame}
% -----------------------------------------
\begin{frame}[plain]
\begin{center}
\includegraphics[height=0.8\textheight]{../images/ISS027-E-020129_lrg.jpg} \\
Image from \href{https://visibleearth.nasa.gov/view.php?id=50671
}{NASA}
\end{center}
\end{frame}
% -----------------------------------------
\begin{frame}{Today: Agglomeration economies}
Gross metropolitan product per capita rises with metro population:
\begin{center}
\includegraphics[height=.65\textheight]{../images/GlaeserGottlieb2009_fig1.pdf}
\end{center}
\href{https://www.sciencedirect.com/science/article/pii/0304393288901687}{Lucas (1988)} on local external economies:
``What can people be paying Manhattan or downtown Chicago rents \textbf{for}, if not being near other people?''
\end{frame}
% -----------------------------------------
\begin{frame}{Today's agenda}
Three big ideas: Cities, spatial equilibrium, and agglomeration economies
\begin{itemize}
\item Measurement: Satellite images, spatial units, and what is a city?
\item Spatial equilibrium in the Rosen-Roback framework
\item Spatial equilibrium and the marginal resident
\item Agglomeration economies (local increasing returns)
\item Developing-economy cities
\end{itemize}
\end{frame}
% -----------------------------------------
\begin{frame}{Data on lights at night}
\begin{itemize}
	\item Defense Meteorological Satellite Program-Operational Linescan System (DMSP-OLS) for 1992-2011
	versus
	Visible Infrared Imaging Radiometer Suite (VIIRS) for 2011 onwards
	\item Lights are more useful for predicting GDP in cross section than time series (Chen and Nordhaus 2019 on both DMSP and VIIRS)
	\item \href{https://www.mdpi.com/2072-4292/11/9/1057}{Chen and Nordhaus (2019)}: high-resolution VIIRS lights better predict MSA GDP than state GDP (urban vs rural; lights do not explain value-added GDP in agriculture and forestry)
	\item \href{https://ideas.repec.org/p/wai/econwp/24-08.html}{Gibson, Kim, Li (2024)}: ``these GDP-luminosity elasticities vary especially by spatial scale and metro status, and also by period and remote sensing source. The elasticities mainly capture extensive margins of luminosity.''
\end{itemize}
\end{frame}
% -----------------------------------------
\begin{frame}{Data and measurement: US geographic units}
\begin{columns}\begin{column}{0.68\textwidth}{\footnotesize
Census block ($\sim$\href{https://www.census.gov/geographies/reference-files/time-series/geo/tallies.html}{8 million}; half unpopulated) \vspace{-1mm}
\begin{itemize}\itemsep-0.2em
    \item Smallest geographic unit used by the US Census
    \item Bounded by streets, roads, streams, or other features
\end{itemize}
Census tract ($\sim$84,000) \vspace{-1mm}
\begin{itemize}\itemsep-0.2em
    \item Designed to have homogeneous characteristics
    \item Usually 1,200-8,000 residents (optimally $\sim$4,000)
	\item This is ``neighborhood'' in most social-science research
\end{itemize}
ZIP code ($\sim$40,000 defined by USPS) \vspace{-1mm}
\begin{itemize}\itemsep-0.2em
	\item Based on postal volumes
\end{itemize}
County (and county equivalents) ($\sim$3,143) \vspace{-1mm}
\begin{itemize}\itemsep-0.2em
    \item Administrative/legal unit for local government services
\end{itemize}
Core-based statistical area ($\sim$926 defined by OMB) \vspace{-1mm}
\begin{itemize}\itemsep-0.2em
    \item Metropolitan and micropolitan statistical areas
	\item Urban core and counties linked by commuting flows
\end{itemize}
}\end{column}
\begin{column}{0.30\textwidth}
\href{https://www.nyc.gov/assets/planning/download/pdf/about/publications/maps/mn-census-tracts-map-rotated.pdf}{%
\includegraphics[width=\textwidth]{../images/IAB_census_tracts.png}}\\ \ \\
\href{https://censusreporter.org/profiles/31000US35620-new-york-newark-jersey-city-ny-nj-metro-area/}{%
\includegraphics[width=\textwidth]{../images/NYC_MSA.png}}
\end{column}\end{columns}
\end{frame}
% -----------------------------------------
\begin{frame}{Modifiable areal unit problem}
Statistics and estimates depend on spatial units
\begin{center}
\includegraphics[height=0.40\textheight]{../images/MAUP-Scale-Effect.jpg}\\
\includegraphics[height=0.40\textheight]{../images/MAUP-Gerrymeandering.png}\\
Images from \href{https://gisgeography.com/maup-modifiable-areal-unit-problem/}{GISGeography}
\end{center}
\end{frame}
% -----------------------------------------
\begin{frame}{What is a city?}
\begin{itemize}
	\item Municipality versus county versus metropolitan area versus commuting zone
	\item An integrated labor market defined by commuting ties? (cf. \href{https://www.aeaweb.org/articles?id=10.1257/aer.20151507}{Monte et al 2018})
	\item What to do absent commuting flows? (\href{https://doi.org/10.1016/j.jue.2019.05.005}{Dingel, Miscio, Davis 2021})
	\item Discretization vs continuous linkages (\href{https://ideas.repec.org/a/oup/restud/v72y2005i4p1077-1106.html}{Duranton and Overman 2005})
	%\item Know the modifiable areal unit problem (MAUP)
	%\item (Related: Are agglomeration economies about size or density?)
\end{itemize}
\begin{columns}
\begin{column}{0.48\textwidth}
\hspace{5mm}
\includegraphics[height=0.55\textheight]{../images/RozenfeldRybskiGabaixMakse2011_fig6.pdf}
\end{column}
\begin{column}{0.48\textwidth}
Today we study agglomeration without geography:
discrete cities are islands without bilateral spatial linkages
\end{column}
\end{columns}
\end{frame}
% -----------------------------------------
\begin{frame}{Spatial equilibrium}
Fundamentally, spatial equilibrium is a no-arbitrage condition.
\href{https://www.aeaweb.org/articles?id=10.1257/jel.47.4.983}{Glaeser and Gottlieb (\textit{JEL} 2009)}:
\begin{quote}
The high mobility of labor leads urban economists to assume a spatial equilibrium, where elevated New York incomes do not imply that New Yorkers are better off. Instead, welfare levels are equalized across space and high incomes are offset by negative urban attributes such as high prices or low amenities.
\end{quote}
\vspace{-4mm}
\begin{itemize}
	\item The benchmark model of spatial equilibrium is dubbed the ``Rosen-Roback'' model, due to the theory of equalizing differences (Sherwin Rosen 1974, 1979) applied to cities for both workers and firms (Jennifer Roback 1982)
	\item I borrow my exposition of Rosen-Roback model from \href{https://scholar.princeton.edu/sites/default/files/zidar/files/zidar_eco524_s2020_lec2.pdf}{Owen Zidar's slides}
\end{itemize}
\end{frame}
% -----------------------------------------
\begin{frame}
\frametitle{Rosen-Roback framework}
Goals
\begin{itemize}
\item How does change in amenity $s$ alter local prices (wages, rents)?
\item Infer the value of amenities
\end{itemize}
Markets
\begin{itemize}
\item Labor: price $w$, quantity $N$
\item Land: price $r$, quantity $L=L^w + L^p$ used by workers and producers
\item Goods: price $p=1$, quantity $X$ [no trade of consequence]
\end{itemize}
Agents
\begin{itemize}
\item Workers (homogeneous, perfectly mobile)
\item Firm (perfectly competitive, constant returns to scale)
\end{itemize}
Indifference conditions
\begin{itemize}
\item Workers have same indirect utility in all locations
\item Firm has zero profit (i.e., unit costs equal 1)
\end{itemize}
\end{frame}
\begin{frame}
\frametitle{Workers: Preferences and budget constraint}
Utility is $u(x, l^c, s)$
\begin{itemize}
\item  $x$ is consumption of private good
\item  $l^c$ is consumption of land
\item  $s$ is amenity
\end{itemize}
Budget constraint is $x + rl^c - w - I = 0$
\begin{itemize}
\item $I$ is non-labor income that is independent of location %(e.g., share of national land portfolio)
\item $w$ is labor income (note: no hours margin)
\end{itemize}
Indirect utility is
\begin{align*}
V(w, r, s)  = \max_{x, l^c} u(x, l^c, s) \text{ s.t. }  x + rl^c - w - I = 0
\end{align*}
Let $\lambda  = \lambda(w, r, s)$ be the marginal utility of a dollar of income, then 
\begin{equation*}
V_w = \lambda >0
\qquad 
V_r = -\lambda l^c <0 
\implies
V_r = - V_w l^c
\quad \text{via Roy's identity}
\end{equation*}
\end{frame}
\begin{frame}
\frametitle{Example: Cobb-Douglas preferences}
Utility is Cobb Douglas over goods and land with an amenity shifter: 
$$u(x, l^c, s)=s^{\theta_W} x^{\gamma} (l^c)^{1-\gamma}$$
\vspace{-7mm}
\begin{itemize}
\item Then $x=\gamma \left(\frac{w + I}{1} \right)$ and $l^c=(1-\gamma)\left(\frac{w + I}{r}\right)$ \medskip
\item Let $\Gamma \equiv \gamma^\gamma (1-\gamma)^{(1-\gamma)}$ so that indirect utility is
\begin{equation*}
V(w, r, s)  = 
\underbrace{\Gamma}_{\text{constant}} 
\underbrace{s^{\theta_W}}_{\text{amenities}} 
\underbrace{1^{-\gamma} r^{-(1-\gamma)}}_{\text{prices}}
\underbrace{(w + I)}_{\text{income}}
\end{equation*}
\item MU of income is $\lambda(w, r, s)$ 
\begin{align*}
V_w &= \lambda = \Gamma s^{\theta_W} r^{-(1-\gamma)}  \\
V_r &= -\lambda l^c =  -\Gamma s^{\theta_W} r^{-(1-\gamma)} (1-\gamma)\left(\frac{w + I}{r}\right) \\
\Rightarrow  V_r &= - V_w l^c 
\end{align*}
\end{itemize}
\end{frame}
\begin{frame}
\frametitle{Firms: Unit cost function}
CRS production with cost function $C(X, w,r,s)$ 
\begin{itemize}
\item  $X$ is output
\item Unit cost $c(w, r, s)=\frac{C(X,w,r,s)}{X}$
\item $L^p$ is total amount of land used by firms
\item $N$ is total employment \medskip
\end{itemize}
From Shepard's Lemma (derivative of cost function wrt factor price equals factor demand), we have
\begin{align*}
c_w &= N/X >0 \\
c_r &= L^p/X >0
\end{align*}
\end{frame}
\begin{frame}
\frametitle{Example: Cobb-Douglas production}
Suppose the production function is
$$X=f(N,L^p,s)= s^{\theta_F} N^\alpha (L^p)^{1-\alpha}$$
Let $\mathcal{A} \equiv \alpha^{-\alpha} (1-\alpha)^{-(1-\alpha)}$.
Then the cost function is
\begin{equation*}
C(X,w,r,s) 
= X (s^{\theta_F})^{-1}w^\alpha r^{1-\alpha} \mathcal{A}
\implies
c(w,r,s)
=(s^{\theta_F})^{-1}w^\alpha r^{1-\alpha} \mathcal{A}
\end{equation*}
Then 
\begin{align*}
C_w(X,w,r,s) = \alpha \frac{\left(X (s^{\theta_F})^{-1}w^\alpha r^{1-\alpha} \mathcal{A} \right)}{w} = N \\
C_r(X,w,r,s) = (1-\alpha) \frac{\left(X (s^{\theta_F})^{-1}w^\alpha r^{1-\alpha} \mathcal{A} \right)}{r} = L^p
\end{align*}
Dividing both sides by $X$ gives:
\begin{equation*}
c_w = N/X >0
\qquad
c_r = L^p/X >0
\end{equation*}
\end{frame}
\begin{frame}
\frametitle{Model recap}
\begin{columns}
\begin{column}{0.49\textwidth}
Workers parameters: $s,\theta_W, \gamma, I$
\begin{itemize}
\item $s$ is level of amenities
\item $\theta_W$ is value of $s$ for utility
\item $\gamma$ is goods share of expenditure
\item $1-\gamma$ is land share
\item $I$ is non-labor income
\end{itemize}
\end{column}
\begin{column}{0.49\textwidth}
Firm Parameters: $s$, ${\theta_F}$, $\alpha$ 
\begin{itemize}
\item $s$ is level of amenities
\item $\theta_F$ is value of $s$ for productivity
\item $\alpha$ is output elasticity of labor
\item $1-\alpha$ is output elasticity of land
\end{itemize}
\vfill
\end{column}
\end{columns}
\begin{center}
Endogenous outcomes:
\begin{itemize}
\item Labor: price $w$, quantity $N$
\item Land: price $r$, quantities $L^w, L^p$ for workers and production
\item Goods: price $p=1$, quantity $X$ 
\end{itemize}
\end{center}
\end{frame}
\begin{frame}
\frametitle{Equilibrium concept: Two key indifference conditions}
 In equilibrium, workers and firms are indifferent across cities with different levels of $s$ and endogenously varying wages $w(s)$ and rents $r(s)$:
\begin{align} 
c(w(s), r(s), s) &= 1 \label{eq_cond_cost} \\
V(w(s), r(s), s) &= V^0 \label{eq_cond_V}
\end{align}
where $V^0$ is the equilibrium level of indirect utility.
\bigskip
Specifically, in our example: \\
\textit{Given $s,\theta_W, \theta_F, \gamma, I, \alpha$, equilibrium is defined by local prices and quantities $\{w,r,N,L^w,L^p,X\}$ such that \eqref{eq_cond_cost} and \eqref{eq_cond_V} hold and land markets clear.}
\bigskip
N.B. We will mainly be focusing on prices: $w(s)$ and $r(s)$.
\end{frame}
%%%%%%%%%%%%%%%%%%%%%%%%%%%%%%%%%%%%%%%%%%%%%%%%%%%%
\subsection{Solving Model}
%%%%%%%%%%%%%%%%%%%%%%%%%%%%%%%%%%%%%%%%%%%%%%%%%%%%
\begin{frame}
\frametitle{Solving for effect of amenity changes on prices}
\begin{itemize}
\item Differentiate \eqref{eq_cond_cost} and \eqref{eq_cond_V} with respect to $s$ and rearrange, we have:
\begin{align*}
\begin{bmatrix}
c_w & c_r \\
V_w & V_r
\end{bmatrix}
\begin{bmatrix}
w'(s) \\
r'(s)
\end{bmatrix} = 
\begin{bmatrix}
-c_s\\
-V_s
\end{bmatrix}
\end{align*}
\item Solving for $w'(s), r'(s)$, we have
\begin{align*}
w'(s) = \frac{V_r c_s - c_r V_s}{c_r V_w - c_w V_r} \\
r'(s) = \frac{V_s c_w - c_s V_w}{c_r V_w - c_w V_r} 
\end{align*}
\item Note we can rewrite
\begin{align*}
c_r V_w - c_w V_r = \lambda L^p/X + \lambda l^c N/X = \lambda L/X =V_w L/X
\end{align*}
\end{itemize}
\end{frame}
\begin{frame}
\frametitle{Aside: example values for matrix elements}
\begin{align*}
c_w &= \alpha \frac{(s^{\theta_F})^{-1}w^\alpha r^{1-\alpha} \mathcal{A}}{w} \\
c_r &= (1-\alpha) \frac{(s^{\theta_F})^{-1}w^\alpha r^{1-\alpha} \mathcal{A}}{r} \\
c_s &= \theta_F \frac{ (s^{\theta_F})^{-1}w^\alpha r^{1-\alpha} \mathcal{A}}{s} \\
V_w &=s^{\theta_W} 1^{-\gamma} r^{-(1-\gamma)}  \Gamma \\
V_r &= -s^{\theta_W} 1^{-\gamma} r^{-(1-\gamma)}\Gamma (1-\gamma)\left(\frac{w + I}{r}\right)   \\
V_s &= \theta_W \frac{\left(  s^{\theta_W} 1^{-\gamma} r^{-(1-\gamma)}\Gamma  \left(w + I\right) \right)}{s}
\end{align*}
\end{frame}
\begin{frame}
\frametitle{Effect of amenity changes on prices}
\begin{itemize}
\item Price changes
\begin{align*}
w'(s) &= \frac{(V_rc_s - c_rV_s) X}{ \lambda L} \\
r'(s) &= \frac{(V_sc_w - c_sV_w) X}{ \lambda L} 
\end{align*}
\item Special cases of interest:
\begin{enumerate}
\item Amenity only valued by consumers: $\theta_F=0 \Rightarrow c_s = 0$
\item Amenity only has productivity effect: $\theta_W=0 \Rightarrow  V_s = 0$
\item Firms use no land $1-\gamma=0$ and amenity is non-productive $\theta_F=0$: $c(w(s))=1$, $c_r = c_s = 0$
\end{enumerate}
\end{itemize}
\end{frame}
%%%%%%%%%%%%%%%%%%%%%%%%%%%%%%%%%%%%%%%%%%%%%%%%%%%%
\section{Comparative Statics and Value of Amenities}
%%%%%%%%%%%%%%%%%%%%%%%%%%%%%%%%%%%%%%%%%%%%%%%%%%%%
\subsection{Price effects under different assumptions about amenities}
\begin{frame}
\frametitle{1. Amenity only valued by consumers: $\theta_F=0 \Rightarrow c_s = 0$}
\begin{itemize}
\item When $c_s = 0$, higher $s$ $\Rightarrow$ higher $r$, lower $w$
\item Workers are willing to pay more in land rents and receive less in wages to have access to higher levels of amenities
\end{itemize}
\begin{figure}
\includegraphics[height=.7\textheight]{../images/Zidar_rosenroback_fig1.pdf}
\end{figure}
\end{frame}
\begin{frame}
\frametitle{2. Amenity only valued by firms: $\theta_W=0 \Rightarrow  V_s = 0$}
\begin{itemize}
\item When $V_s = 0$, higher $s$ $\Rightarrow$ higher $r$ and higher $w$
\item Firms are willing to pay more in land rents and wages to access higher productivity due to amenities
\end{itemize}
\begin{figure}
\includegraphics[height=.7\textheight]{../images/Zidar_rosenroback_fig2.pdf}
\end{figure}
item]{Example: see economies of agglomeration}
\end{frame}
\begin{frame}{3. Firms don't use land nor value amenity}
\begin{itemize}
\item Firms don't use land ($\alpha=1$) nor value amenity ($\theta_F=0$)
\item Only production input is labor and firms are indifferent across locations, so wages must be the same across cities: $c(w(s))=1$\\$\;$\\
\item Since  $c_r = c_s = 0$, 
\begin{align*}
w'(s) &= 0 \\
r'(s) &= \frac{V_sc_w}{- c_wV_r} = \frac{V_s}{ l^c V_w}, \text{ since } V_r = -l^c V_w
\end{align*} 
\item So the rise in total cost of land for a worker living in a city with higher $s$ is 
\begin{align*}
l^c r'(s) &=  \frac{V_s}{ V_w}
\end{align*} 
\end{itemize}
\end{frame}
\begin{frame}{3. Firms don't use land nor value amenity}
\begin{itemize}
\item $\frac{V_s}{ V_w} =$ marginal WTP for a change in $s$ so the marginal value of a change in the amenity is ``fully capitalized" in rents
\end{itemize}
\begin{figure}
\includegraphics[height=.6\textheight]{../images/Zidar_rosenroback_fig3.pdf}
\end{figure}
$\frac{V_s}{ V_w} = \theta_W \frac{\left(w + I\right)}{s}$ is increasing in income, decreasing in level of amenities
\end{frame}
\begin{frame}{Valuing consumer amenities}
\begin{itemize}
\item General case: Start from equal-utility condition $V_0 = V(w(s), r(s), s)$
\begin{align}
0 &= V_ww'(s) + V_r r'(s) + V_s
\nonumber \\
\frac{V_s}{V_w} 
&=
l^cr'(s) - w'(s) \label{eq_WTP}
\end{align}
\item WTP for amenity is extra land cost for consumers plus lower wages
\item Zero-profit condition:
\begin{equation}
c_w w'(s) + c_r r'(s) + c_s = 0 \label{eq_cost_totaldif}
\end{equation}
\item When $c_s = 0$, 
$w'(s) = \frac{-c_r}{c_w} r'(s) = \frac{-L^p}{N} r'(s)$
\item Combine \eqref{eq_WTP} and \eqref{eq_cost_totaldif} to get the WTP of the $N$ people in a given city:
\begin{equation*}
N \frac{V_s}{V_w} = N l^cr'(s) + L^p r'(s)  = L r'(s)
\end{equation*}
Aggregate WTP is how the total value of all land changes as $s$ changes
\end{itemize}
\end{frame}
% -----------------------------------------
\begin{frame}{Inferring and valuing amenities}
\begin{columns}
\begin{column}{0.49\textwidth}
Cobb-Douglas preferences:
$V_0  = 
\Gamma
s^{\theta_W}
r^{-(1-\gamma)}
(w + I)$
implies
$
s^{\theta_W} = \frac{V_0}{\Gamma} \frac{r^{1-\gamma}}{w+I} 
$
\smallskip
More generally,
$\hat{s}_j^{\theta_W} \approx s_y \hat{p}_j - s_w (1-\tau')\hat{w}_j$
where $p$ are all local prices and $\tau'$ is the marginal tax rate,
$s_y$ and $s_w$ are national shares (my bad notation), and $\hat{x}_j = \frac{\textrm{d} x_j}{x}$ are local deviations
{\scriptsize \textcolor{gray}{
Albouy (2012) ``Are Big Cities Bad Places to Live?''
and
Albouy (2016) ``What are cities worth? Land rents, local productivity, and the total value of amenities''
}\par}
\end{column}
\begin{column}{0.40\textwidth}
\includegraphics[width=\textwidth]{../images/Albouy2012_fig1.pdf}
\end{column}
\end{columns}
\end{frame}
% -----------------------------------------
\begin{frame}{What's an amenity?}
Urban economists use the word ``amenity'' in two imperfectly aligned ways
(\href{https://tradediversion.net/2023/11/26/the-two-notions-of-amenities-in-spatial-economics/}{blog post})
\begin{enumerate}
\item Amenities are place-specific services/flows that are not explicitly transacted and hence do not appear in the budget constraint
\item Amenities are place-specific residuals because the researcher lacks expenditure/price data
\end{enumerate}
Traditional view (Diamond and Tolley 1982):
\begin{itemize}
\item Clean air, lack of severe snow storms, and sunny days (Roback 1982)
\end{itemize}
Recent literature on ``consumption amenities''
\begin{itemize}
\item Restaurants and retail (variety-adjusted price indices)
\end{itemize}
If an amenity is a non-tradable with crummy price data, then housing is an amenity in some empirical settings
\end{frame}
% -----------------------------------------
\begin{frame}{Endogenous amenities}
Thus far, $s$ was an exogenous characteristic of a location.
\begin{itemize}
\item Sunshine doesn't respond to population composition
\item Crime rates, school quality, and variety of restaurants are endogenous
\item Endogenous amenities mean endogeneity problems
\item See Milena Almagro's \href{https://m-almagro.github.io/UEA_Summer_School_2023.pdf}{UEA summer school lecture}
\end{itemize}
\end{frame}
% -----------------------------------------
\begin{frame}{Spatial equilibrium and the marginal resident}
Thus far, local labor supply is perfectly elastic
(all workers are indifferent at $V_0$)
\begin{itemize}
\item No notion of welfare or spatial inequality for workers
\item All workers adjust to shocks similarly
\item Incidence of shocks/amenities is on land prices
\end{itemize}
\medskip
The concept of spatial equilibrium is a no-arbitrage condition:
the marginal resident must be indifferent
\begin{itemize}
\item Moretti (2011) and Diamond (2016): discrete-choice problem with idiosyncratic preferences so there are inframarginal residents
\item Inferring and valuing amenities with heterogeneous individuals is harder
\end{itemize}
\end{frame}
% -----------------------------------------
\begin{frame}{The geographic concentration of economic activity}
{\footnotesize People are concentrated. Are industries concentrated? Yes.}
\begin{columns}
\begin{column}{0.49\textwidth}
\includegraphics[width=0.90\textwidth]{../images/Ahlfeldtetal2020_fig2b.pdf}\\
{\scriptsize Figure from \href{https://cep.lse.ac.uk/_NEW/PUBLICATIONS/abstract.asp?index=7318}{Ahlfeldt, Albers \& Behrens (2020)}\par}
\end{column}
\begin{column}{0.49\textwidth}
\includegraphics[width=1.05\textwidth]{../images/Economist_20160416_CNM985.png} \\
\end{column}
\end{columns}
\begin{itemize}{\small
	\item \href{https://doi.org/10.1086/262098}{Ellison and Glaeser (1997)} ``dartboard approach'' to address internal vs external economies
	\item \href{https://doi.org/10.1111/0034-6527.00362}{Duranton and Overman (2005)} for continuous space\par
}\end{itemize}
\end{frame}
% -----------------------------------------
\begin{frame}{Local increasing returns: Urbanization vs localization economies}
Agglomeration economies are the benefits of more people (Marshallian trinity)
\begin{enumerate}
	\item Lower costs of trading goods
	\item Lower costs of finding the right worker
	\item Lower costs of exchanging ideas
\end{enumerate}
\textit{Urbanization} economies are benefits of total scale, regardless of industry,
while 
\textit{localization} economies are benefits of more activity in same industry
(\href{https://doi.org/10.1016/S1574-0080(87)80009-3}{Henderson 1987})
\begin{itemize}
\item Henderson (1974) model with localization economies: ``because cities of different types specialize in the production of different traded goods,''
differences in scale elasticity generate differences in city size
\end{itemize}
Looking at growth effects,
\href{https://doi.org/10.1086/261856}{Glaeser, Kallal, Scheinkman, Shleifer (1992)} contrast
``Marshall-Arrow-Romer'' same-industry knowledge spillovers 
and
``Jane Jacobs'' between-industry knowledge spillovers
\end{frame}
% -----------------------------------------
\begin{frame}{What explains the intense concentration of economic activity?}
{Before we discuss these agglomeration economies, consider the alternative hypothesis:
exogenous ``locational fundamentals'' independent of population\par}
\vspace{3mm}
\begin{columns}
\begin{column}{0.56\textwidth}	
{Places may have advantages, regardless of population, that pull in more people\par}
{First-nature causes for agriculture and mining: what's in the ground?\par}
\includegraphics[height=0.40\textheight]{../images/HolmesStevens2004_fig4.pdf}
\end{column}
\begin{column}{0.43\textwidth}	
\begin{itemize}
\item Plausible:
Half the US population lives in coastal counties (oceans plus Great Lakes), only 13\% of land area
(\href{https://www.jstor.org/stable/40215936}{Rappaport, Sachs 2003})
\item Implausible: Locational fundamentals cannot explain Seattle's Microsoft-led resurgence nor Datang's sock specialization
\end{itemize}
\end{column}
\end{columns}
\end{frame}
% -----------------------------------------
\begin{frame}{Evidence of agglomeration economies}
\begin{itemize}{
	\item Estimate from observed spatial equilibrium, tackling endogeneity and sorting problems
	\item Test for multiple equilibria (sufficiently strong agglomeration forces imply multiple equilibria)
	\item \href{https://www.journals.uchicago.edu/doi/abs/10.1086/653714}{Greenstone, Hornbeck, Moretti (2010)} use ``million-dollar plants'' to estimate agglomeration economies
  (cf. \href{https://onlinelibrary.wiley.com/doi/abs/10.1111/ecin.12339}{Patrick 2016})
}\end{itemize}
See Combes and Gobillon - ``\href{https://www.sciencedirect.com/science/article/pii/B9780444595171000052}{The Empirics of Agglomeration Economies}'' (\textit{Handbook} 2015)
\end{frame}
% -----------------------------------------
\begin{frame}{Estimating agglomeration economies}
Revisit that plot of wages against city size
{\footnotesize (\href{https://www.nber.org/system/files/chapters/c7978/c7978.pdf}{Combes, Duranton, Gobillon, Roux 2010})}
\begin{itemize}
\item Endogeneity problem: exogenous productivity attracts more workers
\item Sorting problem: higher-wage workers may sort into larger cities
\item Instrumental variables for endogenous quantity of labor from geology or history
(e.g., century-ago population, soil quality, climate)
\item[] {\small \textcolor{gray}{Historical IVs requires persistence -- in labor supply, not labor demand}}
\item Sorting problem can be addressed using longitudinal data on workers, introducing worker fixed effects
{\small \textcolor{gray}{Like Abowd, Kramarz, Margolis (1999) regressions; see {de la Roca and Puga (2017)} and {Carry, Kleinman, Nimier-David (2025)}}}
\item One needs to tackle endogeneity and sorting simultaneously
\item CDGR 2010 say density elasticity of wages is about 0.2:
OLS is 0.05; IV is 0.04; worker FEs is 0.033; IV + worker FEs is 0.027
\end{itemize}
\end{frame}
% -----------------------------------------
\begin{frame}{Bleakley and Lin 2012: Portage and Path Dependence}
\begin{columns}
\begin{column}{.45\textwidth}
\includegraphics[width=0.95\textwidth]{../images/BleakleyLin2012_Figure4.pdf}
\end{column}
\begin{column}{.40\textwidth}
{\tiny Table 1: Proximity to Historical Portage Site and Contemporary Population Density\par}
\includegraphics[width=0.95\textwidth]{../images/BleakleyLin2012_Table1a.pdf} \\
\includegraphics[width=0.95\textwidth]{../images/BleakleyLin2012_Table2a.pdf}
\end{column}
\end{columns}
\end{frame}
% -----------------------------------------
\begin{frame}{Davis \& Weinstein: ``Bones, Bombs, and Breakpoints''}
{\small Does a temporary shock have permanent effects? After the Allied bombing in WWII, most cities returned to their rank in the distribution of city sizes within about 15 years \par}
\includegraphics[width=.40\textwidth]{../images/DavisWeinstein2002_fig1.pdf}
\includegraphics[width=.50\textwidth]{../images/DavisWeinstein2002_fig2.pdf}\\
{Also, \href{https://www.sciencedirect.com/science/article/pii/S0304387810000817}{Miguel and Roland (\textit{JDE} 2011)}: ``even the most intense bombing in human history did not generate local poverty traps in Vietnam''\par}
\end{frame}
% -----------------------------------------
\begin{frame}{When and where does history matter?}
\href{https://doi.org/10.1016/j.regsciurbeco.2020.103628}{Lin and Rauch (2022)}:
\begin{itemize}
\item ``with a few important exceptions, major temporary shocks do not appear to permanently affect the fortunes of cities or large regions''
\item ``there is perhaps more evidence of history dependence in the location and scale of city-industries and even more evidence of history dependence in neighborhood sorting and segregation''
\item ``What factors might distinguish city-industries or neighborhoods from regions in making history dependence and multiplicity more empirically relevant?
These factors may provide guidance on when history matters, and when it does not.''
\end{itemize}
\end{frame}
% -----------------------------------------
\begin{frame}{Classic model of agglomeration: Henderson (1974)}
``The Sizes and Types of Cities'' addresses basic, fundamental questions about a system of cities in general equilibrium
\begin{itemize}
\item Why do cities exist? 
\only<2>{\textcolor{gray}{``because there are technological economies of scale in production or consumption''}}
\item Why do cities of different sizes exist?
\only<2>{\textcolor{gray}{
``because cities of different types specialize in the production of different traded goods''}}
\item Are cities too large or too small?
\only<2>{\textcolor{gray}{a stability argument says that cities tend to be too large}}
\end{itemize}
\end{frame}
% -----------------------------------------
\begin{frame}{Optimal city size vs equilibrium city size}
\begin{itemize}
\item If productivity rises with size but congestion costs eventually dominate, then the average return to city size is a single-peaked function
\item Spatial equilibrium equates average returns across cities
\item Henderson (1974): An equilibrium on the rising part of the average-return curve is \textit{unstable}:
adding a person would make that city more attractive
\item Thus, stable equilibria have cities that are too large
\item See \href{https://doi.org/10.1016/j.jue.2018.08.004}{Albouy, Behrens, Robert-Nicoud, Seegert (2019)}
on central planner versus city-level planners versus ``self-organization'' (free migration)
\item Heterogeneous fundamentals mean that average returns differ when marginal returns are equal:
equalizing average returns undoes optimal allocation
\item[] \textcolor{gray}{Optimal sizing is not the same as optimal policy, which introduces regional transfers (trade deficits); see \href{https://doi.org/10.1016/bs.hesreg.2025.06.005}{Fajgelbaum and Gaubert (2025)}}
\end{itemize}
\end{frame}
% -----------------------------------------
\begin{frame}{Developing-economy cities}
\begin{itemize}{\small
\item World Bank projects 2.7 billion more urban residents in developing economies by 2050
\item Cities still require agglomeration and dispersion forces, but the technologies and conditions might differ
\item Gollin, Jedwab, Vollrath ``Urbanization with and without Industrialization'' (2016) on `consumption cities' in resource exporters
\item Jedwab, Loungani, Yezer: cities in rich countries are tall and sprawl; in poor countries they crowd
\item Typically, urban wages are much higher than rural wages
\item \href{https://doi.org/10.1016/j.jue.2020.103301}{Gollin, Kirchberg, Lagakos (2021)}: observed private consumption and amenities are higher in urban areas of 20 SSA countries (they avoid using prices)
\item Henderson and Turner (2020): higher incidence of lifestyle diseases, poorer child health outcomes and greater exposure to
crime
\item See Bryan, Glaeser, Tsivanidis (2019) ``\href{https://doi.org/10.1146/annurev-economics-080218-030303}{Cities in the Developing World}''
}\end{itemize}
\end{frame}
% -----------------------------------------
\begin{frame}{This week has omitted spatial linkages like trade costs}
County-level presence of four industries in 2007
\begin{center}
\includegraphics[height=0.73\textheight]{../images/GervaisJensen_fig1.jpg} \\
{\footnotesize \href{https://doi.org/10.1016/j.jinteco.2019.03.003}{Gervais and Jensen (2019)}, Figure 1}
\end{center}
\end{frame}
% -----------------------------------------
\begin{frame}{Next week}
\begin{itemize}
\item Up next: Quantitative spatial models
\item Read Krugman (1991) before class so I can cover quickly
\end{itemize}
\end{frame}
% -----------------------------------------
\end{document}
