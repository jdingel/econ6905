\documentclass[10pt,notes=hide,aspectratio=169]{beamer}
%Jonathan Dingel; PhD trade course

% PACKAGES
\usepackage{graphics}  % Support for images/figures
\usepackage{graphicx}  % Includes the \resizebox command
\usepackage{url}	   % Includes \urldef and \url commands
\usepackage{soul}      % Includes the underline \ul command
%\usepackage{framed}	   % Includes the \framed command for box around text
\usepackage{booktabs} %\toprule,\bottomrule
%\usepackage{natbib}
\usepackage{bibentry}  % Includes the \nobibliography command
\usepackage{bbm}       %
%\usepackage{pgfpages}  %Supports "notes on second screen" option for beamer
\usepackage{verbatim}  %Supports comments
\usepackage{tikz}		%Supports graphing/drawing
\usepackage{pgfplots} %Supports graphing/drawing
\usepackage{amsfonts}  % Lots of stuff, including \mathbb 
\usepackage{amsmath}   % Standard math package
\usepackage{amsthm}    % Includes the comment functions
\usepackage{physics}

% CUSTOM DEFINITIONS
\def\newblock{} %Get beamer to cooperate with BibTeX
\linespread{1.2}
\hypersetup{backref,pdfpagemode=FullScreen,colorlinks=true,linkcolor=blue,urlcolor=blue}
\newtheorem{proposition}{Proposition}
\newtheorem{assumption}{Assumption}
\newtheorem{condition}{Condition}

% IDENTIFYING INFORMATION
\title{Topics in Trade}
\author{Jonathan I. Dingel}
\date{Fall \the\year}

% BEAMER TEACHING STUFF
\setbeamertemplate{navigation symbols}{}  %Turn off navigation bar

% THEMATIC OPTIONS
\definecolor{columbiablue}{RGB}{185,217,235}  %Columbia blue defined at https://visualidentity.columbia.edu/branding
\definecolor{columbiadarkblue}{RGB}{0,48,135}  %Columbia dark blue defined at https://visualidentity.columbia.edu/branding
\setbeamercovered{transparent=5}
\setbeamercolor{frametitle}{fg=columbiadarkblue}
\setbeamercolor{item}{fg=columbiadarkblue}
\usefonttheme{serif}
\setbeamercolor{button}{bg = white,fg = columbiadarkblue}
\setbeamercolor{button border}{fg = columbiadarkblue}

\setbeamertemplate{footline}{\begin{center}\textcolor{gray}{Dingel -- Topics in Trade -- Week 4 -- \insertframenumber}\end{center}}
\begin{document}
% -----------------------------------------
\begin{frame}[plain]
\begin{center}
\large
\textcolor{columbiadarkblue}{ECON G6905\\
Topics in Trade\\ 
Jonathan Dingel\\
Spring \the\year, Week 2}
\vfill 
\includegraphics[width=0.4\textwidth]{../images/Columbia_logo.png}
\end{center}
\end{frame}
% -----------------------------------------
% -----------------------------------------
\begin{frame}{Today}
\begin{itemize}
	\item The gravity equation
	\item Trade costs
	\item Gains from trade
\end{itemize}
The \textit{Handbook} chapters on the gravity equation and gains from trade are excellent.
Rely on them.
\end{frame}
% -----------------------------------------
\begin{frame}{The gravity equation}
General gravity (Allen Arkolakis Takahashi 2014) [not 2020 \textit{JPE}]:
\begin{equation*}
X_{ij} = G \times S_i \times M_j \times \phi_{ij}
\end{equation*}
Structural gravity (Head and Mayer 2014) [true by $Y_i = \sum_j X_{ij}$]:
\begin{align*}
X_{ij} &= \underbrace{\frac{Y_i}{\Omega_i}}_{S_i} \times \underbrace{\frac{X_j}{\Phi_j}}_{M_j} \times \phi_{ij}\\
\Omega_i &= \sum_{\ell} \frac{X_{\ell}}{\Phi_{\ell}} \phi_{i \ell}
\quad 
\Phi_j  = \sum_{\ell} \frac{Y_{\ell}}{\Omega_{\ell}} \phi_{\ell j}
\end{align*}
Naive gravity (``gold medal error'' of \href{https://www.nber.org/papers/w12516}{Baldwin and Taglioni}):
\begin{equation*}
X_{ij} = G \times Y_i^a \times Y_j^b \times \phi_{ij}
\end{equation*}
Recall last week's Armington model (this is general, is it structural?):
\begin{equation*}
X_{ij}	= \frac{Y_i^{1-\sigma}}{Q_i^{1-\sigma}} \frac{X_j}{P_j^{1-\sigma}}\tau_{ij}^{1-\sigma}
\end{equation*}
\end{frame}
% -----------------------------------------
\begin{frame}{Gravity fits the cross section well}
Naive gravity,
akin to physics's force $\propto \frac{\text{mass}_i \times \text{mass}_j}{\text{distance}^{2}_{ij}}$,
does very well with GDP as mass:
\begin{center}
\includegraphics[width=0.9\textwidth]{../images/HeadMayer2014_fig1.pdf}
\end{center}
\end{frame}
% -----------------------------------------
\begin{frame}{Gravity fits the cross section well}
A broad notion of ``distance'' does well:
\includegraphics[width=0.9\textwidth]{../images/HeadMayer2014_fig2.pdf}
\end{frame}
% -----------------------------------------
\begin{frame}{Is gravity's goodness of fitness impressive? (Lai and Trefler 2002)}
\begin{equation*}
X_{ij} = {\frac{Y_i}{\Omega_i}} \times {\frac{X_j}{\Phi_j}} \times \phi_{ij}
\end{equation*}
\vspace{-4mm}
\begin{itemize}
	\item $Y_i$ is not terribly interesting because $Y_i = \sum_{j} X_{ij}$ is an identity
	\item $\frac{X_{ij}}{X_j} = \frac{S_{i} \phi_{ij}}{\Phi_j} = \frac{S_{i} \phi_{ij}}{\sum_{\ell} S_{\ell} \phi_{\ell j}}$ has substantive content, since it says that preferences are homothetic (budget shares don't depend on $X_j$), but see Deaton and Muellbauer quote
	\item The meat of the structural models is in $\frac{\phi_{ij}}{\Omega_i \Phi_j}$ (IIA and recursive multilateral resistance)
	\item \href{http://www-2.rotman.utoronto.ca/~dtrefler/papers/Lai_Trefler_2002.pdf}{Lai and Trefler (2002)} estimate monopolistic-competition model via NLLS using panel data on manufacturing trade while assuming $\phi_{ij}$ depends only on tariffs
	\item {Lai and Trefler (2002): ``do changes in tariffs over time predict changes in bilateral trade flows? No. The data are completely at odds with the model's core behavioural and general equilibrium predictions about  [tariff changes].''\par}
	\item {Head and Mayer (2014): ``Nevertheless, the standard CES model is too entrenched — partly because it is so useful! — that it will not be abandoned based on one finding.''\par}
\end{itemize}
\end{frame}
% -----------------------------------------
\begin{frame}{Evaluating gravity models based on predicting changes}
\href{https://doi.org/10.1146/annurev-economics-080614-115502}{Kehoe, Pujolas, Rossbach (2017)} on applied general-equilibrium models:
\begin{itemize}
\item ``AGE models do not have a good track record in predicting the impact of trade reforms on production and trade flows by industry''
\item Typical AGE: ``the model exactly matches the data in the base period'' and ``Cobb-Douglas or fixed-coefficient forms, which makes it easy to calibrate factor and demand intensities directly from IO tables'' (cross-sectional fit is a given)
\end{itemize}
You need to get the shocks and the elasticities right.
\begin{itemize}
\item Key question: Which components of inferred/calibrated trade costs will respond to changes in trade policy?
\item Kehoe's path forward: ``need to incorporate product-level data on bilateral trade relations by industry and better model how trade reforms lower bilateral trade costs''
\item Adao, Costinot, Donaldson (2023):
``if the causal impact of policy changes in the researcher's model is correct, then the difference between observed and predicted changes should be equal to the causal impact of other shocks''
\end{itemize}
\end{frame}
% -----------------------------------------
\begin{frame}{Kehoe, Pujolas, Rossbach (2017) on NAFTA}
\begin{itemize}
\item {\small ``if AGE models cannot get one of the largest and most significant trade reforms in recent history correct, then there may be reasons to doubt the reliability of the AGE models currently being used for evaluating trade policy''\par}
\item {\small ``Although the focus of their study is on using their model to disentangle the welfare implications of NAFTA, we can also evaluate the accuracy of the model in matching actual changes in trade flows following the implementation of NAFTA.''\par}
\end{itemize}
\includegraphics[width=1.0\textwidth]{../images/KehoePujolasRossbach_2017_tab2.png}
\end{frame}
% -----------------------------------------
\begin{frame}{Allen, Arkolakis, Takahashi (2020): ``Universal Gravity''}
Setup
\begin{itemize}
\item Models with iceberg bilateral trade costs, CES aggregate demand, CES aggregate supply, exogenous trade deficits
\item Aggregate supply elasticity is non-zero with roundabout production or labor mobility
\item Not covered: models with variable elasticities, multiple factors of production, or intertemporal decisions
\end{itemize}
Results
\begin{itemize}
\item Sufficient conditions for existence, uniqueness, and interiority
\item Counterfactual predictions of these gravity models depend on the elasticities of supply and demand (and observed data)
\item \textit{Local} responses of endogenous variables to trade shocks via matrix inversion
(interpret as shock propagation)
(vs EHA)
\item Country-level supply elasticity of 68 is very high
\end{itemize}
Paper may seem abstract but you will find yourself re-reading it often
\end{frame}
% -----------------------------------------
\begin{frame}{Gravity in urban economics}
Importing the gravity model into urban economics:
\begin{itemize}
	\item Gravity for commuting flows: 
	\href{https://doi.org/10.1016/0191-2615(83)90023-1}{Anas (1983)},
	Ahlfeldt Redding Sturm Wolf (2015), Monte Redding Rossi-Hansberg (2018), Owens Rossi-Hansberg Sarte (2020), Severen (2019), Tsivanidis (2019), Dingel and Tintelnot (2023)
	\item Gravity for consumption in the city:
	Davis Dingel Monras Morales (2019), Allen Arkolakis Li (2015), Miyauchi Nakajima Redding (2021)
\end{itemize}
These settings differ slightly from canonical trade model: 
\begin{itemize}
	\item Model of discrete choice rather than CES demand (see Anderson, de Palma, Thisse book)
	\item Trade flows need not balance due to commuting (workplace income is residential expenditure)
	\item Zeros are far more pervasive (Dingel \& Tintelnot 2023)
	\item Commonalities: estimation with two-way HDFE, recursive market-access terms
\end{itemize}
\end{frame}
% -----------------------------------------
% -----------------------------------------
\begin{frame}{Estimating gravity regressions}
\begin{itemize}
	\item Approximating $\Omega_i$ and $\Phi_j$: ``remoteness'' or ``market potential''
	\item Estimating $\Omega_i$ and $\Phi_j$: NLLS, SILS, or PPML for structural gravity
	\item Fixed effects:
	$\ln X_{ij} = \ln S_i + \ln M_j + \ln \phi_{ij}$
	\item Double ratios (tetrads) [relate to \href{http://www-personal.umich.edu/~alandear/glossary/h.html}{Head-Ries index}]:
	\begin{equation*}
	\frac{X_{ij}/X_{ik}}{X_{\ell j}/X_{\ell k}} = \frac{\phi_{ij}/\phi_{ik}}{\phi_{\ell j}/\phi_{\ell k}} 
	\end{equation*}
	\item Triple ratios (Caliendo \& Parro 2015) [\href{https://tradediversion.net/2020/04/26/do-customs-duties-compound-non-tariff-trade-costs-not-in-the-us/}{not quite right}]:
	\begin{align*}
	\phi_{ij} = [(1+t_{ij})d_{ij}^\delta]^{\epsilon} \quad d_{ij}=d_{ji} \ \forall  i,j \\
	\frac{X_{ij}X_{hi}X_{jh}}{X_{hj}X_{ih}X_{ji}}
	=
	\left(
	\frac{(1+t_{ij})(1+t_{hi})(1+t_{jh})}{(1+t_{hj})(1+t_{ih})(1+t_{ji})} 
	\right)^{\epsilon}
	\end{align*}
\end{itemize}
The keys to informative estimation are (1) not being naive, (2) distinguishing the trade elasticity from reduced-form coefficients, (3) handling zeros appropriately, and (4) recognizing the endogeneity of trade policy
\end{frame}
% -----------------------------------------
\begin{frame}{Know your estimand: the trade elasticity or the distance elasticity?}
\begin{itemize}
\item
Consider the OLS regression with two-way high-dimensional fixed effects given by the CES Armington model ($\epsilon = \sigma -1$):
$$\ln X_{ij} = \ln S_i + \ln M_j - \epsilon \ln \tau_{ij} + u_{ij}$$
\item
If you assume the trade costs are a function of, say, distance with 
$\ln \tau_{ij} = \beta \ln \text{distance}_{ij}$,
then you would estimate
$$\ln X_{ij} = \ln S_i + \ln M_j - \epsilon \beta \ln \text{distance}_{ij} + u_{ij}$$
You will recover $\epsilon \beta$. Do not mistake this for $\epsilon$.
\item
You recover the trade elasticity with observed trade costs and pass-through assumptions.
If $\ln \tau_{ij} = \beta_1 \ln \text{distance}_{ij} + \beta_2 \ln \text{tariff}_{ij}$
and you assume that $\beta_2 = 1$, then $\epsilon \beta_2 = \epsilon$.
\end{itemize}
\end{frame}
% -----------------------------------------
\begin{frame}{Can you distinguish trade costs and preference shifters?}
Consider an Armington model with asymmetric preferences:
\begin{align*}
	U_j &= \left(\sum_{i} \beta_{ij} q_{ij}^{(\sigma-1)/\sigma}\right)^{\sigma/(\sigma-1)}
	\\
	\Rightarrow
	\frac{X_{ij}}{X_j} &= \beta_{ij} \left(\frac{p_{ij}}{P_j}\right)^{1-\sigma} 
	=
	\frac{w_i^{1-\sigma}}{P_j^{1-\sigma}}\beta_{ij}\tau_{ij}^{1-\sigma}
\end{align*}
Now you have a structural gravity setting in which $\phi_{ij}=\beta_{ij}\tau_{ij}^{1-\sigma}$.
\begin{itemize}
	\item Could bilateral preference shifters be correlated with common language, colonial status, import tariffs, or distance? Of course. 
	\item \href{https://www.sciencedirect.com/science/article/pii/S0022199606000225}{Blum and Goldfarb (2006)}: {\small ``Americans are more likely to visit websites from nearby countries, even controlling for language, income, immigrant stock, etc. Furthermore, we show that this effect only holds for taste-dependent digital products, such as music, games, and pornography.''\par}
\end{itemize}
\end{frame}
% -----------------------------------------
\begin{frame}{Logs vs levels and the Poisson PML estimator}
\href{https://personal.lse.ac.uk/tenreyro/LGW.html}{Silva and Tenreyro (REStat 2006)} raise logs vs levels issue:
\begin{align*}
X_{ij} &= \alpha_0 Y_i^{\alpha_1}  Y_j^{\alpha_2}  D_{ij}^{\alpha_3} \eta_{ij}
\\
\ln X_{ij} &= \alpha_0 +{\alpha_1} \ln  Y_i   +{\alpha_2}\ln  Y_j +{\alpha_3} \ln D_{ij} + \ln \eta_{ij}
\end{align*}
\vspace{-8mm}
\begin{itemize}
	\item The levels regression requires $\mathbb{E}\left(\eta_{ij}|Y_i,Y_j,D_{ij}\right)=1$.
	\item What does the logs regression require of $\ln \eta_{ij}$?
\end{itemize}
\smallskip \pause
Stack the fixed effects and log distance in a vector $\mathbf{Z}_{ij}$
and
stack their associated coefficients in a vector $\beta$.
Contrast OLS and PPML first-order conditions:
\begin{align*}
  \text{Ordinary least squares:} \quad
  &
  \sum_{i,j} \left[\ln X_{ij} - \beta \mathbf{Z}_{ij} \right] \mathbf{Z}_{ij}= 0
  \\
  \text{Poisson pseudo maximum likelihood:} \quad
  &
  \sum_{i,j} \left[X_{ij} - \exp\left(\beta \mathbf{Z}_{ij}\right)\right] \mathbf{Z}_{ij}= 0
\end{align*}
The OLS FOC has a log difference; the PPML FOC has a level difference.
{\href{http://www-personal.umich.edu/~ssotelo/research/Sotelo_MPMLE.pdf}{Sotelo (2019)}: ``when using PML methods to estimate gravity models, specifying the dependent variable as shares or as levels amounts to assigning different weights to each importer country.''\par}
\only<2>{}
\end{frame}
% -----------------------------------------
\begin{frame}{Zeros and PPML}
How to handle zeros (on the left side)?
\begin{itemize}
	\item I cannot put $\ln(0)$ on the LHS
	\item We use the PPML estimator to handle zeros
\end{itemize}
\medskip
By the way, one can only generate $X_{ij}=0$ in ``structural gravity'' by $\phi_{ij} = 0$, but this runs up against the triangle inequality for trade costs
\begin{equation*}
X_{ij} = {\frac{Y_i}{\Omega_i}} \times {\frac{X_j}{\Phi_j}} \times \phi_{ij}
\end{equation*}
\medskip
It is very natural to use the PPML estimator for commuting flows,
since they are literally count data and the PPML estimator's FOC coincides with the logit MLE's FOC
(see Dingel and Tintelnot 2023 for more comments).
\end{frame}
% -----------------------------------------
\begin{frame}{PPML estimator and structural gravity}
\href{https://doi.org/10.1016/j.jinteco.2015.05.005}{Fally (2015)}:
The Poisson pseudo-maximum-likelihood estimator automatically satisfies adding-up constraints of structural gravity
\begin{itemize}
\item 
NLLS imposes (using observed $Y_{\ell}$ and $X_{\ell}$)
\begin{equation*}
\hat{\Phi}_j = \sum_{\ell} \frac{Y_{\ell}}{\hat{\Omega}_{\ell}} \hat{\phi}_{\ell j}
\quad
\hat{\Omega}_i = \sum_{\ell} \frac{X_{\ell}}{\hat{\Phi}_{\ell}} \hat{\phi}_{i \ell}
\end{equation*}
\item 
Generally, fixed-effect estimation is consistent with the structural-gravity framework
if we use fitted output ($\hat{Y}_{i} = \sum_{\ell} \hat{X}_{i\ell}$) and fitted expenditures ($\hat{X}_{\ell} = \sum_{i} \hat{X}_{i\ell}$)
instead of observed output and expenditures
\item 
PPMLE's estimated FE deliver model-consistent $\hat{\Phi}$ and $\hat{\Omega}$ because
fitted output equals observed output and fitted expenditures equal observed expenditures
\item
Poisson is the only PML estimator with this property
\end{itemize}
Example:
\href{https://ideas.repec.org/a/eee/inecon/v62y2004i1p53-82.html}{Redding and Venables (2004)}
estimate bilateral gravity regression to recover $\hat{\Phi}$ and $\hat{\Omega}$
and relate them to GDP per capita
\end{frame}
% -----------------------------------------
\begin{frame}{Estimation in practice}
The homework assignment explores practical issues:
\begin{itemize}
\item Speed consequences of how you handle the high-dimensional fixed effects
\item Selection-bias consequences of how you handle the zeros
\item Speed consequences of choice of software (Stata vs R vs Julia)
\end{itemize}
\end{frame}
% -----------------------------------------
\begin{frame}{Trade costs}
Trade costs $\phi_{ij}$ are the frictions that make international and intranational trade distinct (from integrated GE) and interesting, yet we struggle to measure them
\begin{itemize}
	\item Tariffs (easy to define, but go download TRAINS data)
	\item Transportation costs (money + time + trade finance)
	\item Communication costs
	\item Contractual frictions
\end{itemize}
Trade costs are important:
\begin{itemize}
	\item Almost essential to rationalizing observed prices and quantities
	\item Obstfeld and Rogoff (2001) propose that trade frictions key to six puzzles in international macro (\href{https://www.sciencedirect.com/science/article/pii/S0165188916301014}{Eaton, Kortum, Neiman 2016})
	\item Key to evaluating welfare and government investment in transportation infrastructure, from roads to ports
\end{itemize}
\end{frame}
% -----------------------------------------
\begin{frame}{Are trade costs large or is the world integrated?}
Arguments for large trade costs:
\begin{itemize}
	\item Exchange declines dramatically with geographic distance ({\small \href{https://www.mitpressjournals.org/doi/10.1162/rest.90.1.37}{Head and Disdier 2008} for countries, \href{http://faculty.chicagobooth.edu/jonathan.dingel/research/thedeterminantsofqualityspecialization.pdf}{Dingel 2017 table C.1} for US cities})
	\item Large price gaps (from within cities to across countries) aren't arbitraged away
\end{itemize}
Arguments for trade costs not being a big deal:
\begin{itemize}
	\item MFN tariffs are in the single digits for most of world economy
	\item The cost of moving manufactured goods fell 90\% over the twentieth century (Glaeser and Kohlhase 2004)
\end{itemize}
Is $\phi_{ij}$ a good description of international business frictions?
\begin{itemize}
	\item Contrast ad valorem tariffs with specific tariffs
	\item Contrast border barriers with income differences or regulatory differences
\end{itemize}
\end{frame}
% -----------------------------------------
\begin{frame}{Inferring trade costs}
Three strategies:
\begin{itemize}
	\item Measure trade costs directly
	\item Infer trade costs from observed exchange volumes
	\item Infer trade costs from observed price gaps
\end{itemize}
\end{frame}
% -----------------------------------------
\begin{frame}{Direct measurement: Transport prices}
\href{https://www.aeaweb.org/articles?id=10.1257/jep.21.3.131}{Hummels (JEP 2007)} has lots of direct measures:
\begin{center}
\includegraphics[width=.9\textwidth]{../images/Hummels_2007_aircosts}
\end{center}
Classic example is Limao and Venables (2001), who got price quotes from World Bank's freight forwarder for sending a 40-foot container from Baltimore to 64 destination cities.
Land costs more than sea.
\end{frame}
% -----------------------------------------
\begin{frame}{Container shipping}
\begin{itemize}
	\item First used for a Newark-Houston shipment in 1956 %http://pup.princeton.edu/chapters/s8131.html
	\item In 1964, 10 US ports and 3 Australian ports are containerized %http://www.giselarua.com/diffusion-of-containerization.html
	\item By 1977, 68 countries had adopted the technology
\end{itemize}
\begin{center}\begin{figure}
	\includegraphics[width=0.3\textwidth]{../images/container_port} \hfill
	\includegraphics[width=0.3\textwidth]{../images/container_train} \hfill
	\includegraphics[width=0.3\textwidth]{../images/container_truck}
\end{figure}  \end{center}
\end{frame}
% -----------------------------------------
\begin{frame}{Direct measures of trade costs}
See Anderson and van Wincoop (JEL 2004) for survey
\begin{itemize}
	\item Endogenous price quotes for freight and insurance (above plus the US and Australian import data)
	\item UNCTAD TRAINS for tariffs (convert specific to ad-valorem equivalent?) (coverage concerns) (ask about preference utilization)
	\item UNCTAD TRAINS for non-tariff barriers (do quotas bind?) (coverage concerns)
	\item World Bank's Doing Business measures for port/border costs
\end{itemize}
In addition to the limitations of these individual measures,
the overriding concern is that these observables cannot capture all trade costs related to coordination, contracts, intermediaries' market power, uncertainty and just-in-time production, etc
\end{frame}
% -----------------------------------------
\begin{frame}{Inferring from observed exchanges: Gravity residuals}
\href{https://www.aeaweb.org/articles?id=10.1257/aer.91.4.858}{Head and Ries (2001)} suggest backing out the freeness of trade by assuming
$\phi_{ii}=1 \ \forall i$ (normalization) and $\phi_{ij}=\phi_{ji} \ \forall i,j$ (symmetry)
\begin{columns}
\begin{column}{.52\textwidth}
\begin{equation*}
X_{ij} = {\frac{Y_i}{\Omega_i}} \times {\frac{X_j}{\Phi_j}} \times \phi_{ij}
\quad \Rightarrow \quad
\hat{\phi}^k_{ij}=\sqrt{\frac{X^k_{ij}X^k_{ji}}{X^k_{jj}X^k_{ii}}}
\end{equation*}
\begin{itemize}
	\item Requires data on internal trade $X_{ii}^k$, which does not appear in customs records
	\item Requires assumption on tastes and trade elasticity to turn $\phi_{ij}$ into trade costs
	\item See \href{https://doi.org/10.1016/j.eeh.2009.07.001}{Jacks, Meissner, Novy 2010} and \href{https://doi.org/10.1111/j.1465-7295.2011.00439.x}{Novy 2013} for related applications
\end{itemize}
\end{column}
\begin{column}{.45\textwidth}
\includegraphics[width=\textwidth]{../images/Novy2013_tab1.pdf}
\end{column}
\end{columns}
\end{frame}
% -----------------------------------------
\begin{frame}{Inferring from observed exchanges: Feyrer (2009)}
\begin{center}
\includegraphics[width=.8\textwidth]{../images/Feyrer2009_tab1.pdf}
\end{center}
\vspace{-5mm}
\begin{columns}
\begin{column}{.62\textwidth}
\begin{itemize}
	\item The Suez Canal offer shortest Asia-Europe sea route and today handles $\sim 8\%$ of world trade
	\item Egypt closed the Suez Canal 1967-1975
\end{itemize}
\end{column}
\begin{column}{.36\textwidth}
\includegraphics[width=\textwidth]{../images/Feyrer2009_tab2.pdf}
\end{column}
\end{columns}
\end{frame}
% -----------------------------------------
\begin{frame}{Inferring from observed exchanges across products}
\begin{itemize}
\item \href{https://www.mitpressjournals.org/doi/10.1162/rest.2009.11498}{Djankov, Freund, Pham (REStat 2010)} on ``Trading on Time''
\begin{itemize}
	\item Each day of delay at port reduces trade by $\sim 1\%$
	\item Delays have a relatively greater impact on exports of time-sensitive goods, such as perishable crops
\end{itemize}
\item \href{https://www.sciencedirect.com/science/article/pii/S0022199615001403}{Bernhofen, El-Sahlid, Kneller (JIE 2016)} ``Estimating the effects of the container revolution on world trade''
\begin{itemize}
	\item Diff-in-diffs design using staggered intro of container facilities across countries and product variation in container usage
	\item North-North trade shows cumulative ATE of 17\%  after five years
\end{itemize}
\item \href{http://www.jstor.org/stable/25760347}{Horta\c{c}su, Mart\'{i}nez-Jerez, Douglas (2009)} ``The Geography of Trade in Online Transactions: Evidence from eBay and MercadoLibre''
\begin{itemize}
	\item Sample of a quarter million eBay listings scraped Feb-May 2004
	\item Gravity regressions show larger same-city effect for product categories with more buyer dissatisfaction
\end{itemize}
\end{itemize}
\end{frame}
% -----------------------------------------
\begin{frame}{Mis-specified gravity and trade elasticities}
\href{https://doi.org/10.1086/344805}{Yi (JPE 2003)} motivates his paper with two puzzles:
\begin{columns}
\begin{column}{.49\textwidth}
\begin{itemize}
\item Observed trade elasticity (wrt tariffs) of $\sim 20$ much higher than prediction of standard models
\item This elasticity became much higher, non-linearly, around the 1980s. Why?
\end{itemize}
\end{column}
\begin{column}{.5\textwidth}
\includegraphics[width=\textwidth]{../images/Yi2003_fig1b.pdf}
\end{column}
\end{columns}
\vspace{2mm}
Answer is ``vertical specialization'' (tradable intermediate inputs)
\begin{itemize}
	\item Yi (JPE 2003): The possibility of international fragmentation of production raises the trade-to-tariff elasticity.
	\item Yi (AER 2010): Similarly for ``border effect'' estimates
\end{itemize}
\end{frame}
% -----------------------------------------
\begin{frame}{Yi (2003): Vertical specialization story}
Yi (2003) introduces a two-country DFS-style model with vertical specialization. Model lacking without VS misses two puzzle.
\begin{columns}
\begin{column}{.46\textwidth}
\begin{itemize}
	\item Puzzle 1: if goods cross border $N$ times,
	tariff costs is $(1+\tau)^N$ rather than $1+\tau$
	\item Puzzle 2: high tariffs $\to$ intermediates are not traded.
	Elasticity will be initially low (as if $N = 1$) and then suddenly higher (as if $N > 1$).
\end{itemize}
\end{column}
\begin{column}{.52\textwidth}
\includegraphics[width=\textwidth]{../images/Yi2003_fig10.pdf}
\end{column}
\end{columns}
Also see \href{https://www.aeaweb.org/articles?id=10.1257/aer.20150956}{Adao, Costinot, and Donaldson (AER 2017)} on relaxing CES/IIA part of gravity
\end{frame}
% -----------------------------------------
\begin{frame}{Inferring trade costs from price gaps}
If place $i$ exports homogeneous good $u$ to destination $j$, 
the no-arbitrage condition for prices is
$$\ln p_{jt}(u) -\ln p_{it}(u) = \ln \tau_{ijt}(u)$$
\vspace{-6mm}
\begin{itemize}
	\item Need homogeneous products (so there is arbitrage opportunity)
	\item Need to know that $i$ is selling to $j$ (otherwise: only an inequality)
	\item Need competitive market (otherwise: need pass-through rate)
\end{itemize}
Donaldson (2018) on salt in India differentiated by geographic origin tackles 1 and 2 explicitly\\
\href{https://www.nber.org/papers/w21439}{Atkin and Donaldson (2015)} work on 1-3
\begin{itemize}
	\item CPI micro-data from Ethiopia and Nigeria (and US)
	\item Neglecting 1-3 would underestimate cost of distance
	\item Distance effect 4-5 times larger than in US
	\item Intermediate captures the majority of the surplus
\end{itemize}
\end{frame}
% -----------------------------------------
\begin{frame}{Border effects in prices}
\begin{itemize}
	\item Law of one price (LOP): Identical goods sell at same price
	\item \href{https://www.aeaweb.org/articles?id=10.1257/aer.101.6.2450}{Gopinath Gourinchas Hsieh Li (2011)} use price data from a multinational retail chain
\end{itemize}
	\begin{minipage}{.45\textwidth} \begin{center}\begin{figure} \includegraphics[width=1.0\textwidth]{../images/Gopinathetal_2011_fig8_crop}\end{figure} Oregon-Washington \end{center} \end{minipage}
	\begin{minipage}{.45\textwidth} \begin{center}\begin{figure} \includegraphics[width=1.0\textwidth]{../images/Gopinathetal_2011_fig1_crop}\end{figure} Canada-USA \end{center} \end{minipage} \\ \vspace{0.1in}
	{\scriptsize Perrier Sparkling Natural Mineral Water, 25-ounce bottles, log average price. Store distance to the border is positive for Oregon/US, negative for Washington/CA. \par }
Markups or costs? Authors say variation in retail prices related to exchange rate changes due to change in wholesale costs
\end{frame}
% -----------------------------------------
\begin{frame}{Price gaps over time (1/3)}
\begin{center}\includegraphics[width=0.8\textwidth]{../images/Jensen_QJE2007_figure4} \\ \footnotesize{Robert Jensen, ``\href{http://qje.oxfordjournals.org/content/122/3/879.abstract}{The Digital Provide},'' \textit{QJE}, 2007}\end{center}
\end{frame}
% -----------------------------------------
\begin{frame}{Price gaps over time (2/3)}
\begin{minipage}{.47\textwidth}\vspace{0pt}\includegraphics[width=\textwidth]{../images/Steinwender_wp2015_fig1}\end{minipage}
\begin{minipage}{.47\textwidth}\vspace{0pt}\includegraphics[width=\textwidth]{../images/Steinwender_wp2015_fig2}\end{minipage}
\begin{center}\footnotesize{Steinwender, ``\href{https://www.aeaweb.org/articles?id=10.1257/aer.20150681}{Real Effects of Information Frictions: When the States and the Kingdom Became United}'', \textit{AER} 2018} \end{center}
\end{frame}
% -----------------------------------------
\begin{frame}{Price gaps over time (3/3)}
\includegraphics[width=\textwidth]{../images/wired_ragingbull_map.jpg}
\begin{center}\footnotesize{Adler, ``\href{https://www.wired.com/2012/08/ff_wallstreet_trading/}{Raging Bulls}'', \textit{Wired}, 2012} \hfill Budish \textit{et al} ``\href{https://faculty.chicagobooth.edu/eric.budish/research/HFT-FrequentBatchAuctions.pdf}{HFT Arms Race}'', \textit{QJE}, 2015 \end{center}
\end{frame}
% -----------------------------------------
\begin{frame}{``The'' trade elasticity and gains from trade in the Armington model}
\begin{align*}
X_{ij}
&= \frac{w_i^{1-\sigma}}{A_i^{1-\sigma}} \frac{X_j}{P_j^{1-\sigma}}\tau_{ij}^{1-\sigma} 
\qquad \tau_{ii}=1	
\\
&\Rightarrow
\left(\frac{w_i/A_i}{P_i}\right)^{\sigma-1}
=
\frac{X_i}{X_{ii}}
\quad
\Rightarrow
\frac{w_i}{P_i} 
=A_i \left(\frac{X_{ii}}{X_i}\right)^{\frac{-1}{\sigma-1}}
\end{align*}
\begin{itemize}
\item The partial elasticity
$\frac{\partial X_{ij}/X_{jj}}{\partial \tau_{i'j}}$
is a constant
$1-\sigma$ for $i'=i$ and zero otherwise
\item In the ``ACR'' class of models,
``the'' trade elasticity is a common constant
$\epsilon = -\frac{\partial X_{ij}/X_{jj}}{\partial \tau_{i'j}}$ for $i'=i$ and zero otherwise
	\item This is $\theta$ in Eaton \& Kortum (2002) and $\sigma-1$ in Armington. Microfoundations differ.
	\item Compare welfare $w_i/P_i$ in trade equilibrium to autarky welfare
	$$
	\frac{(w_i / P_i)_{\text{trade}}}
	{(w_i / P_i)_{\text{autarky}}}
	=
	\left(\frac{\pi_{ii}}{1}\right)^{-1/\epsilon} 
	$$
	\item This formula says US gains from trade are about 1\% when $\epsilon\approx 5$
\end{itemize}
\end{frame}
% -----------------------------------------
\begin{frame}{Arkolakis, Costinot, and Rodriguez-Clare (AER 2012)}
``New Trade Models, Same Old Gains?'' is a very influential and insightful paper.
\begin{itemize}
\item Within a class of gravity models, $\widehat{\lambda}_{ii}$ and $\epsilon$ are sufficient statistics for welfare analyses of changes in trade costs $\hat{\mathbf{\tau}}$ or market size $\hat{L}$
\begin{equation*}
\widehat{W}_{i}=\widehat{\lambda}_{ii}^{\frac{-1}{\epsilon}}
\end{equation*}
where $W_{i}$ denotes welfare, $\lambda_{ii}$ denotes the domestic share of expenditure, and $\widehat{x}=\frac{x^{\prime}}{x}$ denotes proportional changes
\item Notice that the gains-from-trade result is just a corollary of this since $\widehat{\lambda}_{ii}=\lambda _{ii}$ when moving from autarky to trade
	\item The paper provides sufficient conditions for this result and shows the class contains many models
	\item Optimistic view: welfare predictions of Armington model are more robust than you might have thought
	\item Pessimistic view: within that class of models, micro-level data do not matter
\end{itemize}
\end{frame}
% -----------------------------------------
\begin{frame}{Are these gains too small?}
\href{https://doi.org/10.1016/j.jinteco.2015.07.002}{Ossa (JIE 2015)} extends ACR (2012) environment to multiple sectors
\begin{itemize}
\item While imports in the average industry do not matter too much, imports
in some industries are critical to the functioning of the economy
\item The aggregate formula $G_{j}=\lambda _{j}^{-\frac{1}{\epsilon }}$
extends to $G_{j}=\lambda _{j}^{-\frac{1}{\tilde{\epsilon}_{j}}}$ with
multiple industries, where $1/\tilde{\epsilon}_{j}$ is a weighted average
of the industry-level $1/\epsilon _{s}$
\item Only a few low-elasticity industries are needed to generate large
gains from trade which is missed when using the aggregate formula
\item The industry-level formula predicts about three times larger gains
from trade than the aggregate formula amounting to an average 55.9\%
\end{itemize}
\end{frame}
% -----------------------------------------
\begin{frame}{Estimates of income gains from panel variation in trade costs}
Feyrer (2009, 2018) revisits the Frankel and Romer (1999) idea using 
panel settings with changes in trade costs due to the Suez Canal and rise of air transport
\begin{itemize}
	\item Zero stage: Predicted trade is sum of predicted changes in bilateral trade flows due to Suez shock or rise of air transport
	\item First stage: Total trade on predicted trade
	\item 2SLS regression of GDP per capita on total trade
\end{itemize}
Two papers differ in terms of gradual vs sudden shocks and roles of sea vs air transport \\
\begin{itemize}
	\item Elasticity of GDP per capita wrt trade for Suez is about 0.2
	\item Elasticity of GDP per capita wrt trade for air transport is in 0.5-0.7 range
\end{itemize}
\end{frame}
% -----------------------------------------
\begin{frame}{Whither ``trade theory with numbers''?}
Costinot and Rodriguez-Clare (2014): \\
{\small
new quantitative trade models put more emphasis on transparency and less emphasis on realism. The idea is to construct middle-sized models that are rich enough to speak to first-order features of the data, like the role of country size and geography, yet parsimonious enough so that one can credibly identify its key parameters and understand how their magnitude affects counterfactual analysis.\par}
\noindent {\href{https://www.aeaweb.org/articles?id=10.1257/aer.20101199}{Donaldson (2018)} on Indian railroads:} \\
{\small a little over one-half of the total impact of the railroads estimated in column 1 can be explained by the mechanism of enhanced opportunities to trade according to comparative advantage, represented in the model\par}
\noindent {\href{https://www.annualreviews.org/doi/abs/10.1146/annurev-economics-080213-041015}{Donaldson (2015)} on Feyrer (2009, 2018):} \\
{\small How can we explain the magnitude of the effects in these papers\dots
[LATE, SUTVA, residual OVB]
\dots
or simply the possibility that the stylized and parametric quantitative gravity models—especially those with just one sector and no IO linkages—against which the empirical results here are being compared are too pessimistic about the size of the gains from trade.
\par}
\end{frame}
% -----------------------------------------
\begin{frame}[plain]
Next week: Multiple factors of production
\end{frame}
% -----------------------------------------
\end{document}
