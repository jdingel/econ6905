\documentclass[11pt,notes=hide,aspectratio=169]{beamer}
%Jonathan Dingel; PhD trade course

% PACKAGES
\usepackage{graphics}  % Support for images/figures
\usepackage{graphicx}  % Includes the \resizebox command
\usepackage{url}	   % Includes \urldef and \url commands
\usepackage{soul}      % Includes the underline \ul command
%\usepackage{framed}	   % Includes the \framed command for box around text
\usepackage{booktabs} %\toprule,\bottomrule
%\usepackage{natbib}
\usepackage{bibentry}  % Includes the \nobibliography command
\usepackage{bbm}       %
%\usepackage{pgfpages}  %Supports "notes on second screen" option for beamer
\usepackage{verbatim}  %Supports comments
\usepackage{tikz}		%Supports graphing/drawing
\usepackage{pgfplots} %Supports graphing/drawing
\usepackage{amsfonts}  % Lots of stuff, including \mathbb 
\usepackage{amsmath}   % Standard math package
\usepackage{amsthm}    % Includes the comment functions
\usepackage{physics}

% CUSTOM DEFINITIONS
\def\newblock{} %Get beamer to cooperate with BibTeX
\linespread{1.2}
\hypersetup{backref,pdfpagemode=FullScreen,colorlinks=true,linkcolor=blue,urlcolor=blue}
\newtheorem{proposition}{Proposition}
\newtheorem{assumption}{Assumption}
\newtheorem{condition}{Condition}

% IDENTIFYING INFORMATION
\title{Topics in Trade}
\author{Jonathan I. Dingel}
\date{Fall \the\year}

% BEAMER TEACHING STUFF
\setbeamertemplate{navigation symbols}{}  %Turn off navigation bar

% THEMATIC OPTIONS
\definecolor{columbiablue}{RGB}{185,217,235}  %Columbia blue defined at https://visualidentity.columbia.edu/branding
\definecolor{columbiadarkblue}{RGB}{0,48,135}  %Columbia dark blue defined at https://visualidentity.columbia.edu/branding
\setbeamercovered{transparent=5}
\setbeamercolor{frametitle}{fg=columbiadarkblue}
\setbeamercolor{item}{fg=columbiadarkblue}
\usefonttheme{serif}
\setbeamercolor{button}{bg = white,fg = columbiadarkblue}
\setbeamercolor{button border}{fg = columbiadarkblue}

\setbeamertemplate{footline}{\begin{center}\textcolor{gray}{Dingel -- Topics in Trade -- \semester -- Week 2 -- \insertframenumber}\end{center}}
\begin{document}
% -----------------------------------------
\begin{frame}[plain]
\begin{center}
\large
\textcolor{columbiadarkblue}{ECON G6905\\
Topics in Trade\\ 
Jonathan Dingel\\
\semester, Week 2}
\vfill 
\includegraphics[width=0.5\textwidth]{../images/Columbia_logo.png}
\end{center}
\end{frame}
% -----------------------------------------
\begin{frame}{Outline of today}
\begin{enumerate}
	\item Gains from trade (Samuelson 1939)
	\item Comparative advantage (Deardorff 1980)
	\item Ricardian model with two countries and continuum of goods (Dornbusch Fisher Samuelson 1977)
\end{enumerate}
\end{frame}
% -----------------------------------------
\begin{frame}{Neoclassical environment}
``Neoclassical'' trade models: perfect competition, constant returns to scale, and no distortions
\begin{itemize}
\item There are $n=1,...,N$ countries, each populated by $h=1,...,H_{n}$
households
\item There are $g=1,...,G$ goods
\begin{itemize}
\item Output vector in country $n$:
$y^{n} \equiv \left(y_{1}^{n},...,y_{G}^{n}\right) $
\item Consumption vector of household $h$ in country $n$:
$c^{nh} \equiv \left(c_{1}^{nh},...,c_{G}^{nh}\right) $
\item Goods price vector in country $n$:
$p^{n} \equiv \left(p_{1}^{n},...,p_{G}^{n}\right) $
\end{itemize}
\item There are $f=1,...,F$ factors in fixed supply
\begin{itemize}
\item The endowment vector in country $n$:
$v^{n} \equiv \left(v_{1}^{n},...,v_{F}^{n}\right) $
\item The factor price vector in country $n$:
$w^{n} \equiv \left(w_{1}^{n},...,w_{F}^{n}\right) $
\end{itemize}
\end{itemize}
\end{frame}
% -----------------------------------------
\begin{frame}{Supply and the revenue function}
Revenue function of country $n$ is
\begin{equation*}
r^{n}\left( p^{n},v^{n}\right) =
\max_{y^{n}}\left\{ p^{n} \cdot y^{n}|\left(y^{n},v^{n}\right) \text{ feasible}\right\}
\end{equation*}
Lots of handy properties in a neoclassical environment (see Dixit \& Norman 1980 p.31-36)
\begin{itemize}
	\item Revenue function summarizes all relevant properties of technology
	\item Under perfect competition, $y^n$ maximizes $r^n$
	\item Derivatives w.r.t. goods prices give supply curves
	\begin{equation*}
	\nabla_p r^{n}\left( p^{n},v^{n}\right) = y^{n}\left( p^{n},v^{n}\right)
	\end{equation*}
	\item Derivatives w.r.t endowments give inverse factor demand curves
	\begin{equation*}
	\nabla_v r^{n}\left( p^{n},v^{n}\right) = w^{n}\left( p^{n},v^{n}\right)
	\end{equation*}
\end{itemize}
\end{frame}
% -----------------------------------------
\begin{frame}{Demand and the expenditure function}
Expenditure function for household $h$ in country $n$ with utility function $u^{nh}$ is defined as
\begin{equation*}
e^{nh}\left( p^{n},u^{nh}\right) =
\min_{c^{nh}}\left\{ p^{n} \cdot c^{nh}|\text{ }
u^{nh}\left( c^{nh}\right) \geq u^{nh}\right\}
\end{equation*}
Familiar properties from consumer theory (see Dixit \& Norman 1980 p.59-64)
\begin{itemize}
	\item Optimization implies that $e^{nh}\left( p^{n},u^{nh}\right) = p^{n}\cdot c^{nh}$ so
	\begin{equation*}
	\nabla_p e^{nh}\left(p^{n},u^{nh}\right) = c^{nh}\left(p^{n},u^{nh}\right)
	\end{equation*}
	\item $e^{nh}\left( p,u\right)$ is increasing in $u$
\end{itemize}
\end{frame}
% -----------------------------------------
\begin{frame}{Gains from trade (representative household)}
\begin{itemize}
	\item The revealed-preference argument employs only the revenue and expenditure funtions
	\item Start with case of a single/representative household
	\item Drop the $hn$ notation; use $a$ to denote autarky vectors
\end{itemize}
\end{frame}
% -----------------------------------------
\begin{frame}{Gains from trade (representative household)}
\textit{In a neoclassical trade model with one representative household per country,
all households are (weakly) better off under free trade than autarky.}
Proof:
\begin{align*}
e\left( p,u^{a}\right) &\leq p \cdot c^{a} , &\text{\quad by definition of the expenditure function} \\
&=p \cdot y^{a} , &\text{\quad by market clearing under autarky} \\
&\leq r\left( p,v\right)  , &\text{\quad by definition of the revenue function} \\
&=e\left( p,u\right)  , &\text{\quad by budget and trade balance}
\end{align*}
Since expenditure is increasing in utility, we conclude that $u \geq u^a$.\\
\begin{itemize}
	\item Weak inequalities to accommodate kinks in IC or PPF
	\item Gains from trade relative to autarky does not rank trading equilibria
	\item Draw the two-good case
\end{itemize}
\end{frame}
% -----------------------------------------
\begin{frame}{Gains from trade (lump-sum transfers)}
\begin{itemize}
\item With multiple households, trade is likely to generate winners and losers but we can show the winners win more than the losers lose
\item Formally, there exist feasible domestic lump-sum transfers that make every household better off under free trade than autarky
\item Reintroduce the household superscript notation:
\begin{itemize}
\item $c^{ah}$ and $c^{h}$ denote the vector of consumptions of household $h$ under autarky and free trade
\item $v^{h}$ denotes the vector of endowments of household $h$ under autarky and free trade
\item $u^{ah}$ and $u^{h}$ denote the utility levels of household $h$ under autarky and free trade
\item $\tau ^{h}$ denotes the lump-sum transfer (in trade equilibrium) from government to household $h$ (lump-sum tax if negative)
\end{itemize}
\end{itemize}
\end{frame}
% -----------------------------------------
\begin{frame}{Gains from trade (lump-sum transfers)}
\textit{In a neoclassical trade model with multiple households per country, there
exist domestic lump-sum transfers such that all households are (weakly)
better off under free trade than autarky.}
\begin{itemize}{\small
\item Set transfers such that each household can still afford its autarky
consumption bundle under free trade
\begin{equation*}
\tau ^{h}=\left( p-p^{a}\right) \cdot c^{ah}-\left( w-w^{a}\right) \cdot v^{h}
\end{equation*}
\item These are feasible (government revenue is non-negative)
}\end{itemize}
\begin{align*}
- \sum\nolimits_{h}\tau ^{h} &=&\left( p^{a}-p\right)  \cdot \sum\nolimits_{h}c^{ah}-\left(
w^{a}-w\right)  \cdot \sum\nolimits_{h}v^{h}\text{, by definition of }\tau ^{h} \\
&=&\left( p^{a}-p\right)  \cdot y^{a}-\left( w^{a}-w\right)  \cdot v\text{, market
clearing under autarky} \\
&=&-p \cdot y^{a}+w \cdot v\text{, income equals expenditure under autarky} \\
&\geq &-r\left( p,v\right) +w \cdot v\text{, from definition of revenue function}
\\
&=&0\text{, income equals expenditure under free trade}
\end{align*}
\end{frame}
% -----------------------------------------
\begin{frame}{Gains from trade (commodity and factor taxation)}
\begin{itemize}
	\item Domestic lump-sum transfers are not typically feasible
	\item Let government set specific taxes on goods and factors so that, e.g., the price of good $g$ is $p_g^{\text{consumer}} = p_g + \tau_g$
	\item Set $\tau_g = p^a_g - p_g$ and $\tau_f = w - w^a_f$ so household is indifferent
	\item Government revenue is positive (similar to above):
	\begin{align*}
	T &= \sum\nolimits_g \tau_g \sum\nolimits_h c^{ah}_g + \sum\nolimits_f \tau_f \sum\nolimits_h v^h_f \\
	&= (p^a - p) \cdot \sum_h c^{ah} - (w^a - w) \cdot \sum_h v^h \geq 0
	\end{align*}
	\item Remember that you cannot just rebate the revenue, you need to change a consumer price to achieve the strict improvement (Kemp \& Wan \textit{JIE} 1986)
	\item There's probably a Pareto-improving direction of change in consumer prices in the neighborhood of the autarky price vector (Dixit \& Norman \textit{JIE} 1986)
\end{itemize}
\end{frame}
% -----------------------------------------
\begin{frame}{Introducing comparative advantage}
\begin{itemize}
	\item ``Comparative advantage'' -- differences in autarkic relative marginal costs -- is the basis for trade
	\item If autarkic relative prices are identical, then ``zero trade'' is a free-trade equilibrium allocation at those prices
	\item Theory of comparative advantage (2x2 case): If two countries engage in trade, each will export the good in which it has lower relative marginal cost prior to trade
\end{itemize}
\end{frame}
% -----------------------------------------
\begin{frame}{Law of comparative advantage for free-trade equilibria}
\textit{In a neoclassical trade model with representative households
with autarkic and free-trade prices $p^{na}$ and $p$, 
$\left(p-p^{na}\right) \cdot t^{n} \geq 0$, 
where $t^{n}=y^{n}-c^{n}$ is the vector of country $n$'s net exports.}
\hfill Proof (Deardorff 1980, \href{https://www.jstor.org/stable/40440263}{1994}):
\begin{eqnarray*}
p^{na}\cdot y^{n} &\leq &r\left( p^{na},v^{n}\right) \text{, \ \ by def of revenue function} \\
p^{na}\cdot c^{n} &\geq &e\left( p^{na},u^{n}\right) \text{, \ \ by def of expenditure function} \\
p^{na}\cdot t^{n} &\leq &r\left( p^{na},v^{n}\right) -e\left( p^{na},u^{n}\right) \text{, \ \ by previous two inequalities} \\
e\left( p^{na},u^{n}\right) &\geq &e\left( p^{na},u^{na}\right) \text{,
since }u^{n}\geq u^{na}\text{ and } \frac{\partial e\left( p,u\right)}{\partial u} \geq 0 \\
p^{na}\cdot t^{n} &\leq &r\left( p^{na},v^{n}\right) -e\left(
p^{na},u^{na}\right) \text{, by previous two inequalities} \\
p^{na}\cdot t^{n} &\leq &0\text{, since autarkic income equals autarkic expenditure}
\\
p \cdot t^{n} &=&0\text{, by balanced trade} \\
\left( p-p^{na}\right) \cdot t^{n} &\geq &0\text{, by combining previous two expressions }
\end{eqnarray*}
\end{frame}
% -----------------------------------------
\begin{frame}{Comments on general validity of law of CA}
\begin{itemize}
	\item $\left(p-p^{na}\right) \cdot t^{n}$ is a correlation result because covariance of two vectors is simply their inner product if one of the vectors (i.e., normalized $p-p^{na}$) sums to zero
	\item $\left(p-p^{na}\right) \cdot t^{n}$ depends on both autarky and free-trade prices
	\item Corollaries 3 and 4 of Deardorff (1980) state result in terms of only autarkic prices (requires world market-clearing assumption)
	\item Deardorff (1980) covers the case of costly trade, distinguishing	 consumer price $p^q$, producer price $p^t$, and world price $p^w$
	\item Core of the proof is that $p^{na} \cdot t^n \leq 0$: gains from trade mean that consumption is at most barely attainable under autarky ($p^a y^n \leq p^a c^n$) 
\end{itemize}
\end{frame}
% -----------------------------------------
\begin{frame}{Deardorff (1980) environment}
Notation differs: country $i$, ``natural trade'' $n$, quantity $Q$
{\footnotesize
\begin{align}
(Q^i,T^i) \in F^i \Rightarrow (Q^i+T^i,0) \in F^i &\quad \text{non-negative trade costs}\\
&\text{ local non-satiation } \\
(Q^{ai},0) \in F^i &\quad \text{autarky eqlbm feasible} \\
p^{ai} Q^{ai} \geq p^{ai} Q \ \forall (Q,0) \in F^i &\quad \text{profit maximizing} \\
U^i(Q^{ai}) \geq U^i(Q) \ \forall Q: p^{ai} Q \leq p^{ai} Q^{ai} &\quad \text{utility maximizing} \\
(Q^{ni},T^{ni}) \in F^i &\quad \text{trade eqlbm feasible} \\
p^{qi} Q^{ni} +p^{ti}T^{ni} \geq p^{qi} Q +p^{ti}T \ \forall (Q,T) \in F^i &\quad \text{profit maximizing} \\
U^i(Q^{ni}) \geq U^i(Q) \ \forall Q: p^{qi} Q \leq p^{qi} Q^{ni} &\quad \text{utility maximizing} \\
p^w T^{ni} = 0  &\quad \text{balanced trade} \\
(p^w_g - p^{ti}_g) T_g^{ni} \geq 0 \ \forall g &\quad \text{``natural trade''} \\
\sum_i T^{ni} = 0 &\quad \text{world market clears}
\end{align}
{\footnotesize Turn to proof on pages 948-949}
}
\end{frame}
% -----------------------------------------
\begin{frame}{How should we take comparative advantage to data?}
Canonical $2\times2$ insight we teach in principles classes isn't amenable to empirical investigation.
Now we have a general formulation.
\begin{itemize}
	\item Good news: {``while the classical theory predicts only the direction and not the magnitude of trade, it nonetheless permits one to infer a negative infer a negative correlation between relative costs and net exports'' (Deardorff 1980) \par}
	\item Bad news: {``relative antarky prices are not observable. Almost all countries have engaged in trade throughout history, so that there is no experience with autarky from which to draw data.'' (Deardorff 1984) \par}
	\item Long-standing approach: Use model with observable primitives (technology and factor endowments) to infer autarkic prices. Joint test of CA and model.
\end{itemize}
\end{frame}
% -----------------------------------------
\begin{frame}{Bernhofen and Brown: Sometimes we observe autarky}
\begin{itemize}
	\item Japan had ``sudden and complete opening up to international trade in the 1860s'' due to US military
	\item Bernhofen and Brown use this a natural experiment to test law of comparative advantage
	\item Key prediction is $p^{na} \cdot t^n \leq 0$, but we never simultaneously observe autarky prices $p^a$ and trade-equilibrium net exports $t^n$
	\item ``the comparison of autarky with free trade should be understood as a comparison between two alternative histories, not as a change that takes place over time'' (Helpman and Krugman 1985)
	\item If preferences and technology in 1868-1875 (observed trade years) are same as those in 1858, hope that $p^a$ from 1858 is valid measure of $p^a$ for 1868-1875
	\item Test $p^{na} \cdot t^n \leq 0$ by computing $p^{a,1858} \cdot t^{n,1868}$ (roughly speaking)
\end{itemize}
\end{frame}
% -----------------------------------------
\begin{frame}{Assumptions}
\begin{itemize}
	\item Read Section III of BB (2004) on the assumptions that this is a relevant and valid natural experiment
	\begin{enumerate}
		\item Competitive economy in autarky
		\item Japanese are price takers in international markets
		\item Exports not subsidized
		\item PPF shifts from 1859 to 1868 not biased toward importables (if $\mathbf{p_2^a} = \mathbf{p_1^a} + \mathbf{\epsilon}$, then $\mathbf{\epsilon} \mathbf{T} \leq 0$ is sufficient for $\mathbf{p}_1^a \mathbf{T}\leq 0 \Rightarrow \mathbf{p}_2^a \mathbf{T}\leq 0$)
	\end{enumerate}
	\item What is the test of $p^{na} \cdot t^n \leq 0$?
	\begin{itemize}
		\item Alternative hypothesis $H_1$: $p^{na} \cdot t^n > 0$
		\item Alternative hypothesis $H_2$: $\Pr\left(p^{na} \cdot t^n \leq 0\right)=\frac{1}{2}$
	\end{itemize}
\end{itemize}
\end{frame}
% -----------------------------------------
\begin{frame}{Correlation of $p^{a}$ and $t$ in 1869}
\begin{center}\includegraphics[height=0.9\textheight]{../images/BernhofenBrown2004_fig4.pdf}\end{center}
\end{frame}
% -----------------------------------------
\begin{frame}{Inner product of $p^a$ and $t$, year by year}
\begin{center}\includegraphics[height=0.87\textheight]{../images/BernhofenBrown2004_tab2.pdf}\end{center}
\vspace{-4mm}
{\footnotesize ``The p-value is exactly 1/256, where 1/256 is the probability of obtaining eight heads in eight tosses with a balanced coin.''\par}
\end{frame}
% -----------------------------------------
\begin{frame}{Comments on Bernhofen and Brown (2004)}
\begin{itemize}
	\item What is the autarky price of a good not produced in autarky?
	\item Plot of prices changes $p-p^{a}$ in Figure 4 okay if $p\cdot t = 0$ by balanced trade (check Figure 3)
	\item What is the power of this test in the absence of a competing theory?
	\item Computation of p-value assumes independence of observations
	\item Does $p^a \cdot t$ exhibit a trend?
\end{itemize}
\end{frame}
% -----------------------------------------
\begin{frame}{Taxonomy of neoclassical trade models}
\begin{itemize}
	\item In a neoclassical model, comparative advantage (lower relative autarkic marginal cost) is the basis for trade
	\item Autarky costs might reflect demand or supply differences
	\item Demand differences typically neglected by assumption
	\item Supply-side explanations for autarkic cost differences:
	\begin{itemize}
		\item Technological differences (Ricardian theory)
		\item Factor-endowment differences (Ricardo-Viner and Heckscher-Ohlin)
		\item Increasing returns to scale (beyond neoclassical scope)
	\end{itemize}
	\item In theoretical models,
	the roles of factor proportions and technological differences are typically kept separate:
	\begin{itemize}
		\item Ricardian model assumes one factor of production
		\item Factor-proportions theory typically assumes common production function
	\end{itemize}
\end{itemize}
\end{frame}
% -----------------------------------------
\begin{frame}{Technology vs factors}
Different models for different questions?\footnote{\scriptsize Jones \& Neary (1980): ``positive trade theory uses a variety of models, each one suited to a limited but still important range of questions''}
\begin{itemize}
	\item What is the effect of rising Chinese productivity on US real wages? (DFS 1977, \href{https://tradediversion.net/2011/03/29/ricardo-revisited-back-to-2004/}{its interpretation}, \href{https://faculty.chicagobooth.edu/chang-tai.hsieh/research/hsieh_ossa_jie.pdf}{Hsieh and Ossa 2016})
	\item What are distributional consequences of trade? Need multiple factors
\end{itemize}
Interaction of technology and factors might matter
\begin{itemize}
	\item Does fact of intra-industry trade necessitate increasing returns in theory? No, says \href{https://www.sciencedirect.com/science/article/pii/0022199695013833}{Davis (1995)}.
	\item \href{https://ideas.repec.org/a/eee/inecon/v82y2010i2p152-167.html}{Chor (JIE 2010)} and \href{https://ideas.repec.org/a/eee/inecon/v82y2010i2p137-151.html}{Morrow (JIE 2010)} 
	\item Factor-biased technical change
\end{itemize}
\end{frame}
% -----------------------------------------
\begin{frame}{Canonical Ricardian model of DFS 1977}
\begin{itemize}
	\item Two countries, Home and Foreign; asterisk denotes latter
	\item One factor of production, call it labor, endowed in amounts $L$ and $L^*$ and paid wages $w$ and $w^*$
	\item[] [efficiency units, and see ``Hicksian composite'']
	\item Unit labor costs for good $z$ are $a(z)$ and $a^*(z)$
	\item WLOG, order goods such that $A(z) \equiv \frac{a^*(z)}{a(z)}$ is decreasing
	\item Home has comparative advantage in low-$z$ goods
\end{itemize}
\end{frame}
% -----------------------------------------
\begin{frame}{Recall the concepts and insights of two-good case}
Consider two goods, $z$ and $z'$
\begin{itemize}
	\item Home has \textit{absolute advantage} in $z$ when $a(z) < a^*(z)$
	\item Home has \textit{comparative advantage} in $z$ when its relative autarkic marginal cost is lower: $\frac{a(z)}{a(z')} < \frac{a^*(z)}{a^*(z')}$
\end{itemize}
What is equilibrium pattern of specialization?
\begin{itemize}
	\item For factor markets to clear, Home cannot be least-cost provider of both goods. It is not possible that
	\begin{equation*}w a(z) < w^{*}a^{*}(z) \text{ and } w a(z') < w^*a^*(z')\end{equation*}
	\item If Home has comparative advantage in $z$, it must be that 
	\begin{equation*}\frac{a(z)}{a^*(z)} \leq \frac{w^{*}}{w} \leq \frac{a(z')}{a^*(z')}\end{equation*}
	\item Absolute advantage determines wages; comparative advantage determines specialization
\end{itemize}
\end{frame}
% -----------------------------------------
\begin{frame}{Pattern of specialization in DFS 1977}
\begin{itemize}
	\item Let $p(z)$ denote the price of good $z$ under free trade
	\item Profit maximization and factor-market clearing require
		\vspace{-2mm}
		\begin{equation*}p(z) \leq w a(z)   \text{ and } p(z) \leq w^{*} a^{*}(z) \end{equation*}
		with equality if produced in Home or Foreign, respectively
	\item There exists $\tilde{z}$ such that Home produces all of $z<\tilde{z}$ and Foreign produces all of $z>\tilde{z}$ (proof by contradiction)
	\item Countries specialize according to comparative advantage
	\item Define relative wage $\omega \equiv \frac{w}{w^*}$
	\item Given relative wages, cost-minimizing specialization is $[0,\tilde{z}]$ at Home and $[\tilde{z},1]$ in Foreign such that $A(\tilde{z}) = \omega$
	\item Continuum of $z$ and $A'(\tilde{z})<0$ makes $\tilde{z} = A^{-1}(\omega)$
	\item Second curve in $z$-$\omega$ space requires demand
\end{itemize}
\end{frame}
% -----------------------------------------
\begin{frame}{Cobb-Douglas preferences}
Identical Cobb-Douglas preferences with expenditure shares $b(z)$
\begin{align*}
b\left(z\right) 	 =\frac{p\left(z\right) c\left(z\right) }{wL}
&=b^{*}\left(z\right) =\frac{p^{\ast}\left(z\right) c^{\ast}\left(z\right)}{w^{\ast}L^{\ast}}
\\
\int_{0}^{1}b\left(z\right) dz 
& = \int_{0}^{1}b^{\ast}\left(z\right) dz
=1
\end{align*}
Denote the share of expenditure on Home goods by $\theta\left(\tilde{z}\right)$
\begin{equation*}
\theta \left( \tilde{z}\right) =\int_{0}^{\tilde{z}}b\left( z\right) dz\text{
\ \ \ and \ \ \ }1-\theta \left( \tilde{z}\right) =\int_{\tilde{z}%
}^{1}b\left( z\right) dz
\end{equation*}
Trade balance then requires 
$\theta \left(\tilde{z}\right) w^{*}L^{*}
=\left[1-\theta \left(\tilde{z}\right) \right] wL$,
which implies
\begin{equation*}
\omega =\frac{\theta \left(\tilde{z}\right) }{1-\theta \left(\tilde{z}%
\right) }\frac{L^{*}}{L}\equiv B\left(\tilde{z}\right)
\end{equation*}
\end{frame}
% -----------------------------------------
\begin{frame}{Gains from trade in DFS 1977}
\begin{itemize}
	\item It's a neoclassical model with a free-trade equilibrium,
	so prior results apply: there are gains from trade
	\item Given functional forms, we can speak to magnitudes
	\begin{equation*}
		\ln (U/L) = {\ln w} - \int_{0}^{1} b(z) \ln p(z) dz
	\end{equation*}
	\item Choose $w=1$ in both autarky and trade equilibria
	\begin{equation*}
	 \int_{0}^{1} b(z) \ln a(z) dz \text{ vs } \int_{0}^{\tilde{z}} b(z) \ln a(z) dz + \int_{\tilde{z}}^{1} b(z) \ln \left[w^{*} a^{*}(z)\right] dz 
	\end{equation*}
	\item Connect to \href{http://www-personal.umich.edu/~alandear/glossary/d.html\#DoubleFactoralTermsOfTrade}{double-factoral terms of trade} and dissimilarity as source of GFT
\end{itemize}
\end{frame}
% -----------------------------------------
\begin{frame}{Comparative statics for population growth}
An increase in $L^{*}/L$ moves $B(\tilde{z})$ schedule. See DFS Figure 2:
\begin{itemize}
	\item Equilibrium is a decrease in $\bar{z}$ and increase in $\bar{\omega}$
	\item At initial $\bar{\omega}$, larger $L^{*}/L$ means trade surplus for Home, so its terms of trade must improve
	\item Goods produced at Home before and after shock have no change in price
	\item Goods produced in Foreign before and after shock become cheaper for Home consumers
	\item What about the goods that switch?
	\item Each good is produced using CRS, but akin to country-level DRS
\end{itemize}
\end{frame}
% -----------------------------------------
\begin{frame}{Comparative statics for technical change}
What happens with each of the following shocks?
\begin{itemize}
	\item Uniform global technical progress:  $\textrm{d}\ln a(z) = \textrm{d}\ln a^{*}(z) = x < 0$
	\item Uniform Foreign technical progress: $\textrm{d}\ln a^{*}(z) = x < \textrm{d}\ln a(z) = 0$
	\item Technical transfer: Convergence to $a(z) = a^{*}(z)$
\end{itemize}
\end{frame}
% -----------------------------------------
\begin{frame}{Trade costs}
``Iceberg'' trade costs $g(z) = g < 1$ (in fact, \href{https://ideas.repec.org/p/ces/ceswps/_6881.html}{shipping ice is IRS})
\begin{columns}
\begin{column}{.53\textwidth}
\begin{center}\includegraphics[width=\textwidth]{../images/DornbuschFisherSamuelson1977_fig3.pdf}\end{center}
\end{column}
\begin{column}{.45\textwidth}
\begin{itemize}
	\item Home produces if $w a(z) \leq (1/g) w^{*} a^{*}(z)$
	\item Foreign produces if $w^{*} a^{*}(z) \leq (1/g) w a(z)$
	\item Trade balance is $(1-\lambda)w L  = (1-\lambda^{*}) w^{*}L^{*}$
\end{itemize}
\end{column}
\end{columns}
\end{frame}
% -----------------------------------------
\begin{frame}{DFS with non-homothetic preferences (Matsuyama 2000) in one slide}
Switch to hierarchical demand of Murphy, Shleifer, Vishny (1989):
\begin{itemize}
	\item $z \in [0,\infty)$ and cutoff good $m$ given by $w=A(m)$ with $w^{*}=1$ 
	\item $A(z)$ schedule in Figure 1 same as DFS Figure 1
	\begin{equation*} V = \int_{0}^{\infty} b(z)x(z)dz \text { where } x(z) \in \{0,1\} \end{equation*}
	\item $b(z)/a(z)$ and $b(z)/a^{*}(z)$ decreasing so households ``prioritize'' low-$z$ goods
	\item Expenditure to consume up to $z$ is $E(z)\equiv \int_{0}^{z} p(s)ds$ (monotone in utility).
\end{itemize}
\begin{columns}
\begin{column}{.68\textwidth}
Why might income elasticities of goods be interesting?
\begin{itemize}
	\item Terms of trade might be shaped by global growth
	\item Product cycles in which rich buy innovations first
	\item ``Neutral'' productivity shifts aren't neutral
	\item Scope for normative implications
\end{itemize}
\end{column}
\begin{column}{.30\textwidth}
\includegraphics[width=0.85\textwidth]{../images/Matsuyama2000_fig2.pdf}
\end{column}
\end{columns}
\end{frame}
% -----------------------------------------
\begin{frame}{Wrapping up}
Wilson (Ecma, 1980):
\begin{quote}
{\small The DFS paper represents a significant contribution in demonstrating how one might modify the standard Ricardian model in order to make it more tractable for comparative statics analysis.
Their assumptions are so restrictive, however, that the extent to which their approach can be generalized is not readily apparent.
Besides the possibility of relaxing their assumptions on demand, it is not at all clear from their examples how the analysis would proceed if we wished to allow for more than two countries.\par}
\end{quote}
\vfill
Next week: 
Quantitative Ricardian models that handle many countries
\end{frame}
% -----------------------------------------
\end{document}
